% Gemini theme
% https://github.com/anishathalye/gemini

\documentclass[final]{beamer}

% ====================
% Packages
% ====================

\usepackage[T1]{fontenc}
\usepackage{lmodern}
\usepackage[orientation=portrait, size=a0, scale=1.4]{beamerposter}
\usetheme{gemini}
\usecolortheme{mit}
\usepackage{graphicx}
\usepackage{booktabs}
%\usepackage{tikz}
\usepackage{pgfplots}
\pgfplotsset{compat=1.14}
\usepackage{anyfontsize}
\usepackage{tikz-cd}
\usepackage{quiver}
%\usepackage[dvipsnames]{xcolor}
% ====================
% Lengths
% ====================

% If you have N columns, choose \sepwidth and \colwidth such that
% (N+1)*\sepwidth + N*\colwidth = \paperwidth
\newlength{\sepwidth}
\newlength{\colwidth}
\setlength{\sepwidth}{0.025\paperwidth}
\setlength{\colwidth}{0.3\paperwidth}

\newcommand{\separatorcolumn}{\begin{column}{\sepwidth}\end{column}}

% ====================
% Title
% ====================



%Title: A User-Friendly Introduction to Six-Functor Formalism
%
%Abstract: Instead of delving into abstract and highly general formulations, this poster aims to introduce traditional six-functor formalism in an accessible manner. It presents a user-friendly toolkit designed to simplify the understanding and memorization of six-functor formalisms. Additionally, the poster concludes with various applications of six-functor formalism, demonstrating its versatility and utility in different contexts.
%
%For further details and supplementary materials, please visit my GitHub repository: https://github.com/ramified/personal_handwritten_collection/tree/main/applied_six_functor_formalism


\title{A User-Friendly Introduction to Six-Functor Formalism}

\author{Xiaoxiang Zhou \\ Advisor: Prof. Dr. Thomas Krämer}

\institute[shortinst]{Humboldt-Universität zu Berlin}

% ====================
% Footer (optional)
% ====================

\footercontent{
  \href{https://ramified.github.io/}{https://ramified.github.io/} \hfill
  Ramification Heidelberg 2024 \hfill
  \href{mailto:xiaoxiang.zhou@student.hu-berlin.de}{xiaoxiang.zhou@student.hu-berlin.de}}
% (can be left out to remove footer)

% ====================
% Logo (optional)
% ====================

% use this to include logos on the left and/or right side of the header:
% \logoleft{\includegraphics[height=5cm]{BMS_logo.png}}
% \logoleft{\includegraphics[height=7cm]{logo2.pdf}}

% ====================
% Body
% ====================


%%%%%%%newcommand

\definecolor{BrickRed}{rgb}{0.65, 0.12, 0.21}
\definecolor{BrickBlue}{rgb}{0, 0.1, 0.3}
\DeclareMathOperator{\sky}{\operatorname{sky}}
\DeclareMathOperator{\Hcohom}{\operatorname{H}}
\DeclareMathOperator{\BM}{\operatorname{BM}}
\DeclareMathOperator{\cpt}{\operatorname{c}}
\DeclareMathOperator{\Or}{\operatorname{Or}}
\DeclareMathOperator{\pt}{\operatorname{pt}}
\DeclareMathOperator{\constructable}{\operatorname{cons}}
\DeclareMathOperator{\IC}{\operatorname{IC}}
\DeclareMathOperator{\Perv}{\operatorname{Perv}}
\DeclareMathOperator{\cone}{\operatorname{cone}}
\DeclareMathOperator{\CC}{\operatorname{CC}}
\DeclareMathOperator{\NMD}{\operatorname{NMD}}

\begin{document}

\begin{frame}[t]
\begin{columns}[t]
\separatorcolumn

\begin{column}{\colwidth}



  \begin{block}{A Small Toolkit}
  \begin{itemize}
  \item \textbf{Basic examples.}
  
    For $f^!$, assume $Y,X$ are manifolds of dimension $n$.
    {
  \renewcommand{\arraystretch}{1.2}
  \renewcommand{\tabcolsep}{5mm}
   \begin{table}[]
   \centering
   \begin{tabular}{c|c|c}
  \hline
    & $f:Y \longrightarrow \pt$ & $f: p \hookrightarrow X$ \\ \hline
   $f^*$ & constant sheaf & $\mathcal{F}_p$ \\ \hline
   $Rf_*$ & cohomology & $\sky_p(\mathbb{Q})$ \\ \hline
   $Rf_!$ & cpt supp cohomology & $\sky_p(\mathbb{Q})$ \\ \hline
   $f^!$ & orientation sheaf$[n]$ &  costalk \\ \hline
   \end{tabular}
   \caption{}
   \end{table}
   }
  
  
  \item \textbf{Recollement diagram.}
  
  % https://q.uiver.app/#q=WzAsMyxbMCwwLCJaIl0sWzEsMCwiWCJdLFsyLDAsIlUiXSxbMCwxLCJpIiwwLHsic3R5bGUiOnsidGFpbCI6eyJuYW1lIjoiaG9vayIsInNpZGUiOiJ0b3AifX19XSxbMiwxLCJqIiwyLHsic3R5bGUiOnsidGFpbCI6eyJuYW1lIjoiaG9vayIsInNpZGUiOiJib3R0b20ifX19XV0=
  \[\begin{tikzcd}[ampersand replacement=\&]
  	Z \& X \& U
  	\arrow["i", hook, from=1-1, to=1-2]
  	\arrow["j"', hook', from=1-3, to=1-2]
  \end{tikzcd}\]
  
  % https://q.uiver.app/#q=WzAsMyxbMCwwLCJEKFopIl0sWzEsMCwiRChYKSJdLFsyLDAsIkQoVSkiXSxbMCwxLCJpXyo9aV8hIl0sWzEsMiwial4qPWpeISJdLFsyLDEsImpfISIsMix7ImN1cnZlIjozfV0sWzEsMCwiaV4qIiwyLHsiY3VydmUiOjN9XSxbMiwxLCJSal8qIiwwLHsiY3VydmUiOi0yfV0sWzEsMCwiaV4hIiwwLHsiY3VydmUiOi0yfV1d
  \[\begin{tikzcd}[ampersand replacement=\&]
  	{D(Z)} \& {D(X)} \& {D(U)}
  	\arrow["{i_*=i_!}", from=1-1, to=1-2]
  	\arrow["{i^*}"', curve={height=60pt}, from=1-2, to=1-1]
  	\arrow["{i^!}", curve={height=-60pt}, from=1-2, to=1-1]
  	\arrow["{j^*=j^!}", from=1-2, to=1-3]
  	\arrow["{j_!}"', curve={height=60pt}, from=1-3, to=1-2]
  	\arrow["{Rj_*}", curve={height=-60pt}, from=1-3, to=1-2]
  \end{tikzcd}\]
  % https://q.uiver.app/#q=WzAsNCxbMCwwLCJqXyFqXiogXFxtYXRoY2Fse0Z9Il0sWzEsMCwiXFxtYXRoY2Fse0Z9Il0sWzIsMCwiaV8haV4qXFxtYXRoY2Fse0Z9Il0sWzMsMF0sWzAsMV0sWzEsMl0sWzIsMywiKzEiXV0=
  \[\begin{tikzcd}[ampersand replacement=\&]
  	{j_!j^* \mathcal{F}} \& {\mathcal{F}} \& {i_!i^*\mathcal{F}} \& {}
  	\arrow[from=1-1, to=1-2]
  	\arrow[from=1-2, to=1-3]
  	\arrow["{+1}", from=1-3, to=1-4]
  \end{tikzcd}\]
  
  
  \item \textbf{Compatability among functors.}
  
  % https://q.uiver.app/#q=WzAsMyxbMSwwLCJcXG90aW1lcyJdLFswLDEsImZeKiJdLFsyLDEsImZfISJdLFswLDEsImZeKigtXFxvdGltZXMgLSkiLDIseyJzdHlsZSI6eyJoZWFkIjp7Im5hbWUiOiJub25lIn19fV0sWzAsMiwiXFx0ZXh0e3Byb2ogZm9ybXVsYX0iLDAseyJzdHlsZSI6eyJoZWFkIjp7Im5hbWUiOiJub25lIn19fV0sWzEsMiwiXFx0ZXh0e2Jhc2UgY2hhbmdlfSIsMix7InN0eWxlIjp7ImhlYWQiOnsibmFtZSI6Im5vbmUifX19XV0=
  \[\begin{tikzcd}[ampersand replacement=\&]
  	\& \otimes \\[30mm]
  	{f^*} \&\& {f_!}
  	\arrow["{f^*(-\otimes -)}"', no head, from=1-2, to=2-1]
  	\arrow["{\text{proj formula}}", no head, from=1-2, to=2-3]
  	\arrow["{\text{base change}}"', no head, from=2-1, to=2-3]
  \end{tikzcd}\]
  
  
  \end{itemize}






  \end{block}


  \begin{exampleblock}{A Short List of Applications}

Assuming the six-functor formalism (and everything derived), let \( X \) be a smooth manifold of dimension \( n \).

\begin{enumerate}

\item Define four types of cohomology and the relative cohomology.
   Verify that:
   \begin{equation*}
   \begin{aligned}
    \Hcohom^i_{\cpt}(X;\mathbb{Q}) \cong\;&  \Hcohom^i\left(\bar{X}, \{\infty \};\mathbb{Q}\right)\\ 
     \Hcohom_i^{\BM}(X;\mathbb{Q})\cong\;& \Hcohom^{n-i}(X; \Or_X) \\ 
     \Hcohom_i(X;\mathbb{Q})\cong\;& \Hcohom^{n-i}_{\cpt}(X; \Or_X) \\ 
   \end{aligned}
   \end{equation*}
  Also, define the cup and cap product structures.

\item Using the projection formula, show Poincaré duality:
\begin{equation*}
\begin{aligned}
  \Hcohom^i_{\cpt}(X;\mathbb{Q})^*\cong\;& \Hcohom^{n-i}(X; \Or_X) \\ 
  \Hcohom^i(X;\mathbb{Q})  \cong\;& \Hcohom^{n-i}_{\cpt}(X; \Or_X)^* \\ 
\end{aligned}
\end{equation*}

\item Derive the Gysin sequence for any oriented \( S^k \)-bundle \( \pi: E \longrightarrow B \):
% https://q.uiver.app/#q=WzAsNCxbMCwwLCJIXm4oQikiXSxbMSwwLCJIXm4oRSkiXSxbMiwwLCJIXntuLWt9KEIpIl0sWzMsMF0sWzAsMSwiXFxwaV4qIl0sWzEsMiwiXFxwaV8qIl0sWzIsMywiKzEiXSxbMiwzLCJldV97XFxwaX0iLDJdXQ==
\[\begin{tikzcd}[ampersand replacement=\&, column sep = 15mm]
	{H^n(B)} \& {H^n(E)} \& {H^{n-k}(B)} \& {}
	\arrow["{\pi^*}", from=1-1, to=1-2]
	\arrow["{\pi_*}", from=1-2, to=1-3]
	\arrow["{+1}", from=1-3, to=1-4]
	\arrow["{eu_{\pi}}"', from=1-3, to=1-4]
\end{tikzcd}\]

   Derive the Mayer-Vietoris sequence and the relative cohomology sequence, and verify the equivalence of different cohomology groups.

\item Compute the upper shriek for singular spaces.
\end{enumerate}


  \end{exampleblock}

\end{column}

\separatorcolumn

\begin{column}{\colwidth}

  \begin{block}{Perverse Sheaf}

We will mix the usage of sheaves and complexes. For simplicity, let us fix a Whitney stratification $\mathcal{S}$:
\newcommand\largesubset{\mathrel{\scalebox{1.5}[1]{\(\subset\)}}}
$$\varnothing \overset{\;\,U_0}{\largesubset} Z_0 \overset{\;\,U_1}{\largesubset}\cdots  \overset{\;\,U_n}{\largesubset} Z_n = X$$
Denote $D_{\constructable,\mathcal{S}}^{b}(X)$ as the category of constructible sheaves over $X$ with respect to $\mathcal{S}$.

\heading{Definition}

Roughly speaking, a perverse sheaf is a type of sheaf that lies between $\pi^* \mathbb{Q}$ and $\pi^! \mathbb{Q}$. More rigorously, a perverse sheaf is a complex that belongs to the heart of the perverse $t$-structure. 

We say that $\mathcal{F} \in D_{\constructable,\mathcal{S}}^{b}(X)$ is perverse if 
$$
\begin{cases}
\mathcal{H}^i\left(\iota_{U_j}^* \mathcal{F}\right) = 0, & \text{for any } i > -j\\
\mathcal{H}^i\left(\iota_{U_j}^! \mathcal{F}\right) = 0, & \text{for any } i < -j\\
\end{cases}
$$




\heading{Deligne's construction}
Any local system $\mathcal{L}$ supported on $U_i$ can be converted into a perverse sheaf through truncations. This process is known as \textbf{Deligne's construction}, and the resulting perverse sheaf is called the intersection cohomology complex, or the IC sheaf, denoted by $\IC(\mathcal{L})$. IC sheaves are the simple objects in the category $\Perv_{\mathcal{S}}(X)$.

To determine whether a complex $\mathcal{F}$ is {\color{BrickRed}perverse} or an {\color{BrickBlue}IC sheaf}, one simply needs to complete {\color{BrickRed}Table 2}.

  \end{block}
  
  \begin{block}{Nearby Cycle}
A perverse sheaf may not be so ``perverse", but a nearby cycle is definitely ``nearby".

% https://q.uiver.app/#q=WzAsOSxbMCwxLCJcXHswXFx9Il0sWzEsMSwiXFxtYXRoYmJ7Q30iXSxbMiwxLCJcXG1hdGhiYntDfV57XFx0aW1lc30iXSxbMywxLCJcXHdpZGV0aWxkZXtcXG1hdGhiYntDfX1ee1xcdGltZXN9Il0sWzgsMCwiXFxtYXRoY2Fse0Z9Il0sWzQsMSwiXFxtYXRoYmJ7Q30iXSxbNSwwLCJpXipcXG1hdGhjYWx7Rn0iXSxbNiwwLCJcXHBzaSBcXG1hdGhjYWx7Rn0iXSxbNywwLCJcXHZhcnBoaSBcXG1hdGhjYWx7Rn0iXSxbMCwxLCJpIiwwLHsic3R5bGUiOnsidGFpbCI6eyJuYW1lIjoiaG9vayIsInNpZGUiOiJ0b3AifX19XSxbMiwxLCJqIiwyLHsic3R5bGUiOnsidGFpbCI6eyJuYW1lIjoiaG9vayIsInNpZGUiOiJib3R0b20ifX19XSxbMywyLCJwIiwyXSxbMSw0LCIiLDIseyJzdHlsZSI6eyJoZWFkIjp7Im5hbWUiOiJub25lIn19fV0sWzMsNSwiXFxjb25nIiwxLHsic3R5bGUiOnsiYm9keSI6eyJuYW1lIjoibm9uZSJ9LCJoZWFkIjp7Im5hbWUiOiJub25lIn19fV0sWzYsMCwiIiwwLHsic3R5bGUiOnsiaGVhZCI6eyJuYW1lIjoibm9uZSJ9fX1dLFs3LDAsIiIsMCx7InN0eWxlIjp7ImhlYWQiOnsibmFtZSI6Im5vbmUifX19XSxbOCwwLCIiLDAseyJzdHlsZSI6eyJoZWFkIjp7Im5hbWUiOiJub25lIn19fV1d
\[\begin{tikzcd}[ampersand replacement=\&]
	\&\&\&\&[-30mm]\&[-270mm] {\color{BrickRed}i^*\mathcal{F}} \&[-35mm] {\color{BrickRed}\psi \mathcal{F}} \&[-35mm] {\color{BrickRed}\varphi \mathcal{F}} \& {\mathcal{F}} \\[-15mm]
	{\{0\}} \& {\mathbb{C}} \& {\mathbb{C}^{\times}} \& {\widetilde{\mathbb{C}}^{\times}} \& {\mathbb{C}}
	\arrow[no head,color=BrickRed, from=1-6, to=2-1]
	\arrow[no head,color=BrickRed, from=1-7, to=2-1]
	\arrow[no head,color=BrickRed, from=1-8, to=2-1]
	\arrow["i", hook, from=2-1, to=2-2]
	\arrow[no head, from=2-2, to=1-9]
	\arrow["j"', hook', from=2-3, to=2-2]
	\arrow["p"', from=2-4, to=2-3]
	\arrow["\cong"{description}, draw=none, from=2-4, to=2-5]
\end{tikzcd}\]
Given $\mathcal{F} \in D^b(\mathbb{C})$, one can construct the \textbf{nearby cycle} 
$${\color{BrickRed}\psi \mathcal{F}}:= i^* Rj_* p_* p^* j^* \mathcal{F} \in D^b(\{0\}),$$
 which can be roughly viewed as the fiber $\mathcal{F}_x$ for $x$ sufficiently close to $0$. By quotienting out the \textbf{non-vanishing cycle} {\color{BrickRed}$i^*\mathcal{F}$}, one obtains the \textbf{vanishing cycle} $${\color{BrickRed}\varphi \mathcal{F}}:= \cone \left[ i^*\mathcal{F} \overset{sp}{\longrightarrow} \psi \mathcal{F} \right] \in D^b(\{0\}).$$
 
 
% https://q.uiver.app/#q=WzAsOSxbMCwyLCJcXHswXFx9Il0sWzEsMiwiRCJdLFsyLDIsIkReKiJdLFszLDIsIlxcd2lkZXRpbGRle0R9XioiXSxbMCwxLCJYXzAiXSxbMSwxLCJYIl0sWzIsMSwiWF4qIl0sWzMsMSwiXFx3aWRldGlsZGV7WH1eKiJdLFs0LDAsIlxcbWF0aGNhbHtGfSJdLFs3LDNdLFs2LDJdLFs1LDFdLFs0LDBdLFs0LDUsImkiLDAseyJzdHlsZSI6eyJ0YWlsIjp7Im5hbWUiOiJob29rIiwic2lkZSI6InRvcCJ9fX1dLFs2LDUsImoiLDIseyJzdHlsZSI6eyJ0YWlsIjp7Im5hbWUiOiJob29rIiwic2lkZSI6ImJvdHRvbSJ9fX1dLFsyLDEsIiIsMSx7InN0eWxlIjp7InRhaWwiOnsibmFtZSI6Imhvb2siLCJzaWRlIjoiYm90dG9tIn19fV0sWzAsMSwiIiwxLHsic3R5bGUiOnsidGFpbCI6eyJuYW1lIjoiaG9vayIsInNpZGUiOiJ0b3AifX19XSxbNyw2LCJwIiwyXSxbMywyXSxbNSw4LCIiLDIseyJzdHlsZSI6eyJoZWFkIjp7Im5hbWUiOiJub25lIn19fV1d
\[\begin{tikzcd}[ampersand replacement=\&]
	\&\&\&\&[-150mm] {\mathcal{F}} \\[-15mm]
	{X_0} \& X \& {X^*} \& {\widetilde{X}^*} \\
	{\{0\}} \& D \& {D^*} \& {\widetilde{D}^*}
	\arrow["i", hook, from=2-1, to=2-2]
	\arrow[from=2-1, to=3-1]
	\arrow[no head, from=2-2, to=1-5]
	\arrow[from=2-2, to=3-2]
	\arrow["j"', hook', from=2-3, to=2-2]
	\arrow[from=2-3, to=3-3]
	\arrow["p"', from=2-4, to=2-3]
	\arrow[from=2-4, to=3-4]
	\arrow[hook, from=3-1, to=3-2]
	\arrow[hook', from=3-3, to=3-2]
	\arrow[from=3-4, to=3-3]
\end{tikzcd}\]
  In general, $\mathcal{C}$ can be replaced by any disk $\mathcal{D}$, as the problem is local, and $\mathcal{F}$ can be a sheaf over any space $X$ over $\mathcal{D}$.
  
 The same construction yields a distinguished triangle in $D^b(X_0)$:
 % https://q.uiver.app/#q=WzAsNCxbMCwwLCJpXiogXFxtYXRoY2Fse0Z9Il0sWzEsMCwiXFxwc2lfZiBcXG1hdGhjYWx7Rn0iXSxbMiwwLCJcXHZhcnBoaV9mIFxcbWF0aGNhbHtGfSJdLFszLDBdLFswLDFdLFsxLDJdLFsyLDMsIisxIl1d
 \[\begin{tikzcd}[ampersand replacement=\&]
 	{i^* \mathcal{F}} \& {\psi_f \mathcal{F}} \& {\varphi_f \mathcal{F}} \& {}
 	\arrow[from=1-1, to=1-2]
 	\arrow[from=1-2, to=1-3]
 	\arrow["{+1}", from=1-3, to=1-4]
 \end{tikzcd}\]
  \end{block}




\end{column}

\separatorcolumn

\begin{column}{\colwidth}

  \begin{alertblock}{NMD and CC}
\heading{Normal Morse Data}  
We work with a fixed complex variety embedding $X \subseteq \mathbb{C}^n$, equipped with a Whitney stratification $\mathcal{S}$. Let $S \subseteq X$ be a connected component of some $U_i$. Fix $x_0 \in S$, and let $N$ be a normal slice of $S$ at $x_0$.

For any sheaf $\mathcal{F} \in D^b_{\constructable, \mathcal{S}}(X)$, the normal Morse data is defined as
$$\NMD(\mathcal{F}, S):= \left( \varphi_{g|_{N \cap X}} \left(\mathcal{F}|_{N \cap X}\right)  \right)_{x_0} [-1]$$
where $g: \mathbb{C}^n \longrightarrow \mathbb{C}$ is a holomorphic function, and $f:= Re(g)$ such that 
\begin{itemize}
\item $g(x_0)=0$;
\item $df_{x_0} \in T^*_S \mathbb{C}^n$, $df_{x_0} \notin T^*_{S'} \mathbb{C}^n$ for any $S'\neq S$;
\item $x_0$ is a non-degenerate critical point of $f|_{S}$.
\end{itemize}  
  
\heading{Characteristic Cycle}
With normal Morse data, one can define the characteristic cycle
$$\CC(\mathcal{F}):= \sum_S m_S [T_S^* \mathbb{C}^n] \in H_{2n}^{\BM}\Big(\operatornamewithlimits{\cup}_S T_S^* \mathbb{C}^n\Big),$$
where
$$m_S:= \chi\left(\NMD(\mathcal{F},S)\rule{0mm}{10mm}[-\dim S]\right) \in \mathbb{Z}.$$

%Notably, $\CC(\mathcal{F})$ does not depend on the stratification $\mathcal{S}$.

The characteristic cycle can be computed when the geometry is well-understood. For example, we can compute $\CC(\IC(\underline{\mathbb{Q}}_{X \smallsetminus \{0\}}))$ when $X \subseteq \mathbb{C}^m$ is a cone over a smooth hyperplane.

  \end{alertblock}

 
  \begin{block}{References}
  $\;$\\[-30mm] $\,$
    \nocite{*}
    \footnotesize{\bibliographystyle{plain}\bibliography{poster}}
    
    {
      \renewcommand{\arraystretch}{1.2}
      \renewcommand{\tabcolsep}{5mm}
      \newcommand{\pervcross}{\textcolor{BrickRed}{$\boldsymbol{\times}$}  }
      \newcommand{\ICcross}{\textcolor{BrickBlue}{$\boldsymbol{\times}$}  }
    \begin{table}[]
    \centering
    \begin{tabular}{|c|c|c|c|c|c|c|c|}
    \hline
               $\mathcal{H}^i(-)$                  & $-3$ & $-2$ & $-1$ & $\;0\;$ & $\;1\;$ & $\;2\;$ & $\;3\;$ \\ \hline
    $\iota_{U_2}^*\mathcal{F} = \iota_{U_2}^!\mathcal{F}$ & \pervcross   &      & \pervcross     &  \pervcross   &  \pervcross   & \pervcross    &  \pervcross   \\ \hline
    $\iota_{U_1}^*\mathcal{F}$ &      &      &  \ICcross    &  \pervcross   &  \pervcross   &  \pervcross   &  \pervcross   \\ \hline
    $\iota_{U_1}^!\mathcal{F}$ &  \pervcross    &  \pervcross    &  \ICcross    &     &     &     &     \\ \hline
    $\iota_{Z_0}^*\mathcal{F}$ &      &      &      &  \ICcross   &   \pervcross  &   \pervcross  &  \pervcross   \\ \hline
    $\iota_{Z_0}^!\mathcal{F}$ &   \pervcross   &  \pervcross    &  \pervcross    &  \ICcross   &     &     &     \\ \hline
    \end{tabular}
    \caption{Sheaf verification for $\dim_{\mathbb{C}} X = 2$.}
    \end{table}
    }

  \end{block}

\end{column}

\separatorcolumn
\end{columns}
\end{frame}

\end{document}
