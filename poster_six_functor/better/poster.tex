% Gemini theme
% https://github.com/anishathalye/gemini

\documentclass[final]{beamer}

% ====================
% Packages
% ====================

\usepackage[T1]{fontenc}
\usepackage{lmodern}
\usepackage[orientation=portrait, size=a0, scale=1.4]{beamerposter}
\usetheme{gemini}
\usecolortheme{mit}
\usepackage{graphicx}
\usepackage{booktabs}
%\usepackage{tikz}
\usepackage{pgfplots}
\pgfplotsset{compat=1.14}
\usepackage{anyfontsize}
\usepackage{tikz-cd}
\usepackage{quiver}
%\usepackage[dvipsnames]{xcolor}
% ====================
% Lengths
% ====================

% If you have N columns, choose \sepwidth and \colwidth such that
% (N+1)*\sepwidth + N*\colwidth = \paperwidth
\newlength{\sepwidth}
\newlength{\colwidth}
\setlength{\sepwidth}{0.025\paperwidth}
\setlength{\colwidth}{0.3\paperwidth}

\newcommand{\separatorcolumn}{\begin{column}{\sepwidth}\end{column}}

% ====================
% Title
% ====================



%Title: A User-Friendly Introduction to Six-Functor Formalism
%
%Abstract: Instead of delving into abstract and highly general formulations, this poster aims to introduce traditional six-functor formalism in an accessible manner. It presents a user-friendly toolkit designed to simplify the understanding and memorization of six-functor formalisms. Additionally, the poster concludes with various applications of six-functor formalism, demonstrating its versatility and utility in different contexts.
%
%For further details and supplementary materials, please visit my GitHub repository: https://github.com/ramified/personal_handwritten_collection/tree/main/applied_six_functor_formalism


\title{A User-Friendly Introduction to Six-Functor Formalism}

\author{Xiaoxiang Zhou \\ Advisor: Prof. Dr. Thomas Krämer}

\institute[shortinst]{Humboldt-Universität zu Berlin}

% ====================
% Footer (optional)
% ====================

\footercontent{
  \href{https://ramified.github.io/}{https://ramified.github.io/} \hfill
  Ramification Heidelberg 2024 \hfill
  \href{mailto:xiaoxiang.zhou@student.hu-berlin.de}{xiaoxiang.zhou@student.hu-berlin.de}}
% (can be left out to remove footer)

% ====================
% Logo (optional)
% ====================

% use this to include logos on the left and/or right side of the header:
% \logoright{\includegraphics[height=7cm]{logo1.pdf}}
% \logoleft{\includegraphics[height=7cm]{logo2.pdf}}

% ====================
% Body
% ====================


%%%%%%%newcommand
\DeclareMathOperator{\sky}{\operatorname{sky}}
\DeclareMathOperator{\Hcohom}{\operatorname{H}}
\DeclareMathOperator{\BM}{\operatorname{BM}}
\DeclareMathOperator{\cpt}{\operatorname{c}}
\DeclareMathOperator{\Or}{\operatorname{Or}}
\DeclareMathOperator{\pt}{\operatorname{pt}}
\DeclareMathOperator{\constructable}{\operatorname{cons}}
\DeclareMathOperator{\IC}{\operatorname{IC}}
\DeclareMathOperator{\Perv}{\operatorname{Perv}}

\begin{document}

\begin{frame}[t]
\begin{columns}[t]
\separatorcolumn

\begin{column}{\colwidth}



  \begin{block}{A Small Toolkit}
  \begin{itemize}
  \item \textbf{Basic examples.}
  
    For $f^!$, assume $Y,X$ are manifolds of dimension $n$.
    {
  \renewcommand{\arraystretch}{1.2}
  \renewcommand{\tabcolsep}{5mm}
   \begin{table}[]
   \centering
   \begin{tabular}{c|c|c}
  \hline
    & $f:Y \longrightarrow \pt$ & $f: p \hookrightarrow X$ \\ \hline
   $f^*$ & constant sheaf & $\mathcal{F}_p$ \\ \hline
   $Rf_*$ & cohomology & $\sky_p(\mathbb{Q})$ \\ \hline
   $Rf_!$ & cpt supp cohomology & $\sky_p(\mathbb{Q})$ \\ \hline
   $f^!$ & orientation sheaf$[n]$ &  $\mathcal{F}_p[n]$ \\ \hline
   \end{tabular}
   \end{table}
   }
  
  
  \item \textbf{Recollement diagram.}
  
  % https://q.uiver.app/#q=WzAsMyxbMCwwLCJaIl0sWzEsMCwiWCJdLFsyLDAsIlUiXSxbMCwxLCJpIiwwLHsic3R5bGUiOnsidGFpbCI6eyJuYW1lIjoiaG9vayIsInNpZGUiOiJ0b3AifX19XSxbMiwxLCJqIiwyLHsic3R5bGUiOnsidGFpbCI6eyJuYW1lIjoiaG9vayIsInNpZGUiOiJib3R0b20ifX19XV0=
  \[\begin{tikzcd}[ampersand replacement=\&]
  	Z \& X \& U
  	\arrow["i", hook, from=1-1, to=1-2]
  	\arrow["j"', hook', from=1-3, to=1-2]
  \end{tikzcd}\]
  
  % https://q.uiver.app/#q=WzAsMyxbMCwwLCJEKFopIl0sWzEsMCwiRChYKSJdLFsyLDAsIkQoVSkiXSxbMCwxLCJpXyo9aV8hIl0sWzEsMiwial4qPWpeISJdLFsyLDEsImpfISIsMix7ImN1cnZlIjozfV0sWzEsMCwiaV4qIiwyLHsiY3VydmUiOjN9XSxbMiwxLCJSal8qIiwwLHsiY3VydmUiOi0yfV0sWzEsMCwiaV4hIiwwLHsiY3VydmUiOi0yfV1d
  \[\begin{tikzcd}[ampersand replacement=\&]
  	{D(Z)} \& {D(X)} \& {D(U)}
  	\arrow["{i_*=i_!}", from=1-1, to=1-2]
  	\arrow["{i^*}"', curve={height=60pt}, from=1-2, to=1-1]
  	\arrow["{i^!}", curve={height=-60pt}, from=1-2, to=1-1]
  	\arrow["{j^*=j^!}", from=1-2, to=1-3]
  	\arrow["{j_!}"', curve={height=60pt}, from=1-3, to=1-2]
  	\arrow["{Rj_*}", curve={height=-60pt}, from=1-3, to=1-2]
  \end{tikzcd}\]
  % https://q.uiver.app/#q=WzAsNCxbMCwwLCJqXyFqXiogXFxtYXRoY2Fse0Z9Il0sWzEsMCwiXFxtYXRoY2Fse0Z9Il0sWzIsMCwiaV8haV4qXFxtYXRoY2Fse0Z9Il0sWzMsMF0sWzAsMV0sWzEsMl0sWzIsMywiKzEiXV0=
  \[\begin{tikzcd}[ampersand replacement=\&]
  	{j_!j^* \mathcal{F}} \& {\mathcal{F}} \& {i_!i^*\mathcal{F}} \& {}
  	\arrow[from=1-1, to=1-2]
  	\arrow[from=1-2, to=1-3]
  	\arrow["{+1}", from=1-3, to=1-4]
  \end{tikzcd}\]
  
  
  \item \textbf{Compatability among functors.}
  
  % https://q.uiver.app/#q=WzAsMyxbMSwwLCJcXG90aW1lcyJdLFswLDEsImZeKiJdLFsyLDEsImZfISJdLFswLDEsImZeKigtXFxvdGltZXMgLSkiLDIseyJzdHlsZSI6eyJoZWFkIjp7Im5hbWUiOiJub25lIn19fV0sWzAsMiwiXFx0ZXh0e3Byb2ogZm9ybXVsYX0iLDAseyJzdHlsZSI6eyJoZWFkIjp7Im5hbWUiOiJub25lIn19fV0sWzEsMiwiXFx0ZXh0e2Jhc2UgY2hhbmdlfSIsMix7InN0eWxlIjp7ImhlYWQiOnsibmFtZSI6Im5vbmUifX19XV0=
  \[\begin{tikzcd}[ampersand replacement=\&]
  	\& \otimes \\[30mm]
  	{f^*} \&\& {f_!}
  	\arrow["{f^*(-\otimes -)}"', no head, from=1-2, to=2-1]
  	\arrow["{\text{proj formula}}", no head, from=1-2, to=2-3]
  	\arrow["{\text{base change}}"', no head, from=2-1, to=2-3]
  \end{tikzcd}\]
  
  
  \end{itemize}






  \end{block}


  \begin{alertblock}{A Short List of Applications}

Assuming the six-functor formalism (and everything derived), let \( X \) be a smooth manifold of dimension \( n \).

\begin{enumerate}

\item Define four types of cohomology and the relative cohomology.
   Verify that:
$$\Hcohom^i_{\cpt}(X;\mathbb{Q}) \cong \Hcohom^i\left(\bar{X}, \{\infty \};\mathbb{Q}\right)$$
$$\Hcohom_i^{\BM}(X;\mathbb{Q}) \cong \Hcohom^{n-i}(X; \Or_X)$$
$$\Hcohom_i(X;\mathbb{Q}) \cong \Hcohom^{n-i}_{\cpt}(X; \Or_X)$$
  Also, define the cup and cap product structures.

\item Using the projection formula, show Poincaré duality:
$$\Hcohom^i_{\cpt}(X;\mathbb{Q})^* \cong \Hcohom^{n-i}(X; \Or_X)$$
$$\Hcohom^i(X;\mathbb{Q}) \cong \Hcohom^{n-i}_{\cpt}(X; \Or_X)^*$$


\item Derive the Gysin sequence for any oriented \( S^k \)-bundle \( \pi: E \longrightarrow B \):
% https://q.uiver.app/#q=WzAsNCxbMCwwLCJIXm4oQikiXSxbMSwwLCJIXm4oRSkiXSxbMiwwLCJIXntuLWt9KEIpIl0sWzMsMF0sWzAsMSwiXFxwaV4qIl0sWzEsMiwiXFxwaV8qIl0sWzIsMywiKzEiXSxbMiwzLCJldV97XFxwaX0iLDJdXQ==
\[\begin{tikzcd}[ampersand replacement=\&, column sep = 15mm]
	{H^n(B)} \& {H^n(E)} \& {H^{n-k}(B)} \& {}
	\arrow["{\pi^*}", from=1-1, to=1-2]
	\arrow["{\pi_*}", from=1-2, to=1-3]
	\arrow["{+1}", from=1-3, to=1-4]
	\arrow["{eu_{\pi}}"', from=1-3, to=1-4]
\end{tikzcd}\]

   Derive the Mayer-Vietoris sequence and the relative cohomology sequence, and verify the equivalence of different cohomology groups.

\item Compute the upper shriek for singular spaces.
\end{enumerate}


  \end{alertblock}

\end{column}

\separatorcolumn

\begin{column}{\colwidth}

  \begin{block}{Perverse Sheaf}

We will mix the usage of sheaves and complexes. For simplicity, let us fix a stratification $\mathcal{S}$:
\newcommand\largesubset{\mathrel{\scalebox{1.5}[1]{\(\subset\)}}}
$$\varnothing \overset{\;\,U_0}{\largesubset} Z_0 \overset{\;\,U_1}{\largesubset}\cdots  \overset{\;\,U_n}{\largesubset} Z_n = X$$
Denote $D_{\constructable,\mathcal{S}}^{b}(X)$ as the category of constructible sheaves over $X$ with respect to $\mathcal{S}$.

\heading{Definition}

Roughly speaking, a perverse sheaf is a type of sheaf that lies between $\pi^* \mathbb{Q}$ and $\pi^! \mathbb{Q}$. More rigorously, a perverse sheaf is a complex that belongs to the heart of the perverse $t$-structure. 

We say that $\mathcal{F} \in D_{\constructable,\mathcal{S}}^{b}(X)$ is perverse if 
$$
\begin{cases}
\mathcal{H}^i\left(\iota_{U_j}^* \mathcal{F}\right) = 0, & \text{for any } i > -j\\
\mathcal{H}^i\left(\iota_{U_j}^! \mathcal{F}\right) = 0, & \text{for any } i < -j\\
\end{cases}
$$


To determine whether a complex $\mathcal{F}$ is perverse, one simply needs to complete the following table:
{
  \renewcommand{\arraystretch}{1.2}
  \renewcommand{\tabcolsep}{5mm}
  \definecolor{BrickRed}{rgb}{0.8, 0.25, 0.33}
  \definecolor{BrickBlue}{rgb}{0, 0.1, 0.3}
  \newcommand{\pervcross}{\textcolor{BrickRed}{$\boldsymbol{\times}$}  }
  \newcommand{\ICcross}{\textcolor{BrickBlue}{$\boldsymbol{\times}$}  }
\begin{table}[]
\centering
\begin{tabular}{|c|c|c|c|c|c|c|c|}
\hline
           $\mathcal{H}^i(-)$                  & $-3$ & $-2$ & $-1$ & $\;0\;$ & $\;1\;$ & $\;2\;$ & $\;3\;$ \\ \hline
$\iota_{U_2}^*\mathcal{F}$ & \pervcross   &      & \pervcross     &  \pervcross   &  \pervcross   & \pervcross    &  \pervcross   \\ \hline
$\iota_{U_1}^*\mathcal{F}$ &      &      &  \ICcross    &  \pervcross   &  \pervcross   &  \pervcross   &  \pervcross   \\ \hline
$\iota_{U_1}^!\mathcal{F}$ &  \pervcross    &  \pervcross    &  \ICcross    &     &     &     &     \\ \hline
$\iota_{Z_0}^*\mathcal{F}$ &      &      &      &  \ICcross   &   \pervcross  &   \pervcross  &  \pervcross   \\ \hline
$\iota_{Z_0}^!\mathcal{F}$ &   \pervcross   &  \pervcross    &  \pervcross    &  \ICcross   &     &     &     \\ \hline
\end{tabular}
\end{table}
}
\heading{Deligne's construction}
Any local system $\mathcal{L}$ supported on $U_i$ can be converted into a perverse sheaf through truncations. This process is known as \textbf{Deligne's construction}, and the resulting perverse sheaf is called the intersection cohomology complex, or the IC sheaf, denoted by $\IC(\mathcal{L})$. IC sheaves are the simple objects in the category $\Perv_{\mathcal{S}}(X)$.

  \end{block}

  \begin{block}{Fusce aliquam magna velit}

    Et rutrum ex euismod vel. Pellentesque ultricies, velit in fermentum
    vestibulum, lectus nisi pretium nibh, sit amet aliquam lectus augue vel
    velit. Suspendisse rhoncus massa porttitor augue feugiat molestie. Sed
    molestie ut orci nec malesuada. Sed ultricies feugiat est fringilla
    posuere.

    \begin{figure}
      \centering
      \begin{tikzpicture}
        \begin{axis}[
            scale only axis,
            no markers,
            domain=0:2*pi,
            samples=100,
            axis lines=center,
            axis line style={-},
            ticks=none]
          \addplot[red] {sin(deg(x))};
          \addplot[blue] {cos(deg(x))};
        \end{axis}
      \end{tikzpicture}
      \caption{Another figure caption.}
    \end{figure}

  \end{block}


\end{column}

\separatorcolumn

\begin{column}{\colwidth}

  \begin{exampleblock}{A highlighted block containing some math}

    A different kind of highlighted block.

    $$
    \int_{-\infty}^{\infty} e^{-x^2}\,dx = \sqrt{\pi}
    $$

    Interdum et malesuada fames $\{1, 4, 9, \ldots\}$ ac ante ipsum primis in
    faucibus. Cras eleifend dolor eu nulla suscipit suscipit. Sed lobortis non
    felis id vulputate.

    \heading{A heading inside a block}

    Praesent consectetur mi $x^2 + y^2$ metus, nec vestibulum justo viverra
    nec. Proin eget nulla pretium, egestas magna aliquam, mollis neque. Vivamus
    dictum $\mathbf{u}^\intercal\mathbf{v}$ sagittis odio, vel porta erat
    congue sed. Maecenas ut dolor quis arcu auctor porttitor.

    \heading{Another heading inside a block}

    Sed augue erat, scelerisque a purus ultricies, placerat porttitor neque.
    Donec $P(y \mid x)$ fermentum consectetur $\nabla_x P(y \mid x)$ sapien
    sagittis egestas. Duis eget leo euismod nunc viverra imperdiet nec id
    justo.

  \end{exampleblock}

 
  \begin{block}{References}

    \nocite{*}
    \footnotesize{\bibliographystyle{plain}\bibliography{poster}}

  \end{block}

\end{column}

\separatorcolumn
\end{columns}
\end{frame}

\end{document}
