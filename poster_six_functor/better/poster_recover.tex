% Gemini theme
% https://github.com/anishathalye/gemini

\documentclass[final]{beamer}

% ====================
% Packages
% ====================

\usepackage[T1]{fontenc}
\usepackage{lmodern}
\usepackage[orientation=portrait, size=a0, scale=1.4]{beamerposter}
\usetheme{gemini}
\usecolortheme{mit}
\usepackage{graphicx}
\usepackage{booktabs}
%\usepackage{tikz}
\usepackage{pgfplots}
\pgfplotsset{compat=1.14}
\usepackage{anyfontsize}
\usepackage{tikz-cd}
\usepackage{quiver}

% ====================
% Lengths
% ====================

% If you have N columns, choose \sepwidth and \colwidth such that
% (N+1)*\sepwidth + N*\colwidth = \paperwidth
\newlength{\sepwidth}
\newlength{\colwidth}
\setlength{\sepwidth}{0.025\paperwidth}
\setlength{\colwidth}{0.3\paperwidth}

\newcommand{\separatorcolumn}{\begin{column}{\sepwidth}\end{column}}

% ====================
% Title
% ====================



%Title: A User-Friendly Introduction to Six-Functor Formalism
%
%Abstract: Instead of delving into abstract and highly general formulations, this poster aims to introduce traditional six-functor formalism in an accessible manner. It presents a user-friendly toolkit designed to simplify the understanding and memorization of six-functor formalisms. Additionally, the poster concludes with various applications of six-functor formalism, demonstrating its versatility and utility in different contexts.
%
%For further details and supplementary materials, please visit my GitHub repository: https://github.com/ramified/personal_handwritten_collection/tree/main/applied_six_functor_formalism


\title{A User-Friendly Introduction to Six-Functor Formalism}

\author{Xiaoxiang Zhou}

\institute[shortinst]{Humboldt-Universität zu Berlin}

% ====================
% Footer (optional)
% ====================

\footercontent{
  \href{https://ramified.github.io/}{https://ramified.github.io/} \hfill
  Ramification Heidelberg 2024 \hfill
  \href{mailto:xiaoxiang.zhou@student.hu-berlin.de}{xiaoxiang.zhou@student.hu-berlin.de}}
% (can be left out to remove footer)

% ====================
% Logo (optional)
% ====================

% use this to include logos on the left and/or right side of the header:
% \logoright{\includegraphics[height=7cm]{logo1.pdf}}
% \logoleft{\includegraphics[height=7cm]{logo2.pdf}}

% ====================
% Body
% ====================


%%%%%%%newcommand
\DeclareMathOperator{\sky}{\operatorname{sky}}
\DeclareMathOperator{\Hcohom}{\operatorname{H}}
\DeclareMathOperator{\BM}{\operatorname{BM}}
\DeclareMathOperator{\cpt}{\operatorname{c}}
\DeclareMathOperator{\Or}{\operatorname{Or}}
\DeclareMathOperator{\pt}{\operatorname{pt}}



\begin{document}

\begin{frame}[t]
\begin{columns}[t]
\separatorcolumn

\begin{column}{\colwidth}



  \begin{block}{A Small Toolkit}
  \begin{itemize}
  \item Basic example.
  \item Open-closed formalism.
  \item Compatability among functors.
  \end{itemize}
  For $f^!$, assume $Y,X$ are manifolds of dimension $n$.
  {
\renewcommand{\arraystretch}{1.2}
\renewcommand{\tabcolsep}{5mm}
 \begin{table}[]
 \centering
 \begin{tabular}{c|c|c}
\hline
  & $f:Y \longrightarrow \pt$ & $f: p \hookrightarrow X$ \\ \hline
 $f^*$ & constant sheaf & $\mathcal{F}_p$ \\ \hline
 $Rf_*$ & cohomology & $\sky_p(\mathbb{Q})$ \\ \hline
 $Rf_!$ & cpt supp cohomology & $\sky_p(\mathbb{Q})$ \\ \hline
 $f^!$ & orientation sheaf$[n]$ &  $\mathcal{F}_p[n]$ \\ \hline
 \end{tabular}
 \end{table}
 }
% https://q.uiver.app/#q=WzAsMyxbMCwwLCJaIl0sWzEsMCwiWCJdLFsyLDAsIlUiXSxbMCwxLCJpIiwwLHsic3R5bGUiOnsidGFpbCI6eyJuYW1lIjoiaG9vayIsInNpZGUiOiJ0b3AifX19XSxbMiwxLCJqIiwyLHsic3R5bGUiOnsidGFpbCI6eyJuYW1lIjoiaG9vayIsInNpZGUiOiJib3R0b20ifX19XV0=
\[\begin{tikzcd}[ampersand replacement=\&]
	Z \& X \& U
	\arrow["i", hook, from=1-1, to=1-2]
	\arrow["j"', hook', from=1-3, to=1-2]
\end{tikzcd}\]

% https://q.uiver.app/#q=WzAsMyxbMCwwLCJEKFopIl0sWzEsMCwiRChYKSJdLFsyLDAsIkQoVSkiXSxbMCwxLCJpXyo9aV8hIl0sWzEsMiwial4qPWpeISJdLFsyLDEsImpfISIsMix7ImN1cnZlIjozfV0sWzEsMCwiaV4qIiwyLHsiY3VydmUiOjN9XSxbMiwxLCJSal8qIiwwLHsiY3VydmUiOi0yfV0sWzEsMCwiaV4hIiwwLHsiY3VydmUiOi0yfV1d
\[\begin{tikzcd}[ampersand replacement=\&]
	{D(Z)} \& {D(X)} \& {D(U)}
	\arrow["{i_*=i_!}", from=1-1, to=1-2]
	\arrow["{i^*}"', curve={height=60pt}, from=1-2, to=1-1]
	\arrow["{i^!}", curve={height=-60pt}, from=1-2, to=1-1]
	\arrow["{j^*=j^!}", from=1-2, to=1-3]
	\arrow["{j_!}"', curve={height=60pt}, from=1-3, to=1-2]
	\arrow["{Rj_*}", curve={height=-60pt}, from=1-3, to=1-2]
\end{tikzcd}\]
% https://q.uiver.app/#q=WzAsNCxbMCwwLCJqXyFqXiogXFxtYXRoY2Fse0Z9Il0sWzEsMCwiXFxtYXRoY2Fse0Z9Il0sWzIsMCwiaV8haV4qXFxtYXRoY2Fse0Z9Il0sWzMsMF0sWzAsMV0sWzEsMl0sWzIsMywiKzEiXV0=
\[\begin{tikzcd}[ampersand replacement=\&]
	{j_!j^* \mathcal{F}} \& {\mathcal{F}} \& {i_!i^*\mathcal{F}} \& {}
	\arrow[from=1-1, to=1-2]
	\arrow[from=1-2, to=1-3]
	\arrow["{+1}", from=1-3, to=1-4]
\end{tikzcd}\]


% https://q.uiver.app/#q=WzAsMyxbMSwwLCJcXG90aW1lcyJdLFswLDEsImZeKiJdLFsyLDEsImZfISJdLFswLDEsImZeKigtXFxvdGltZXMgLSkiLDIseyJzdHlsZSI6eyJoZWFkIjp7Im5hbWUiOiJub25lIn19fV0sWzAsMiwiXFx0ZXh0e3Byb2ogZm9ybXVsYX0iLDAseyJzdHlsZSI6eyJoZWFkIjp7Im5hbWUiOiJub25lIn19fV0sWzEsMiwiXFx0ZXh0e2Jhc2UgY2hhbmdlfSIsMix7InN0eWxlIjp7ImhlYWQiOnsibmFtZSI6Im5vbmUifX19XV0=
\[\begin{tikzcd}[ampersand replacement=\&]
	\& \otimes \\[30mm]
	{f^*} \&\& {f_!}
	\arrow["{f^*(-\otimes -)}"', no head, from=1-2, to=2-1]
	\arrow["{\text{proj formula}}", no head, from=1-2, to=2-3]
	\arrow["{\text{base change}}"', no head, from=2-1, to=2-3]
\end{tikzcd}\]

  \end{block}


  \begin{alertblock}{A Short List of Applications}

Assuming the six-functor formalism (and everything derived), let \( X \) be a smooth manifold of dimension \( n \).

\begin{enumerate}

\item Define four types of cohomology and the relative cohomology.
   Verify that:
$$\Hcohom^i_{\cpt}(X;\mathbb{Q}) \cong \Hcohom^i\left(\bar{X}, \{\infty \};\mathbb{Q}\right)$$
$$\Hcohom_i^{\BM}(X;\mathbb{Q}) \cong \Hcohom^{n-i}(X; \Or_X)$$
$$\Hcohom_i(X;\mathbb{Q}) \cong \Hcohom^{n-i}_{\cpt}(X; \Or_X)$$
  Also, define the cup and cap product structures.

\item Using the projection formula, show Poincaré duality:
$$\Hcohom^i_{\cpt}(X;\mathbb{Q})^* \cong \Hcohom^{n-i}(X; \Or_X)$$
$$\Hcohom^i(X;\mathbb{Q}) \cong \Hcohom^{n-i}_{\cpt}(X; \Or_X)^*$$


\item Derive the Gysin sequence for any oriented \( S^k \)-bundle \( \pi: E \longrightarrow B \):
% https://q.uiver.app/#q=WzAsNCxbMCwwLCJIXm4oQikiXSxbMSwwLCJIXm4oRSkiXSxbMiwwLCJIXntuLWt9KEIpIl0sWzMsMF0sWzAsMSwiXFxwaV4qIl0sWzEsMiwiXFxwaV8qIl0sWzIsMywiKzEiXSxbMiwzLCJldV97XFxwaX0iLDJdXQ==
\[\begin{tikzcd}[ampersand replacement=\&, column sep = 15mm]
	{H^n(B)} \& {H^n(E)} \& {H^{n-k}(B)} \& {}
	\arrow["{\pi^*}", from=1-1, to=1-2]
	\arrow["{\pi_*}", from=1-2, to=1-3]
	\arrow["{+1}", from=1-3, to=1-4]
	\arrow["{eu_{\pi}}"', from=1-3, to=1-4]
\end{tikzcd}\]

   Derive the Mayer-Vietoris sequence and the relative cohomology sequence, and verify the equivalence of different cohomology groups.

\item Compute the upper shriek for singular spaces.
\end{enumerate}


  \end{alertblock}

\end{column}

\separatorcolumn

\begin{column}{\colwidth}

  \begin{block}{Perverse Sheaf}

    Vivamus congue volutpat elit non semper. Praesent molestie nec erat ac
    interdum. In quis suscipit erat. \textbf{Phasellus mauris felis, molestie
    ac pharetra quis}, tempus nec ante. Donec finibus ante vel purus mollis
    fermentum. Sed felis mi, pharetra eget nibh a, feugiat eleifend dolor. Nam
    mollis condimentum purus quis sodales. Nullam eu felis eu nulla eleifend
    bibendum nec eu lorem. Vivamus felis velit, volutpat ut facilisis ac,
    commodo in metus.

    \begin{enumerate}
      \item \textbf{Morbi mauris purus}, egestas at vehicula et, convallis
        accumsan orci. Orci varius natoque penatibus et magnis dis parturient
        montes, nascetur ridiculus mus.
      \item \textbf{Cras vehicula blandit urna ut maximus}. Aliquam blandit nec
        massa ac sollicitudin. Curabitur cursus, metus nec imperdiet bibendum,
        velit lectus faucibus dolor, quis gravida metus mauris gravida turpis.
      \item \textbf{Vestibulum et massa diam}. Phasellus fermentum augue non
        nulla accumsan, non rhoncus lectus condimentum.
    \end{enumerate}

  \end{block}

  \begin{block}{Fusce aliquam magna velit}

    Et rutrum ex euismod vel. Pellentesque ultricies, velit in fermentum
    vestibulum, lectus nisi pretium nibh, sit amet aliquam lectus augue vel
    velit. Suspendisse rhoncus massa porttitor augue feugiat molestie. Sed
    molestie ut orci nec malesuada. Sed ultricies feugiat est fringilla
    posuere.

    \begin{figure}
      \centering
      \begin{tikzpicture}
        \begin{axis}[
            scale only axis,
            no markers,
            domain=0:2*pi,
            samples=100,
            axis lines=center,
            axis line style={-},
            ticks=none]
          \addplot[red] {sin(deg(x))};
          \addplot[blue] {cos(deg(x))};
        \end{axis}
      \end{tikzpicture}
      \caption{Another figure caption.}
    \end{figure}

  \end{block}

  \begin{block}{Nam cursus consequat egestas}

    Nulla eget sem quam. Ut aliquam volutpat nisi vestibulum convallis. Nunc a
    lectus et eros facilisis hendrerit eu non urna. Interdum et malesuada fames
    ac ante \textit{ipsum primis} in faucibus. Etiam sit amet velit eget sem
    euismod tristique. Praesent enim erat, porta vel mattis sed, pharetra sed
    ipsum. Morbi commodo condimentum massa, \textit{tempus venenatis} massa
    hendrerit quis. Maecenas sed porta est. Praesent mollis interdum lectus,
    sit amet sollicitudin risus tincidunt non.

    Etiam sit amet tempus lorem, aliquet condimentum velit. Donec et nibh
    consequat, sagittis ex eget, dictum orci. Etiam quis semper ante. Ut eu
    mauris purus. Proin nec consectetur ligula. Mauris pretium molestie
    ullamcorper. Integer nisi neque, aliquet et odio non, sagittis porta justo.

    \begin{itemize}
      \item \textbf{Sed consequat} id ante vel efficitur. Praesent congue massa
        sed est scelerisque, elementum mollis augue iaculis.
        \begin{itemize}
          \item In sed est finibus, vulputate
            nunc gravida, pulvinar lorem. In maximus nunc dolor, sed auctor eros
            porttitor quis.
          \item Fusce ornare dignissim nisi. Nam sit amet risus vel lacus
            tempor tincidunt eu a arcu.
          \item Donec rhoncus vestibulum erat, quis aliquam leo
            gravida egestas.
        \end{itemize}
      \item \textbf{Sed luctus, elit sit amet} dictum maximus, diam dolor
        faucibus purus, sed lobortis justo erat id turpis.
      \item \textbf{Pellentesque facilisis dolor in leo} bibendum congue.
        Maecenas congue finibus justo, vitae eleifend urna facilisis at.
    \end{itemize}

  \end{block}

\end{column}

\separatorcolumn

\begin{column}{\colwidth}

  \begin{exampleblock}{A highlighted block containing some math}

    A different kind of highlighted block.

    $$
    \int_{-\infty}^{\infty} e^{-x^2}\,dx = \sqrt{\pi}
    $$

    Interdum et malesuada fames $\{1, 4, 9, \ldots\}$ ac ante ipsum primis in
    faucibus. Cras eleifend dolor eu nulla suscipit suscipit. Sed lobortis non
    felis id vulputate.

    \heading{A heading inside a block}

    Praesent consectetur mi $x^2 + y^2$ metus, nec vestibulum justo viverra
    nec. Proin eget nulla pretium, egestas magna aliquam, mollis neque. Vivamus
    dictum $\mathbf{u}^\intercal\mathbf{v}$ sagittis odio, vel porta erat
    congue sed. Maecenas ut dolor quis arcu auctor porttitor.

    \heading{Another heading inside a block}

    Sed augue erat, scelerisque a purus ultricies, placerat porttitor neque.
    Donec $P(y \mid x)$ fermentum consectetur $\nabla_x P(y \mid x)$ sapien
    sagittis egestas. Duis eget leo euismod nunc viverra imperdiet nec id
    justo.

  \end{exampleblock}

  \begin{block}{Nullam vel erat at velit convallis laoreet}

    Class aptent taciti sociosqu ad litora torquent per conubia nostra, per
    inceptos himenaeos. Phasellus libero enim, gravida sed erat sit amet,
    scelerisque congue diam. Fusce dapibus dui ut augue pulvinar iaculis.



    Donec quis posuere ligula. Nunc feugiat elit a mi malesuada consequat. Sed
    imperdiet augue ac nibh aliquet tristique. Aenean eu tortor vulputate,
    eleifend lorem in, dictum urna. Proin auctor ante in augue tincidunt
    tempor. Proin pellentesque vulputate odio, ac gravida nulla posuere
    efficitur. Aenean at velit vel dolor blandit molestie. Mauris laoreet
    commodo quam, non luctus nibh ullamcorper in. Class aptent taciti sociosqu
    ad litora torquent per conubia nostra, per inceptos himenaeos.

    Nulla varius finibus volutpat. Mauris molestie lorem tincidunt, iaculis
    libero at, gravida ante. Phasellus at felis eu neque suscipit suscipit.
    Integer ullamcorper, dui nec pretium ornare, urna dolor consequat libero,
    in feugiat elit lorem euismod lacus. Pellentesque sit amet dolor mollis,
    auctor urna non, tempus sem.

  \end{block}

  \begin{block}{References}

    \nocite{*}
    \footnotesize{\bibliographystyle{plain}\bibliography{poster}}

  \end{block}

\end{column}

\separatorcolumn
\end{columns}
\end{frame}

\end{document}
