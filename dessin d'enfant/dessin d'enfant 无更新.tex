
\documentclass[reqno,11pt]{amsart}

%\usepackage{color,graphicx}
%\usepackage{mathrsfs,amsbsy}
\usepackage{amssymb}
\usepackage{amsmath}
\usepackage{amsfonts}
\usepackage{bm}
\usepackage{graphicx}
\usepackage{amsthm}
\usepackage{enumerate}
\usepackage[mathscr]{eucal}
\usepackage{float}
\usepackage{longtable}
\usepackage{mathrsfs}
\usepackage{multicol}
\usepackage[all,pdf]{xy}
\usepackage[a4paper,left=3cm,right=3cm]{geometry}

%\usepackage[notcite,notref]{showkeys}

% showkeys  make label explicit on the paper

\makeatletter
\@namedef{subjclassname@2010}{%
  \textup{2010} Mathematics Subject Classification}
\makeatother

\numberwithin{equation}{section}

\theoremstyle{plain}
\newtheorem{theorem}{Theorem}[section]
\newtheorem{lemma}[theorem]{Lemma}
\newtheorem{proposition}[theorem]{Proposition}
\newtheorem{corollary}[theorem]{Corollary}
\newtheorem{claim}[theorem]{Claim}
\newtheorem{defn}[theorem]{Definition}
\newtheorem{ques}[theorem]{Question}
\newtheorem*{fact}{Facts}
\newtheorem{eg}[theorem]{Example}

\theoremstyle{plain}
\newtheorem{thmsub}{Theorem}[subsection]
\newtheorem{lemmasub}[thmsub]{Lemma}
\newtheorem{corollarysub}[thmsub]{Corollary}
\newtheorem{propositionsub}[thmsub]{Proposition}
\newtheorem{defnsub}[thmsub]{Definition}

\numberwithin{equation}{section}


\theoremstyle{remark}

\newtheorem{remark}[theorem]{Remark}
\newtheorem{remarks}{Remarks}
\newcommand*\widebar[1]{%
	\hbox{%
		\vbox{%
			\hrule height 0.5pt % The actual bar
			\kern0.6ex%         % Distance between bar and symbol
			\hbox{%
				\kern 0em%      % Shortening on the left side
				\ensuremath{#1}%
				\kern 0em%      % Shortening on the right side
			}%
		}%
	}%
}
\renewcommand\thefootnote{\fnsymbol{footnote}}
%dont use number as footnote symbol, use this command to change

\DeclareMathOperator{\supp}{supp}
\DeclareMathOperator{\dist}{dist}
\DeclareMathOperator{\vol}{vol}
\DeclareMathOperator{\diag}{diag}
\DeclareMathOperator{\tr}{tr}
\DeclareMathOperator{\Gal}{\operatorname{Gal}}

\begin{document}
\date{}

\title
{Dessin d'enfant: an Introduction}


\author{Xiaoxiang Zhou}
\address{School of Mathematical Sciences\\
University of Science and Technology of China\\
Hefei, 230026\\ P.R. China\\} 
\email{email:xx352229@mail.ustc.edu.cn}





\begin{abstract}
In this talk, we will talk about the relation between Belyi map and dessin d'enfant, and than extract imformations from the dessin.

Contents: section 4.1-4.3 and some examples from section 4.6.
\end{abstract}



\maketitle
%%%%%%%%%%%%%%%%%%%%%%%%%%%%%%%%%%%%%%%%%%%%%%%%%%%%%%%%%%%%%%%%%%%%%%%%%%%%%%%%%%%%%%%%%%%%%

%%%%%%%%%%%%%%%%%%%%%%%%%%%%%%%%%%%%%%%%%%%%%%%%%%%%%%%%%%%%%%%%%%%%%%%%%%%%%%%%%%%%%%%%%%%%%

 Last time, we talked about the Belyi's Theorem:
\begin{theorem}[Thm 3.1]
	Let $S$ be a cpt RS, then $S$ is defined over $\bar{\mathbb{Q}}$ iff $S$ admits a Belyi fct.
\end{theorem}
   
This time, we talked a specific Belyi fct (ramified at $0,1,\infty$), and
\begin{itemize}
	\item combine it with a kind of special graph (on $S$);
	\item extract information from this graph.
\end{itemize}

A black box is useful for us to familiar with Belyi fct (But not actually used in this talk):
\begin{center}
	\fbox{\textbf{Prop 3.34:} Belyi fcts are defined over $\bar{\mathbb{Q}}$.}
	%%%%更逼真的盒子
\end{center}
\begin{remark}
	We can talk about the Galois group $\Gal(\bar{\mathbb{Q}} / \mathbb{Q})$ acts on $\{S \longrightarrow \mathbb{P}^1 \}$.
	%%%%%%%%画图
\end{remark}
\begin{eg}
	When $S=\mathbb{P}^1$, then an Belyi fct $f(z)$ is an rational fct with coefficient in $\bar{\mathbb{Q}}$ such that $f$ maps any zero or pole of $f'(z)$ to $0,1,\infty$. In short,
	
	\begin{enumerate}[1.]
		\item  $f(z) \in \bar{\mathbb{Q}}(z)$;
		\item For any $z_0$ such that $f'(z_0)=0 \text{ or } \infty$, $f(z_0)=0,1 \text{ or } \infty$.
	\end{enumerate}

For example,
\begin{itemize}
	\item $f(z)=z^n$;
	\item $\displaystyle f(z)=-\frac{256}{27}z^3(z-1)$;
	\item $\displaystyle f(z)=\frac{3+i}{5}z^3(z-1)^2 \left(z-\frac{4}{25}(4+3i) \right)$;
	\item $\displaystyle f(z)=\frac{4}{27} \frac{(1-z+z^2)^3}{z^2(z-1)^2}$;
	\item $\displaystyle f(z)=C\frac{z^4 (z-1)^2}{z+\frac{9+2\sqrt{10}}{18}}$.
\end{itemize}
\end{eg}
\section{What is a dessin d'enfants? / Quel est un dessin d'enfants ?}

We postponed the abstract definition of the dessin d'enfant. A better question: How to draw a dessin d'enfants from a Belyi fct?
%%%%%%%%Picture

\begin{proposition}
	\item the graph $D$ is drawed on the RS $S$;
	\item $D$ is bicolored;
	\item $X \smallsetminus D$ is union of finitely many topo discs;
	\item $D$ is connected.
\end{proposition}

Abstractly, we have the following concept:
\begin{defn}[Def 4.1]
	 A dessin d'enfant, or simply a dessin, is a pair
	 $(X, \mathcal{D})$ where $X$ is an oriented compact topological surface, and
	 $\mathcal{D} \subset X$ is a finite graph such that: 
	 	\begin{enumerate}[(i)]
	 		\item  $\mathcal{D}$ is connected.
	 		\item  $\mathcal{D}$ is bicoloured, i.e. the vertices have been given either white or black colour and vertices connected by an edge have different colours.
	 		\item  $X \smallsetminus \mathcal{D}$ is the union of finitely many topological discs, which we call faces of  $\mathcal{D}$.
	 	\end{enumerate}
\end{defn}
\begin{eg}
	example pictures
\end{eg}
\begin{proposition}[Prop 4.20]
	%%%%%%Pictures
\end{proposition}
\begin{eg}
	example pictures
\end{eg}
\begin{remark}
	Belyi maps have some kind of rigid: they're decided by dessins (be viewed as skeleton of Belyi fcts)
\end{remark}
\begin{proof}[Proof of Prop.]
	\begin{enumerate}[1.]
		\item Given a pair $(X,D)$, we need
		\begin{itemize}
			\item give a RS structure of $X$;
			\item give a fct $f:X\longrightarrow \mathbb{P}^1$.
		\end{itemize}
	$f$ gives a RS structure of X (Riemann extension)
	\item  compability
	\item $$(S,f) \rightarrow (S,D_f) \rightarrow (S,f_{D_f}) \qquad (S,f) \sim (S,f_{D_f})?$$
	$$(X,D) \rightarrow (X_D,f_D) \rightarrow (X_D,D_{f_D}) \qquad (X,D) \sim (X_D,D_{f_D})?$$
	\end{enumerate}
\end{proof}
\section{Extract informations from the correspondence}
\subsection{basic information}
	\begin{longtable}{c|c}
	\hline
	$\{\text{Belyi fct}\}/\sim$ & $\{\text{Dessin d'enfants}\}/\sim$	\\
	
	\hline
	\endhead
	\hline
	$\{\text{Belyi fct}\}/\sim$ & $\{\text{Dessin d'enfants}\}/\sim$	\\
	
	\hline
	\endfirsthead	
	\hline
	\endfoot
	\hline		
	\caption{Correspondence}
	\endlastfoot
	
	$\# f^{-1}(0) + \# f^{-1}(1)$ & $v$\\
	$\# f^{-1}(\infty)$ & $f$ \\
	$\deg f$ & $e$\\
	$2-2g(S)$ & $v+f-e$\\
	Ram index of $x \in f^{-1}(0)$ & $\#$ black dots adjecent to $x$\\
	Ram index of $x \in f^{-1}(1)$ & $\#$ white dots adjecent to $x$\\
	Ram index of $x \in f^{-1}(\infty)$ & $\frac{1}{2}$ $\#$ sides of face\\
	
	
	
	\hline								
\end{longtable}
\newpage
\begin{longtable}{c|c|c}
	\hline
	$\{\text{Belyi fct}\}/\sim$ & $\{\text{Dessin d'enfants}\}/\sim$	& $\{\text{perm rep pair}\}/\sim $\\
	
	\hline
	\endhead
	\hline
	$\{\text{Belyi fct}\}/\sim$ & $\{\text{Dessin d'enfants}\}/\sim$	& $\{\text{perm rep pair}\}/\sim $\\
	
	\hline
	\endfirsthead	
	\hline
	\endfoot
	\hline		
	%		\caption{Correspondence}
	\endlastfoot
	
	$\# f^{-1}(0) $ & $\#$ $\{$white dots$\}$ &$\#$ $\{$cycles of  $\sigma_0\}$\\
	$ \# f^{-1}(1)  $ & $\#$ $\{$black dots$\}$&$\#$ $\{$cycles of  $\sigma_1\}$\\
	$\# f^{-1}(\infty)$ & $f$ &$\#$ $\{$cycles of  $\sigma_1\sigma_0\}$\\
	$\deg f$ & $e$ & $N=\#$ $\{$ cycles of  $Id\}$\\
	$2-2g(S)$ & $v+f-e$ & $\#\{\ldots \sigma_0\}$+$\#\{\ldots \sigma_1\}$+$\#\{\ldots \sigma_1\sigma_0\}$-N\\
	Ram index of $x \in f^{-1}(0)$ & $\#$ $\{$black dots adjacent to $x\}$ & length of a cycle on $\sigma_0$\\
	Ram index of $x \in f^{-1}(1)$ & $\#$ $\{$white dots adjacent to $x\}$ & length of a cycle on $\sigma_1$\\
	Ram index of $x \in f^{-1}(\infty)$ & $\frac{1}{2}$ $\#$ $\{$sides of face$\}$ & length of a cycle on $\sigma_0\sigma_1$\\
	\hline
	
\end{longtable}
\subsection{monodromy}
\subsection{Galois action}
\subsection{construct new from old}
%%%%%%%%%%%%%%%%%%%%%%%%%%%%%%%%%%%%%%%%%%%%%%%%%%%%%%%%%%%%%%%%%%%%%%%%%%

 




%%%%%%%%%%%%%%%%%%%%%%%%%%%%%%%%%%%%%%%%%%%%%%%%%%%%%%%%%%%%%%%%%%%%%%%%%%%%%%%%%%%%%%%%%%%%%%%






\end{document}




