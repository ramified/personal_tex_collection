
\documentclass{amsart}
%Typical documenttypes: article/book
%some examples:
%\documentclass[reqno,11pt]{book}   %%%for books
%\documentclass[]{minimal}			%%%for Minimal Working Example


%for beamers, you have to change a lot. Especially, remove the package enumitem!!!



%%%%%%%%%%%%%%%%%%%% setting for fast compiling

%\special{dvipdfmx:config z 0}		% no compression

\includeonly{chapters/chapter9}		% In practice, use an empty document called "chapter9"	% usually for printing books






%%%%%%%%%%%%%%%%%%%% here we include packages

%%%basic packages for math articles

\usepackage{amssymb}
\usepackage{amsthm}
\usepackage{amsmath}
\usepackage{amsfonts}
\usepackage[shortlabels]{enumitem}	% It supersedes both enumerate and mdwlist. The package option shortlabels is included to configure the labels like in enumerate.

%%%packages for special symbols
\usepackage{pifont}					% Access to PostScript standard Symbol and Dingbats fonts
\usepackage{wasysym}				% additional characters
\usepackage{bm}						% bold fonts: \bm{...}
\usepackage{extarrows}				% may be replaced by tikz-cd
%\usepackage{unicode-math}			% unicode maths for math fonts, now I don't know how to include it

%%%basic packages for fancy electronic documents
\usepackage[colorlinks]{hyperref}
\usepackage[table,hyperref]{xcolor} 			% before tikz-cd. 
%\usepackage[table,hyperref,monochrome]{xcolor}	% disable colored output (black and white)
\definecolor{darkblue}{rgb}{0.0,0.0,0.3}
\hypersetup{colorlinks,breaklinks,
            linkcolor=darkblue,urlcolor=darkblue,
            anchorcolor=darkblue,citecolor=darkblue}

%%%packages for figures and tables (general setting)
\usepackage{float}				%Improved interface for floating objects
\usepackage{caption,subcaption}
\usepackage{adjustbox}			% for me it is usually used in tables 
\usepackage{stackengine}		%baseline changes

%%%packages for commutative diagrams
\usepackage{tikz-cd}
%\usepackage{quiver}			% see https://q.uiver.app/

%%%packages for pictures
\usepackage[width=0.5,tiewidth=0.7]{strands}
\usepackage{graphicx}			% Enhanced support for graphics

%%%packages for tables and general settings
\usepackage{array}
\usepackage{makecell}
\usepackage{multicol}
\usepackage{multirow}
\usepackage{diagbox}

%%%packages for ToC, LoF and LoT







 %https://tex.stackexchange.com/questions/58852/possible-incompatibility-with-enumitem










%%%%%%%%%%%%%%%%%%%% here we include theoremstyles

\numberwithin{equation}{section}

\theoremstyle{plain}
\newtheorem{theorem}{Theorem}[section]

\newtheorem{setting}[theorem]{Setting}
\newtheorem{definition}[theorem]{Definition}
\newtheorem{lemma}[theorem]{Lemma}
\newtheorem{proposition}[theorem]{Proposition}
\newtheorem{corollary}[theorem]{Corollary}
\newtheorem{conjecture}[theorem]{Conjecture}

\newtheorem{claim}[theorem]{Claim}
\newtheorem{eg}[theorem]{Example}
\newtheorem{ex}[theorem]{Exercise}
\newtheorem{fact}[theorem]{Fact}
\newtheorem{ques}[theorem]{Question}
\newtheorem{warning}[theorem]{Warning}



\newtheorem*{bbox}{Black box}
\newtheorem*{notation}{Conventions and Notations}


\numberwithin{equation}{section}


\theoremstyle{remark}

\newtheorem{remark}[theorem]{Remark}
\newtheorem*{remarks}{Remarks}

%%% for important theorems
%\newtheoremstyle{theoremletter}{4mm}{1mm}{\itshape}{ }{\bfseries}{}{ }{}
%\theoremstyle{theoremletter}
%\newtheorem{theoremA}{Theorem}
%\renewcommand{\thetheoremA}{A}
%\newtheorem{theoremB}{Theorem}
%\renewcommand{\thetheoremB}{B}







%%%%%%%%%%%%%%%%%%%% here we declare some symbols

%%%%%%%DeclareMathOperator
%see here for why newcommand is better for DeclareMathOperator: https://tex.stackexchange.com/questions/67506/newcommand-vs-declaremathoperator

%%%%%basic symbols. Keep them!

%%%symbols for sets and maps
\DeclareMathOperator{\pt}{\operatorname{pt}}	%points. Other possibilities are \{pt\}, ...
\DeclareMathOperator{\Id}{\operatorname{Id}}	%identity in groups.
\DeclareMathOperator{\Img}{\operatorname{Im}}

\DeclareMathOperator{\Ob}{\operatorname{Ob}}
\DeclareMathOperator{\Mor}{\operatorname{Mor}}	%difference of Mor and Hom: Hom is usually for abelian categories
\DeclareMathOperator{\Hom}{\operatorname{Hom}}	\DeclareMathOperator{\End}{\operatorname{End}}
\DeclareMathOperator{\Aut}{\operatorname{Aut}}

%%%symbols for linear algebras and 
%%linear algebras
\DeclareMathOperator{\tr}{\operatorname{tr}}
\DeclareMathOperator{\diag}{\operatorname{diag}}	%for diagonal matrices

%%abstract algebras
\DeclareMathOperator{\ord}{\operatorname{ord}}
\DeclareMathOperator{\gr}{\operatorname{gr}}
\DeclareMathOperator{\Frac}{\operatorname{Frac}}

%%%symbols for basic geometries
\DeclareMathOperator{\vol}{\operatorname{vol}}	%volume
\DeclareMathOperator{\dist}{\operatorname{dist}}
\DeclareMathOperator{\supp}{\operatorname{supp}}

%%%symbols for category
%%names of categories
\DeclareMathOperator{\Mod}{\operatorname{Mod}}
\DeclareMathOperator{\Vect}{\operatorname{Vect}}
\DeclareMathOperator{\rep}{\operatorname{rep}} %usually rep means the category of finite dimensional representations, while Rep means the category of representations.
\DeclareMathOperator{\Rep}{\operatorname{Rep}}


%%%symbols for homological algebras
\DeclareMathOperator{\Tor}{\operatorname{Tor}}
\DeclareMathOperator{\Ext}{\operatorname{Ext}}
\DeclareMathOperator{\gldim}{\operatorname{gl.dim}}
\DeclareMathOperator{\projdim}{\operatorname{proj.dim}}
\DeclareMathOperator{\injdim}{\operatorname{inj.dim}}
\DeclareMathOperator{\rad}{\operatorname{rad}}


%%%symbols for algebraic groups
\DeclareMathOperator{\GL}{\operatorname{GL}}
\DeclareMathOperator{\SL}{\operatorname{SL}}

%%%symbols for typical varieties
\DeclareMathOperator{\Gr}{\operatorname{Gr}}
\DeclareMathOperator{\Flag}{\operatorname{Flag}}

%%%symbols for basic algebraic geometry
\DeclareMathOperator{\Spec}{\operatorname{Spec}}
\DeclareMathOperator{\Coh}{\operatorname{Coh}}
\newcommand{\Dcoh}{\mathcal{D}_{\operatorname{Coh}}}%%%This one shows the difference between \DeclareMathOperator and \newcommand
\DeclareMathOperator{\Pic}{\operatorname{Pic}}
\DeclareMathOperator{\Jac}{\operatorname{Jac}}

%%%%%advanced symbols. Choose the part you need!

%%%symbols for algebraic representation theory
\DeclareMathOperator{\Irr}{\operatorname{Irr}}
\DeclareMathOperator{\ind}{\operatorname{ind}}	%\ind(Q) means the set of  equivalence classes of finite dimensional indecomposable representations
\DeclareMathOperator{\Res}{\operatorname{Res}}
\DeclareMathOperator{\Ind}{\operatorname{Ind}}
\DeclareMathOperator{\cInd}{\operatorname{c-Ind}}


%%%symbols for algebraic topology
\DeclareMathOperator{\EGG}{\operatorname{E}\!}
\DeclareMathOperator{\BGG}{\operatorname{B}\!}

\DeclareMathOperator{\chern}{\operatorname{ch}^{*}}
\DeclareMathOperator{\Td}{\operatorname{Td}}
\DeclareMathOperator{\AS}{\operatorname{AS}}	%Atiyah--Segal completion theorem 

%%%symbols for Auslander--Reiten theory 
\DeclareMathOperator{\Modup}{\overline{\operatorname{mod}}}
\DeclareMathOperator{\Moddown}{\underline{\operatorname{mod}}}
\DeclareMathOperator{\Homup}{\overline{\operatorname{Hom}}}
\DeclareMathOperator{\Homdown}{\underline{\operatorname{Hom}}}


%%%symbols for operad
\DeclareMathOperator{\Com}{\operatorname{\mathcal{C}om}}
\DeclareMathOperator{\Ass}{\operatorname{\mathcal{A}ss}}
\DeclareMathOperator{\Lie}{\operatorname{\mathcal{L}ie}}
\DeclareMathOperator{\calEnd}{\operatorname{\mathcal{E}nd}} %cal=\mathcal


%%%%%personal symbols. Use at your own risk!

%%%symbols only for master thesis
\DeclareMathOperator{\ptt}{\operatorname{par}}	%the partition map
\DeclareMathOperator{\str}{\operatorname{str}}	%strict case
\DeclareMathOperator{\RRep}{\widetilde{\operatorname{Rep}}}
\DeclareMathOperator{\Rpt}{\operatorname{R}}
\DeclareMathOperator{\Rptc}{\operatorname{\mathcal{R}}}
\DeclareMathOperator{\Spt}{\operatorname{S}}
\DeclareMathOperator{\Sptc}{\operatorname{\mathcal{S}}}
\DeclareMathOperator{\Kcurl}{\operatorname{\mathcal{K}}}
\DeclareMathOperator{\Hcurl}{\operatorname{\mathcal{H}}}
\DeclareMathOperator{\eu}{\operatorname{eu}}
\DeclareMathOperator{\Eu}{\operatorname{Eu}}
\DeclareMathOperator{\dimv}{\operatorname{\underline{\mathbf{dim}}}}
\DeclareMathOperator{\St}{\mathcal{Z}}

%%%%%symbols which haven't been classified. Add your own math operators here!


\DeclareMathOperator{\Modr}{\operatorname{-Mod}}





%%%%%%%newcommand

%%%basic symbols
\newcommand{\norm}[1]{\Vert{#1}\Vert}

%%%symbols only for master thesis
\newcommand{\dimvec}[1]{\mathbf{#1}}
\newcommand{\abdimvec}[1]{|\dimvec{#1}|}
\newcommand{\ftdimvec}[1]{\underline{\dimvec{#1}}}

\newcommand{\absgp}[1]{\mathbb{#1}}
\newcommand{\WWd}{\absgp{W}_{\abdimvec{d}}}
\newcommand{\Wd}{W_{\dimvec{d}}}
\newcommand{\MinWd}{\operatorname{Min}(\absgp{W}_{\abdimvec{d}},W_{\dimvec{d}})}
\newcommand{\Compd}{\operatorname{Comp}_{\dimvec{d}}}
\newcommand{\Shuffled}{\operatorname{Shuffle}_{\dimvec{d}}}

\newcommand{\Omcell}{\Omega}
\newcommand{\OOmcell}{\boldsymbol{\Omega}}
\newcommand{\Vcell}{\mathcal{V}}
\newcommand{\VVcell}{\boldsymbol{\mathcal{V}}}
\newcommand{\Ocell}{\mathcal{O}}
\newcommand{\OOcell}{\boldsymbol{\mathcal{O}}}
\newcommand{\preimage}[1]{\widetilde{#1}}
\newcommand{\orde}{\operatorname{ord}_e}
\newcommand{\fakestar}{*}

%as the subscription of Hom
\newcommand{\Alggp}{\text{-Alg gp}}
\newcommand{\Gal}{\operatorname{Gal}}






%%%%%%%%%%%%%%%%%%%% here we make some blocks for special features. 

%%%% todo notes %%%%
\usepackage[colorinlistoftodos,textsize=footnotesize]{todonotes}
\setlength{\marginparwidth}{2.5cm}
\newcommand{\leftnote}[1]{\reversemarginpar\marginnote{\footnotesize #1}}
\newcommand{\rightnote}[1]{\normalmarginpar\marginnote{\footnotesize #1}\reversemarginpar}









%%%%%%%%%%%%%%%%%%%% here we make some global settings. Understand everything here before you make a document!

\usepackage[a4paper,left=3.3cm,right=3.3cm,bottom=4cm]{geometry}
\usepackage{indentfirst}	% Indent first paragraph after section header

\setcounter{tocdepth}{2}


%https://latexref.xyz/_005cparindent-_0026-_005cparskip.html
\setlength{\parindent}{15pt}	
\setlength{\parskip}{3pt plus5pt}

%\setlength\intextsep{0cm}
%\setlength\textfloatsep{0cm}
\def\arraystretch{1}
%\setcounter{secnumdepth}{3}

\allowdisplaybreaks

\makeatletter
% we use \prefix@<level> only if it is defined
\renewcommand{\@seccntformat}[1]{%
  \ifcsname prefix@#1\endcsname
    \csname prefix@#1\endcsname
  \else
    \csname the#1\endcsname\quad
  \fi}
% define \prefix@subsection
\newcommand\prefix@subsection{}
\makeatother


\begin{document}

% The beginning depends on the documentclass. Rewrite this part if you use different documentclass!
\date{\today}

\title
{Research Statement
}
\author{Xiaoxiang Zhou}
\address{Institut für Mathematik\\
Humboldt-Universität zu Berlin\\
Berlin, 12489\\ Germany\\} 
\email{email:xiaoxiang.zhou@hu-berlin.de}



\maketitle



My research lies in algebraic geometry and geometric representation theory, with a focus on geometric approaches to hidden combinatorial and representation-theoretic structures. My work ranges from affine pavings of quiver varieties and equivariant $K$-theory of Steinberg varieties to the geometry of characteristic cycles and perverse sheaves on abelian varieties. Throughout I have been motivated by the richness of algebraic and representation-theoretic phenomena that naturally emerge from geometry. By combining tools from algebraic geometry, topology, and Auslander–Reiten theory in novel ways, my work builds new connections between different areas from a unifying perspective that guides my current and future research.

\vspace{6mm}

\section{Master’s Research}
\subsection{\href{https://doi.org/10.1016/j.jpaa.2025.107953}{Affine pavings of quiver partial flag varieties}}$\,$\\[-5mm]

%\textcolor{red}{TODO: Why not say a few words about the notion of affine pavings / cellular decomposition in algebraic geometry? Give a few other examples where cell decompositions play a role, and explain why and where quiver partial flag varieties are interesting. Also you should quote some of the results in other Dynkin types that you generalize in this paper: Mention a few names and/or give references to put things into context!}

In my master’s thesis, I studied quiver partial flag varieties, which parameterize subrepresentations of a fixed indecomposable representation of a quiver. These varieties may be viewed as quiver analogues of Springer fibers, whose geometry controls the representation theory of Weyl groups via their top Borel–Moore homology. Although the geometry of Springer fibers can be quite involved, they often admit affine pavings, as do quiver partial flag varieties.

A complex algebraic variety $X$ has an affine paving if $X$ has a filtration
$$\varnothing= X_0 \subset X_1 \subset \cdots \subset X_d=X$$
with $X_i$ closed and $X_{i+1} \setminus X_i$ isomorphic to some affine space $\mathbb{A}^k_{\mathbb{C}}$. This notion plays a similar role in algebraic geometry as cellular decompositions do in topology, and implies nice properties about the cohomology of varieties, for example the vanishing of cohomology in odd degrees.
 
Affine pavings have been constructed in many cases, as for Grassmannians \cite[Theorem 4.1]{3264}, flag varieties \cite[Theorem 9.9.5]{Dmoduleperv}, as well as certain Springer fibers \cite{E7paving}, quiver Grassmannians \cite[Theorem 4]{irelli2019cell}, and quiver flag varieties \cite[Theorem 1.2]{maksimau2019flag}. 

In my article, we extend these results by constructing affine pavings for all quiver partial flag varieties of Dynkin type and of affine types $\widetilde{A}$ and $\widetilde{D}$. Our construction is based on a systematic stratification of the partial flag varieties together with Auslander–Reiten combinatorics, which reduces the geometric problem to explicit combinatorial data.

\vspace{6mm}

\subsection{\href{https://github.com/ramified/master_thesis/raw/main/master_thesis_Xiaoxiang_Zhou.pdf}{Equivariant $K$-theory of Steinberg varieties}}$\,$\\[-5mm]

%\textcolor{red}{TODO: Say briefly where Steinberg varieties arise. Was your computation new? What was known before? Again include more context.}

The Steinberg variety $\St$ was introduced in \cite{steinberg1976desingularization} and, in type $A$, consists of triples of a nilpotent operator and two flags fixed by this operator. Its top Borel–Moore homology carries a convolution product identifying it with the group algebra of the Weyl group \cite{kazhdan1980topological}. Lusztig later showed \cite{lusztig1985equivariant} that the affine Hecke algebra is naturally realized as the $G\times\mathbb{C}^\times$-equivariant $K$-theory of $\St$, a result often referred to as the Kazhdan–Lusztig isomorphism. Khovanov--Lauda \cite{https://doi.org/10.48550/arxiv.0803.4121} and Rouquier \cite{https://doi.org/10.48550/arxiv.0812.5023} subsequently introduced an algebra, called the quiver Hecke algebra or KLR algebra, in order to categorify quantum groups. Varagnolo–Vasserot \cite{varagnolo2011canonical} identified these algebras with the $G$-equivariant cohomology of a quiver version $\St_{\mathbf d}$ of the Steinberg variety.

Motivated by these ideas, we compute the equivariant $K$-theory of Steinberg varieties, yielding a $K$-theoretic analogue of the KLR algebra. Using the formulas for pullback, proper pushforward, and tensor product in $K$-theory, we obtain an explicit basis together with the convolution product, described combinatorially via strand diagrams. Furthermore, we discussed the Atiyah--Segal completion theorem, which links $K$-theory and cohomology and allows the Chern class and Todd class to appear explicitly in illustrative examples.

This project sparked my interest in the geometrical representation theory. Even now, I am still amazed by how the Euler class appears as a correction term in the excess base change formula and how this topological object can be explicitly expressed to capture rich representation-theoretic information.

\vspace{6mm}

\section{PhD Research}

\subsection{\href{https://github.com/ramified/personal_tex_collection/raw/main/PhD_thesis_raw_data/reorganized_version/re_subvarieties_in_abelian_variety.pdf}{Characteristic cycles of perverse sheaves on abelian varieties}}$\,$\\[-5mm]

During my PhD, I changed gears towards algebraic geometry. In my thesis I am studying the geometry of perverse sheaves on complex abelian varieties. In particular, I am investigating the correspondence between characteristic cycles and Weyl group orbits of dominant weights that arise from the Tannakian description of perverse sheaves in Krämer's work \cite{Kr20}. The cycles that appear are Lagrangian cones inside the cotangent bundle to the abelian varieties, and as such they can be written as formal sums of conormal varieties to certain subvarieties $Z^{(m)}$ of the abelian variety. I have provided a purely geometric construction of these subvarieties and have studied their basic properties. Among other things, I have computed their dimensions, their homology classes, and have investigated when these varieties are irreducible. 


The dimensions and homology classes of these subvarieties are determined by their total Chern--Mather class. Using the Möbius function in combination with the Pontryagin product, we can derive an explicit combinatorial expression for the Chern--Mather class. Regarding irreducibility, the constructions in \cite[2.c]{Kr20} imply that the irreducible components correspond precisely to the orbits of the monodromy group, whose structure we will examine in detail in the next section.

\vspace{6mm}

\subsection{\href{https://github.com/ramified/personal_tex_collection/raw/main/PhD_thesis_raw_data/reorganized_version/re_subvarieties_in_abelian_variety.pdf}{Monodromy groups of conormal Gauss maps}}$\,$\\[-5mm]

Let $A$ be a complex abelian variety and $Z \subset A$ be a non-degenerated closed subvariety.
More recently I have studied the conormal Gauss maps $\gamma_Z$ that are obtained by projecting from conormal varieties to the fiber direction, using that abelian varieties have trivial cotangent bundle. Gauss maps of this type have a long history in algebraic geometry starting from the work of Ran \cite{Ran84}, Zak \cite{Zak93} and others. 

In the case at hand, these conormal Gauss maps are generically finite covers. One may view the fiber of  $\gamma_Z$ as a cluster of stars projected onto a celestial dome. As the fiber direction varies, these points shift, tracing out paths much like stars drifting across the night sky. The constraints that govern them are subtle, like the imagined lines that shape constellations. And in the long arc of variation, monodromy emerges—like the slow turning that replaces Kochab with Polaris among the stars.

The study of monodromy groups is already very interesting from a purely geometric viewpoint, and substantial progress has been made in many cases: for instance, for Weierstrass points on a general curve \cite{EH87}, for the Prym map $\mathcal{R}_6 \longrightarrow \mathcal{A}_5$ \cite[Theorem 4.2]{Don92}, and numerous other cases \cite{Harris79}. Typically, monodromy groups are as large as possible, unless constrained by intrinsic geometric structures. In the context of conormal Gauss maps, establishing criteria for the ``bigness" of the monodromy group would yield important advances in our understanding of the Tannaka groups of perverse sheaves on abelian varieties, with further applications in moduli theory and in arithmetic geometry along the lines of Lawrence--Sawin \cite{LS25} and Javanpeykar--Krämer--Lehn--Maculan \cite{JKLM22}.

Can we find cases where the monodromy group is not as big as expected? There are two situations where this can occur, though they are considered reducible and hence excluded from our focus. First, if $Z$ is invariant under a nontrivial translation (i.e., $Z = Z + p$ for some $p \neq 0$), then the geometry effectively descends to a quotient variety, and the monodromy is governed by that simpler setting. Second, if $Z$ can be written as a sum $Z = X + Y$ of positive-dimensional subvarieties, the monodromy group is often constrained by the structure of the summands. To study genuinely new phenomena, we therefore restrict attention to ``primitive" subvarieties that do not arise in these ways.

The earliest known example occurs when $Z$ is the Fano surface of lines on a smooth cubic threefold, embedded into its Albanese variety; in this case, the monodromy group agrees with the Weyl group of type $E_6$. Recently, we have found a new family of examples in which the monodromy groups are significantly smaller. In these cases, the abelian variety $A$ is isogenous to either $E_{\mathfrak{i}}^{\oplus n}$ or $E_{\rho}^{\oplus n}$, and the subvariety $Z$ is invariant under a certain ``rotation" in $A$. Consequently, the monodromy group is strictly smaller than the associated Weyl group, and the subvarieties associated to Weyl group orbits become highly reducible.

Focusing on curves, one naturally asks whether all instances with small monodromy groups have been found. I have proposed a conjecture asserting that this is indeed the case, which can also be formulated equivalently in terms of the Gaussian curvature of the curve. We have checked the conjecture in the most Prym case, especially when the cover has degree $2$ or when the base curve is non-hyperelliptic; in these situations, the statement holds true. We are still working toward a proof, and a more thorough understanding is likely to reveal deeper underlying structures.

\vspace{6mm}


%\textcolor{red}{Feel free to change the above, but the text needs to have some motivation and context}

\section{Future Research Directions}

%In my future research I plan to deepen the geometric understanding of representation-theoretic structures and to connect my current work to broader themes in geometric representation theory, particularly those surrounding Hecke algebras, Schubert varieties, and Langlands duality.

My proposed future work at Strasbourg will build on my earlier exploration of algebraic cycles arising from perverse sheaves on abelian varieties, and will orient toward generalized cohomology of concrete complex algebraic geometry, in line with the research program of Lie Fu. In particular, I plan the following directions:

\subsection{Cycles, Hodge-Theoretic Invariants, and Functorial Methods}$\,$\\[-5mm]

A central theme of my PhD research is to understand algebraic cycles on abelian varieties as concretely as possible: I have computed the dimensions and homology classes of the subvarieties I constructed, but their position in the Chow group remains mysterious. A natural next step is to study their Hodge-theoretic and motivic invariants. Recent work such as \cite{FuLie24} provides refined invariants that recover and unify the results of Göttsche--Soergel, Göttsche, and Belmans--Fu--Krug, expressing the invariants of $\operatorname{Hilb}^n(S)$ in terms of the corresponding invariants of $S$. I would like to investigate whether analogous invariants can be computed for my family of subvarieties $Z^{(m)}$ in terms of the invariants of the original subvariety $Z$, thereby providing a more structural understanding of these cycles beyond their homological classes.

This direction resonates with my earlier training. In my master’s thesis on equivariant $K$-theory of Steinberg varieties, I became familiar with how the categorical tools translates into explicit geometric formulas. During my first year of PhD study, I deepened this perspective by working through the theory of perverse sheaves, mixed Hodge modules, and the decomposition theorem in examples. As part of this self-study, I computed characteristic cycles of certain $\operatorname{IC}$-sheaves as practice, using the six-functor formalism to reduce the problem to concrete topological calculations. Although these explorations are not part of my thesis, they continue to shape my intuition for how abstract functorial frameworks can yield surprisingly concrete information. 

%\subsection{Algebraic structures in equivariant $K$-theory of Steinberg varieties}$\,$\\[-5mm]
% 
%
%I plan to revisit my earlier work on the equivariant $K$-theory of Steinberg varieties with the goal of identifying the “big algebra’’ in $K_0$. Beginning with the type A full flag variety, I aim to clarify the relationship between this algebra structure, convolution geometry, and the geometry of Schubert varieties. Understanding this interaction may shed new light on Hecke algebras from a geometric viewpoint, with potential relevance to the geometric Langlands program.

\subsection{Characteristic cycles on Schubert varieties}$\,$\\[-5mm]

A second research direction is to extend my work on characteristic cycles from abelian varieties to Schubert varieties. Since the $K$-theory of the Steinberg variety and the category of perverse sheaves are related by the Bezrukavnikov equivalence, which is the categorification of the Kazhdan–Lusztig isomorphism, I plan to describe characteristic cycles of perverse sheaves on Schubert varieties and trace their images under the equivalence. This project connects naturally to the Springer correspondence, the geometry of conic Lagrangians, and Langlands correspondence.

\section{Outlook}

Across my projects, my goal is to develop geometric frameworks that reveal the underlying combinatorial and representation-theoretic structures. The environment led by Lie Fu at Strasbourg, with its emphasis on generalized cohomology on concrete geometrical objects, offers an ideal environment for advancing these ideas. I look forward to pursuing new ideas at the interface of algebraic geometry, representation theory, and combinatorics, contributing to the group’s research activities, and learning from interactions with members such as Mauro Porta, Drago\c{s} Fr\u{a}\c{t}il\u{a} and Lucas Toury.

%\textcolor{red}{Maybe also mention some other people at ISTA or in his group?}
\nocite{Zhou25}

\bibliographystyle{alpha}
\bibliography{reference}

% Remember to protect the uppercase of people's name and LaTeX symbols
\end{document}