
\documentclass{ctexart}
%此处以后,可以开始用\usepackage{*}添加要用的各类Latex宏包
\usepackage{amsmath} %推荐使用的数学公式宏包,因为可以用align环境更好地排版行间公式
\usepackage{amsfonts}
\usepackage{amssymb}
\usepackage{geometry} %调整页面的页边距
\geometry{left=2.5cm,right=2.5cm,top=2cm,bottom=3cm}%具体的页边距设置
%\usepackage{graphicx} %插入图片的宏包
%\usepackage{lineno,hyperref}  % 显示行号,超链接
%\usepackage{multirow} %插入表格时需要用的合并行的功能
%\usepackage{amsthm} %插入定理、证明的宏包
%\usepackage{enumerate} %插入列表的宏包
%\usepackage{enumitem} %插入列举项目的宏包
%\usepackage[linesnumbered,lined,boxed]{algorithm2e} %插入算法的宏包
\usepackage{CJKfntef} %汉字加点
\newtheorem{re}{注}
\begin{document}
%正文部分,包括:题目、摘要、关键字、脚注、章节、参考文献,等等

\title{小测参考题目2} %添加标题

\author{周潇翔} %添加作者

%\date{} %LaTeX会自动生成日期,如果不需要就加这一步将日期去掉


	\maketitle %制作封面
	
%	\tableofcontents %加入目录,包括页码(非必需)

	

	\section{Exercise1} 
可以从以下任选一题.
\begin{enumerate}
	\item 设$\mathbb{R}$上的Lipchitz函数$f(x)$满足性质:对任意$x \in \mathbb{R}$,
	$$\lim\limits_{n \rightarrow \infty}n[f(x+\frac{1}{n})-f(x)]=0$$
	证明$f(x)$在$\mathbb{R}$上右可导.
\item 设$f(x)$在$[a,b]$上可导.若$f(a)=0,\forall x \in (a,b), f(x)>0$,则对任意的$n,m \in \mathbb{N}$,存在$\xi,\eta \in (a,b)$,使得
$$\frac{f'(\xi)}{f(\xi)}=\frac{m}{n}\frac{f'(\eta)}{f(\eta)}$$
\item 设$f(x)$在$\mathbb{R}$上可导.记Dirichlet函数:
\begin{equation*}\label{eq:dirichlet}
	D(x)=
	\begin{cases}
	1, & \text{if } x \in \mathbb{Q}; \\
	0, & \text{if } x \notin \mathbb{Q};
	\end{cases}
\end{equation*}
试求函数$g(x):=f(x)D(x)$的连续点集与可导点集.

你可使用如下记号:($a \in \mathbb{R}$)
$$f^{-1}(a)=\{x \in \mathbb{R}|f(x)=a\}$$
$$(f')^{-1}(a)=\{x \in \mathbb{R}|f'(x)=a\}$$
\begin{re}
	考试时$f(x)$可取成某些具体的例子,比如说
	$$f(x)=x^3(x+1)^2(x+2)$$
\end{re}
\item 判断正误:若$\mathbb{R}$上的可导函数$f$满足$f'(0)\neq 0$,则
存在$\delta>0$,使得$f(x)$于$(-\delta,\delta)$单调.\\
如果命题正确,请说明理由;若不正确,请举出具体例子.
\begin{re}
	若补充条件“导函数连续”,则命题成立.
\end{re}
\end{enumerate}

	\section{Exercise2} 
		我们在这里考虑方程$\tan x=x$的解.
\begin{enumerate}

	\item

证明:

对$\forall n \in \mathbb{N}$,方程$\tan x=x$于$(n\pi-\pi /2, n\pi+\pi /2)$有且恰好只有一个解.

记这个解为$x_n$.
	
		\item
证明:
	
	$$\lim\limits_{n \rightarrow \infty}x_n-n\pi=\pi/2$$
	
			\item
	
计算极限:

$$\lim\limits_{n \rightarrow \infty}n(x_n-n\pi-\pi/2)$$
\end{enumerate}
\begin{re}
想考Sard定理的简化版本的,但是还没有简单的做法.
\end{re}

\end{document}