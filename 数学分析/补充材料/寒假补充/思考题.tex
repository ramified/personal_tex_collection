\documentclass[11pt,A4paper,oneside]{amsart}

%\usepackage{color,graphicx}
%\usepackage{mathrsfs,amsbsy}
\usepackage{ctex}
\usepackage{amssymb}
\usepackage{amsmath}
\usepackage{amsfonts}
\usepackage{fancyhdr}
\usepackage[unicode, bookmarksnumbered]{hyperref}	% 启动超链接和 PDF 文档信息所需
\usepackage{graphicx}
\usepackage{amsthm}
\usepackage{enumerate}
\usepackage[mathscr]{eucal}
\usepackage{mathrsfs}
\usepackage{verbatim}

%\usepackage[notcite,notref]{showkeys}

% showkeys  make label explicit on the paper

\makeatletter
\@namedef{subjclassname@2010}{%
  \textup{2010} Mathematics Subject Classification}
\makeatother

\numberwithin{equation}{section}

\theoremstyle{plain}
\newtheorem{theorem}{Theorem}[section]
\newtheorem{lemma}[theorem]{Lemma}
\newtheorem{proposition}[theorem]{Proposition}
\newtheorem{corollary}[theorem]{Corollary}
\newtheorem{claim}[theorem]{Claim}
\newtheorem{defn}[theorem]{Definition}

\theoremstyle{plain}
\newtheorem{thmsub}{Theorem}[subsection]
\newtheorem{lemmasub}[thmsub]{Lemma}
\newtheorem{corollarysub}[thmsub]{Corollary}
\newtheorem{propositionsub}[thmsub]{Proposition}
\newtheorem{defnsub}[thmsub]{Definition}

\numberwithin{equation}{section}


\theoremstyle{remark}
\newtheorem{remark}[theorem]{Remark}
\newtheorem{remarks}{Remarks}


%\renewcommand\thefootnote{\fnsymbol{footnote}}
%dont use number as footnote symbol, use this command to change

\DeclareMathOperator{\supp}{supp}
\DeclareMathOperator{\dist}{dist}
\DeclareMathOperator{\vol}{vol}
\DeclareMathOperator{\diag}{diag}
\DeclareMathOperator{\tr}{tr}


\begin{document}

\title[]{\LARGE 思考题}


\author[]{\large 周潇翔}
\address{School of Mathematical Sciences\\
University of Science and Technology of China\\
Hefei, 230026\\ P.R. China\\}
\email{xx352229@mail.ustc.edu.cn}
\maketitle




\begin{abstract}
寒假列一些思考题.
\end{abstract}




%%%%%%%%%%%%%%%%%%%%%%%%%%%%%%%%%%%%%%%%%%%%%%%%%%%%%%%%%%%%%%%%%%%%%%%%%%%%%%%%%%%%%%%%%%%%%




\section{Lie代数方向}
设$E$为$n$维内积空间.我们称$E$中的容许集$M$为$E$中满足下列条件的集合:
\begin{itemize}
	\item $M=\{e_1,\ldots,e_n\}$为$E$的一组基,且$\lVert e_i\rVert = 1$.
	\item 对任意$1 \leqslant i<j\leqslant n$,我们有$4(e_i,e_j)^2 \in \{0,1,2,3\}$.
\end{itemize}
现对$E$中的某一个容许集$M$,我们可以按照如下方式构建图$\Gamma_M$:
\begin{itemize}
	\item 图的顶点取为$M$中的元素
	\item 两个顶点$e_i,e_j$之间的边数$=4(e_i,e_j)^2$
\end{itemize}
试确定所有可能的$\Gamma_M$(更精确地说,是在图的同构意义下确定.思考:如何定义图的同构?)
\begin{remark}\
	\begin{enumerate}
		\item 这道题需要你至少知道内积空间的知识(习题课讲过,可见李尚志的最后一章),但是这道题较为开放,且背后隐含着有限维半单Lie代数的分类理论,完整的解答详见\cite[p58]{Hum78}.
		\item 你可以试着从$n=1,2,3$开始做起.这部分直观且结论相当漂亮.
		\item 这是毛天乐助教在线代习题课上给我们出的思考题,留给我们寒假思考.我还没有完整思考出来.
	\end{enumerate}
\end{remark}
%%%%%%%%%%%%%%%%%%%%%%%%%%%%%%%%%%%%%%%%%%%%%%%%%%%%%%%%%%%%%%%%%%%%%%%%%%%%%%%%%%%%%%%%%%%%%
\section{发红包的题}
打算给做出来下面这道题的同学一点小奖励。

\subsection{近世代数+线性代数}
记
$$SL(2,\mathbb{Z}):=\left\{
\begin{pmatrix}
a & b \\
c & d
\end{pmatrix}\bigg| a,b,c,d \in \mathbb{Z}, ad-bc=1 \right\}$$
$$A:=\left\{
\begin{pmatrix}
24a+1 & b \\
24c & 24d+1
\end{pmatrix} \in SL(2,\mathbb{Z}) \bigg| a,b,c,d \in \mathbb{Z} \right\}$$
证明$A$为$SL(2,\mathbb{Z})$的子群,并求$(SL(2,\mathbb{Z}):A)$的值.

请将此值转换为中文并在其后添加“科大红包”即为支付宝的吱口令.

例如:如这个值为2018,则吱口令就为“二零一八科大红包”.



%%%%%%%%%%%%%%%%%%%%%%%%%%%%%%%%%%%%%%%%%%%%%%%%%%%%%%%%%%%%%%%%%%%%%%%%%%






%%%%%%%%%%%%%%%%%%%%%%%%%%%%%%%%%%%%%%%%%%%%%%%%%%%%%%%%%%%%%%%%%%%%%%%%%%%%%%%%%%%%%%%%%%%%%%%




\begin{thebibliography}{99}


\bibitem{Hum78}%
James E. Humphreys, Introduction to Lie algebras and representation theory, Graduate
Texts in Mathematics, vol. 9, Springer-Verlag, New York-Berlin, 1978, Second printing,
revised. MR 499562





\end{thebibliography}


\end{document}




