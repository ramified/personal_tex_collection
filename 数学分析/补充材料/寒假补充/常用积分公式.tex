\documentclass[11pt,A4paper,oneside]{amsart}

%\usepackage{color,graphicx}
%\usepackage{mathrsfs,amsbsy}
\usepackage{ctex}
\usepackage{amssymb}
\usepackage{amsmath}
\usepackage{amsfonts}
\usepackage{fancyhdr}
\usepackage[unicode, bookmarksnumbered]{hyperref}	% 启动超链接和 PDF 文档信息所需
\usepackage{graphicx}
\usepackage{amsthm}
\usepackage{enumerate}
\usepackage[mathscr]{eucal}
\usepackage{mathrsfs}
\usepackage{verbatim}

%\usepackage[notcite,notref]{showkeys}

% showkeys  make label explicit on the paper

\makeatletter
\@namedef{subjclassname@2010}{%
  \textup{2010} Mathematics Subject Classification}
\makeatother

\numberwithin{equation}{section}

\theoremstyle{plain}
\newtheorem{theorem}{Theorem}[section]
\newtheorem{lemma}[theorem]{Lemma}
\newtheorem{proposition}[theorem]{Proposition}
\newtheorem{corollary}[theorem]{Corollary}
\newtheorem{claim}[theorem]{Claim}
\newtheorem{defn}[theorem]{Definition}

\theoremstyle{plain}
\newtheorem{thmsub}{Theorem}[subsection]
\newtheorem{lemmasub}[thmsub]{Lemma}
\newtheorem{corollarysub}[thmsub]{Corollary}
\newtheorem{propositionsub}[thmsub]{Proposition}
\newtheorem{defnsub}[thmsub]{Definition}

\numberwithin{equation}{section}


\theoremstyle{remark}
\newtheorem{remark}[theorem]{Remark}
\newtheorem{remarks}{Remarks}


%\renewcommand\thefootnote{\fnsymbol{footnote}}
%dont use number as footnote symbol, use this command to change

\DeclareMathOperator{\supp}{supp}
\DeclareMathOperator{\dist}{dist}
\DeclareMathOperator{\vol}{vol}
\DeclareMathOperator{\diag}{diag}
\DeclareMathOperator{\tr}{tr}


\begin{document}

\title[]{\LARGE 补A3:常用积分}


\author[]{\large 周潇翔}
\address{School of Mathematical Sciences\\
University of Science and Technology of China\\
Hefei, 230026\\ P.R. China\\}
\email{xx352229@mail.ustc.edu.cn}
\maketitle




\begin{abstract}
列一下A3中证明的非平凡的积分公式供概率论临时使用。
\end{abstract}




%%%%%%%%%%%%%%%%%%%%%%%%%%%%%%%%%%%%%%%%%%%%%%%%%%%%%%%%%%%%%%%%%%%%%%%%%%%%%%%%%%%%%%%%%%%%%




\begin{itemize}
	\item Gauss积分:$\displaystyle \int_{0}^{+\infty} e^{-x^2}dx= \frac{\sqrt{\pi}}{2}$\\
	概率论记忆法:$\displaystyle \int_{-\infty}^{+\infty} \frac{1}{\sqrt{2\pi}}e^{-\frac{x^2}{2}}dx= 1$
	\item Dirichlet积分:$\displaystyle \int_{0}^{+\infty} \frac{\sin x}{x} dx=\frac{\pi}{2}$
	\item Laplace积分:$\displaystyle I(\beta):=\int_{0}^{+\infty} \frac{\cos \beta x}{x^2+\alpha^2} dx\qquad 
	J(\beta):=\int_{0}^{+\infty} \frac{x\sin \beta x}{x^2+\alpha^2} dx$
	\item Fresnel积分:$\displaystyle \int_{0}^{+\infty} \sin x^2dx= \int_{0}^{+\infty} \cos x^2dx=\frac{1}{2}\sqrt{\frac{\pi}{2}}$
	\item $\Gamma$函数:$\Gamma(p)\Gamma(1-p)=\frac{\pi}{\sin p\pi},\quad \Gamma(1/2)=\sqrt{\pi},\quad \lim\limits_{x \rightarrow +\infty} \frac{x^a \Gamma(x)}{\Gamma(x+a)}=1$\\
	$B$函数:$B(p,q)=\frac{\Gamma(p+q)}{\Gamma(p)\Gamma(q)}$
\end{itemize}
就我个人经历,概率论中我用到的主要是Gauss积分、$\Gamma$函数与$B$函数。



%%%%%%%%%%%%%%%%%%%%%%%%%%%%%%%%%%%%%%%%%%%%%%%%%%%%%%%%%%%%%%%%%%%%%%%%%%%%%%%%%%%%%%%%%%%%%





%%%%%%%%%%%%%%%%%%%%%%%%%%%%%%%%%%%%%%%%%%%%%%%%%%%%%%%%%%%%%%%%%%%%%%%%%%






%%%%%%%%%%%%%%%%%%%%%%%%%%%%%%%%%%%%%%%%%%%%%%%%%%%%%%%%%%%%%%%%%%%%%%%%%%%%%%%%%%%%%%%%%%%%%%%




%\begin{thebibliography}{99}


%\bibitem{AF12}%
%Antunes, P., Freitas, P.: Optimal spectral rectangles and lattice ellipses. \emph{Proc. Royal Soc. London Ser. A.} \textbf{469} (2012), 20120492.





%\end{thebibliography}


\end{document}




