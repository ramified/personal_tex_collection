
\documentclass{ctexart}

%\usepackage{color,graphicx}
%\usepackage{mathrsfs,amsbsy}
\usepackage{color}
\usepackage{CJK}
\usepackage{amssymb}
\usepackage{amsmath}
\usepackage{amsfonts}
\usepackage{graphicx}
\usepackage{amsthm}
\usepackage{enumerate}
\usepackage[mathscr]{eucal}
\usepackage{mathrsfs}
\usepackage{verbatim}
\usepackage{mathtools}
\usepackage{geometry} %调整页面的页边距
\geometry{top=2cm,bottom=3cm,left=5cm,right=5cm}
%\usepackage[notcite,notref]{showkeys}

% showkeys  make label explicit on the paper

\makeatletter
\@namedef{subjclassname@2010}{%
  \textup{2010} Mathematics Subject Classification}
\makeatother

\numberwithin{equation}{section}

\theoremstyle{plain}
\newtheorem{theorem}{Theorem}[section]
\newtheorem{lemma}[theorem]{Lemma}
\newtheorem{proposition}[theorem]{Proposition}
\newtheorem{corollary}[theorem]{Corollary}
\newtheorem{claim}[theorem]{Claim}
\newtheorem{defn}[theorem]{Definition}

\theoremstyle{plain}
\newtheorem{thmsub}{Theorem}[subsection]
\newtheorem{lemmasub}[thmsub]{Lemma}
\newtheorem{corollarysub}[thmsub]{Corollary}
\newtheorem{propositionsub}[thmsub]{Proposition}
\newtheorem{defnsub}[thmsub]{Definition}

\numberwithin{equation}{section}


\theoremstyle{remark}
\newtheorem{remark}[theorem]{Remark}
\newtheorem{remarks}{Remarks}


\renewcommand\thefootnote{\fnsymbol{footnote}}
%dont use number as footnote symbol, use this command to change

\DeclareMathOperator{\supp}{supp}
\DeclareMathOperator{\dist}{dist}
\DeclareMathOperator{\vol}{vol}
\DeclareMathOperator{\diag}{diag}
\DeclareMathOperator{\tr}{tr}

\begin{document}
\date{}

\title
{拓扑与近世代数补充:PSet2}


\author{2019.3.24}
%\address{School of Mathematical Sciences\\
%University of Science and Technology of China\\
%Hefei, 230026\\ P.R. China\\} 
%\email{email:xx352229@mail.ustc.edu.cn}
\maketitle




%\begin{abstract}
%\end{abstract}




%%%%%%%%%%%%%%%%%%%%%%%%%%%%%%%%%%%%%%%%%%%%%%%%%%%%%%%%%%%%%%%%%%%%%%%%%%%%%%%%%%%%%%%%%%%%%
\section{有趣的群作用}
设$\{u,v\} \in \mathbb{C}$为$\mathbb{C}$的一组$\mathbb{R}$-基,考虑$G:=\mathbb{Z}u \oplus \mathbb{Z}v$在$\mathbb{C}$上的群作用:
$$G \times \mathbb{C} \longrightarrow \mathbb{C} \qquad (g,z) \longmapsto g+z$$
另外,考虑
$$SL(2,\mathbb{Z}):=\left\{\;\gamma=\begin{pmatrix}
a & b\\
c & d
\end{pmatrix}\Bigg|ad-bc=1\right\}$$
在上半平面
$$\mathcal{H}:=\{z \in \mathbb{C} \mid Im z>0\}$$
上的作用:
$$SL(2,\mathbb{Z}) \times \mathcal{H}\longrightarrow \mathcal{H} \qquad (\gamma,z) \longmapsto \frac{az+b}{cz+d}$$
试问这些作用对某个元素的稳定子群、作用的基本区域、商掉这些作用后得到的商空间(取商拓扑)拓扑同胚于什么?是否是紧的?
\begin{remark}
	这两个群作用有一些关系,如果有机会的话可以讲一下。另外可以讲一下格点的性质(近世代数)。
\end{remark}
\section{矩阵群的性质}
填表:\\[1cm]
\begin{tabular}{|c|c|c|c|c|c|}
	\hline 
$G$	& 连通分支个数 & 紧性 & 中心 & \textcolor{red}{$G/Z(G)$是否为单群} & \textcolor{red}{极大环面}\\ 
	\hline 
$GL_n(\mathbb{R})$	&  &  &  & & \\ 
	\hline 
$SL_n(\mathbb{R})$	&  &  &  & & \\ 
	\hline 
$O(n)$	&  &  &  &  &\\ 
	\hline 
$SO(n)$	&  &  &  &  &\\ 
	\hline 
$U(n)$	&  &  &  &  &\\ 
	\hline 
$SU(n)$	&  &  &  &  &\\ 
	\hline 
\textcolor{red}{$GL_n(\mathbb{C})$}	&  &  &  &  &\\ 
\hline 
\textcolor{red}{$SL_n(\mathbb{C})$}	&  &  &  &  &\\ 
\hline 
\end{tabular} 
\\[1cm]
注:红色的是可补充内容。横向还可以加:标准型、切空间维数等。纵向可以加:$Sp(n,\mathbb{C}),Sp(n,\mathbb{R}),O(p,q),SL_n(Z),PSL_n(\mathbb{R})$等等。




             





%%%%%%%%%%%%%%%%%%%%%%%%%%%%%%%%%%%%%%%%%%%%%%%%%%%%%%%%%%%%%%%%%%%%%%%%%%%%%%%%%%%%%%%%%%%%%

 
   


%%%%%%%%%%%%%%%%%%%%%%%%%%%%%%%%%%%%%%%%%%%%%%%%%%%%%%%%%%%%%%%%%%%%%%%%%%

 




%%%%%%%%%%%%%%%%%%%%%%%%%%%%%%%%%%%%%%%%%%%%%%%%%%%%%%%%%%%%%%%%%%%%%%%%%%%%%%%%%%%%%%%%%%%%%%%




%\begin{thebibliography}{99}

 
%\bibitem{AF12}%
%Antunes, P., Freitas, P.: Optimal spectral rectangles and lattice ellipses. \emph{Proc. Royal Soc. London Ser. A.} \textbf{469} (2012), 20120492.


  

%\end{thebibliography}


\end{document}




