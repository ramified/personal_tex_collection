
\documentclass{ctexart}
%此处以后,可以开始用\usepackage{*}添加要用的各类Latex宏包
\usepackage{amsmath} %推荐使用的数学公式宏包,因为可以用align环境更好地排版行间公式
\usepackage{amsfonts}
\usepackage{amssymb}
\usepackage{amsthm}
\usepackage{geometry} %调整页面的页边距
\geometry{left=2.5cm,right=2.5cm,top=2cm,bottom=3cm}%具体的页边距设置
%\usepackage{graphicx} %插入图片的宏包
%\usepackage{lineno,hyperref}  % 显示行号,超链接
%\usepackage{multirow} %插入表格时需要用的合并行的功能
%\usepackage{amsthm} %插入定理、证明的宏包
%\usepackage{enumerate} %插入列表的宏包
%\usepackage{enumitem} %插入列举项目的宏包
%\usepackage[linesnumbered,lined,boxed]{algorithm2e} %插入算法的宏包

\newtheorem{re}{注}
\newtheorem{jie}{【解答】}
\begin{document}
	%正文部分,包括:题目、摘要、关键字、脚注、章节、参考文献,等等
	
	\title{补充材料} %添加标题
	
	\author{周潇翔} %添加作者
	
	%\date{} %LaTeX会自动生成日期,如果不需要就加这一步将日期去掉
	
	
	\maketitle %制作封面
	
	%	\tableofcontents %加入目录,包括页码(非必需)
	注意:本材料只适合志愿基础方向,且将课程内的材料学扎实的同学,只是作为一个引子,让你们自行去探索那深奥的数学世界。不适合在考前复习使用!
	
	
\section{第一章}
		
之前我说过你们看完谢惠民后还有很多的材料可以补充,在此列下:

1.点集拓扑的基本概念。

这部分对你们了解实数完备性的六个等价定理大有裨益。比如,为什么是“闭”区间套定理?有限覆盖为什么是用“开”覆盖来覆盖的?为什么覆盖的是“有界”“闭”区间?等价定理中,有几个是与点集拓扑有着极为密切的关系的。

这块可以看《数学分析教程》的第8章8.1-8.5,那里介绍了$\mathbb{R}^n$中的点集拓扑基本概念:开集、闭集、紧集、连通性,$\mathbb{R}^2$中的集合是最直观的例子。当然你们可以看在一般的度量空间上这些概念的定义(参考napkin2、4、5节)。

2.数项级数

就像你们高中学数列必学求和公式一样,对数列的求和取极限也是一个自然的问题,这个问题强化你对无穷小量更加精细的感觉。现在你去看的话,只需要搞清楚比较判别法,了解最常被用于比较的三个数项级数:等比级数、调和级数、p级数即可。

这块可以看《数学分析教程》的第14章

3.集合论

其实你们代基的内容就已经比较全了,这里重点强调一下集合的势的概念。其实给出一些非常难证的结论后,几乎随便给一个集合,它的势都应该是能比较容易地算出来的。衡量集合的大小,第一步就是算这个集合的势。

材料在之后会给出。

4.复数

基本的内容你们高中竞赛的话会有接触,不过复数若加上极限运算的话,就会有更多奇妙的性质。

可以看Ahlfors的《Complex Analysis》1-20页,这部分是复数的基本定义和基本几何性质,包括球极投影;同样可以看Stein的《Fourier Analysis》p23-26的练习,但这部分比较困难(因为你们没学函数项级数)。

\section{第二章}

关于第二章的补充材料:(给学数分觉得没学满足的同学使用)

1.拓扑学。

你们应该听过“同胚”这个词,就是连续映射+可逆+逆映射亦连续。而在一般的拓扑空间中(其实度量空间就差不多够用了),连续的定义就是“开集的原像是开集”(在度量空间中的定义与你们学到的类似,可以证明两种定义等价)。拓扑学培养的直观有助于你们了解老师说的“球极投影”和“有界闭区间到无界区间的推广”。

另外,有界闭区间上连续函数的练习题也给你们提供了拓扑学最基本的例子,所以还是相当有趣的。

可以看《数学分析教程》的第8章,也可以看尤承业的《基础拓扑学讲义》。

2.泛函分析。

数分中主要考虑的是某一个连续函数函数,而当你们考虑所有$[a,b]$上的连续函数全体构成的集合时,就有一些单个函数很难看出的性质。它是一个无穷维的实线性空间,并且还带有范数(范数就是“绝对值”的推广,范数诱导距离,两个函数之间可以定义“距离”的概念),这样子的处理方式是泛函分析的思维方式,亦别有一番趣味。

想看的话得先看一些线性代数的知识。知道线性空间、线性映射、二次型、内积即可。

3.函数逼近论。

你们学到连续函数,就有用很多好的函数去逼近连续函数的例子。比如局部常值函数、分段线性函数、连续可微函数、无穷阶可微函数、多项式函数(逼近$[a,b]$上的连续函数)。这块内容会随着你们的学习不断地加深理解,不一定现在就要学会。但你们要知道用一个好的函数去逼近一个“差”的函数,可以更容易地得到一些非平凡的性质。

这一块和上一块有交集,不过是一个相对集中的专题,可以慢慢思考。

4.动力系统

关于混沌的那一节,本质上是动力系统的导入,感兴趣的同学可以看看《数学:维度漫步2:chaos》的科普(影片),可以去看丁同仁的《常微分方程教程》第八章(微分方程课上8.3,8.4也会跳过),另外可以找黄文老师去探索这部分的内容。(我不是很了解,但我看科普之后还是觉得这一块挺好玩的)

\section{第三、四章}
1.全微分

凑全微分是一项很重要的技能(虽然大部分的题都是用肉眼看),你们构造辅助函数运用中值定理某种方式上也是在凑微分。算不定积分、解简单的ODE(常微分方程)这块的理论能帮助你们去寻找辅助函数(或者凑全微分)。这块对你们学力学也有帮助。

这块推荐看下《数学分析教程》的5.2(5.3和5.4目前知道下结论就行),还有丁同仁的《常微分方程教程》第二章,偏计算的内容,理解起来不算困难。

2.偏导数(参看《数学分析教程》的9.1)

看了下大部分的内容还是需要用到一些偏导的记号的,其实这个不难,也就相当于“含参函数的求导”,细心些计算就好。多变量函数的微分学重点还是在微分。

掌握这个之后,就可以看很多很多的方向了,比如说:

1)复分析

你们可以考虑在复平面上类似地定义导数和可微函数的概念,这时在定义域可导的函数称为“解析函数”,它比实轴上的可微函数性质要好的多的多(比如这个函数无穷阶可微)。你们可以在这其中发现为什么$e^x$具有那么好的性质,将数分中学到的概念在复数域中做推广,结论可能比数分中要漂亮得多。

2)微分几何中对曲线、曲面的研究

考虑一条参数化曲线,假如这个参数化是可微的,那你们就可以应用求导这个工具来研究这条曲线了。从某种意义上说,这是微分学和解析几何的交汇处。对曲面的探索需要一些偏导数的知识。

这一块可以看《数学分析教程》的9.5,还可以看刘世平老师的微分几何讲义(http://staff.ustc.edu.cn/~spliu/Teach\_DG2017.html),老师还是写的很细致的。

3)应用于热学

看了偏导数之后再看宏观的热力学定律就不会被符号搞得晕头转向了。

3.微分

这个词背后有丰富的几何。想法是用线性空间来与一点处逼近这个函数,在这里线性代数进入分析。当然这个词在一维情况下是很难看清楚的。

这个我还不知道隐函数应该推荐什么。。

4.隐函数定理和隐映射定理

这一块可以看《数学分析教程》的9.7和9.8,不过不容易看懂。目前知道一下结论就行,你们会在之后的课程中不断地遇到此定理(微分方程、微分流形、泛函分析)

5.微分方程的最大值原理

若$f'' \geqslant0$,则$f$在边界点处取到最大值。最大值原理是这个结论的一个推广。这本书可以看第一节了,里面的分析技巧还是有一定难度的。

\section{第五章}
这一章没有什么内容,就是形式计算。

\section{第六章}

1.测度论和Lebesgue积分论

这一部分老师已经在课上科普了很多次了,与可积性理论关系密切,这里就不班门弄斧了。感兴趣的同学可以直接找任广斌老师问相关的内容。

2.Fourier分析

简单来说,你可以用这章学到的知识来算Fourier系数,可以定义$\mathbb{R}$上的卷积,可以用某些test function的卷积来反推函数的性质。定积分的估计手段相当精妙,这里推荐Stein的Fourier分析前4章。基本上把前三章吃透,估计的手段对数分就足够用了。(书上的许多题都有这个背景,练习题6.2.7-8,练习题6.3.1(2),6.4.11-12,问题6.4.2	)

3.内积、范数、逼近论。这也是这周习题课的主要内容。

4.多重积分。

二重积分的观点可以很容易看懂三道题:练习题6.4.9-10,上课的一道例题。另外可以算$\int_{0}^{+\infty}e^{-x^2/2}dx$。这个积分式是有用的,因为概率论中,正态分布函数$\displaystyle\frac{1}{\sqrt{2\pi}}e^{-x^2/2}$具有极其重要的地位。
		
5.特殊函数。

你们已经见到了许多由积分定义的函数,并会简单地研究他们的性质。许多这些函数已经超越了初等函数的范围,比如说$\Gamma$函数和$B$函数。真正可以将这些函数的性质讲清楚,需要用到含参变量积分的知识。
		
\section{第七章}
1.微分几何

真正学懂如何算弧长、面积、体积得等到微分几何。目前可以直观些将就着看第一节。

2.物理嘛,下学期力热就会学到。

3.漂亮的不等式。这块高中做竞赛的同学肯定比较熟。试着将它们推广到积分形式。(面积原理在说明基本的(拿来比较的)级数收敛时非常有用。)。

4.阶乘推广就是$\Gamma$函数,Stirling公式推广就是$\Gamma$函数的渐进展开,这些内容在Alfors第五章上会有清楚的阐述。
	
	
\end{document}
