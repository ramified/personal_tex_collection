\documentclass[pdf]{beamer}
\mode<presentation>{
	\usetheme{Ilmenau}

}
	\usecolortheme{dolphin}
%\usepackage{color,graphicx}
%\usepackage{mathrsfs,amsbsy}
\usepackage{bm}
\usepackage{booktabs}
\usepackage[UTF8,noindent]{ctexcap}
\usepackage{amssymb}
\usepackage{amsmath}
\usepackage{amsfonts}
\usepackage{array}
\usepackage{fancyhdr}
\usepackage{hhline}
%\usepackage[unicode, bookmarksnumbered]{hyperref}	% 启动超链接和 PDF 文档信息所需
\usepackage{graphicx}
\usepackage{amsthm}
\usepackage{indentfirst}
\usepackage{enumerate}
\usepackage[mathscr]{eucal}
\usepackage{mathrsfs}
\usepackage{verbatim}
\usepackage{wrapfig}
%\usepackage{geometry} %调整页面的页边距
\usepackage{pifont}
%\geometry{left=2.5cm,right=2.5cm,top=2cm,bottom=2.5cm}%具体的页边距设置
%\usepackage[notcite,notref]{showkeys}

% showkeys  make label explicit on the paper

%\makeatletter
%\@namedef{subjclassname@2010}{%
%  \textup{2010} Mathematics Subject Classification}
%\makeatother

\usepackage{tikz}
\usetikzlibrary{shapes.geometric, arrows}
\usetikzlibrary{er}

\numberwithin{equation}{section}

\theoremstyle{plain}
%\newtheorem{theorem}{Theorem}[section]
%\newtheorem{lemma}[theorem]{Lemma}
\newtheorem{proposition}[theorem]{Proposition}
%\newtheorem{corollary}[theorem]{Corollary}
\newtheorem{claim}[theorem]{Claim}
\newtheorem{defn}[theorem]{Definition}
%\newtheorem{example}[theorem]{Example}

\theoremstyle{plain}
\newtheorem{exercise}{Exercise}[section]

\theoremstyle{plain}
\newtheorem{thmsub}{Theorem}[subsection]
\newtheorem{lemmasub}[thmsub]{Lemma}
\newtheorem{corollarysub}[thmsub]{Corollary}
\newtheorem{propositionsub}[thmsub]{Proposition}
\newtheorem{defnsub}[thmsub]{Definition}

%\numberwithin{equation}{section}


\theoremstyle{remark}
\newtheorem{remark}[theorem]{Remark}
\newtheorem{remarks}{Remarks}
\newtheorem{ex}[theorem]{Exercise}
\newtheorem{question}[theorem]{Question}

\newcommand*{\thick}[1]{\text{\boldmath$#1$}}
\newcommand*{\cir}[1]{\;$\ding{19#1}$\;}%临时使用
\newcommand*{\norm}[1]{\lVert#1\rVert}

%\renewcommand\thefootnote{\fnsymbol{footnote}}
%dont use number as footnote symbol, use this command to change

\DeclareMathOperator{\supp}{supp}
\DeclareMathOperator{\dist}{dist}
\DeclareMathOperator{\vol}{vol}
\DeclareMathOperator{\diag}{diag}
\DeclareMathOperator{\tr}{tr}
\setlength{\parindent}{2em}



\title{数学分析讲座}
\author{周潇翔}
\institute[USTC]{University of Science and Technology of China}
\date{\today}
\subject{Q\&A}
\keywords{数学分析,介绍,考试}

\begin{document}
	\begin{frame}
	\titlepage
	\end{frame}
\begin{frame}
\begin{abstract}
这是12月8号数学分析讲座的ppt (不包含A2,A3).主要借鉴了USTC基础数学修课指南(\url{https://www.zhangjy9610.me/USTC/ustcmathplan1.pdf}).
\end{abstract}

讲座中的内容仅代表本人在最近的观点,仅做参考。另外文中的大部分信息均无严格调查,均不严谨,许多应该加“大部分”等修饰语的地方,为了行文的方便没有加。

本讲座与中法班无关,因为中法班上的是《分析》。
\end{frame}
\begin{frame}
我叫周潇翔,来自福建,目前是大四基础数学方向,近期准备申请(所以没有认真做这份ppt).
\begin{itemize}
	\item QQ:1051686409
	\item 主页:\url{http://home.ustc.edu.cn/~xx352229}
	\item 邮箱:xx352229@mail.ustc.edu.cn
\end{itemize}
\end{frame}
\begin{frame}{目录}
	这个讲座的目的,说得直白些,就是这三个问题:
	\tableofcontents
	这几个问题的难度是层层下降的。
\end{frame}




\section{如何学懂}
\begin{frame}{目录}
\tableofcontents[currentsection]
\end{frame}
\begin{frame}{原因}
	我到现在也没有完全学懂,就只从个人角度谈谈。
	
	为什么学不懂:
	\begin{itemize}
		\item \textbf{局部}:从未见过的定义和定理
		\item \textbf{整体}:无法理解全课逻辑,离散的知识点
		\item \textbf{前置}课程缺失:逻辑语言、代数语言
		\item 在\textbf{后继}课程中不断被使用和推广
	\end{itemize}
\end{frame}
\begin{frame}{数分在基础课程中的地位}
\tikzstyle{startstop} = [rectangle,rounded corners, minimum width=1.5cm,minimum height=0.5cm,text centered, draw=black,fill=red!30]
\tikzstyle{io} = [trapezium, trapezium left angle = 70,trapezium right angle=110,minimum width=1.5cm,minimum height=0.5cm,text centered,draw=black,fill=blue!30]
\tikzstyle{process} = [rectangle,minimum width=1.5cm,minimum height=0.5cm,text centered,text width =1.7cm,draw=black,fill=orange!30]
\tikzstyle{process2} = [rectangle,minimum width=1.7cm,minimum height=0.7cm,text centered,text width =1.7cm,draw=black,fill=green!50]
\tikzstyle{process3} = [rectangle,minimum width=1.5cm,minimum height=0.5cm,text centered,text width =1.7cm,draw=black,fill=yellow!70]
\tikzstyle{arrow} = [thick,->,>=stealth]
\tikzstyle{arrow2} = [thick,<->,>=stealth]
\hspace*{-2em}
\begin{tikzpicture}[node distance=1cm]
\node (start) [startstop] {Start};
\node (input1) [io,below of=start] {分析};
\node (input2) [io,below of=input1,yshift=-0.6cm] {几何};
\node (input3) [io,below of=input2,yshift=-0.6cm] {代数};
\node (process1) [process2,right of=input1,xshift=1cm] {数学分析};
\node (process2a) [process3,right of=process1,yshift=-0.8cm,xshift=1.4cm] {微分几何};
\node (process2b) [process3,below of=process2a,yshift=0.2cm] {拓扑学};
\node (process3a) [process,below of=process2b,yshift=-0.6cm] {近世代数};
\node (process1a) [process3,right of=process1,xshift=3.8cm,yshift=0.8cm] {实分析};
\node (process1b) [process3,below of=process1a,yshift=0.2cm] {复分析};
\node (process2c) [process,below of=process1b,yshift=0.2cm] {黎曼几何};
\node (process2d) [process,below of=process2c,yshift=0.2cm] {微分流形};
\node (process3b) [process,below of=process2d,yshift=0.2cm] {交换代数};
\node (process3c) [process,below of=process3b,yshift=0.2cm] {代数数论};
\node (process3d) [process,below of=process3c,yshift=0.2cm] {表示论};
\node (process2) [process3,below of=process1,yshift=-1.4cm] {线性代数};
\node (process1c) [process,right of=process1a,xshift=1.8cm] {泛函分析};
\node (process1d) [process,below of=process1c,yshift=0.2cm] {微分方程};
\node (process2e) [process,below of=process1d,yshift=0.2cm] {黎曼面};
\node (process2f) [process,below of=process2e,yshift=-0.6cm] {代数几何};
\node (process3e) [process,below of=process2f,yshift=0.2cm] {同调代数};
\draw [arrow2] (process2e) -- (process2f);
\draw [arrow2] (process1) -- (process2);
\draw [arrow2] (process2) |- (process3a);
\draw [arrow] (process1) |- (process1a);
\draw [arrow] (process1) -- (process1b);
\draw [arrow] (process1) -- (process2a);
\draw [arrow] (2.3cm,-1.4cm) |- (process2b);
\draw [arrow] (process2a) -- (process2c);
\draw [arrow] (process2b) -- (process2d);
\draw [arrow] (process2d) -- (process2c);
\draw [arrow] (process3a) -- (process3b);
\draw [arrow] (process3a) -- (process3c);
\draw [arrow] (process3a) -- (process3d);
\draw [arrow] (process1a) -- (process1c);
\draw [arrow] (process1c) -- (process1d);
\draw [arrow] (process1b) -- (process2e);
\draw [arrow] (process2d) -- (process2e);
\draw [arrow] (process2d) -- (process2f);
\draw [arrow] (process3e) -- (process2f);
\draw [arrow] (process3b) -- (process2f);
\end{tikzpicture}
\end{frame}

\begin{frame}{解决方案}

\begin{itemize}
	\item 抠\textbf{细节},“新生需要学会的第一件事就是用严格的数学语
言去刻画证明中的任何结论”
	\item 观\textbf{大略},画流程框图等
	\item 选好合适的习题集,如谢惠民
	\item 多\textbf{交流},不要把同学当成纯粹的竞争对手
	\item 保持好\textbf{心态},“无论遇到什么困难都不要怕,加油,奥利给!”
	\item 有自己的理解(不求与众不同),并在学习过程中不断改进自己的理解。
\end{itemize}
\end{frame}

\begin{frame}{如何问助教}
\begin{itemize}
	\item \textbf{不会就问},不要自卑(又不会扣你平时分,怕什么怕)
	\item 主动找助教约答疑/在答疑课上\textbf{当面}问,线上回答非常耗时
	\item 问的时候顺便讲讲自己对这个问题理解到了什么程度(做到哪一步),让助教清楚了解到你的\textbf{需求}(节省大家的时间)
	\item 助教做不出来/不能当面做出来都很正常,不要因此瞧不起助教
	\item 若是助教讲完了自己还不懂,就直接说,让助教\textbf{再讲一遍}/给别的建议(掌握知识才是最重要的)
\end{itemize}
\end{frame}
\begin{frame}{问什么样的问题?}
助教答出的概率:问题类型
\begin{itemize}
\item 90\%:作业题(助教不愿意回答的话就让他讲一个思路相同的变形)
\item 85\%:史济怀上的题
\item 60\%:其他数分书上的题
\item 30\%:高级课程题,道听途说的题,自己瞎编的题(不属于助教的答疑范围,所以问的时候要讲究策略,不要期待能得到答案)
\end{itemize}
\end{frame}

\section{如何做题}
\begin{frame}{目录}
\tableofcontents[currentsection]
\end{frame}
\begin{frame}{定义与定理}

定义:框架+限定性条件(+目的)
\begin{example}
	一个实数列$\{x_n\}$称为\textbf{Cauchy列},如果对任意$\varepsilon >0$,存在$N \in \mathbb{N}$,使得对任意的$m,n >N$,有$|x_m-x_n| < \varepsilon$
	\begin{itemize}
		\item 框架:实数列$\{x_n\}$
		\item 限定性条件:任意$\varepsilon >0$,存在$N \in \mathbb{N}$,使得对任意的$m,n >N$,有$|x_m-x_n| < \varepsilon$
		\item 目的:尝试内蕴地定义数列的收敛
	\end{itemize}
\end{example}


\end{frame}
\begin{frame}
 定理:对象+(本质)条件+(技术性)条件+结论
 
 \phantom{定理:}(+\textbf{应用}+证明+背景)
	\begin{example}
	设函数$f$在有限区间$[a,b]$上有界,则$f$在$[a,b]$上Riemann可积当且仅当$D(f)$是零测集。
	
	\begin{itemize}
		\item 对象:$[a,b]$上函数$f$
		\item (技术性)条件:$f$有界
		\item (本质)条件+结论:$f$ Riemann可积、$D(f)$是零测集
		\item 应用:对具体的函数计算$D(f)$得到(不)可积性
		\item 理解:对Riemann可积的性质有了清晰的刻画
		\item 证明思路:连续部分的振幅小+震荡部分的测度小$\rightarrow$收敛
	\end{itemize}
\end{example}
\end{frame}
\begin{frame}{"大定理"}
\begin{itemize}
	\item Taylor展开
	\item Lebesgue可积性定理
	\item 隐函数/逆映射定理
	\item Fourier分析的结论
\end{itemize}
这是我觉得的最重要,也是最难证的几个定理吧。别的定理请大部分做到能自己手推
\end{frame}
\begin{frame}{作业题}
\begin{itemize}
	\item 直接硬肝
	\item (有思路)自己的尝试中漏了什么条件?
	\item[] \phantom{(有思路)}没有这个条件有什么后果?(找反例)
	\item (完全没思路)画个图,对某个具体的函数思考 
	\item[] \phantom{(完全没思路)}返回本节课本,看下该节讲了什么内容
	\item (完成)这个结论漂亮吗?漂亮在何处?
	\item[] \phantom{(完全)}简洁性、应用广、内涵深刻…
\end{itemize}
\end{frame}
\begin{frame}
\begin{example}
用实数完备性的六个等价命题证明有界闭区间上连续函数的性质(有界、介值、最值)
\begin{itemize}
	\item 有界闭区间改成无界?开区间?$\longrightarrow$ (不对)找反例
	\item 直观上为什么找不到反例?(画图)你的图被什么束缚住了?
	\item 你的思路有哪些?$\longrightarrow$区间套点、Lebesgue膨胀胀点(上确界/单调收敛)、反证对每一点找开邻域(构成开覆盖)、找一列点取子列$\longrightarrow$这些思路为何只对有界闭区间成立?
	\item 连续函数的性质有什么推广?(有界闭区间$\rightarrow$紧+连通)
	\item[] 实数完备性的六个等价命题有什么推广?
	
\end{itemize}
\end{example}


\end{frame}

\begin{frame}{A1经典例子}
常见的例子:
\begin{itemize}
	\item Riemann函数、Dirichlet函数
	\item $x^{\alpha}\sin \frac{1}{x^{\beta}}$
	\item $e^{-1/x^2}$
	\item $\sqrt{x}, (1+\frac{1}{x})^x$
	\item 单调函数,凸函数
\end{itemize}
\end{frame}

\begin{frame}{如何掌握例子?}
\begin{figure}[th]
	\begin{minipage}[t]{.1\textwidth}
	\end{minipage}
	\begin{minipage}[t]{.85\textwidth}
		\begin{itemize}
			\setlength{\itemindent}{-3em}
			\item \textbf{直观}:这些例子是什么函数,图像如何
			\item \textbf{性质}:这个例子具有...性质
			\item[] \phantom{作用:}例如,$[A,B]$上的单调函数至多只有可数个间断点,且每一个间断点处都是跳跃间断点。
			\item \textbf{作用}:例子满足A条件,但不满足B条件,说明A不能推B
			\item[] \phantom{作用:}例子不满足定理中的结论B.这是因为例子不满足定理中的A条件,故不能应用该定理推出结论B.
			
			\item \textbf{再造}:构造正确的几何直观,用以替代之前有偏差的直观
			\item[]\phantom{作用:}例如,如何在图像上理解函数在某一点处极限存在、连续、可导?
		\end{itemize}
	\end{minipage}
\end{figure}

\end{frame}



\section{如何考试}
\begin{frame}{目录}
\tableofcontents[currentsection]
\end{frame}
\begin{frame}{自我评估}
\begin{itemize}
	\item 2.0:及时完成作业,及时交作业(态度)
	\item 3.3:史济怀课后习题
	\item 3.6:史济怀课后问题
	\item 4.0:谢惠民、应试技能
	\item 4.3:运气和应试技能
\end{itemize}
没有必要强求4.3的结果,但一定要有学习的态度!
\end{frame}
\begin{frame}{小测反思}
上一次的数分考试,自己考得如何?在哪个环节出了问题,是否可以在之后改进?“缺啥补啥”
\begin{itemize}
	\item 基本概念不熟
	\item 计算粗心
	\item 定理不记得
	\item 没有及时交作业,考前补作业/作业题忘记如何做了
	\item 时间来不及/没有细心检查
	\item 心态爆炸
\end{itemize}

\end{frame}
\begin{frame}{期末考前,考中,考后}
\begin{itemize}
	\item 考前1-2周:作业交齐后领回,开始复习,把平时成绩问清楚
	\item 考前3天:做个整体的总结(抓大放小),有问题问助教和老师
	\item 考前1天:把之前做过的题看看,包括书上例题、作业、小测题、习题课的题
	\item 考试时:心态放平,允许自己失分,认真即可
	\item 考后:及时查卷,总结失误(不要在自己确实不对的地方去缠助教给分)
\end{itemize}

\end{frame}

\begin{frame}{Q \& A}
\begin{center}
	Thank you!
	
	Questions \& Answers?
\end{center}



\end{frame}






%%%%%%%%%%%%%%%%%%%%%%%%%%%%%%%%%%%%%%%%%%%%%%%%%%%%%%%%%%%%%%%%%%%%%%%%%%




%%%%%%%%%%%%%%%%%%%%%%%%%%%%%%%%%%%%%%%%%%%%%%%%%%%%%%%%%%%%%%%%%%%%%%%%%%%%%%%%%%%%%%%%%%%%%%%







\end{document}




