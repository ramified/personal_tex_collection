
\documentclass{ctexart}
%此处以后,可以开始用\usepackage{*}添加要用的各类Latex宏包
\usepackage{amsmath} %推荐使用的数学公式宏包,因为可以用align环境更好地排版行间公式
\usepackage{amsfonts}
\usepackage{amssymb}
\usepackage{amsthm}
\usepackage{mathtools}
\usepackage{geometry} %调整页面的页边距
\geometry{left=2.5cm,right=2.5cm,top=2cm,bottom=3cm}%具体的页边距设置
%\usepackage{graphicx} %插入图片的宏包
%\usepackage{lineno,hyperref}  % 显示行号,超链接
%\usepackage{multirow} %插入表格时需要用的合并行的功能
%\usepackage{amsthm} %插入定理、证明的宏包
%\usepackage{enumerate} %插入列表的宏包
%\usepackage{enumitem} %插入列举项目的宏包
%\usepackage[linesnumbered,lined,boxed]{algorithm2e} %插入算法的宏包
\newcommand{\myiint}{\int\hspace{-0.3cm}\int}
\newtheorem{re}{注}
\newtheorem{jie}{【解答】}
\begin{document}
	%正文部分,包括:题目、摘要、关键字、脚注、章节、参考文献,等等
	
	\title{微分几何复习} %添加标题
	
	\author{周潇翔} %添加作者
	
	%\date{} %LaTeX会自动生成日期,如果不需要就加这一步将日期去掉
	
	
	\maketitle %制作封面
	
	%	\tableofcontents %加入目录,包括页码(非必需)
	
	
\section{期中}
\begin{enumerate}
	

	\item	$$(\log \det (A))' = Tr (A^{-1} \frac{dA}{dt})$$
		
	\item	$$\frac{A(\Sigma \cap B_x(r))}{\pi r^2}=c_1+c_2 r+c_3 r^2+o(r^2)$$

\item 曲率为常数的平面曲线 \\
(1)$k(s)\equiv 0 \Leftrightarrow \vec{r}(s)$是直线\\
(2)$k(s)\equiv a \neq 0 \Leftrightarrow \vec{r}(s)$是半径为$|\frac{1}{a}|$的圆

\item$\vec{r}(s)(k>0)$落在某平面上的充要条件为$\tau =0$

\item 必考:利用Frenet公式导出某些简单的几何命题

例:法平面过定点$\rightarrow$球面曲线

例:一般螺线:切向量与某固定方向成定角的非直线螺线$(k>0)$.

证明:非直线曲线为一般螺线$\Leftrightarrow \frac{\tau}{k}=c
$

\item 考虑$\vec{r}(s):[a,b]\rightarrow E^4$(Minkowski空间),有
$$\begin{pmatrix}
e_1\\ \vdots\\ \vdots \\ e_n
\end{pmatrix}=\begin{pmatrix}
0 & k_1(s) & & 0\\
-k_1(s) &&\ddots&\\
& \ddots &&k_{n-1}(s)\\
0& & -k_{n-1}(s)& 0
\end{pmatrix}
\begin{pmatrix}
e_1\\ \vdots\\ \vdots \\ e_n
\end{pmatrix}$$

\item 等温参数网 直纹面推可展曲面:$k=0; (a',b,b')=0;$切平面重合

\item 全脐点曲面的分类.例:$E^3$中全为平点的曲面只能是平面.

\item 渐进方向与渐近曲线:$L(du)^2+2Mdudv+N(dv)^2=0$

\item 曲率线与曲率线网:($F=M=0$)
$$\frac{d\vec{n}}{ds}=-\lambda(s)\frac{d\vec{r}}{ds}$$
$$\left|\begin{matrix}
(du^2)^2 & -du^1du^2 & (du^1)^2 \\
g_{11} & g_{12} & g_{22}\\
h_{11} & h_{12} & h_{22}
\end{matrix}\right|=0$$
曲面$\Sigma$的参数曲线网为曲率线网的充要条件:$F=M=0$

\item 命题:\uppercase\expandafter{\romannumeral1}、\uppercase\expandafter{\romannumeral2}、\uppercase\expandafter{\romannumeral3}不是独立的,有
$\uppercase\expandafter{\romannumeral3}-2H\uppercase\expandafter{\romannumeral2}+K\uppercase\expandafter{\romannumeral1}$

\item 可展曲面:直纹面+
\begin{itemize}
	\item 直母线上各点的切平面重合(即法向量平行)
	\item Gauss曲率为0
	\item 与平面等距
\end{itemize}
Gauss曲率为0且无脐点,则$\Sigma$必为可展曲面\\
可展曲面的分类:柱面、锥面、切线面\\
例:曲面$\Sigma$, $\vec{r}(u,v)$上的曲线$C$为曲率线$\Leftrightarrow C$上每点曲面法线所生成的直纹面$\tilde{\Sigma}$为可展曲面
\end{enumerate}
\section{期末}
\begin{itemize}
	\item 曲率、挠率
	\item 第一、二基本形式、Christoffel符号
	\item Weingarden变换
	\item Gauss映射的像集
	\item Gauss曲率、平均曲率、法曲率、主曲率、测地曲率
	\item 椭圆点、抛物点、平点、脐点
	\item 渐近线、曲率线、测地线
	\item 旋转面、直纹面、可展曲面、全脐点曲面、极小曲面
	\item 等温参数系、曲率线网、测地法坐标系、测地极坐标系、测地平行坐标系
	\item 协变导数、协变微分、平行移动、向量平移产生的角差
	\item 曲面上的Laplace算子及其局部坐标表示
	\item Gauss-Bonnet公式
	\item 弧长泛函、能量泛函、面积泛函\\
	将上述的概念在正交活动标架下再算一遍。
	
\end{itemize}
关于整体曲线:
\begin{itemize}
	\item 旋转指数 $\displaystyle\quad \frac{1}{2\pi} \int_{0}^{l}k(s)ds=\pm1$
	\item 等周不等式$\displaystyle\quad \left(\frac{L^2}{4\pi}-A\right)\geqslant 0$
	\item 凸曲线
	\item 支撑函数,Minkowski问题
\end{itemize}
关于整体曲面:
\begin{itemize}
	\item Gauss-Bonnet $\displaystyle\quad \int_{D}KdA+\int_{\partial D}k_g ds +\sum \alpha_i =2\pi \chi(D)$\\[0.3cm]
	应用:指数定理 $\quad I(\nu) =\chi (\Sigma) =2(1-g)$\\[0.15cm]
	\phantom{应用:}Jacobi定理 $\displaystyle\quad \int_{D}dA=\int_{\partial D} d(\arctan \frac{\tau}{k})=\int_{\partial D}k_gd\rho =2\pi$
	\item 紧致曲面的Gauss映射
	\begin{itemize}
		\item $\displaystyle \int_{\Sigma}KdA=4\pi(1-g)$
		\item $\displaystyle\int_{\Sigma_+}KdA \geqslant 4\pi \quad \Rightarrow \quad \int_{C}kds \geqslant 2\pi$
		\item $\displaystyle\int_{\Sigma}|K|dA \geqslant 4\pi(1+g)$
		\item $\displaystyle\int_{\Sigma}H^2dA \geqslant 4\pi$
		\item $\displaystyle\int_{T^2}H^2dA \geqslant 2\pi^2$
	\end{itemize}
	\item 凸曲面(卵形面)
	\begin{itemize}
		\item Gauss映射与卵形面
		\begin{itemize}
			\item 紧致曲面存在点$p, K(p)>0$
			\item 凸 $\Rightarrow K \geqslant 0$恒
			\item $K>0$恒 $\Rightarrow$ 凸,且Gauss映射为一一映射
		\end{itemize}
		\item 积分公式
		\begin{itemize}
			\item $\displaystyle\int_{\Sigma}HdA=\int_{\Sigma}K\varphi dA \qquad \int_{\Sigma}dA=\int_{\Sigma}H\varphi dA$
			\item 此外,$\displaystyle\int_{\Sigma}ndA=\int_{\Sigma}HndA=\int_{\Sigma}KndA=0$
			\item $\displaystyle H \equiv C \Rightarrow \int_{\Sigma}(H^2-K)\varphi dA=0 \qquad \frac{H}{K} \equiv C \Rightarrow \int_{\Sigma}\frac{H^2-K}{K}dA=0$
		\end{itemize}
		\item 刚性
		$$\det (h_{\alpha\beta}-\bar{h}_{\alpha\beta})=2K-(\bar{h}_{11}h_{22}+h_{11}\bar{h}_{22}-2h_{12}\bar{h}_{12})$$
		$$0\leqslant \int \varphi \det(h_{\alpha\beta}-\bar{h}_{\alpha\beta})dA =2 \int(H-\bar{H})dA$$
	\end{itemize}
\end{itemize}


	\begin{minipage}[t]{.5\textwidth}
	\centering
\begin{equation*}
\begin{aligned}
&r(u,v)\\
\Rightarrow&r_u,r_v,E,F,G\\
\Rightarrow&r_u \wedge r_v,n\\
\Rightarrow&r_{uu},r_{uv},r_{vv},L,M,N\\
\Rightarrow&b_1^1,b_1^2,b_2^1,b_2^2\\
\Rightarrow&K,H,k_1,k_2,W(e_i)=k_ie_i\\
\Rightarrow&\text{渐近线:}L(du)^2+2Mdudv+N(dv)^2=0\\
&\text{曲率线:}\left|\begin{matrix}
(du^2)^2 & -du^1du^2 & (du^1)^2 \\
g_{11} & g_{12} & g_{22}\\
h_{11} & h_{12} & h_{22}
\end{matrix}\right|=0
\end{aligned}
\end{equation*}
\begin{equation*}
\begin{aligned}
&de_1=w_1^2e_2+w_1^3e_3\\
&\vec{k}=\frac{de_1}{ds}=\frac{w_1^2}{ds}e_2+\frac{w_1^3}{ds}e_3=k_ge_2+k_ne_3\\
&\frac{d^2r}{ds^2}=\left(\frac{d^2u^{\gamma}}{ds^2}+\Gamma_{\alpha \beta}^{\gamma}\frac{du^{\alpha}}{ds}\frac{du^{\beta}}{ds}\right)r_{\gamma}+b_{\alpha \beta}\frac{du^{\alpha}}{ds}\frac{du^{\beta}}{ds}n\\
\Rightarrow&\frac{d^2u^{\gamma}}{ds^2}+\Gamma_{\alpha \beta}^{\gamma}\frac{du^{\alpha}}{ds}\frac{du^{\beta}}{ds}=0\\
\end{aligned}
\end{equation*}
\begin{equation*}
\begin{aligned}
k_g&=\frac{d\theta}{ds}+\frac{w_1^2}{ds}\\
&=\frac{d\theta}{ds}+(k_g)_u \cos \theta+(k_g)_v \sin \theta\\
&=\frac{d\theta}{ds}-\frac{(\sqrt{E})_v}{\sqrt{EG}}\cos \theta + \frac{(\sqrt{G})_u}{\sqrt{EG}}\sin \theta
\end{aligned}
\end{equation*}
\begin{equation*}
\begin{aligned}
&I=w^1w^1+w^2w^2\\
\Rightarrow&w^1,w^2,w^1 \wedge w^2 (\Rightarrow dA)\\
\Rightarrow&dw^1,dw^2\\
\Rightarrow&w_1^2=\frac{dw^1}{w^1 \wedge w^2}w^1+\frac{dw^2}{w^1 \wedge w^2}w^2\\
\Rightarrow&K=-\frac{dw_1^2}{w^1 \wedge w^2}\\
\Rightarrow&KdA=-dw_1^2\\
\Rightarrow&\displaystyle \myiint_{D}KdA=-\oint_{\partial D}w_1^2 = \oint_{\partial D} d\theta - \oint_{\partial D}k_g ds \\
\text{角差}:&\beta(l)-\beta(0)=\oint_{\partial D}d\beta =-\oint_{\partial D}w_1^2=\myiint_DKdA\\
\text{一般}:&\myiint_{D}KdA+\oint_{\partial D}k_gds + \sum \alpha_i =2\pi \chi(D)\\
&\myiint_{\Sigma}KdA=2\pi \chi(\Sigma) \qquad \chi=2(1-g)
\end{aligned}
\end{equation*}



\end{minipage}
	\begin{minipage}[t]{.5\textwidth}
	\centering
\begin{equation*}
\begin{aligned}
&\text{自然标架运动方程}
\begin{dcases}
\frac{\partial \vec{r}}{\partial u^{\alpha}}=\vec{r}_{\alpha}\\
\frac{\partial \vec{r}_{\alpha}}{\partial u^{\beta}}=\Gamma_{\alpha \beta}^{\gamma}\vec{r}_{\gamma}+b_{\alpha \beta} \vec{n}\\
\frac{\partial \vec{n}}{\partial u^{\alpha}}=-b_{\alpha}^{\phantom{1}\beta}\vec{r}_{\beta}
\end{dcases}\\
&\text{外微分法}
\begin{dcases}
d\vec{r}=\vec{r}_{\alpha}du^{\alpha}\\
d\vec{r}_{\alpha}=\Gamma_{\alpha \beta}^{\gamma}du^{\beta}\vec{r}_{\gamma}+b_{\alpha \beta}du^{\beta}\vec{n}\\
d\vec{n}=-b_{\alpha}^{\phantom{1}\beta}du^{\alpha}\vec{r}_{\beta}
\end{dcases}\\
&\text{运动方程:}\begin{cases}
dr=w^{\alpha}e_{\alpha}\\
de_i=w_i^j e_j
\end{cases}\\
&\text{结构方程:}\begin{cases}
dw^{\alpha}=w^{\beta} \wedge w_{\beta}^{\alpha}\\
dw_i^j=w_i^k \wedge w_k^j
\end{cases}\\
&\displaystyle \Gamma_{\alpha \beta}^{\gamma}=\frac{1}{2} g^{\gamma \delta} \left\{ \frac{\partial g_{\alpha \delta}}{\partial u_{\beta}}+\frac{\partial g_{\beta \delta}}{\partial u_{\alpha}}-\frac{\partial g_{\alpha\beta }}{\partial u_{\delta}} \right\}\\
&D\vec{r}_{\alpha}=\Gamma_{\alpha \beta}^{\gamma}du^{\beta} \vec{r}_{\gamma}\\
&D_{\vec{r}_{\beta}}\vec{r}_{\alpha}=\Gamma_{\alpha \beta}^{\gamma}\vec{r}_{\gamma}\\
&\begin{cases}
\nabla f=\sum_{\alpha} f_{\alpha} e_{\alpha}\\
df=f_{\alpha} w^{\alpha}\\
D\nabla f:=Df_1e_1+Df_2e_2\\
\phantom{11111}=(df_1-f_2 w_1^2)e_1+(df_2+f_1 w_1^2)e_2\\
??=Df_{\alpha}w^{\alpha}
\end{cases}\\
\end{aligned}
\end{equation*}

$$
\begin{pmatrix}
Df_1\\Df_2
\end{pmatrix}
\begin{pmatrix}
f_{11} & f_{12} \\
f_{21} & f_{22}
\end{pmatrix}
\begin{pmatrix}
w^1\\w^2
\end{pmatrix}
$$

\begin{equation*}
\begin{aligned}
\Delta f=&f_{11}+f_{22}\\
\Delta_{\Sigma}f=&\frac{1}{\sqrt{\det g}}\frac{\partial}{\partial u^{\alpha}}\left(\sqrt{\det g} \frac{\partial f}{\partial u^{\beta}}g^{\alpha\beta}\right)\\
div_{\mu}X=&\sum \frac{1}{f} \partial_i(fX^i)\\
\mu=&\sqrt{\det g}\; du^1\!\!\wedge du^2\\
\Delta_{\Sigma}f=& div_{\Sigma} \nabla f
\end{aligned}
\end{equation*}

\end{minipage}
\end{document}
