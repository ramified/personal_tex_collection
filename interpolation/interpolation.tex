
\documentclass[reqno,UTF8]{amsart}
%Typical documenttypes: article/book
%some examples:
%\documentclass[reqno,11pt]{book}   %%%for books
%\documentclass[]{minimal}			%%%for Minimal Working Example


%for beamers, you have to change a lot. Especially, remove the package enumitem!!!



%%%%%%%%%%%%%%%%%%%% setting for fast compiling

%\special{dvipdfmx:config z 0}		% no compression

\includeonly{chapters/chapter9}		% In practice, use an empty document called "chapter9"	% usually for printing books






%%%%%%%%%%%%%%%%%%%% here we include packages

%%%basic packages for math articles
\usepackage{amssymb}
\usepackage{amsthm}
\usepackage{amsmath}
\usepackage{amsfonts}
\usepackage[shortlabels]{enumitem}	% It supersedes both enumerate and mdwlist. The package option shortlabels is included to configure the labels like in enumerate.

%%%packages for special symbols
\usepackage{pifont}					% Access to PostScript standard Symbol and Dingbats fonts
\usepackage{wasysym}				% additional characters
\usepackage{bm}						% bold fonts: \bm{...}
\usepackage{extarrows}				% may be replaced by tikz-cd
%\usepackage{unicode-math}			% unicode maths for math fonts, now I don't know how to include it
%\usepackage{ctex}					% Chinese characters, huge difference.


%%%basic packages for fancy electronic documents
\usepackage[colorlinks]{hyperref}
\usepackage[table,hyperref]{xcolor} 			% before tikz-cd. 
%\usepackage[table,hyperref,monochrome]{xcolor}	% disable colored output (black and white)

%%%packages for figures and tables (general setting)
\usepackage{float}				%Improved interface for floating objects
\usepackage{caption,subcaption}
\usepackage{adjustbox}			% for me it is usually used in tables 
\usepackage{stackengine}		%baseline changes

%%%packages for commutative diagrams
\usepackage{tikz-cd}
%\usepackage{quiver}			% see https://q.uiver.app/

%%%packages for pictures
\usepackage[width=0.5,tiewidth=0.7]{strands}
\usepackage{graphicx}			% Enhanced support for graphics

%%%packages for tables and general settings
\usepackage{array}
\usepackage{makecell}
\usepackage{multicol}
\usepackage{multirow}
\usepackage{diagbox}
\usepackage{longtable}

%%%packages for ToC, LoF and LoT







 %https://tex.stackexchange.com/questions/58852/possible-incompatibility-with-enumitem










%%%%%%%%%%%%%%%%%%%% here we include theoremstyles

\numberwithin{equation}{section}

\theoremstyle{plain}
\newtheorem{theorem}{Theorem}[section]

\newtheorem{setting}[theorem]{Setting}
\newtheorem{definition}[theorem]{Definition}
\newtheorem{lemma}[theorem]{Lemma}
\newtheorem{proposition}[theorem]{Proposition}
\newtheorem{corollary}[theorem]{Corollary}
\newtheorem{conjecture}[theorem]{Conjecture}

\newtheorem{claim}[theorem]{Claim}
\newtheorem{eg}[theorem]{Example}
\newtheorem{ex}[theorem]{Exercise}
\newtheorem{fact}[theorem]{Fact}
\newtheorem{ques}[theorem]{Question}
\newtheorem{warning}[theorem]{Warning}



\newtheorem*{bbox}{Black box}
\newtheorem*{notation}{Conventions and Notations}


\numberwithin{equation}{section}


\theoremstyle{remark}

\newtheorem{remark}[theorem]{Remark}
\newtheorem*{remarks}{Remarks}

%%% for important theorems
%\newtheoremstyle{theoremletter}{4mm}{1mm}{\itshape}{ }{\bfseries}{}{ }{}
%\theoremstyle{theoremletter}
%\newtheorem{theoremA}{Theorem}
%\renewcommand{\thetheoremA}{A}
%\newtheorem{theoremB}{Theorem}
%\renewcommand{\thetheoremB}{B}







%%%%%%%%%%%%%%%%%%%% here we declare some symbols

%%%%%%%DeclareMathOperator
%see here for why newcommand is better for DeclareMathOperator: https://tex.stackexchange.com/questions/67506/newcommand-vs-declaremathoperator

%%%%%basic symbols. Keep them!

%%%symbols for sets and maps
\DeclareMathOperator{\pt}{\operatorname{pt}}	%points. Other possibilities are \{pt\}, \{*\}, pt, * ...
\DeclareMathOperator{\Id}{\operatorname{Id}}	%identity in groups.
\DeclareMathOperator{\Img}{\operatorname{Im}}

\DeclareMathOperator{\Ob}{\operatorname{Ob}}
\DeclareMathOperator{\Mor}{\operatorname{Mor}}	%difference of Mor and Hom: Hom is usually for abelian categories
\DeclareMathOperator{\Hom}{\operatorname{Hom}}	\DeclareMathOperator{\End}{\operatorname{End}}
\DeclareMathOperator{\Aut}{\operatorname{Aut}}

%%%symbols for linear algebras and 
%%linear algebras
\DeclareMathOperator{\tr}{\operatorname{tr}}
\DeclareMathOperator{\diag}{\operatorname{diag}}	%for diagonal matrices

%%abstract algebras
\DeclareMathOperator{\ord}{\operatorname{ord}}
\DeclareMathOperator{\gr}{\operatorname{gr}}
\DeclareMathOperator{\Frac}{\operatorname{Frac}}

%%%symbols for basic geometries
\DeclareMathOperator{\vol}{\operatorname{vol}}	%volume
\DeclareMathOperator{\dist}{\operatorname{dist}}
\DeclareMathOperator{\supp}{\operatorname{supp}}

%%%symbols for category
%%names of categories
\DeclareMathOperator{\Mod}{\operatorname{Mod}}
\DeclareMathOperator{\Vect}{\operatorname{Vect}}
\DeclareMathOperator{\rep}{\operatorname{rep}} %usually rep means the category of finite dimensional representations, while Rep means the category of representations.
\DeclareMathOperator{\Rep}{\operatorname{Rep}}


%%%symbols for homological algebras
\DeclareMathOperator{\Tor}{\operatorname{Tor}}
\DeclareMathOperator{\Ext}{\operatorname{Ext}}
\DeclareMathOperator{\gldim}{\operatorname{gl.dim}}
\DeclareMathOperator{\projdim}{\operatorname{proj.dim}}
\DeclareMathOperator{\injdim}{\operatorname{inj.dim}}
\DeclareMathOperator{\rad}{\operatorname{rad}}


%%%symbols for algebraic groups
\DeclareMathOperator{\GL}{\operatorname{GL}}
\DeclareMathOperator{\SL}{\operatorname{SL}}

%%%symbols for typical varieties
\DeclareMathOperator{\Gr}{\operatorname{Gr}}
\DeclareMathOperator{\Flag}{\operatorname{Flag}}

%%%symbols for basic algebraic geometry
\DeclareMathOperator{\Spec}{\operatorname{Spec}}
\DeclareMathOperator{\Coh}{\operatorname{Coh}}
\newcommand{\Dcoh}{\mathcal{D}_{\operatorname{Coh}}}%%%This one shows the difference between \DeclareMathOperator and \newcommand
\DeclareMathOperator{\Pic}{\operatorname{Pic}}
\DeclareMathOperator{\Jac}{\operatorname{Jac}}

%%%%%advanced symbols. Choose the part you need!

%%%symbols for algebraic representation theory
\DeclareMathOperator{\Irr}{\operatorname{Irr}}
\DeclareMathOperator{\ind}{\operatorname{ind}}	%\ind(Q) means the set of  equivalence classes of finite dimensional indecomposable representations
\DeclareMathOperator{\Res}{\operatorname{Res}}
\DeclareMathOperator{\Ind}{\operatorname{Ind}}
\DeclareMathOperator{\cInd}{\operatorname{c-Ind}}


%%%symbols for algebraic topology
\DeclareMathOperator{\EGG}{\operatorname{E}\!}
\DeclareMathOperator{\BGG}{\operatorname{B}\!}

\DeclareMathOperator{\chern}{\operatorname{ch}^{*}}
\DeclareMathOperator{\Td}{\operatorname{Td}}
\DeclareMathOperator{\AS}{\operatorname{AS}}	%Atiyah--Segal completion theorem 

%%%symbols for Auslander--Reiten theory 
\DeclareMathOperator{\Modup}{\overline{\operatorname{mod}}}
\DeclareMathOperator{\Moddown}{\underline{\operatorname{mod}}}
\DeclareMathOperator{\Homup}{\overline{\operatorname{Hom}}}
\DeclareMathOperator{\Homdown}{\underline{\operatorname{Hom}}}


%%%symbols for operad
\DeclareMathOperator{\Com}{\operatorname{\mathcal{C}om}}
\DeclareMathOperator{\Ass}{\operatorname{\mathcal{A}ss}}
\DeclareMathOperator{\Lie}{\operatorname{\mathcal{L}ie}}
\DeclareMathOperator{\calEnd}{\operatorname{\mathcal{E}nd}} %cal=\mathcal


%%%%%personal symbols. Use at your own risk!

%%%symbols only for master thesis
\DeclareMathOperator{\ptt}{\operatorname{par}}	%the partition map
\DeclareMathOperator{\str}{\operatorname{str}}	%strict case
\DeclareMathOperator{\RRep}{\widetilde{\operatorname{Rep}}}
\DeclareMathOperator{\Rpt}{\operatorname{R}}
\DeclareMathOperator{\Rptc}{\operatorname{\mathcal{R}}}
\DeclareMathOperator{\Spt}{\operatorname{S}}
\DeclareMathOperator{\Sptc}{\operatorname{\mathcal{S}}}
\DeclareMathOperator{\Kcurl}{\operatorname{\mathcal{K}}}
\DeclareMathOperator{\Hcurl}{\operatorname{\mathcal{H}}}
\DeclareMathOperator{\eu}{\operatorname{eu}}
\DeclareMathOperator{\Eu}{\operatorname{Eu}}
\DeclareMathOperator{\dimv}{\operatorname{\underline{\mathbf{dim}}}}
\DeclareMathOperator{\St}{\mathcal{Z}}

%%%%%symbols which haven't been classified. Add your own math operators here!

\DeclareSymbolFont{bbold}{U}{bbold}{m}{n}
\DeclareSymbolFontAlphabet{\mathbbold}{bbold}
\DeclareMathOperator{\indicator}{\mathbbold{1}}
%https://tex.stackexchange.com/questions/488/blackboard-bold-characters/3260

\DeclareMathOperator{\Real}{\operatorname{Re}}
\DeclareMathOperator{\ev}{\operatorname{ev}}
\DeclareMathOperator{\LHS}{\operatorname{LHS}}
\DeclareMathOperator{\RHS}{\operatorname{RHS}}
\makeatletter
\newcommand*\rel@kern[1]{\kern#1\dimexpr\macc@kerna}
\newcommand*\widebar[1]{%
  \begingroup
  \def\mathaccent##1##2{%
    \rel@kern{0.8}%
    \overline{\rel@kern{-0.8}\macc@nucleus\rel@kern{0.2}}%
    \rel@kern{-0.2}%
  }%
  \macc@depth\@ne
  \let\math@bgroup\@empty \let\math@egroup\macc@set@skewchar
  \mathsurround\z@ \frozen@everymath{\mathgroup\macc@group\relax}%
  \macc@set@skewchar\relax
  \let\mathaccentV\macc@nested@a
  \macc@nested@a\relax111{#1}%
  \endgroup
}
\makeatother



%%%%%%%newcommand

%%%basic symbols
\newcommand{\norm}[1]{\Vert{#1}\Vert}

%%%symbols only for master thesis
\newcommand{\dimvec}[1]{\mathbf{#1}}
\newcommand{\abdimvec}[1]{|\dimvec{#1}|}
\newcommand{\ftdimvec}[1]{\underline{\dimvec{#1}}}

\newcommand{\absgp}[1]{\mathbb{#1}}
\newcommand{\WWd}{\absgp{W}_{\abdimvec{d}}}
\newcommand{\Wd}{W_{\dimvec{d}}}
\newcommand{\MinWd}{\operatorname{Min}(\absgp{W}_{\abdimvec{d}},W_{\dimvec{d}})}
\newcommand{\Compd}{\operatorname{Comp}_{\dimvec{d}}}
\newcommand{\Shuffled}{\operatorname{Shuffle}_{\dimvec{d}}}

\newcommand{\Omcell}{\Omega}
\newcommand{\OOmcell}{\boldsymbol{\Omega}}
\newcommand{\Vcell}{\mathcal{V}}
\newcommand{\VVcell}{\boldsymbol{\mathcal{V}}}
\newcommand{\Ocell}{\mathcal{O}}
\newcommand{\OOcell}{\boldsymbol{\mathcal{O}}}
\newcommand{\preimage}[1]{\widetilde{#1}}
\newcommand{\orde}{\operatorname{ord}_e}
\newcommand{\fakestar}{*}

%as the subscription of Hom
\newcommand{\Alggp}{\text{-Alg gp}}







%%%%%%%%%%%%%%%%%%%% here we make some blocks for special features. 

%%%% todo notes %%%%
\usepackage[colorinlistoftodos,textsize=footnotesize]{todonotes}
\setlength{\marginparwidth}{2.5cm}
\newcommand{\leftnote}[1]{\reversemarginpar\marginnote{\footnotesize #1}}
\newcommand{\rightnote}[1]{\normalmarginpar\marginnote{\footnotesize #1}\reversemarginpar}









%%%%%%%%%%%%%%%%%%%% here we make some global settings. Understand everything here before you make a document!

\usepackage[a4paper,left=3cm,right=3cm,bottom=4cm]{geometry}
\usepackage{indentfirst}	% Indent first paragraph after section header

\setcounter{tocdepth}{2}


%https://latexref.xyz/_005cparindent-_0026-_005cparskip.html
\setlength{\parindent}{15pt}	
\setlength{\parskip}{0pt plus1pt}

%\setlength\intextsep{0cm}
%\setlength\textfloatsep{0cm}
\def\arraystretch{1}
%\setcounter{secnumdepth}{3}

\allowdisplaybreaks


\begin{document}

% The beginning depends on the documentclass. Rewrite this part if you use different documentclass!
\date{\today}

\title
{Calderón's complex interpolation method
}
\author{empty}
\address{School of Mathematical Sciences\\
University of Bonn\\
Bonn, 53115\\ Germany\\} 
\email{email:xx352229@mail.ustc.edu.cn}


\maketitle
\tableofcontents


\section{Calderón's complex interpolation method (Due to Alberto Calderón)}


In this section, we will try to show that
\begin{equation}\label{eq: Fourier middle}
\mathcal{F} \left( L^p(\Omega) \right) \subseteq L^q(\Omega) \quad \text{ for } 1 \le p \le 2, \; \frac{1}{p}+ \frac{1}{q}=1
\end{equation}
under the help of complex interpolation method. Surprisingly, this method stems from a theorem in complex analysis, call the three-lines theorem.

\begin{theorem}[Three lines theorem, due to Hadamard]\label{thm:three_lines}
Let
\begin{equation*}
\begin{aligned}
  \Omega:&\!=\; \left\{ z \in \mathbb{C} : 0 < \Real <1 \right\} \\ 
  E:& \text{Banach space} \\
  f:&\; \widebar{\Omega} \longrightarrow E \text{ is bounded, continuous, and $f|_{\Omega}$ is holomorphic.}\\
\end{aligned}
\end{equation*}
For $0 \le \theta \le 1$, define
$$M_{\theta}(f):= \sup_{t \in \mathbb{R}} \norm{f(r+it)} < + \infty,$$
then
$$M_{\theta}(f) \le \left(M_{0}(f)\right)^{\theta}\left(M_{1}(f)\right)^{1-\theta}.$$
This is equivalent to say, the function 
$$[0,1] \longrightarrow \mathbb{R} \qquad \theta \longmapsto \log M_{\theta}(f)$$
is convex.
\end{theorem}
\begin{remark}
We say $f: \Omega \longrightarrow E$ is \textbf{holomorphic} if $f$ satisfies Riemann-Cauchy equation. Equivalently, $f: \Omega \longrightarrow E$ is holomorphic if for any $\phi \in E'$, the composition
$$\Omega \stackrel{f}{\longrightarrow} E \stackrel{\phi}{\longrightarrow} \mathbb{C}$$
is holomorphic.
\end{remark}

The proof use the Phragmén--Lindelöf method. Before the proof, let me recall the maximum principle.

\begin{theorem}[Maximum principle for holomorphic functions]
Let $\Omega$ be a bounded open subset of $\mathbb{C}$, $f: \widebar{\Omega} \longrightarrow \mathbb{C}$ be continuous and holomorphic (in $\Omega$). Then for any $z \in \widebar{\Omega}$,
$$\norm{f(z)}_E \le \sup_{w \in \partial \Omega} |f(w)|. $$
\end{theorem}

\begin{proof}[Proof of Theorem \ref{thm:three_lines}, by Phragmén--Lindelöf method]\hspace{3mm}\\
\paragraph*{\underline{\textbf{Step 1}}}Prove the theorem for $E=\mathbb{C}$, $M_0(f)=M_1(f)=1$ case. In this case, we need to show
$$|f(z)| \le 1 \qquad \text{ for any } z \in \Omega.$$\\[-3mm]
\underline{\textbf{Idea}}: introduce a multiplicative factor $e^{\frac{z^2-1}{n}}$ to ``subdue" the growth of $f$, so that we can use maximal principle to get the bound.

Let $f_n(z):= e^{\frac{z^2-1}{n}} f(z)$, then there exists $R>0$ (depend on $n$), such that 
$$|f_n(z)| \le 1 \qquad \text{ for } z \in \widebar{\Omega}, \;\; |\Img z| \ge R.$$ 
By the maximal principle for $\left\{ z \in \widebar{\Omega}: |\Img z| \le R  \right\}$,
$$|f_n(z)| \le 1 \qquad \text{ for } z \in \widebar{\Omega}, \;\; |\Img z| \le R.$$ 
Therefore, $\norm{f_n} \le 1$. As a result,
$$|f(z)| = \lim_{n \rightarrow \infty} |f_n(z)| \le 1.$$ 


\paragraph*{\underline{\textbf{Step 2}}}Prove the theorem for $E=\mathbb{C}$ case. \\[-3mm]

Define 
$$g: \widebar{\Omega} \longrightarrow \mathbb{C} \qquad g(z):= M_0(f)^{z-1} M_1(f)^{-z} f(z),$$
and apply $g$ to Step 1.

\paragraph*{\underline{\textbf{Step 3}}}General case.\\[-3mm]

For $\phi \in E'$, $\norm{\phi}_{E'}\le 1$, define $h_{\phi}:= \phi \circ f$:
$$h_{\phi}: \widebar{\Omega} \stackrel{f}{\longrightarrow} E \stackrel{\phi}{\longrightarrow} \mathbb{C},$$
then
$$|h_{\phi}(z)|=\left|\phi\left(\rule{0mm}{3.1mm}f(z)\right)\right| \le \norm{\phi}_{E'} \norm{f(z)}_{E} \le \norm{f(z)}_{E}.$$
Apply $h_{\phi}$ to Step 2, we get 
$$|h_{\phi}(z)| \le M_{0}(h_{\phi})^{\theta}M_{1}(h_{\phi})^{1-\theta} \le M_{0}(f)^{\theta}M_{1}(f)^{1-\theta} \qquad \text{ for any } z \in \widebar{\Omega}, \Real z = \theta,$$
so
$$\norm{f(z)}_E = \sup_{\substack{\phi \in E'\\ \norm{\phi} \le 1}} \left| \left< \phi, f(z) \right> \right| = \sup_{\substack{\phi \in E'\\ \norm{\phi} \le 1}} \left| h_{\phi}(z) \right| \le M_{0}(f)^{\theta}M_{1}(f)^{1-\theta}.$$
\end{proof}

Somewhat surprising, Theorem \ref{thm:three_lines} offers us a way to construct ``a continuous deformation between two Banach spaces". Intuitively, these intermediate spaces must lie in the sum of these two Banach spaces. First, we try to give a norm to this ambiance space.

\begin{proposition}
Let $E_0$, $E_1$ be two Banach spaces contained in some topological vector space $V$.??? Then $E_0 \oplus E_1$ is a Banach space with norm
$$\norm{(x,y)} = \norm{x}_{E_0} + \norm{y}_{E_1},$$
$E_0 \oplus E_1$ is a Banach space with norm
$$\norm{x+y} = \inf_{\xi \in E_0 \cap E_1} \left\{ \norm{x+ \xi}_{E_0} + \norm{y-\xi}_{E_1} \right\}.$$
\end{proposition}

\begin{proof}
Conditions on norm are relatively easy to check, but I don't know how to show completeness.
\end{proof}

\begin{lemma}
The injection $j: E_0 \hookrightarrow E_0 + E_1$ is continuous of norm $\le 1$.
\end{lemma}
\begin{proof}
We have the estimation
$$\norm{x+0}_{E_0+E_1} = \inf_{\xi \in E_0 \cap E_1} \left\{ \norm{x+ \xi}_{E_0} + \norm{-\xi}_{E_1} \right\} \le \norm{x}_{E_0}.$$
\end{proof}
\begin{warning}
The injection $j$ may be not topological embedding, i.e., $E_0 \hookrightarrow \Img j$ may be not homeomorphism.
\end{warning}

\begin{definition}[Interpolation spaces]
For two Banach spaces $E_0$, $E_1$ contained in some vector space $V$, we define the space
$$\mathcal{H}: = \mathcal{H}(E_0,E_1) := \left\{ f: \widebar{\Omega} \longrightarrow E_0 +E_1  \;\middle|\; 
\begin{aligned}
  & \text{$f$ is continuous and bounded} \\ 
  & \text{$f|_{\Omega}$ is holomorphic} \\ 
  & f(it) \in E_0,\; f(1+it) \in E_1, \text{ for any } t \in \mathbb{R} \\   
\end{aligned}
 \right\}$$
 to be the Banach space with norm
 $$\norm{f}_{\mathcal{H}}:= \max \left( M_0(f), M_1(f)  \right)= \norm{|f|}_{\infty}.$$
 For $0 < \theta <1$, define the \textbf{interpolation space}
 $$E_{\theta}:= [E_0,E_1]_{\theta}:= \mathcal{H}(E_0,E_1)/\left\{ f \in \mathcal{H}: f(\theta)=0 \right\},$$
 i.e., the image of the map
 $$\ev_{\theta}: \mathcal{H}(E_0,E_1) \longrightarrow E_0+E_1 \qquad f \longmapsto f(\theta).$$
 Notice that we only take the norm of $E_{\theta}$ as the residue norm of $\mathcal{H}$, instead of the subspace norm of $E_0+E_1$.
\end{definition}
\begin{ques}
Are these two norms the same norm?
\end{ques}
\begin{remark}
It is natural to set $\theta=0$, and guess $[E_0,E_1]_0=E_0$, but this is false in general. Consider $E_0=\mathbb{C}, E_1=0$, then
$$\mathcal{H}(E_0,E_1)=0 \quad\Longrightarrow\quad [E_0,E_1]_{\theta}=0 \quad \text{ for any } \theta.$$ 
\end{remark}

Here we list some immediate properties of the interpolation spaces.
\begin{lemma}
$[E_0,E_1]_{\theta}=[E_1,E_0]_{1-\theta}$.
\end{lemma}
\begin{lemma}
For $\xi \in [E_0,E_1]_{\theta}$,
$$\norm{\xi}_{\theta}= \inf_{\substack{f\in \mathcal{H}\\ \bar{f}=\xi}} \left\{ M_0(f)^{1-\theta} M_1(f)^{\theta} \right\}.$$
\end{lemma}
\begin{proof}
For the easy direction,
$$\LHS = \inf_{\substack{f\in \mathcal{H}\\ \bar{f}=\xi}} \norm{f}_{\mathcal{H}}= \inf_{\substack{f\in \mathcal{H}\\ \bar{f}=\xi}} \left\{ \max \left(\rule{0mm}{3.2mm}M_0(f), M_1(f) \right)\right\} \geq \RHS.$$
To show $\RHS \le \LHS$, one needs to show that
$$\text{ For } f \in \mathcal{H}, \text{ there exists } g \in \mathcal{H}, g(\theta)=f(\theta), \text{ such that } \norm{g}_{\mathcal{H}} \le  M_0(f)^{1-\theta} M_1(f)^{\theta}.$$
Then $g(z):= M_0(f)^{z-1} M_1(f)^{-z}f(z)$ satisfy the condition. ???
\end{proof}

The next theorem gives us a perfect example.

\begin{theorem}[Riesz-Thorin interpolation theorem]\label{thm:interpolation_L_p}
Let $(X,\mathcal{A}, \mu)$ be a $\sigma$-finite space, $1 \le p < q \le \infty$, $0 \le \theta \le 1$, $p'$, $q'$ be the conjugate indices of $p$, $q$. Let $r \in \mathbb{R}$ such that $\frac{1}{r}= \frac{1-\theta}{p}+\frac{\theta}{q}$, then
$$\left[ L^p(X), L^q(X) \right]_{\theta}\cong L^r(X)$$
as Banach spaces.
\end{theorem}

\begin{eg}$\,$

When $q= \infty$, $r= \frac{1}{1-\theta} \cdot p$.

When $\theta=\frac{1}{2}$, $\frac{2}{r}=\frac{1}{p}+\frac{1}{q}$, $(0,p,r,q)$ is a harmonic range.
\end{eg}

\begin{proof}
We do the case $q < +\infty$. Let
$$L^0(X):= \left\{ \text{ measurable functions } \right\} / \text{ null functions }$$
be an ambiance space, and $f \in L^r(X)$ be a representative (i.e., a function). We need three steps:

\paragraph*{\underline{\textbf{Step 1}}}Let $f \in L^r(X)$, construct $\phi \in \mathcal{H}\left(\rule{0mm}{3.2mm}L^p(X), L^q(X)\right)$ such that $\phi(\theta)=f$.\\[-3mm]

For this, define
$$\phi: \widebar{\Omega} \longrightarrow L^p(X) + L^q(X) 
\qquad \phi(z)= \frac{f(-)}{|f(-)|} |f(-)|^{ r\left(\textstyle \frac{1-z}{p}+\frac{z}{q} \right)} \indicator_{\{|f|>0\}}.$$
 We need to verify:
 \begin{itemize}
 \item For a fixed $z$, $\phi(z) \in L^p(X)+L^q(X)$;
 \item $\phi \in \mathcal{H}\left(\rule{0mm}{3.2mm}L^p(X), L^q(X)\right)$;
 \item $\phi(\theta)=f$.
 \end{itemize}
 
\paragraph*{\underline{\textbf{Step 2}}}For $\phi \in \mathcal{H}\left(\rule{0mm}{3.2mm}L^p(X), L^q(X)\right)$, show that $\phi(\theta) \in L^r(X)$.\\[-3mm]

For proving this, we need a fact from the duality theory:

\begin{fact}
Given $h \in L^0(X)$ and $r$, $r'$ as conjugate indices. If for all simple functions\footnote{For simple functions, we mean the function of form $$\sum_{\text{fin}} a_i \indicator_{A_i}$$ where all $A_i$ are measurable sets with finite measure.} $g$ we have
$$h \cdot g \in L^1, \qquad \int |h \cdot g| d \mu \le C \cdot \norm{g}_{r'},$$
then $h \in L^r$ and $\norm{h}_{L^r} \le C$.
\end{fact}

From this fact, one needs to estimate
$$\int |\phi(\theta) \cdot g| d \mu \le C \cdot \norm{g}_{r'}$$
for any simple function $g$. Now fix $g$, define
\begin{equation*}
\begin{aligned}
  &\psi: \widebar{\Omega} \longrightarrow L^{p'}(X) + L^{q'}(X)\;\; &&  \psi(z)= \frac{g(-)}{|g(-)|} |g(-)|^{ r'\left(\textstyle \frac{1-z}{p'}+\frac{z}{q'} \right)} \indicator_{\{|g|>0\}}\\ 
  & H: \widebar{\Omega} \longrightarrow \mathbb{C} && H(z):= \int_{L^1} \phi(z)\psi(z)d\mu.\\ 
\end{aligned}
\end{equation*}

???
\paragraph*{\underline{\textbf{Step 3}}}For $\xi \in \left[ L^p(X), L^q(X) \right]_{\theta}$, show that $\norm{\xi}_{\theta}=\norm{\xi}_{L^r}$.\\[-3mm]

???
\end{proof}

Finally, we state the main theorem of this section. The inclusion \ref{eq: Fourier middle} is a natural corollary of Theorem \ref{thm:interpolation_operator}.
\begin{theorem}[Abstract Riesz-Thorin]\label{thm:interpolation_operator}
Given $E_0$, $E_1$; $F_0$, $F_1$ two pairs of Banach spaces as before???, $0 < \theta < 1$. Suppose $T: E_0 +E_1 \longrightarrow F_0 + F_1$ is linear with
$$T(E_0) \subseteq F_0, \quad T(E_1) \subseteq F_1,$$
then
$$T([E_0,E_1]_{\theta}) \subseteq [F_0,F_1]_{\theta}.$$
Moreover, if $T|_{E_0}$, $T|_{E_1}$ are bounded, then $T|_{E_{\theta}}$ is bounded, and 
$$\norm{T}_{\theta} \le \norm{T}_0^{1-\theta} \norm{T}_1^{\theta}.$$
\end{theorem}
\begin{proof}
Let $xi \in [E_0,E_1]_{\theta}$, we need to show $T(\xi) \in [F_0,F_1]_{\theta}$, and give an estimation of $T(\xi)$. For any $\varepsilon > 0$, we choose $f \in \mathcal{H}(E_0,E_1)$, $\bar{f}=\xi$ such that 
$$\norm{\xi}_{\theta} \le \norm{f}_{\mathcal{H }} \le \norm{\xi}_{\theta} + \varepsilon.$$
Then $T(f) \in \mathcal{H}(F_0,F_1)$ ($\Rightarrow T(\xi) \in [F_0,F_1]_{\theta}$), and 
\begin{equation*}
\begin{aligned}
  \;& M_0(T(f)) \le \norm{T}_0 M_0(f) \qquad M_1(T(f)) \le \norm{T}_1 M_1(f)\\ 
  \Longrightarrow\;& M_{\theta}(T(f)) \le \norm{T}_0^{1-\theta} \norm{T}_1^{\theta} M_0(f)^{1-\theta} M_1(f)^{\theta} \\
   \;& \phantom{M_{\theta}(T(f))} \le \norm{T}_0^{1-\theta} \norm{T}_1^{\theta} \left( \norm{\xi}_{\theta} +\varepsilon \right)
\end{aligned}
\end{equation*}
Let $\varepsilon \rightarrow 0$, we get the bound.
\end{proof}


% Remember to protect the uppercase of people's name and LaTeX symbols

\end{document}