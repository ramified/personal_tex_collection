
\documentclass{ctexart}

%\usepackage{color,graphicx}
%\usepackage{mathrsfs,amsbsy}
\usepackage{CJK}
\usepackage{amssymb}
\usepackage{amsmath}
\usepackage{amsfonts}
\usepackage{graphicx}
\usepackage{amsthm}
\usepackage{enumerate}
\usepackage[mathscr]{eucal}
\usepackage{mathrsfs}
\usepackage{verbatim}

%\usepackage[notcite,notref]{showkeys}

% showkeys  make label explicit on the paper

\makeatletter
\@namedef{subjclassname@2010}{%
  \textup{2010} Mathematics Subject Classification}
\makeatother

\numberwithin{equation}{section}

\theoremstyle{plain}
\newtheorem{theorem}{Theorem}[section]
\newtheorem{lemma}[theorem]{Lemma}
\newtheorem{proposition}[theorem]{Proposition}
\newtheorem{corollary}[theorem]{Corollary}
\newtheorem{claim}[theorem]{Claim}
\newtheorem{defn}[theorem]{Definition}

\theoremstyle{plain}
\newtheorem{thmsub}{Theorem}[subsection]
\newtheorem{lemmasub}[thmsub]{Lemma}
\newtheorem{corollarysub}[thmsub]{Corollary}
\newtheorem{propositionsub}[thmsub]{Proposition}
\newtheorem{defnsub}[thmsub]{Definition}

\numberwithin{equation}{section}


\theoremstyle{remark}
\newtheorem{remark}[theorem]{Remark}
\newtheorem{remarks}{Remarks}


\renewcommand\thefootnote{\fnsymbol{footnote}}
%dont use number as footnote symbol, use this command to change

\DeclareMathOperator{\supp}{supp}
\DeclareMathOperator{\dist}{dist}
\DeclareMathOperator{\vol}{vol}
\DeclareMathOperator{\diag}{diag}
\DeclareMathOperator{\tr}{tr}

\begin{document}
\date{}

\title
{二阶椭圆PDE作业}


\author{第一周}
%\address{School of Mathematical Sciences\\
%University of Science and Technology of China\\
%Hefei, 230026\\ P.R. China\\} 
%\email{email:xx352229@mail.ustc.edu.cn}
\maketitle




%\begin{abstract}
%\end{abstract}




%%%%%%%%%%%%%%%%%%%%%%%%%%%%%%%%%%%%%%%%%%%%%%%%%%%%%%%%%%%%%%%%%%%%%%%%%%%%%%%%%%%%%%%%%%%%%


\section{Exercise 1}
取$H=H_0^2(\Omega), u \in H_0^2(\Omega),$则$\lVert u \rVert_{H}=\lVert \Delta u \rVert_{L^2(\Omega)}$.再取
$$a(u,v):=\int_{\Omega} \Delta u \Delta v dx$$
$$F:H_0^1(\Omega)\longrightarrow \mathbb{R} \qquad v \longmapsto \int_{\Omega}fvdx$$
  则有
  \begin{itemize}
  	\item $F \in H^{-1}(\Omega)$
  	\item \begin{equation*}
  		\begin{aligned}
  			|a(u,v)| &\leqslant \int_{\Omega} |\Delta u||\Delta v| dx\\
  			&\leqslant C \lVert\Delta u\rVert_{L^2}\lVert\Delta v\rVert_{L^2}\\
  			&\leqslant \tilde{C} \lVert u\rVert_{H_0^2(\Omega)}\lVert v\rVert_{H_0^2(\Omega)}
  		\end{aligned}
  	\end{equation*}
  	  	\item \begin{equation*}
  	\begin{aligned}
  	a(u,u) &=\int_{\Omega} \Delta u \Delta u dx\\
  	&= \lVert\Delta u\rVert_{L^2(\Omega)}\\
  	&\geqslant C'\lVert u\rVert_{H_0^2(\Omega)}
  	\end{aligned}
  	\end{equation*}
  \end{itemize}
则由lax-milgram引理,我们有唯一的$u \in H,$使得
$a(u,v)=\left<F,v\right>$,i.e.
$$\int_{\Omega} \Delta u \Delta v dx=\int_{\Omega}fvdx \text{ for any } v \in H_0^2(\Omega)$$
故此双调和方程存在唯一的弱解.
\section{Exercise 2}
回忆
$$W^{1,p}(B_1)=\{u \in L^p(B_1) \mid u_{x_i} \in L^p(B_1)\}$$
我们有

\begin{minipage}[t]{.4\textwidth}
\centering
\begin{equation*}
\begin{aligned}
	& u(x)=|x|^{-\alpha} \in L^p(B_1)\\
	\Leftrightarrow & \int_{B_1}|x|^{-\alpha}dV < +\infty\\
	\Leftrightarrow & \int_{0}^{1}r^{n-1}r^{-\alpha p}dr < +\infty\\
	\Leftrightarrow & (n-1)-\alpha p >-1\\
	\Leftrightarrow & \alpha < \frac{n}{p}
\end{aligned}
\end{equation*}
\begin{equation*}
\begin{aligned}
u_{x_i}(x)&=-\frac{\alpha}{2}|x|^{-\alpha-2}\frac{\partial |x|^2}{\partial x_i}\\
&=-\alpha |x|^{-\alpha-2}x_i
\end{aligned}
\end{equation*}
\end{minipage}
\begin{minipage}[t]{.6\textwidth}
\centering
\begin{equation*}
\begin{aligned}
& u_{x_i}(x)=-\alpha |x|^{-\alpha-2}x_i \in L^p(B_1)\\
\Leftrightarrow & \int_{B_1} \alpha^p (|x|^{-2\alpha -4}|x_i|^2)^{p/2} dV< +\infty\\
\Leftrightarrow & \sum_{i=1}^{n}\int_{B_1} (|x|^{-2\alpha -4}|x_i|^2)^{p/2} dV< +\infty\\
\Leftrightarrow & \int_{B_1} (|x|^{-\alpha -1})^p dV< +\infty\\
\Leftrightarrow	& u(x)=|x|^{-\alpha-1} \in L^p(B_1)\\
\Leftrightarrow & \alpha < \frac{n-p}{p}
\end{aligned}
\end{equation*}
\end{minipage}
故
$$u(x)=|x|^{-\alpha} \in W^{1,p}(B_1) \Longleftrightarrow \alpha < \frac{n-p}{p}$$

\begin{equation*}
\begin{aligned}
内容...
\end{aligned}
\end{equation*}
             





%%%%%%%%%%%%%%%%%%%%%%%%%%%%%%%%%%%%%%%%%%%%%%%%%%%%%%%%%%%%%%%%%%%%%%%%%%%%%%%%%%%%%%%%%%%%%

 
   


%%%%%%%%%%%%%%%%%%%%%%%%%%%%%%%%%%%%%%%%%%%%%%%%%%%%%%%%%%%%%%%%%%%%%%%%%%

 




%%%%%%%%%%%%%%%%%%%%%%%%%%%%%%%%%%%%%%%%%%%%%%%%%%%%%%%%%%%%%%%%%%%%%%%%%%%%%%%%%%%%%%%%%%%%%%%




%\begin{thebibliography}{99}

 
%\bibitem{AF12}%
%Antunes, P., Freitas, P.: Optimal spectral rectangles and lattice ellipses. \emph{Proc. Royal Soc. London Ser. A.} \textbf{469} (2012), 20120492.


  

%\end{thebibliography}


\end{document}




