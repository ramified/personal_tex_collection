\documentclass{ctexart}
\usepackage{amsmath}
\usepackage{amsthm}
\usepackage{amssymb}
\usepackage{delarray}%编排数组、矩阵
\usepackage{mathtools}
\usepackage{extarrows}
\usepackage{xcolor}
\usepackage{geometry} %调整页面的页边距
\usepackage{hyperref}
\geometry{left=2.5cm,right=2.5cm,top=3cm,bottom=3cm}
\title{我看过的书单}
\usepackage{tikz}
\usetikzlibrary{shapes.geometric, arrows, arrows.meta, fit, backgrounds, decorations.pathmorphing, petri, calc}
\date{\today}

\DeclareMathOperator{\Id}{\operatorname{Id}}
\DeclareMathOperator{\supp}{supp}
\DeclareMathOperator{\dist}{dist}
\DeclareMathOperator{\vol}{vol}
\DeclareMathOperator{\diag}{diag}
\DeclareMathOperator{\tr}{tr}
\DeclareMathOperator{\cha}{\operatorname{char}}
\DeclareMathOperator{\Proj}{\operatorname{Proj}}
\DeclareMathOperator{\rank}{\operatorname{rank}}
\DeclareMathOperator{\Ker}{\operatorname{Ker}}
\DeclareMathOperator{\coker}{\operatorname{Coker}}
\DeclareMathOperator{\Img}{\operatorname{Im}}
\DeclareMathOperator{\tor}{\operatorname{tor}}

\newcommand{\Hom}{\operatorname{Hom}}
\newcommand{\Ext}{\operatorname{Ext}}
\newcommand{\Spec}{\operatorname{Spec}}
\newcommand{\Pic}{\operatorname{Pic}}
\newcommand{\Jac}{\operatorname{Jac}}
\newcommand{\MaxSpec}{\operatorname{MaxSpec}}
\newcommand{\End}{\operatorname{End}}
\newcommand{\Mod}{\operatorname{\textbf{Mod}}}
\newcommand{\Gal}{\operatorname{Gal}}
\newcommand{\Grp}{\operatorname{\textbf{Grp}}}
\newcommand{\Set}{\operatorname{\textbf{Set}}}
\newcommand{\Stab}{\operatorname{Stab}}
\newcommand{\ord}{\operatorname{ord}}
\definecolor{rede}{rgb}{1,0.97647,0.97647}
\definecolor{browne}{rgb}{0.619608,0.258824,0}
\begin{document}
	\author{周潇翔}
\maketitle
\section{already}
2020年4月21日时做,按熟悉程度降序

\begin{itemize}
\item \cite[Chap 1-6]{ahlfors1979complex}:复分析
\item \cite[Chap 1-18.7]{vakil2017rising}:代数几何(大量跳步)
\item \cite{humphreys2012introduction}:Naive的理论
\item 课程讲义:王作勤老师的微分流形和刘世平老师的微分几何,还有王作勤老师的辛几何前4节
\item 古老的无法摆上台面的书:数学分析教程,谢惠民,李尚志等...
\item \cite{bott2013differential}:大部分直观理解解决了\\[0cm]
\item \cite[1-8]{morandi2012field}:Galois理论
\item \cite{tate1974the}:椭圆曲线的算术,只看了Mordell定理证明需要的部分作为大研,还速读了\cite{silverman1992rational}
\item \cite[Chap 1-5]{fermat2013dream}:好书,就是经常卡
\item \cite[Chap 1,2]{周蜀林2005PDE}:补过的材料
\item \cite[Chap 1-4,7,8]{Li2019modularform}:模形式入门书
\item ?[Chap 10]GTM133.J.Harris.-.Algebraic.Geometry.a.first.course
\item \cite[Chap 1-4]{冯克勤2000代数数论}:代数数论(一日速成)
\end{itemize}
\section{fully forgotten}
按熟悉度降序.
\begin{itemize}
	\item \cite{梅加强2000黎曼曲面讲义,forster2012lectures,springer1981introduction,anvari2009automorphisms}:黎曼面的材料看的很杂,基本没有覆盖过某一本的证明细节
	\item \cite[6]{alperin2012groups}:上课教材
	\item \cite[Chap 1-4,6]{陈亚浙1991}:上课教材,追过细节,不记得了.类似的有\cite{周蜀林2019Schauder}
	\item \cite{atiyah2018introduction}:题目没做,内容看过.\cite[1-5]{altman2013term}
	\item \cite{bump1998automorphic}:自守形式,旁听过课以这本书为教材,没看
	\item \cite{howe1995perspectives}:不变量理论,当时需要的部分(Schur-Weyl对偶)没看懂
\end{itemize}
\section{ongoing}
强烈的毕设需求。。。
\begin{itemize}
	\item \cite[Chap 1-4]{bruinier20081}:正在看第6章.大量跳步.
	\item \cite{klein2003lectures,klein1892vorlesungen}毕设的书
	\item \cite{mumford1974abelian}:目标是看懂$\sigma$函数的基本性质
	\item \cite{shimura1971introduction}:老师推荐的,目标是看懂模方程
	\item \cite{shurman1997geometry}:对应的补充材料,看到第2章。学习如何用现代语言描述Klein的理论。
	\item \cite{mumford1995algebraic}\cite{reid1988undergraduate}:目标是看懂27条直线。
	
\end{itemize}
\section{in future}
按渴望程度排序
\begin{itemize}
	\item \cite{moduli1991mathematical},想知道Klein四次曲线的知识.
	\item \cite{bruinier20081}:\cite{bruinier20081}的后续.
	\item \cite{fricke1897vorlesungen}:毕设相关,据说写的不好
	\item Mline教授的数论全家桶:\cite{milne2006elliptic,milne1986jacobian,milne2005introduction}
\end{itemize}
\section{spectral sequence}
关于自己读过的谱序列的书的推荐,本来想放每周例子的,想想没啥内容就不放了。
\begin{itemize}
	\item \cite[Chap 3]{bott2013differential}这本书用谱序列计算了一些球面的非平凡同伦群。
	\item \cite[1.7,23.3]{vakil2017rising}介绍了谱序列的大定理和应用(就算把它作为黑箱也已经有很多很有意思的结论了),特别是导出函子诱导的双复形的谱序列。不过这里要推荐\href{https://www.3blue1brown.com/content/blog/exact-sequence-picturebook/PuzzlingThroughExactSequences.pdf}{Puzzling through exact sequences: A bedtime story with pictures},建议中文推广名:宝宝的谱序列。
	\item \href{https://www.cnblogs.com/XiongRuiMath/p/14992978.html}{Spectral Sequence, My Homological Saw}:似乎是相对完整的谱序列讲义?只是图片似乎没有Vakil那么有吸引力。
\end{itemize}
作为补充,我得收集一下Snake lemma的各类证明:
\begin{itemize}
	\item diagram chasing element by element:基本方法
	\item by universal properties of kernal and cokernal:我从没真正搞定过,不过似乎可以按照\href{https://www.3blue1brown.com/content/blog/exact-sequence-picturebook/PuzzlingThroughExactSequences.pdf}{Vakil的睡前小故事}中的图片来按图索骥?
	\item by spectral sequence:用这种方式可以有效地看出所有的条件用在了什么地方,以及可以做什么方式的推广。参见\cite[1.7.5]{vakil2017rising},``大炮打蚊子"的方式也是一种享受。
\end{itemize}
\renewcommand\refname{{\textbf{参考文献}}}
\bibliography{reference}	
\bibliographystyle{ieeetr}


\end{document}