
\documentclass[UTF8]{amsart}
%Typical documenttypes: article/book
%some examples:
%\documentclass[reqno,11pt]{book}   %%%for books
%\documentclass[]{minimal}			%%%for Minimal Working Example


%for beamers, you have to change a lot. Especially, remove the package enumitem!!!



%%%%%%%%%%%%%%%%%%%% setting for fast compiling

%\special{dvipdfmx:config z 0}		% no compression

\includeonly{chapters/chapter9}		% In practice, use an empty document called "chapter9"	% usually for printing books






%%%%%%%%%%%%%%%%%%%% here we include packages

%%%basic packages for math articles
\usepackage{amssymb}
\usepackage{amsthm}
\usepackage{amsmath}
\usepackage{amsfonts}
\usepackage[shortlabels]{enumitem}	% It supersedes both enumerate and mdwlist. The package option shortlabels is included to configure the labels like in enumerate.

%%%packages for special symbols
\usepackage{pifont}					% Access to PostScript standard Symbol and Dingbats fonts
\usepackage{wasysym}				% additional characters
\usepackage{bm}						% bold fonts: \bm{...}
\usepackage{extarrows}				% may be replaced by tikz-cd
%\usepackage{unicode-math}			% unicode maths for math fonts, now I don't know how to include it
%\usepackage{ctex}					% Chinese characters, huge difference.


%%%basic packages for fancy electronic documents
\usepackage[colorlinks]{hyperref}
\usepackage[table,hyperref]{xcolor} 			% before tikz-cd. 
%\usepackage[table,hyperref,monochrome]{xcolor}	% disable colored output (black and white)

%%%packages for figures and tables (general setting)
\usepackage{float}				%Improved interface for floating objects
\usepackage{caption,subcaption}
\usepackage{adjustbox}			% for me it is usually used in tables 
\usepackage{stackengine}		%baseline changes

%%%packages for commutative diagrams
\usepackage{tikz-cd}
\usepackage{quiver}			% see https://q.uiver.app/

%%%packages for pictures
\usepackage[width=0.5,tiewidth=0.7]{strands}
\usepackage{graphicx}			% Enhanced support for graphics

%%%packages for tables and general settings
\usepackage{array}
\usepackage{makecell}
\usepackage{multicol}
\usepackage{multirow}
\usepackage{diagbox}
\usepackage{longtable}

%%%packages for ToC, LoF and LoT







 %https://tex.stackexchange.com/questions/58852/possible-incompatibility-with-enumitem
%https://tex.stackexchange.com/questions/43008/absolute-value-symbols
\usepackage{physics}








%%%%%%%%%%%%%%%%%%%% here we include theoremstyles

\numberwithin{equation}{section}

\theoremstyle{plain}
\newtheorem{theorem}{Theorem}[section]

\newtheorem{setting}[theorem]{Setting}
\newtheorem{definition}[theorem]{Definition}
\newtheorem{lemma}[theorem]{Lemma}
\newtheorem{proposition}[theorem]{Proposition}
\newtheorem{corollary}[theorem]{Corollary}
\newtheorem{conjecture}[theorem]{Conjecture}

\newtheorem{claim}[theorem]{Claim}
\newtheorem{eg}[theorem]{Example}
\newtheorem{ex}[theorem]{Exercise}
\newtheorem{fact}[theorem]{Fact}
\newtheorem{question}[theorem]{Question}
\newtheorem{warning}[theorem]{Warning}



\newtheorem*{bbox}{Black box}
\newtheorem*{notation}{Conventions and Notations}


\numberwithin{equation}{section}


\theoremstyle{remark}

\newtheorem{remark}[theorem]{Remark}
\newtheorem*{remarks}{Remarks}

%%% for important theorems
%\newtheoremstyle{theoremletter}{4mm}{1mm}{\itshape}{ }{\bfseries}{}{ }{}
%\theoremstyle{theoremletter}
%\newtheorem{theoremA}{Theorem}
%\renewcommand{\thetheoremA}{A}
%\newtheorem{theoremB}{Theorem}
%\renewcommand{\thetheoremB}{B}







%%%%%%%%%%%%%%%%%%%% here we declare some symbols

%%%%%%%DeclareMathOperator
%see here for why newcommand is better for DeclareMathOperator: https://tex.stackexchange.com/questions/67506/newcommand-vs-declaremathoperator

%%%%%basic symbols. Keep them!

%%%symbols for sets and maps
\DeclareMathOperator{\pt}{\operatorname{pt}}	%points. Other possibilities are \{pt\}, \{*\}, pt, * ...
\DeclareMathOperator{\Id}{\operatorname{Id}}	%identity in groups.
\DeclareMathOperator{\Img}{\operatorname{Im}}

\DeclareMathOperator{\Ob}{\operatorname{Ob}}
\DeclareMathOperator{\Mor}{\operatorname{Mor}}	%difference of Mor and Hom: Hom is usually for abelian categories
\DeclareMathOperator{\Hom}{\operatorname{Hom}}	\DeclareMathOperator{\End}{\operatorname{End}}
\DeclareMathOperator{\Aut}{\operatorname{Aut}}

%%%symbols for linear algebras and 
%%linear algebras
%\DeclareMathOperator{\tr}{\operatorname{tr}}
\DeclareMathOperator{\diag}{\operatorname{diag}}	%for diagonal matrices

%%abstract algebras
\DeclareMathOperator{\ord}{\operatorname{ord}}
\DeclareMathOperator{\gr}{\operatorname{gr}}
\DeclareMathOperator{\Frac}{\operatorname{Frac}}
\DeclareMathOperator{\Gal}{\operatorname{Gal}}

%%%symbols for basic geometries
\DeclareMathOperator{\vol}{\operatorname{vol}}	%volume
\DeclareMathOperator{\dist}{\operatorname{dist}}
\DeclareMathOperator{\supp}{\operatorname{supp}}

%%%symbols for category
%%names of categories
\DeclareMathOperator{\Mod}{\operatorname{Mod}}
\DeclareMathOperator{\Vect}{\operatorname{Vect}}


%%%symbols for homological algebras
\DeclareMathOperator{\Tor}{\operatorname{Tor}}
\DeclareMathOperator{\Ext}{\operatorname{Ext}}
\DeclareMathOperator{\gldim}{\operatorname{gl.dim}}
\DeclareMathOperator{\projdim}{\operatorname{proj.dim}}
\DeclareMathOperator{\injdim}{\operatorname{inj.dim}}
\DeclareMathOperator{\rad}{\operatorname{rad}}


%%%symbols for algebraic groups
\DeclareMathOperator{\GL}{\operatorname{GL}}
\DeclareMathOperator{\SL}{\operatorname{SL}}

%%%symbols for typical varieties
\DeclareMathOperator{\Gr}{\operatorname{Gr}}
\DeclareMathOperator{\Flag}{\operatorname{Flag}}

%%%symbols for basic algebraic geometry
\DeclareMathOperator{\Spec}{\operatorname{Spec}}
\DeclareMathOperator{\Coh}{\operatorname{Coh}}
\newcommand{\Dcoh}{\mathcal{D}_{\operatorname{Coh}}}%%%This one shows the difference between \DeclareMathOperator and \newcommand
\DeclareMathOperator{\Pic}{\operatorname{Pic}}
\DeclareMathOperator{\Jac}{\operatorname{Jac}}

%%%%%advanced symbols. Choose the part you need!

%%%symbols for algebraic representation theory
\DeclareMathOperator{\ind}{\operatorname{ind}}	%\ind(Q) means the set of  equivalence classes of finite dimensional indecomposable representations
%\DeclareMathOperator{\Res}{\operatorname{Res}}
\DeclareMathOperator{\Ind}{\operatorname{Ind}}
\DeclareMathOperator{\cInd}{\operatorname{c-Ind}}


\DeclareMathOperator{\Rep}{\operatorname{Rep}}
\DeclareMathOperator{\rep}{\operatorname{rep}} %usually rep means the category of finite dimensional representations, while Rep means the category of representations.
\DeclareMathOperator{\Irr}{\operatorname{Irr}}
\DeclareMathOperator{\irr}{\operatorname{irr}}
\DeclareMathOperator{\Adm}{\operatorname{\Pi}}
\DeclareMathOperator{\Char}{\operatorname{Char}}
\DeclareMathOperator{\WDrep}{\operatorname{WDrep}}

%%%symbols for algebraic topology
\DeclareMathOperator{\EGG}{\operatorname{E}\!}
\DeclareMathOperator{\BGG}{\operatorname{B}\!}

\DeclareMathOperator{\chern}{\operatorname{ch}^{*}}
\DeclareMathOperator{\Td}{\operatorname{Td}}
\DeclareMathOperator{\AS}{\operatorname{AS}}	%Atiyah--Segal completion theorem 

%%%symbols for Auslander--Reiten theory 
\DeclareMathOperator{\Modup}{\overline{\operatorname{mod}}}
\DeclareMathOperator{\Moddown}{\underline{\operatorname{mod}}}
\DeclareMathOperator{\Homup}{\overline{\operatorname{Hom}}}
\DeclareMathOperator{\Homdown}{\underline{\operatorname{Hom}}}


%%%symbols for operad
\DeclareMathOperator{\Com}{\operatorname{\mathcal{C}om}}
\DeclareMathOperator{\Ass}{\operatorname{\mathcal{A}ss}}
\DeclareMathOperator{\Lie}{\operatorname{\mathcal{L}ie}}
\DeclareMathOperator{\calEnd}{\operatorname{\mathcal{E}nd}} %cal=\mathcal


%%%%%personal symbols. Use at your own risk!

%%%symbols only for master thesis
\DeclareMathOperator{\ptt}{\operatorname{par}}	%the partition map
\DeclareMathOperator{\str}{\operatorname{str}}	%strict case
\DeclareMathOperator{\RRep}{\widetilde{\operatorname{Rep}}}
\DeclareMathOperator{\Rpt}{\operatorname{R}}
\DeclareMathOperator{\Rptc}{\operatorname{\mathcal{R}}}
\DeclareMathOperator{\Spt}{\operatorname{S}}
\DeclareMathOperator{\Sptc}{\operatorname{\mathcal{S}}}
\DeclareMathOperator{\Kcurl}{\operatorname{\mathcal{K}}}
\DeclareMathOperator{\Hcurl}{\operatorname{\mathcal{H}}}
\DeclareMathOperator{\eu}{\operatorname{eu}}
\DeclareMathOperator{\Eu}{\operatorname{Eu}}
\DeclareMathOperator{\dimv}{\operatorname{\underline{\mathbf{dim}}}}
\DeclareMathOperator{\St}{\mathcal{Z}}

%%%%%symbols which haven't been classified. Add your own math operators here!

\DeclareMathOperator{\SO}{\operatorname{SO}}
\DeclareMathOperator{\Modr}{\operatorname{-Mod}}
\DeclareMathOperator{\alg}{\operatorname{alg}}
\DeclareMathOperator{\ab}{\operatorname{ab}}
\DeclareMathOperator{\ur}{\operatorname{ur}}
\DeclareMathOperator{\Art}{\operatorname{Art}}
\DeclareMathOperator{\wt}{\operatorname{wt}}
\DeclareMathOperator{\fin}{\operatorname{fin}}
\DeclareMathOperator{\character}{\operatorname{char}}
\DeclareMathOperator{\poly}{\operatorname{poly}}
\DeclareMathOperator{\Frob}{\operatorname{Frob}}


%%%%%%%newcommand

%%%basic symbols
%\newcommand{\norm}[1]{\Vert{#1}\Vert}

%%%symbols only for master thesis
\newcommand{\dimvec}[1]{\mathbf{#1}}
\newcommand{\abdimvec}[1]{|\dimvec{#1}|}
\newcommand{\ftdimvec}[1]{\underline{\dimvec{#1}}}

\newcommand{\absgp}[1]{\mathbb{#1}}
\newcommand{\WWd}{\absgp{W}_{\abdimvec{d}}}
\newcommand{\Wd}{W_{\dimvec{d}}}
\newcommand{\MinWd}{\operatorname{Min}(\absgp{W}_{\abdimvec{d}},W_{\dimvec{d}})}
\newcommand{\Compd}{\operatorname{Comp}_{\dimvec{d}}}
\newcommand{\Shuffled}{\operatorname{Shuffle}_{\dimvec{d}}}

\newcommand{\Omcell}{\Omega}
\newcommand{\OOmcell}{\boldsymbol{\Omega}}
\newcommand{\Vcell}{\mathcal{V}}
\newcommand{\VVcell}{\boldsymbol{\mathcal{V}}}
\newcommand{\Ocell}{\mathcal{O}}
\newcommand{\OOcell}{\boldsymbol{\mathcal{O}}}
\newcommand{\preimage}[1]{\widetilde{#1}}
\newcommand{\orde}{\operatorname{ord}_e}
\newcommand{\fakestar}{*}

%as the subscription of Hom
\newcommand{\Alggp}{\text{-Alg gp}}

%as automorphic representations
\newcommand{\Acusp}{\mathcal{A}_{\operatorname{cusp}}}
\newcommand{\Acuspkn}{\mathcal{A}_{\operatorname{cusp},k,\eta}}



%%%%%%%%%%%%%%%%%%%% here we make some blocks for special features. 

%%%% todo notes %%%%
\usepackage[colorinlistoftodos,textsize=footnotesize]{todonotes}
\setlength{\marginparwidth}{2.5cm}
\newcommand{\leftnote}[1]{\reversemarginpar\marginnote{\footnotesize #1}}
\newcommand{\rightnote}[1]{\normalmarginpar\marginnote{\footnotesize #1}\reversemarginpar}









%%%%%%%%%%%%%%%%%%%% here we make some global settings. Understand everything here before you make a document!

\usepackage[a4paper,left=3cm,right=3cm,bottom=4cm]{geometry}
\usepackage{indentfirst}	% Indent first paragraph after section header

\setcounter{tocdepth}{2}


%https://latexref.xyz/_005cparindent-_0026-_005cparskip.html
\setlength{\parindent}{15pt}	
\setlength{\parskip}{0pt plus1pt}

%\setlength\intextsep{0cm}
%\setlength\textfloatsep{0cm}
\def\arraystretch{1.2}
%\setcounter{secnumdepth}{3}

\allowdisplaybreaks


\begin{document}

% The beginning depends on the documentclass. Rewrite this part if you use different documentclass!
\date{\today}

\title
{a crash introduction to Langlands Correspondence
}
\author{Xiaoxiang Zhou}
\address{School of Mathematical Sciences\\
University of Bonn\\
Bonn, 53115\\ Germany\\} 
\email{email:xx352229@mail.ustc.edu.cn}

\begin{abstract}
In these notes, we explore various versions of the Langlands correspondence, placing particular emphasis on modular forms, automorphic forms, and automorphic representations.
\end{abstract}
\maketitle
\tableofcontents


\section{Introduction}
These notes represent a faithful record of my talk at KleinAG. I have intentionally omitted sections that were not addressed during the actual presentation, making these notes somewhat incomplete. Readers may refer to my handwritten notes \cite{weeklyupdate} for a more expressive and detailed account.

I want to acknowledge that there is nothing original in my presentation. I appreciate the organizers, the attentive audience, and fellow speakers for helping identify my mistakes. Please feel free to continue pointing out any more errors or issues.

Introducing the Langlands correspondence can often be a challenging and intricate endeavor. It encompasses numerous versions, spanning from local to global, from one dimension to $n$ dimensions, and from $\GL_n$ to non-split groups. Today's talk is structured into four parts, each focusing on a specific version of Langlands correspondence, as outlined below:
%https://tex.stackexchange.com/questions/289295/column-alignment-in-tikzcd

% https://q.uiver.app/#q=WzAsOCxbMCwwLCJcXElycl97XFxtYXRoYmJ7Q319XFxsZWZ0KFxcR0xfbihGKVxccnVsZXswbW19ezMuMG1tfVxccmlnaHQpIl0sWzEsMCwiXFxXRHJlcF97XFxzdWJzdGFja3tuXFx0ZXh0ey1cXCF9XFxkaW0gXFxcXCBcXHRleHR7RnJvYiBzc319fShXX0YpIl0sWzAsMSwiXFxDaGFyX3tcXG1hdGhiYntDfSwgXFxhbGd9IFxcIVxcbGVmdChGXntcXHRpbWVzfSBcXGJhY2tzbGFzaCBcXG1hdGhiYntBfV9GXntcXHRpbWVzfSBcXHJpZ2h0KSJdLFsxLDEsIlxcQ2hhcl97XFxiYXJ7XFxtYXRoYmJ7UX19X3B9XFwhIFxcbGVmdChcXEdhbW1hIFxccmlnaHQpICsgXFx0ZXh0e2RlIFJoYW19Il0sWzAsMiwiXFxQaV97XFxBY3VzcCxrLFxcZXRhfSBcXCFcXGxlZnQoXFxHTF8yKFxcbWF0aGJie0F9X3tcXG1hdGhiYntRfX0pIFxccnVsZXswbW19ezMuMG1tfVxccmlnaHQpIl0sWzEsMiwiXFxJcnJfe1xcYmFye1xcbWF0aGJie1F9fV9wLCAyXFx0ZXh0ey1cXCF9XFxkaW19XFwhIFxcbGVmdChcXEdhbW1hIFxccmlnaHQpICsgXFx0ZXh0e21vZHVsYXJ9Il0sWzAsMywiXFxQaV97XFxBY3VzcCxrLFxcZXRhfSBcXCFcXGxlZnQoR19EKFxcbWF0aGJie0F9X3tcXG1hdGhiYntRfX0pIFxccnVsZXswbW19ezMuMG1tfVxccmlnaHQpIl0sWzEsMywiXFxjZG90cyJdLFswLDEsIjE6MSIsMCx7InN0eWxlIjp7InRhaWwiOnsibmFtZSI6ImFycm93aGVhZCJ9fX1dLFsyLDMsIjE6MSIsMCx7InN0eWxlIjp7InRhaWwiOnsibmFtZSI6ImFycm93aGVhZCJ9fX1dLFs0LDUsIkVTIl0sWzYsN11d
\[\begin{tikzcd}[column sep=3.15em,row sep=0mm
    ,/tikz/column 1/.append style={anchor=base east}
    ,/tikz/column 2/.append style={anchor=base west}
    ]
	{\Irr_{\mathbb{C}}\left(\GL_n(F)\rule{0mm}{3.0mm}\right)} & {\WDrep_{\substack{n\text{-\!}\dim \\ \text{Frob ss}}}(W_F)} \\
	{\Char_{\mathbb{C}, \alg} \!\left(F^{\times} \backslash \mathbb{A}_F^{\times} \right)} & {\Char_{\overline{\mathbb{Q}}_p}\! \left(\Gamma_F \right) \hspace{4.3mm} + \text{de Rham}} \\
	{\Pi_{\Acusp,k,\eta} \!\left(\GL_2(\mathbb{A}_{\mathbb{Q}}) \rule{0mm}{3.0mm}\right)} & {\Irr_{\overline{\mathbb{Q}}_p, 2\text{-\!}\dim}\! \left(\Gamma_F \right) + \text{modular}} \\
	{\Pi_{\Acusp,k,\eta} \!\left(G_D(\mathbb{A}_{\mathbb{Q}}) \rule{0mm}{3.0mm}\right)} & \hspace{7mm}\cdots
	\arrow["{1:1}", tail reversed, from=1-1, to=1-2]
	\arrow["{1:1}", tail reversed, from=2-1, to=2-2]
	\arrow["ES", from=3-1, to=3-2]
	\arrow[from=4-1, to=4-2]
\end{tikzcd}\]

Before discussing these correspondings, let us fix some notations.

\begin{setting}$\,$

In Section \ref{sec:NALLC}, $F$ is a non-Archimedean local field with integral ring $O_F$ and residue field $\kappa_F$. Within this context, we also make use of the absolute Galois group $\Gamma_F$ and the Weil group $W_F$ associated with $F$.

Moving on to Section \ref{sec:GLC_dim1}, we shift our focus to a number field, still denoted as $F$, with its integral ring denoted as $O_F$. For each place $v$ of $F$, we equip with three complete local rings, namely, $O_v$, $F_v$ and $\kappa_v$. The absolute Galois group of $F$ remains denoted as $\Gamma_F$.

In Section \ref{sec:GLC_nonsplit}, $F$ will be a totally real field for simplicity.

We will use the following abbreviations for representations:

\begin{table}[ht]
\centering
\begin{tabular}{|c|r|}
\hline
$\Rep$ & smooth representation \\
$\Irr$ & irreducible smooth representation \\
$\Pi$ &\hspace{5mm} admissible irreducible smooth representation \\
$\Char$ & $1$-dim smooth representation \\
$\WDrep$ & Weil--Deligne representation \\
$\Acusp$ & cuspidal automorphic form \\
\hline
\end{tabular}
\end{table}

\end{setting}

For the definition of smooth/irreducible/admissible/Weil--Deligne representation, see \cite{Modlift22} or (partially)\cite[\href{https://github.com/ramified/personal_handwritten_collection/raw/main/weeklyupdate/2022.04.17_preliminary_facts_of_reps_of_p-adic_groups.pdf}{22.04.17}]{weeklyupdate}.

\section{Non-Archimedean Local Field Case}\label{sec:NALLC}

Read \cite[\href{https://github.com/ramified/personal_handwritten_collection/raw/main/Langlands/GL_case.pdf}{$\GL_n$-case}]{weeklyupdate}. You may assume $F=\mathbb{Q}_p$ if you are not familiar with local fields.

In this instance, the Langlands correspondence is notably explicit, allowing for the classification of representations on both sides. Notably, it simplifies to a linear algebra task when considering the L-parameters of $\GL_{2,\mathbb{R}}$.

\section{Global Langlands Correspondence, $n=1$}\label{sec:GLC_dim1}

To state the global Langlands correspondence, we rely on the concepts of adèles and idèles, which gather all the local information. A brief introduction to adèles and idèles can be found in \cite[\href{https://github.com/ramified/personal_handwritten_collection/raw/main/weeklyupdate/2022.08.28_global_field.pdf}{21.08.28}]{weeklyupdate}.

Observe that
$$\mathbb{Q}^{\times} \backslash \raisebox{1mm}{$\mathbb{A}_{\mathbb{Q}}^{\times}$} /\, \mathbb{R}_{>0} \;\cong\; \widehat{\mathbb{Z}}^{\times} \;\cong\; \Gal \left( \mathbb{Q}^{\ab}/\mathbb{Q} \right) \!:= \Gamma_{\mathbb{Q}}^{\ab}.$$
In fact, we have Artin reciprocity:
$$\Art : \raisebox{-1mm}{$F^{\times}$} \backslash \raisebox{1mm}{$\mathbb{A}_{F}^{\times}$} /\, \raisebox{-1mm}{$\overline{\left(F_{\infty}^{\times}\right)^{\circ}}$} \;\cong\; \Gamma_{F}^{\ab},$$
which gives us global Langlands correspondence for $n=1$:
% https://q.uiver.app/#q=WzAsNSxbMCwwLCJcXENoYXJfe1xcbWF0aGJie0N9LCBcXGFsZywgXFx3dCAwfSBcXCFcXGxlZnQoIFxccmFpc2Vib3h7LTAuNW1tfXskRl57XFx0aW1lc30kfSBcXGJhY2tzbGFzaCBcXHJhaXNlYm94ezAuNW1tfXskXFxtYXRoYmJ7QX1fe0Z9XntcXHRpbWVzfSR9IFxccmlnaHQpIl0sWzAsMSwiXFxDaGFyX3tcXG1hdGhiYntDfSwgXFxhbGd9IFxcIVxcbGVmdCggXFxyYWlzZWJveHstMC41bW19eyRGXntcXHRpbWVzfSR9IFxcYmFja3NsYXNoIFxccmFpc2Vib3h7MC41bW19eyRcXG1hdGhiYntBfV97Rn1ee1xcdGltZXN9JH0gXFxyaWdodCkiXSxbMSwwLCJcXENoYXJfe1xcbWF0aGJie0N9fSBcXCFcXGxlZnQoXFxHYW1tYV97Rn0gXFxyaWdodCkiXSxbMSwxLCJcXENoYXJfe1xcb3ZlcmxpbmV7XFxtYXRoYmJ7UX19X3B9XFwhXFwhXFwhXFxsZWZ0KFxcR2FtbWFfe0Z9IFxccmlnaHQpIl0sWzIsMSwiXFx0ZXh0e2RlIFJoYW19Il0sWzAsMiwiIiwwLHsic3R5bGUiOnsidGFpbCI6eyJuYW1lIjoiYXJyb3doZWFkIn19fV0sWzAsMSwiIiwxLHsic3R5bGUiOnsidGFpbCI6eyJuYW1lIjoiaG9vayIsInNpZGUiOiJib3R0b20ifX19XSxbMSwzLCJcXHRleHR7dHdpc3R9IiwwLHsic3R5bGUiOnsidGFpbCI6eyJuYW1lIjoiYXJyb3doZWFkIn19fV0sWzIsMywiXFx0ZXh0e3R3aXN0fSIsMCx7InN0eWxlIjp7InRhaWwiOnsibmFtZSI6Imhvb2siLCJzaWRlIjoiYm90dG9tIn19fV0sWzMsNCwiKyIsMSx7InN0eWxlIjp7ImJvZHkiOnsibmFtZSI6Im5vbmUifSwiaGVhZCI6eyJuYW1lIjoibm9uZSJ9fX1dXQ==
\[\begin{tikzcd}[column sep=20mm,row sep=5mm
    ,/tikz/column 1/.append style={anchor=base east}
    ,/tikz/column 2/.append style={anchor=base west}
    ]
	{\hspace{-5mm}\Char_{\mathbb{C}, \alg, \wt 0} \!\left( \raisebox{-0.5mm}{$F^{\times}$} \backslash \raisebox{0.5mm}{$\mathbb{A}_{F}^{\times}$} \right)} & {\Char_{\mathbb{C}} \!\left(\Gamma_{F} \right)} & [-18mm] \\
	{\Char_{\mathbb{C}, \alg} \!\left( \raisebox{-0.5mm}{$F^{\times}$} \backslash \raisebox{0.5mm}{$\mathbb{A}_{F}^{\times}$} \right)} & {\Char_{\overline{\mathbb{Q}}_p}\!\!\!\left(\Gamma_{F} \right)} & {\text{de Rham}}
	\arrow[tail reversed, from=1-1, to=1-2]
	\arrow[hook', from=1-1, to=2-1]
	\arrow["{\text{twist}}", tail reversed, from=2-1, to=2-2]
	\arrow["{\text{twist}}", hook', from=1-2, to=2-2]
	\arrow["{+}"{description}, draw=none, from=2-2, to=2-3]
\end{tikzcd}\]
For more information about the twist, see \cite[\href{https://github.com/ramified/personal_handwritten_collection/blob/main/Langlands/3.1_Galois_representation.pdf}{Galois representation}]{weeklyupdate}.(???Wait for updating)

\section{Adèlic Modular Forms}\label{sec:GLC_GL2}
In this section, we want to discuss global Langlands correspondence for $\GL_2$. The route is as follows:
$$\text{moduli space} \;\rightsquigarrow\; \text{MF} \;\rightsquigarrow\; \Acuspkn \;\rightsquigarrow\; \Pi_{\Acusp,k,\eta} \;\rightsquigarrow\; \text{GLC}$$
\subsection{Moduli space}
Recall:
{
\setlength\arraycolsep{1pt}
\renewcommand{\arraystretch}{0.6}
 
% https://q.uiver.app/#q=WzAsNixbMCwwLCJcXEdhbW1hKE4pIl0sWzEsMCwiXFxHYW1tYV8xKE4pIl0sWzIsMCwiXFxHTF8yKFxcbWF0aGJie1p9KSJdLFswLDEsIlxcYmVnaW57cG1hdHJpeH0gMSAmIDAgXFxcXCAwICYgMSBcXGVuZHtwbWF0cml4fSJdLFsyLDEsIlxcR0xfMihcXG1hdGhiYntafS9OXFxtYXRoYmJ7Wn0pIl0sWzEsMSwiXFxiZWdpbntwbWF0cml4fSAxICYgKiBcXFxcIDAgJiAqIFxcZW5ke3BtYXRyaXh9Il0sWzAsMSwiXFxzdWJzZXQiLDEseyJzdHlsZSI6eyJib2R5Ijp7Im5hbWUiOiJub25lIn0sImhlYWQiOnsibmFtZSI6Im5vbmUifX19XSxbMSwyLCJcXHN1YnNldCIsMSx7InN0eWxlIjp7ImJvZHkiOnsibmFtZSI6Im5vbmUifSwiaGVhZCI6eyJuYW1lIjoibm9uZSJ9fX1dLFszLDUsIlxcc3Vic2V0IiwxLHsic3R5bGUiOnsiYm9keSI6eyJuYW1lIjoibm9uZSJ9LCJoZWFkIjp7Im5hbWUiOiJub25lIn19fV0sWzUsNCwiXFxzdWJzZXQiLDEseyJzdHlsZSI6eyJib2R5Ijp7Im5hbWUiOiJub25lIn0sImhlYWQiOnsibmFtZSI6Im5vbmUifX19XSxbMCwzXSxbMSw1XSxbMiw0LCJcXHRleHR7bm90IHN1cmp9Il1d
\[\begin{tikzcd}[ampersand replacement=\&]
	{\Gamma(N)} \& {\Gamma_1(N)} \& {\GL_2(\mathbb{Z})} \\
	{\begin{pmatrix} 1 & 0 \\ 0 & 1 \end{pmatrix}} \& {\begin{pmatrix} 1 & * \\ 0 & * \end{pmatrix}} \& {\GL_2(\mathbb{Z}/N\mathbb{Z})}
	\arrow["\subset"{description}, draw=none, from=1-1, to=1-2]
	\arrow["\subset"{description}, draw=none, from=1-2, to=1-3]
	\arrow["\subset"{description}, draw=none, from=2-1, to=2-2]
	\arrow["\subset"{description}, draw=none, from=2-2, to=2-3]
	\arrow[from=1-1, to=2-1]
	\arrow[from=1-2, to=2-2]
	\arrow["{\text{not surj}}", from=1-3, to=2-3]
\end{tikzcd}\]

One can define subgroups of $\GL_2(\widehat{\mathbb{Z}})$ in a similar way:

% https://q.uiver.app/#q=WzAsNixbMCwwLCJcXHdpZGVoYXR7XFxHYW1tYShOKX0iXSxbMSwwLCJcXHdpZGVoYXR7XFxHYW1tYV8xKE4pfSJdLFsyLDAsIlxcR0xfMihcXHdpZGVoYXR7XFxtYXRoYmJ7Wn19KSJdLFswLDEsIlxcYmVnaW57cG1hdHJpeH0gMSAmIDAgXFxcXCAwICYgMSBcXGVuZHtwbWF0cml4fSJdLFsyLDEsIlxcR0xfMihcXG1hdGhiYntafS9OXFxtYXRoYmJ7Wn0pIl0sWzEsMSwiXFxiZWdpbntwbWF0cml4fSAxICYgKiBcXFxcIDAgJiAqIFxcZW5ke3BtYXRyaXh9Il0sWzAsMSwiXFxzdWJzZXQiLDEseyJzdHlsZSI6eyJib2R5Ijp7Im5hbWUiOiJub25lIn0sImhlYWQiOnsibmFtZSI6Im5vbmUifX19XSxbMSwyLCJcXHN1YnNldCIsMSx7InN0eWxlIjp7ImJvZHkiOnsibmFtZSI6Im5vbmUifSwiaGVhZCI6eyJuYW1lIjoibm9uZSJ9fX1dLFszLDUsIlxcc3Vic2V0IiwxLHsic3R5bGUiOnsiYm9keSI6eyJuYW1lIjoibm9uZSJ9LCJoZWFkIjp7Im5hbWUiOiJub25lIn19fV0sWzUsNCwiXFxzdWJzZXQiLDEseyJzdHlsZSI6eyJib2R5Ijp7Im5hbWUiOiJub25lIn0sImhlYWQiOnsibmFtZSI6Im5vbmUifX19XSxbMCwzXSxbMSw1XSxbMiw0LCJcXHRleHR7bm90IHN1cmp9Il1d
\[\begin{tikzcd}[ampersand replacement=\&]
	{\widehat{\Gamma(N)}} \& {\widehat{\Gamma_1(N)}} \& {\GL_2(\widehat{\mathbb{Z}})} \\
	{\begin{pmatrix} 1 & 0 \\ 0 & 1 \end{pmatrix}} \& {\begin{pmatrix} 1 & * \\ 0 & * \end{pmatrix}} \& {\GL_2(\mathbb{Z}/N\mathbb{Z})}
	\arrow["\subset"{description}, draw=none, from=1-1, to=1-2]
	\arrow["\subset"{description}, draw=none, from=1-2, to=1-3]
	\arrow["\subset"{description}, draw=none, from=2-1, to=2-2]
	\arrow["\subset"{description}, draw=none, from=2-2, to=2-3]
	\arrow[from=1-1, to=2-1]
	\arrow[from=1-2, to=2-2]
	\arrow["{\text{not surj}}", from=1-3, to=2-3]
\end{tikzcd}\]
}

\begin{proposition}
As a topological space,\\
$$\raisebox{-1mm}{$\GL_2(\mathbb{Q})$} \backslash \raisebox{1mm}{$\GL_2(\mathbb{A}_{\mathbb{Q}})$} /\, \raisebox{-1mm}{$\widehat{\Gamma_1(N)} \cdot \mathbb{R}^{\times} \cdot \SO_2$} \;\cong\; \raisebox{-1mm}{$\Gamma_1(N)$} \backslash \raisebox{1mm}{$\mathcal{H}^{\pm}$}.$$
\\As a result, the moduli space can be realized adèlically.
\end{proposition}
\begin{proof}
We use the strong approximation theorem\footnote{See \cite{StrongapproximationConrad,3057117} for the sketch of proof.} for $\SL_2$:
$$\SL_2(\mathbb{A}_{\mathbb{Q}, \fin}) = \SL_2(\mathbb{Q})\cdot \widehat{\Gamma_1(N)}_{\det=1}.$$
With this in hand, one can show that
$$\GL_2(\mathbb{A}_{\mathbb{Q}, \fin}) = \GL_2(\mathbb{Q})\cdot \widehat{\Gamma_1(N)}._{\phantom{\det=1}}$$
Therefore,
\begin{equation*}
\begin{aligned}
  \;&  \raisebox{-1mm}{$\GL_2(\mathbb{Q})$} \backslash \raisebox{1mm}{$\GL_2(\mathbb{A}_{\mathbb{Q}})$} /\, \raisebox{-1mm}{$\widehat{\Gamma_1(N)} \cdot \mathbb{R}^{\times} \cdot \SO_2$}\\ 
  \cong\;&  \raisebox{-1mm}{$\GL_2(\mathbb{Q})$} \backslash
  \left( \raisebox{1mm}{$\GL_2(\mathbb{A}_{\mathbb{Q},\fin})$} /\, \raisebox{-1mm}{$\widehat{\Gamma_1(N)}$}
  \;\times\;
  \raisebox{1mm}{$\GL_2(\mathbb{R})$} /\, \raisebox{-1mm}{$\mathbb{R}^{\times} \cdot \SO_2$}
  \right)\\ 
  \cong\;&  \raisebox{-1mm}{$\GL_2(\mathbb{Q})$} \backslash
  \left( \raisebox{1mm}{$\GL_2(\mathbb{Q})\cdot \widehat{\Gamma_1(N)}$} /\, \raisebox{-1mm}{$\widehat{\Gamma_1(N)}$}
  \;\times\;
  \raisebox{1mm}{$\GL_2(\mathbb{R})$} /\, \raisebox{-1mm}{$\mathbb{R}^{\times} \cdot \SO_2$}
  \right)\\ 
  \cong\;&  \raisebox{-1mm}{$\GL_2(\mathbb{Q})$} \backslash
  \left( \raisebox{1mm}{$\GL_2(\mathbb{Q})$} /\, \raisebox{-1mm}{$\Gamma_1(N)$}
  \;\times\;
  \mathcal{H}^{\pm}
  \right)\\ 
  \cong\;&  \left(\raisebox{-1mm}{$\Gamma_1(N)$} \backslash
   \raisebox{1mm}{$\GL_2(\mathbb{Q})$}\right)
  \;\times_{\GL_2(\mathbb{Q})}\;
  \mathcal{H}^{\pm}
\\ 
  \cong\;& \raisebox{-1mm}{$\Gamma_1(N)$} \backslash \raisebox{1mm}{$\mathcal{H}^{\pm}$}. \\ 
\end{aligned}
\end{equation*}
\end{proof}
\begin{remark}
One don't have strong approximation theorem for $\GL_2$. In fact, for $N \geqslant 2$,
{
\setlength\arraycolsep{1pt}
\renewcommand{\arraystretch}{0.6}
\begin{equation*}
\begin{aligned}
  \mathbb{A}_{\mathbb{Q}, \fin}^{\times} =\;&  \bigsqcup_{t \in I_N} \mathbb{Q}^{\times} \cdot t \cdot \ker \chi_N\\ 
  \GL_2(\mathbb{A}_{\mathbb{Q}, \fin}) =\;&  \bigsqcup_{t \in I_N} \GL_2(\mathbb{Q}) \cdot 
  \begin{pmatrix}
  1 & \\ & t
  \end{pmatrix} \cdot \widehat{\Gamma(N)}\\ 
\end{aligned}
\end{equation*}
}
where
\begin{equation*}
\begin{aligned}
  &\chi_N: \widehat{\mathbb{Z}}^{\times} \longrightarrow (\mathbb{Z}/N\mathbb{Z})^{\times} \qquad (a_n)_n \longmapsto a_N  \\ 
  I_N:= \raisebox{-1mm}{$\{\pm 1\}$} \backslash \raisebox{1mm}{$\widehat{\mathbb{Z}}^{\times}$}&\!/ \raisebox{-1mm}{$\ker \chi_N$} \;\cong\; \raisebox{-1mm}{$\{\pm 1\}$} \backslash \raisebox{1mm}{$(\mathbb{Z}/N\mathbb{Z})^{\times}$}
  \qquad \# I_N= \begin{cases}
  1, & N=2, \\
  \phi(N)/2, & N>2.
  \end{cases}\\  
\end{aligned}
\end{equation*}
Using the same method, one would get
$$\raisebox{-1mm}{$\GL_2(\mathbb{Q})$} \backslash \raisebox{1mm}{$\GL_2(\mathbb{A}_{\mathbb{Q}})$} /\, \raisebox{-1mm}{$\widehat{\Gamma(N)} \cdot \mathbb{R}^{\times} \cdot \SO_2$} \;\cong\; \bigsqcup_{t \in I_N} \raisebox{-1mm}{$\Gamma(N)$} \backslash \raisebox{1mm}{$\mathcal{H}^{\pm}$}.$$
You may need the following fact during the proof:
{
\setlength\arraycolsep{1pt}
\renewcommand{\arraystretch}{0.6}
\begin{equation*}
\begin{aligned}
  \;& \GL_2(\mathbb{Q}) \cap \begin{pmatrix}
  1 & \\ & t
  \end{pmatrix} \widehat{\Gamma(N)} 
  \begin{pmatrix}
    1 & \\ & t
    \end{pmatrix}^{-1} \\ 
  =\;& \GL_2(\mathbb{Q}) \cap \widehat{\Gamma(N)} \\
  =\;& \Gamma(N).
\end{aligned}
\end{equation*}
}
\end{remark}

\subsection{Adèlic cuspidal modular forms}

In this subsection, we define modular form in an adèlic way.

\begin{definition}[Cuspidal modular form $S_{M_2(\mathbb{Q}), k, \eta}$]
For $k \geqslant 2$, $\eta \in \mathbb{Z}$, let
{
\setlength\arraycolsep{1pt}
\renewcommand{\arraystretch}{0.6}
$$j_{k,\eta}(\gamma):= (\det \gamma)^{\eta-1} (ci+d)^{k} \qquad \gamma=\begin{pmatrix}
a & b \\ c & d
\end{pmatrix} \in \GL_2(\mathbb{R}).$$
}
We define the space of cuspidal modular form 
$$S_{M_2(\mathbb{Q}), k, \eta}:= \left\{
\begin{aligned}
  &\phi\colon \raisebox{-1mm}{$\GL_2(\mathbb{Q})$} \backslash \raisebox{1mm}{$\GL_2(\mathbb{A}_{\mathbb{Q}})$} \longrightarrow \mathbb{C} \quad\text{ as functions}   \\ 
  & \qquad\text{such that {\upshape(1) to (4)} are ture} \\
\end{aligned}
\right\}$$
\begingroup
\upshape
%\setlist{itemsep=-0.4em}
\renewcommand\labelenumi{(\theenumi)}
\begin{enumerate}
\item (continuity) There exists an open subset $U_{\fin} \leqslant \GL_2(\mathbb{A}_{\mathbb{Q},\fin})$ such that 
$$\phi(g\gamma)= \phi(g) \phantom{j_{k,\eta}(\gamma)^{-1}} \qquad \text{ for any } \gamma \in U_{\fin}.\hspace{8mm}$$
\item (automorphy)
$$\phi(g\gamma)= j_{k,\eta}(\gamma)^{-1}\phi(g) \qquad \text{ for any } \gamma \in \mathbb{R}^{\times} \cdot \SO_2.$$
This formula can also be formulated as $$j_{k,\eta}(\gamma'\gamma)^{-1}\phi(g\gamma)=j_{k,\eta}(\gamma')^{-1}\phi(g).$$
\item (holomorphy) For any $g \in \GL_2(\mathbb{A}_{\mathbb{Q}})$, the function
$$f_{\phi,g}: \mathcal{H}^{\pm} \longrightarrow \mathbb{C} \qquad \gamma i \longmapsto \gamma(g\gamma)j_{k,\eta}(\gamma)$$
is holomorphic.
\item[] (holomorphic at $\infty$) $f_{\phi,g}(\tau) \abs{\Img \tau}^{\frac{k}{2}}$ is bounded.
\item (cuspidal condition) For any $g \in \GL_2(\mathbb{A}_{\mathbb{Q}})$, 
{
\setlength\arraycolsep{1pt}
\renewcommand{\arraystretch}{0.6}
$$\int_{\mathbb{Q} \backslash \mathbb{A}_{\mathbb{Q}}} \phi \! \left( \begin{pmatrix}
1 & x \\ 0 & 1 
\end{pmatrix} \rule{0mm}{5mm}g \right) dx =0.$$
}
\end{enumerate}
\endgroup
\end{definition}
\begin{eg}
When $U_{\fin}= \widehat{\Gamma_1(N)}$, one has isomorphism
$$S_{M_2(\mathbb{Q}), k, \eta}^{\widehat{\Gamma_1(N)}} \cong S_k\!\left(\rule{0mm}{3mm}\Gamma_1(N)\right) \qquad \phi \longmapsto f_{\phi, \Id},$$
where
$$S_k\!\left(\rule{0mm}{3mm}\Gamma_1(N)\right)= \left\{ f: \mathcal{H}^{\pm} \longrightarrow \mathbb{C} \;\middle|\; 
\begin{aligned}
  & f(\gamma z)= (c\tau +d)^k f(z) \quad \text{ for any } \gamma \in \Gamma_1(N) \\ 
  & f \text{ has zeros in the cusps } + \cdots \\ 
\end{aligned}
  \right\}.$$
\end{eg}
\begin{remark}
The integer $k$ works as the weight while the subgroup $U_{\fin}$ works as the level. The integer $\eta$ is not too important: one has isomorphism
$$S_{M_2(\mathbb{Q}), k, \eta} \longrightarrow S_{M_2(\mathbb{Q}), k, \eta-1} \qquad \phi(-) \longmapsto \phi(-) \cdot \abs{\det(-)}_{\mathbb{A}_{\mathbb{Q}^{\times}}}$$
which shifts the weight $\eta$.
\end{remark}


\subsection{Automorphic forms and automorphic representations}

In this subsection, we introduce the space of cuspidal automorphic forms $\Acuspkn$ and the space of cuspidal automorphic representations $\Pi_{\Acusp,k,\eta}$.

\begin{definition}
For $k \geqslant 2$, $\eta \in \mathbb{Z}$, the space of cuspidal automorphic forms of weight $(k,\eta)$ is defined as the minimal $\GL_2(\mathbb{A}_{\mathbb{Q}})$ representation containing $S_{M_2(\mathbb{Q}), k, \eta}$, i.e.,
$$\Acuspkn = \left< S_{M_2(\mathbb{Q}), k, \eta} \right>_{\Rep_{\mathbb{C}}\left( \GL_2(\mathbb{A}_{\mathbb{Q}}) \right)}$$
\end{definition}

%tex.stackexchange.com/questions/230360/consistency-in-the-appearances-of-loops-from-rectangular-nodes
%https://tex.stackexchange.com/questions/405149/setting-the-name-of-matrix-in-tikz-cd
% https://q.uiver.app/#q=WzAsNSxbMCwwLCJcXGxlZnRcXHtcXHBoaVxcY29sb24gXFxyYWlzZWJveHstMW1tfXskXFxHTF8yKFxcbWF0aGJie1F9KSR9IFxcYmFja3NsYXNoIFxccmFpc2Vib3h7MW1tfXskXFxHTF8yKFxcbWF0aGJie0F9X3tcXG1hdGhiYntRfX0pJH0gXFxsb25ncmlnaHRhcnJvdyBcXG1hdGhiYntDfVxccmlnaHRcXH0iXSxbMCwxLCJcXEFjdXNwa24iXSxbMCwyLCIgU197TV8yKFxcbWF0aGJie1F9KSwgaywgXFxldGF9Il0sWzEsMSwiXFx0ZXh0eyBjdXNwaWRhbCBhdXRvbW9ycGhpYyBmb3JtcyBvZiB3ZWlnaHQgfSAoayxcXGV0YSkiXSxbMSwyLCJcXHRleHR7IGFkw6hsaWMgbW9kdWxhciBmb3JtcyBvZiB3ZWlnaHQgfSAoayxcXGV0YSkiXSxbMiwxLCJcXHN1YnNldCIsMSx7InN0eWxlIjp7ImJvZHkiOnsibmFtZSI6Im5vbmUifSwiaGVhZCI6eyJuYW1lIjoibm9uZSJ9fX1dLFsxLDAsIlxcc3Vic2V0IiwxLHsic3R5bGUiOnsiYm9keSI6eyJuYW1lIjoibm9uZSJ9LCJoZWFkIjp7Im5hbWUiOiJub25lIn19fV1d
\[\begin{tikzcd}[ampersand replacement=\&, column sep={30mm,between origins}, row sep={10mm,between origins}, execute at end picture={
%    \filldraw[red] (\tikzcdmatrixname-1-1.172) circle[radius=1pt];
%    \filldraw[green] (\tikzcdmatrixname-1-1.169) circle[radius=1pt];
    \path[->] (\tikzcdmatrixname-1-1.169) edge  [out=100, in=170,distance=5mm] node[anchor=east] {$\GL_2(\mathbb{Q})$}(\tikzcdmatrixname-1-1.172);
  }]
	{\left\{\phi\colon \raisebox{-1mm}{$\GL_2(\mathbb{Q})$} \backslash \raisebox{1mm}{$\GL_2(\mathbb{A}_{\mathbb{Q}})$} \longrightarrow \mathbb{C}\right\}}  \\
	\Acuspkn \& {\text{ \textbf{cuspidal automorphic forms} of weight } (k,\eta)} \\
	{ S_{M_2(\mathbb{Q}), k, \eta}} \& {\text{ \textbf{adèlic modular forms} of weight } (k,\eta)}
	\arrow["\subset"{description}, sloped, draw=none, from=3-1, to=2-1]
	\arrow["\subset"{description}, sloped, draw=none, from=2-1, to=1-1]
\end{tikzcd}\]

\begin{remarks}\
\begin{enumerate}[1.]
\item (see \cite[Remark 4.14]{Modlift22}) People have defined the space of cuspidal automorphic forms, denoted as $\Acusp$, which encompasses a broader range of elements compared to the definitions provided earlier. One get
$$\Acusp \;\supsetneqq\;\; \bigoplus_{\substack{k \geqslant 2 \\ \eta \in \mathbb{Z}}} \Acuspkn,$$
where Maass forms (a special case of regular algebraic cuspidal automorphic forms) and weight-$1$ modular forms (a special case of regular algebraic cuspidal automorphic forms) are missing on the right hand side.
\item (see \cite[Fact 4.12]{Modlift22}) $\Acuspkn$ can be written as direct sums of irreducible admissible representations of $\GL_2(\mathbb{A}_{\mathbb{Q},\fin})$,\footnote{I'm not sure if replacing $\GL_2(\mathbb{A}_{\mathbb{Q},\fin})$ with $\GL_2(\mathbb{A}_{\mathbb{Q}})$ is possible, but I don't think using $\GL_2(\mathbb{A}_{\mathbb{Q},\fin})$ here sounds natural.} i.e.,\footnote{In this document, I consistently use symbols $\pi \in \Pi$ and $\varphi \in \Phi$ to ensure clarity and prevent any confusion between their respective memberships.}
$$\Acuspkn = \bigoplus_{i \in I} \pi_i \qquad \pi_i \in \Pi \!\left(\GL_2(\mathbb{A}_{\mathbb{Q},\fin}\rule{0mm}{3.0mm}) \right).$$
\end{enumerate}
\end{remarks}

\begin{definition}
We call
$$\Pi_{\Acusp,k,\eta}:= \left\{ \pi_i \middle| i \in I \right\} \subseteq \Pi \!\left(\GL_2(\mathbb{A}_{\mathbb{Q},\fin}\rule{0mm}{3.0mm}) \right)$$
as the set of \textbf{cuspidal automorphic representations} of weight $(k,\eta)$.
\end{definition}
\begin{remark}
We did not delve into the Hecke operator theory \cite[4.6-4.7]{Modlift22}, strong multiplicity one \cite[4.15]{Modlift22}, and the theory of newforms \cite[4.16]{Modlift22} in this discussion. Interested readers are encouraged to explore these topics independently.
\end{remark}
\begin{remark}
To generalize the above results to $\GL_{2,F}$, substitute $\mathbb{Q}$ with $\mathbb{F}$. If any issues arise, apply $D=M_2(F)$ in the following section to observe the generalization.
\end{remark}

\subsection{Global Langlands correspondence for $\GL_{2,F}$}
In this subsection, we present the Eichler–Shimura theorem without providing a proof. For more information about global Langlands correspondence, see the discussion in \href{https://mathoverflow.net/questions/127157/status-of-global-langlands-conjecture-for-mathrmgl-2-over-mathbbq}{Mathoverflow:127157}.

\begin{theorem}[{Eichler–Shimura, \cite[4.20]{Modlift22}}]$\,$

Fix $\pi \in \Pi_{\Acusp,k,\eta}\!\left(\GL_2(\mathbb{A}_{F}\rule{0mm}{3.0mm}) \right)$, and take $L$ as some CM-field containing all eigenvalues of Hecke operators. For any finite place $\lambda$ of $L$, there exists $\varphi_{\lambda}(\pi) \in \Irr_{\overline{\mathbb{Q}}_p, 2\text{-\!}\dim}\! \left(\Gamma_F \right)$ such that 

\begingroup
\upshape
%\setlist{itemsep=-0.4em}
\renewcommand\labelenumi{(\theenumi)}
\begin{enumerate}[1)]
\item If $\pi_v$ is unramified and $\character \kappa_v \neq \character \kappa_{\lambda}$, then $\varphi_{\lambda}(\pi)\big|_{G_{F_v}}$ is unramified, and 
$$\character \poly (\Frob) = X^2 -t_vX + (\#\kappa_v) s_v,$$
where $t_v$ and $s_v$ are the eigenvalues of $T_v$ and $S_v$.
\item It is compatible with the local Langlands correspondence in both the cases when $l \neq p$ and when $l = p$.
\item $\varphi_{\lambda}(\pi)$ is geometric.\footnote{See \cite[2.28]{Modlift22} for the definition of geometric representations.} 
\item For any $\lambda | \infty$ satisfying $F_v \cong \mathbb{R}$, if we denote $\Gamma_{F_v}:=\{1,\sigma_v\} \subseteq \Gamma_{F}$, then
$$\det \left( \rule{0mm}{3.0mm}\varphi_{\lambda}(\pi)(\sigma_v) \right)=-1.$$
\item $\{ \varphi_{\lambda}(\pi) \}_{\lambda}$ forms a strictly compatible system.\footnote{See \cite[2.32]{Modlift22} for the definition of a strictly compatible system.}
\end{enumerate}
\endgroup
\end{theorem}
\begin{definition}
$\varphi \in \Irr_{\overline{\mathbb{Q}}_p, 2\text{-\!}\dim}(\Gamma_F)$ is modular, if $\varphi=\varphi_{\lambda}(\pi)$ for some $\pi,\lambda$.
\end{definition}
\begin{question}
Are all geometric representations modular?
\end{question}

\section{Adèlic Modular Forms on Quaternion Algebras}\label{sec:GLC_nonsplit}

In this section, we try to generalize all the results in Section \ref{sec:GLC_GL2} to quaternion algebras. In another word, we are trying to do global Langlands correspondence for inner forms of $\GL_2$.

For simplicity, $F$ is a totally real field in the whole section.

\subsection{Quaternion algebras}

\href{https://www.zhihu.com/column/c_1338102029848887297}{Quaternion algebras}

%\include{chapters/chapter9}

%\nocite{Eberhardt2022Koszul}	% cite articles which are not cited in the document yet

% Remember to protect the uppercase of people's name and LaTeX symbols

\bibliographystyle{plain}
\bibliography{reference}
\end{document}