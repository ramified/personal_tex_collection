
\documentclass[11pt,oneside]{amsart}

%\usepackage{color,graphicx}
%\usepackage{mathrsfs,amsbsy}

\usepackage{CJK}
\usepackage{amssymb}
\usepackage{amsmath}
\usepackage{amsfonts}
\usepackage{bbm}
\usepackage[unicode, bookmarksnumbered]{hyperref}	% 启动超链接和 PDF 文档信息所需
\usepackage{graphicx}
\usepackage{amsthm}
\usepackage{enumerate}
\usepackage[mathscr]{eucal}
\usepackage{mathrsfs}
\usepackage{verbatim}
\usepackage{tikz-cd}
\usepackage{mathtools}
\usepackage{geometry} %调整页面的页边距
\geometry{top=3cm,bottom=3cm}
%\usepackage[notcite,notref]{showkeys}
\usepackage{url}
% showkeys  make label explicit on the paper

\makeatletter
\@namedef{subjclassname@2010}{%
	\textup{2010} Mathematics Subject Classification}
\makeatother

\numberwithin{equation}{section}

\theoremstyle{plain}
\newtheorem{theorem}{Theorem}[section]
\newtheorem{lemma}[theorem]{Lemma}
\newtheorem{proposition}[theorem]{Proposition}
\newtheorem{corollary}[theorem]{Corollary}
\newtheorem{claim}[theorem]{Claim}
\newtheorem{defn}[theorem]{Definition}
\newtheorem*{fact}{Fact}
\newtheorem*{idea}{Idea}

\theoremstyle{plain}
\newtheorem{thmsub}{Theorem}[subsection]
\newtheorem{lemmasub}[thmsub]{Lemma}
\newtheorem{corollarysub}[thmsub]{Corollary}
\newtheorem{propositionsub}[thmsub]{Proposition}
\newtheorem{defnsub}[thmsub]{Definition}

\numberwithin{equation}{section}


\theoremstyle{remark}
\newtheorem{remark}[theorem]{Remark}
\newtheorem{remarks}{Remarks}


\newcommand{\elli}{,\ldots,}
\newcommand{\exist}{ there exists }
\newcommand{\st}{ such that }
\newcommand{\rad}{\operatorname{rad}}
\newcommand{\nil}{\operatorname{nil}}
\newcommand{\Max}{\operatorname{Max}}
\newcommand{\Spec}{\operatorname{Spec}}
\newcommand{\Hom}{\operatorname{Hom}}
\newcommand{\End}{\operatorname{End}}
\newcommand{\im}{\operatorname{Im}\;}
\renewcommand\thefootnote{\fnsymbol{footnote}}
%dont use number as footnote symbol, use this command to change

\DeclareMathOperator{\supp}{supp}
\DeclareMathOperator{\dist}{dist}
\DeclareMathOperator{\vol}{vol}
\DeclareMathOperator{\diag}{diag}
\DeclareMathOperator{\tr}{tr}

\begin{document}
	\date{}
	
	\title
	{Schur-Weyl Duality}
	
	
\author{Xiaoxiang Zhou, 1901930052}
\address{School of Mathematical Sciences\\
University of Science and Technology of China\\
Hefei, 230026\\ P.R. China\\} 
\email{email:xx352229@mail.ustc.edu.cn}

	
	
	
	
	
	\begin{abstract}
		This article mainly concerns about the Schur-Weyl Duality. We will prove it by using the Double Centralizer Theorem. After that, we will give some commentary about the relations to other field including the Invariant Theory.
		
	\end{abstract}
	
	
	
	\maketitle
	%%%%%%%%%%%%%%%%%%%%%%%%%%%%%%%%%%%%%%%%%%%%%%%%%%%%%%%%%%%%%%%%%%%%%%%%%%%%%%%%%%%%%%%%%%%%%
			Suppose $V$ is a $\mathbb{C}$-linear space, we consider two group actions on $V^{\otimes n}$:
	$$\hspace{-1cm}GL(V) \text{ \rotatebox[origin=c]{270}{$\circlearrowleft$} } V^{\otimes n} \qquad \hspace{1cm} \mathcal{A}(v_1\elli v_n) =(\mathcal{A}v_1\elli \mathcal{A}v_n)$$
	$$\hspace{1.5cm}V^{\otimes n}\text{ \rotatebox[origin=c]{90}{$\circlearrowleft$} }S_n\qquad g(v_1\elli v_n) =(v_{g^{-1}(1)}\elli v_{g^{-1}(n)})$$
	Notice that these two actions commutes each other, we can abbriviate it as
	\begin{center}
	\begin{tikzcd}
		GL(V) \arrow[r, "\rho_1"] & \End (V^{\otimes n}) & S_n \arrow[l, "\rho_2"']
	\end{tikzcd}
	\end{center}

	
	We state the central theorem, which connects the irreducible representations of these two groups:
	\begin{theorem}
		For $V^{\otimes n}$, we have the decomposition
		$$V^{\otimes n} \cong \bigoplus_{\lambda \in \Lambda} V_{\lambda} \otimes \mathbb{S}_{\lambda}V$$
		where $V_{\lambda}$ runs over all irreducible representations of $S_n$, 
		$$\mathbb{S}_{\lambda}V:=\Hom_{\mathbb{C}[S_n]}(V_{\lambda},V^{\otimes n}) $$
		is 0 or the irreducible representation of $GL(V)$.
	\end{theorem}
	We will need some knowledges from the Representation Theory, which can be founded in \cite{alperin2012groups}
	
	And recall the following theorems:
	\begin{lemma}\label{lemma:perp}
		Suppose $G$ is a finite group(or compact Lie group), $V$ is a representation of $G$, while $W$ is its subrepresentation. Then\exist a subrepresentation $W^{\perp}$ of $V$, \st 
		$$V\cong W \oplus W^{\perp}$$
		is the isomorphism of representations. 
	\end{lemma}
 
 \begin{theorem}[Maschke’s Theorem]
 	Suppose $A$ is finite dimensional $\mathbb{C}$-algebra, then $A$ has finitely many irreducible finite dimensional representations $V_i$ up to isomorphism, then
 	$$A \cong \bigoplus_{i=1}^n \End_{\mathbbm{C}}(V_i)$$
 \end{theorem}
	\noindent\underline{\bf{Acknowledgement}}: This essay roughly follows notes from the reading group \cite{SWduality}. I would like to thank Prof. Zuoqin Wang for inviting me the book \cite{howe1995perspectives}. I also thank Prof. Xiaowu Chen, Ye Ma, Zhiyuan Chen and the anonymous referee for giving me some corrections and helping me to get through some details in the proof.
	\section{The Double Centralizer Theorem}
	
	Denote
	\begin{itemize}
		\item $A$ is a semisimple $\mathbbm{k}$-algebra with $dim_{\mathbbm{k}}A < + \infty$.
		\item $E$ is an $A$-module of $dim_{\mathbbm{k}}E < + \infty$.
		\item $B:=\End_A(E)=\{\mathcal{B}:E\longrightarrow E \mid \mathcal{B} \text{ is an $A$-invariant map}\}$
		\item $V_1 \elli V_k$ are all the irreducible representations of $A$. 
	\end{itemize}
then we have
\begin{itemize}
	\item $B$ is semisimple.
	\item the space
	$$W_i:=\Hom_A (V_i,E) \qquad (1\leqslant i \leqslant k)$$ 
	are irreducible representations of $B$ or 0.\\
	Moreover, we have the decompositions of $E$:
	$$E = \bigoplus_{i=1}^k (V_i \otimes W_i)$$
	and a description of $B$:
	$$B=\bigoplus_{i=1}^k \End(W_i).$$
\end{itemize}
	\begin{remark}
		When the representation
		$$\rho:A\longrightarrow \End_{\mathbbm{k}}(E)$$
		is faithful (i.e. $\rho$ is injective), then we can view $A$ as a subspace of $\End_{\mathbbm{k}}(E)$, and $B$ as the centralizer of $A$. In this condition we can show even more: $W_i$ is never 0, and $A=\End_B(E)$ where
		$$\End_B(E):=\{\mathcal{A}:E\longrightarrow E \mid \mathcal{A} \text{ is a $B$-invariant map}\}$$
		this is why we call it the ``Double Centralizer Theorem".
		
		This special case ($\rho$ is faithful) of the Double Centralizer Theorem is well illustrated in \cite[6]{SWduality}. In this situation, the decomposition
		$$E = \bigoplus_{i=1}^k V_i \otimes W_i$$
		gives a bijection between the irreducible representations of $A$ and the irreducible representations of $B$.
		
		However, this version of theorem is not strong enough to prove the Schur-Weyl theorem. In the proof of Schur-Weyl theorem, we study the representation of $\mathbb{C}[S_n]$, and this representation is not faithful.
		
		 For example, the representation $\mathbb{C}[S_n]\longrightarrow \End_{\mathbb{C}}(\mathbb{C}^n)$ is not faithful when $n>3$, for
		$$\dim_{\mathbb{C}}\mathbb{C}[S_n]=n! \quad \text{while} \quad \dim_{\mathbb{C}}\End_{\mathbb{C}}(\mathbb{C}^n) =n^2$$
	\end{remark}
	\begin{proof}
		We divide it into two steps.
		\subsection*{\underline{Step1}}
		We will show that, for a fixed $i$, if $W_i\neq 0$ ,then $W_i$ is an irreducible representation of $B$.
	
		We define the (left)$B$-action:
		$$B \text{ \rotatebox[origin=c]{270}{$\circlearrowleft$} } W_i=\Hom_A (V_i,E) \qquad \mathcal{B}(f)=\mathcal{B} \circ f.$$
		This is well-defined because of the following diagram:
		\begin{center}
			\begin{tikzcd}
				V_i \arrow[d, "\mathcal{A}"'] \arrow[r, "f"] & E \arrow[d, "\mathcal{A}"'] \arrow[r, "\mathcal{B}"] & E \arrow[d, "\mathcal{A}"'] \\
				V_i \arrow[r, "f"]                           & E \arrow[r, "\mathcal{B}"]                           & E                          
			\end{tikzcd}
		\end{center}
		 To show that $W_i$ is irreducible, we only need:
		 \begin{fact}
		 	Suppose $f_1,f_2 \in W_i, f_1\neq 0.$ Then \exist $\mathcal{B} \in B$ \st 
		 	$$f_2=\mathcal{B}(f_1)=\mathcal{B} \circ f_1$$
		 \end{fact}
	 \begin{idea}
	 	When $w=f_1(v), \mathcal{B}(w)$ can be only defined by
	 	$$\mathcal{B}(w)=\mathcal{B}(f_1(v)) = \mathcal{B} \circ f_1(v)=f_2(v)$$
	 	So we only need to worry about elements not in $\im f_1$.
	 \end{idea}
 \begin{proof}[Proof of the fact]
 	Choose $v\neq 0 \in V_i,$ then $Av=V_i$ because $V_i$ is irreducible representation of $A$. From this,
 	\begin{itemize}
 		\item $f_1,f_2$ is uniquely defined by $f_1(v), f_2(v)$.
 		\item $f_1 \neq 0 \Longrightarrow f_1(v) \neq 0$.
 		\item $\im f_1=f_1(V_i)=f_1(Av)=A(f_1(v))$ is $A$-invariant.
 	\end{itemize}
 By the lemma \ref{lemma:perp}, we can decompose $E$ into
 $$E=\im f_1 \oplus (\im f_1)^{\perp}$$
 \end{proof}
then we can easily define
$$\mathcal{B}:E \longrightarrow E \qquad f_1(v) \longmapsto f_2(v) \quad v' \in (\im f_1)^{\perp} \longmapsto v'$$
Now $\mathcal{B} \in B, f_2=\mathcal{B} \circ f_1$.
		\subsection*{\underline{Step2}} 
		What remains are the simple but interesting algebraic calculations. We will show them step by step:
		\begin{itemize}
			\item $E \cong \bigoplus_{i=1}^n (V_i \otimes_{\mathbbm{k}}W_i)$.
			\item $B \cong \bigoplus_{i=1}^n \End(W_i)$, thus semisimple.
			\item When $\rho$ is faithful, $W_i$ is nonzero and $A \cong \End_B(E)$
		\end{itemize}
	\begin{minipage}[t]{.39\textwidth}
	\begin{equation*}
\begin{aligned}
&\\
E&\cong A \otimes_A E\\
&\cong \bigoplus_{i=1}^n (\End_{\mathbbm{k}}(V_i) \otimes_A E)\\
&\cong \bigoplus_{i=1}^n (V_i \otimes_{\mathbbm{k}} V_i^* \otimes_A E)\\
&\cong \bigoplus_{i=1}^n (V_i \otimes_{\mathbbm{k}} \Hom_A (V_i,E))\\
&\cong\bigoplus_{i=1}^n (V_i \otimes_{\mathbbm{k}}W_i)\\
\end{aligned}
\end{equation*}
	\end{minipage}
	\begin{minipage}[t]{.01\textwidth}
		 \begin{equation*}
	\begin{array}{c||c}
	&	\\
	&\\
	&\\
	
		&	\\
	&\\
	&\\
		&	\\
	&\\
	&\\
		&	\\
	&\\
	&\\
		&	\\
	&\\
	&\\
	\end{array}
	\end{equation*}
\end{minipage}
	\begin{minipage}[t]{.55\textwidth}
	\begin{equation*}
	\begin{aligned}
	&\\[-0.2cm]
	B&\cong \End_A(E)\\
	&\cong \Hom_A(E,E)\\
	&\cong \Hom_A(\textstyle\bigoplus_{i=1}^n V_i \otimes_{\mathbbm{k}}W_i,E)\\
	&\cong \bigoplus_{i=1}^n\Hom_A( V_i \otimes_{\mathbbm{k}}W_i,E)\\
	&\mathop{\cong}\limits^{(*)}\bigoplus_{i=1}^n\Hom_A( W_i \otimes_{\mathbbm{k}}V_i,E)\\
	&\cong\bigoplus_{i=1}^n\Hom_{\mathbbm{k}}( W_i, \Hom_A(V_i,E))\\
	&\cong\bigoplus_{i=1}^n\Hom_{\mathbbm{k}}( W_i, W_i)\cong\bigoplus_{i=1}^n\End_{\mathbbm{k}}(W_i)\\
	\end{aligned}
	\end{equation*}
\end{minipage}

\begin{remark}
	Though $W_i=\Hom_A (V_i,E)$, $V_i\neq\Hom_A (W_i,E)$ in general. So we can't skip (*) to get ``$B\cong \bigoplus_{i=1}^n\Hom_A( V_i \otimes_{\mathbbm{k}}W_i,E) \cong A$". But we do have $V_i \cong \Hom_B(W_i,E)$ when $W_i \neq 0$ because of the isomorphism
	$$E \cong \bigoplus_{i=1}^{n}(W_{i} \otimes \operatorname{Hom}_{B}(W_{i}, E)) \cong \bigoplus_{i=1}^{n}(W_{i} \otimes V_{i})$$
\end{remark}	
Now let us suppose $\rho$ is faithful. Because $V$ is a faithful representation of $A$, it must contain all irreducible representations of $A$. So there is always a nonzero map $V_i \longrightarrow V$, i.e. $W_i \neq 0$.

In a similar way, we calculate
	\begin{equation*}
\begin{aligned}
A&\cong\bigoplus_{i=1}^n\End_{\mathbbm{k}}(V_i)\\
&\cong\bigoplus_{i=1}^n\Hom_{\mathbbm{k}}( V_i, V_i)\\
&\cong\bigoplus_{i=1}^n\Hom_{\mathbbm{k}}( V_i, \Hom_B(W_i,E))\\
&\mathop{\cong}\bigoplus_{i=1}^n\Hom_B( V_i \otimes_{\mathbbm{k}}W_i,E)\\
&\cong \End_B(E)\\
\end{aligned}
\end{equation*}
	\end{proof}

	
	
	\section{proof of the Schur-Weyl Duality} 
	Recall the statement of the theorem:
		\begin{theorem}
		For $V^{\otimes n}$, we have the decomposition
		$$V^{\otimes n} \cong \bigoplus_{\lambda \in \Lambda} V_{\lambda} \otimes \mathbb{S}_{\lambda}V$$
		where $V_{\lambda}$ runs over all irreducible representations of $S_n$, 
		$$\mathbb{S}_{\lambda}V:=\Hom_{\mathbb{C}[S_n]}(V_{\lambda},V^{\otimes n}) $$
		is 0 or an irreducible representation of $GL(V)$.
	\end{theorem}
\begin{proof}
	We will use the Double Centralizer Theorem by applying
	\begin{itemize}
		\item $E=V^{\otimes n}$
		\item $A=\mathbb{C}[S_n]\quad$ (semisimple by the Maschke's Theorem)\\[-0.4cm]
		\item $B= \tilde{\rho}\big(\mathcal{U}(\mathfrak{gl}(V))\big)$ where $\tilde{\rho}$ is induced by the group action
		$$\rho:GL(V) \text{ \rotatebox[origin=c]{270}{$\circlearrowleft$} } V^{\otimes n}$$
		through the following isomorphism:
		\begin{equation*}
		\begin{aligned}
		\Hom_{Grp}(GL(V),(\End(V^{\otimes n})^{\times}))& \cong \Hom_{LieAlg}(\mathfrak{gl}(V),\End(V^{\otimes n}))\\
		& \cong 
		\Hom_{Alg}(\mathcal{U}(\mathfrak{gl}(V)),\End(V^{\otimes n}))\\
		\end{aligned}
		\end{equation*}
	\end{itemize}
		What remains to prove is listed as follows.
		\begin{itemize}
			\item $$\End_{\mathbb{C}[S_n]}(V^{\otimes n})= \im \tilde{\rho} =\left<\im \rho\right>_{Alg}$$\\[-0.2cm]
			\item Any irreducible representation of $B$ is a irreducible representation of $GL(V)$.
		\end{itemize}
	
	\noindent\underline{$\End_{\mathbb{C}[S_n]}(V^{\otimes n})= \im \tilde{\rho}=\tilde{\rho}\big(\mathcal{U}(\mathfrak{gl}(V))\big)$}\\[0cm]
	
	$``\supseteq"$: For any $X \in \mathfrak{gl}(V)$, the action of $X$ on $V^{\otimes n}$ is 
	$$X(v_1 \otimes \cdots \otimes v_n)=\sum_{i=1}^{n}v_1 \otimes \cdots \otimes Xv_i\otimes \cdots \otimes v_n$$
	thus
	$$\tilde{\rho}(X)=\sum_{i=1}^{n} Id \otimes \cdots \otimes X\otimes \cdots \otimes Id \in \End_{\mathbb{C}[S_n]}(V^{\otimes n})$$
	We get $\im \tilde{\rho} \subseteq \End_{\mathbb{C}[S_n]}(V^{\otimes n})$.\\
	
	$``\subseteq"$: Abbrieviate
	$$X^{\otimes n} = X \otimes X \otimes \cdots \otimes X \in \End_{\mathbb{C}[S_n]}(V^{\otimes n})$$
	We know that any elementary symmetric polynomial can be expressed as the polynomial of 
	$$p_j=x_1^j+x_2^j+\cdots +x_n^j.$$
	Especially, \exist a polynomial $\mathcal{P}$\st 
	$$\prod_{i=1}^{n}x_i=\mathcal{P}(p_1\elli p_n).$$
	Then we have
	$$X^{\otimes n}= \mathcal{P}\big(\,\tilde{\rho}(X),\tilde{\rho}(X^2)\elli \tilde{\rho}(X^n)\,\big) \in \im \tilde{\rho}$$
	Moreover, the set $\{X^{\otimes n}\mid X\in \End(V)\}$ span
	$$\operatorname{Sym}^{n} \End(V) \simeq\left(\End(V)^{\otimes n}\right)^{S_{n}} \simeq\left(\End\left(V^{\otimes n}\right)\right)^{S_{n}}=\End_{\mathbb{C}\left[S_{n}\right]}\left(V^{\otimes n}\right)$$
	So we get $\End_{\mathbb{C}[S_n]}(V^{\otimes n})\subseteq \im \tilde{\rho}$.\\
	
	(The fact that $\operatorname{Sym}^{n} \End(V)$ was spanned by $\{X^{\otimes n}\}$ is due to the polarization theorem, the technique similar to the construction of the inner product from a normed vector space with the parallelogram law.)\\
	
	\noindent\underline{$\im \tilde{\rho} =\left<\im \rho\right>_{Alg}$}\\
	
	$``\supseteq"$: \\[-0.8cm]
	\begin{equation*}
	\begin{aligned}
	\,& \text{for any $g \in GL(V)$, $g$ commutes with $S_n$}\\
		\Rightarrow\,& \rho(g) \in \End_{\mathbb{C}[S_n]}(V^{\otimes n}) =\im \tilde{\rho}\\
		\Rightarrow\,& \left<\im \rho\right>_{Alg} \subseteq \im \tilde{\rho}\\
	\end{aligned}
	\end{equation*}\\
	
	$``\subseteq"$: For any $X \in \End(V),$ we want to show $X^{\otimes n} \in \left<\im \rho\right>_{Alg}$.
	\begin{equation*}
	\begin{aligned}
	& \det (X+tI) \neq 0 \text{ for all but finite } t \in \mathbb{C}\\
	\Rightarrow&(X+tI)^{\otimes n} \in \im \rho \text{ for all but finite } t \in \mathbb{C}\\
	\Rightarrow&(X+tI)^{\otimes n} \in \left<\im \rho\right>_{Alg} \text{ for all } t \in \mathbb{C}\\
	\Rightarrow&X^{\otimes n} \in \left<\im \rho\right>_{Alg}\\
	\end{aligned}
	\end{equation*}\\
	
	\noindent\underline{Any irreducible representation of $B$ is a irreducible representation of $GL(V)$.}\\[0.2cm]
	$B=\left<\im \rho\right>_{Alg} = \rho(\mathbb{C}[GL(V)])$, so
	\begin{itemize}
		\item Any representation of $B$ is a representation of $GL(V)$.
		\item Any irreducible representation of $B$ is a irreducible representation of $GL(V)$.
	\end{itemize}
\end{proof}
	As a Corrollary,
	\begin{corollary}[]\cite[Thm 2.4.2]{howe1995perspectives}
		The algebras spanned by the images of $G L(V)$ and of $S_{k},$ each acting on $V^{\otimes n}$ as described in the beginning of this essay, are mutual centralizer in $\End(V^{\otimes n})$.
	\end{corollary}
	
	\section{Commentary} 
	\begin{remark}
		This theorem is of much interest because it connects the representation of symmetic group and the representation of general linear group.
		
		For the symmetric group part, we have an algorithm (though finicky) To obtain all its irreducible representations Using the Young tableau.(This gives us many examples of the Schur-Weyl Duality, for reference,\cite[Example 7.0.3]{SWduality}. The algorithm can be founded in \cite[5]{SWduality}) For a vivid introduction about the Young tableau, you can see \cite{对称群表示}.
		
		For the general linear group part, we can generalize it to other classical groups including $O_n, Sp_{2n}$.(for reference,\cite[3.4]{SWduality2})
	\end{remark}
	The Schur-Weyl duality is closely connected to the invariant theory. In \cite{calderon1989local}, the author gives an equivalent propositions of the Schur-Weyl duality Theorem:
	\begin{theorem}[$(GL_{n}, GL_{m})$-duality]
		Let $U,V$ be two linear spaces of dimension $n,m$. 
		we consider two group actions on $U \times V$:
		$$\hspace{-1cm}GL(U) \text{ \rotatebox[origin=c]{270}{$\circlearrowleft$} } U \times V \qquad \hspace{1cm} g(u \otimes v)=g(u) \otimes v$$
		$$\hspace{-1cm}GL(V) \text{ \rotatebox[origin=c]{270}{$\circlearrowleft$} } U \times V \qquad \hspace{1cm} g(u \otimes v)=u \otimes g(v)$$
		Notice that these two actions commutes each other.
		
		Denote $\mathcal{S}(U \otimes V)$ to be the symmetric algebra of $U \otimes V$, then we have a decomposition
		$$\mathcal{S}(U \otimes V)=\sum_{D} \rho_{U}^{D} \otimes \rho_{V}^{D}$$
		of $GL(U) \otimes GL(V)$-modules where $\rho_{U}^{D}$ is some representaion of $GL(U)$, $\rho_{V}^{D}$ is some representaion of $GL(V)$.
	\end{theorem}
	You can see the equivalence from \cite[2.4.5]{howe1995perspectives}.
	
	These theories are related to the invariant theory because the invariant theory studies the invariants of a group action, and the decomposition offers a method to find these invariants.
	
	
	%%%%%%%%%%%%%%%%%%%%%%%%%%%%%%%%%%%%%%%%%%%%%%%%%%%%%%%%%%%%%%%%%%%%%%%%%%%%%%%%%%%%%%%%%%%%%
	
	
	
	
	
	%%%%%%%%%%%%%%%%%%%%%%%%%%%%%%%%%%%%%%%%%%%%%%%%%%%%%%%%%%%%%%%%%%%%%%%%%%
	
	
	
	
	
	
	%%%%%%%%%%%%%%%%%%%%%%%%%%%%%%%%%%%%%%%%%%%%%%%%%%%%%%%%%%%%%%%%%%%%%%%%%%%%%%%%%%%%%%%%%%%%%%%
	
	
\bibliography{../reference}	
\bibliographystyle{plain}
\nocite{howe1995perspectives}
\nocite{weyl1946classical}
\nocite{对称群表示}
\nocite{humphreys2012introduction}
	
\end{document}




