\documentclass[border={0pt 0pt 0pt 0pt}]{standalone}
\usepackage{tikz-cd,tikz-3dplot} 
\usepackage{ctex}
\usepackage{CJKfntef}
\usetikzlibrary {calc,positioning,shapes.misc,graphs,decorations.pathreplacing}
\tikzset{
	double -latex/.style args={#1 colored by #2 and #3}{    
		-latex,line width=#1,#2,
		postaction={draw,-latex,#3,line width=2*(#1)/3,shorten <=(#1)/6,shorten >=4.5*(#1)/6},
	},
}
\begin{document}
\begin{tikzpicture}[
double={blue!5!white!95!cyan}, double distance between line centers=2pt,
node distance=20mm,
terminal/.style={
	% The shape:
	rectangle,minimum size=6mm, minimum width=10mm,rounded corners=0mm,
	% The rest
	thick,draw=violet!50,
	top color=white,bottom color=violet!20,
	font=\ttfamily},
smaller/.style={
	% The shape:
rectangle,minimum size=4mm,rounded corners=0mm,
% The rest
thick,draw=red!50,
top color=white,bottom color=red!20,
font=\ttfamily},
every new ->/.style={shorten >=1pt},
>={Implies[fill=blue!70!cyan]},thick,blue!70!cyan,text=black,
decoration={brace,raise=2pt,amplitude=5mm,aspect=1},
%arrows = {-Latex[length=5pt 3 0]}
]

\node[smaller,yshift=-.48cm,xshift=1.3cm,font=\footnotesize]{$F_1^{v_1},F_2^{v_2}, F_3^{v_3}$};
\node (quan) [terminal] {全轨道形式};
\node[smaller,right=of quan,yshift=-.49cm,xshift=1.8cm,font=\footnotesize]{$F_1$};
\node (group) [terminal,right=of quan,xshift=0.5cm] {轨道形式} ;
\node[smaller,below=of quan,yshift=-.49cm,xshift=1.5cm,font=\footnotesize]{$\displaystyle \frac{F_2^{v_2}}{F_1^{v_1}}$};
\node (RS) [terminal,below=of quan,align=center] {$\Gamma$-不变函数\\$f(\gamma z)=f(z)$};
\node[smaller,below=of group ,yshift=-.55cm,xshift=0.8cm,font=\footnotesize]{$F_2,F_3$};
\node (sym) [terminal,below=of group] {轨道形式};
\path (quan) edge [->,double]node[right,yshift=-.2cm]{\scriptsize 商} (RS);
\path (group) edge [->,double]node[right,yshift=-.2cm]{\scriptsize 协变函子} (sym);
\draw [decorate,double] (sym.west) -- (group.west);
\path ($(group.west)+(-5.5mm,0)$) edge [->,double]node[above]{\scriptsize 表达} (quan.east);

%\path (shier) edge [<->,double]node[left]{\parbox{1.5cm}{\small 辅助方程}} ($(Bri1.north)+(10mm,0)$);
%\path ($(Bri1.south)+(10mm,0)$) edge [<->,double]node[left]{\parbox{2.5cm}{\small Tschirnhaus变换}} (qua);
%\path (jie.south west) edge [<->,double]node[right]{\parbox{1.5cm}{\small J-B约化}} (qua.north east);
\end{tikzpicture}

\end{document}

%\tiny
%\scriptsize
%\footnotesize
%\small
%\normalsize
%\large
%\Large
%\LARGE
%\huge
%\Huge