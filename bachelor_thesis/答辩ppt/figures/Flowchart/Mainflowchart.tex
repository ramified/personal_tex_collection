\documentclass[border={0pt 0pt 0pt 0pt}]{standalone}
\usepackage{tikz-cd,tikz-3dplot} 
\usepackage{ctex}
\usepackage{CJKfntef}
\usetikzlibrary {calc,positioning,shapes.misc,graphs}
\begin{document}
\begin{tikzpicture}[
double distance between line centers=3pt,
node distance=15mm,
terminal/.style={
	% The shape:
	rectangle,minimum size=6mm,rounded corners=3mm,
	% The rest
	very thick,draw=black!50,
	top color=white,bottom color=black!20,
	font=\ttfamily},
every new ->/.style={shorten >=1pt},
>={Implies},thick,black!50,text=black,
%arrows = {-Latex[length=5pt 3 0]}
]

\node (mod) [terminal] {模方程$j(\tau)$的解};
\node (shier) [terminal,below=of mod,yshift=0.5cm] {正二十面体方程$Z_5(z)=Z_0$的解};
\node (Bri) [coordinate,below=of shier] {};
\node (Bri1) [terminal,left=of Bri,xshift=1.7cm] {Brioschi预解式};
\node (qua) [terminal,below=of Bri] {五次方程解};
\node (jie) [terminal,right=of Bri1] {(6.2)的解};
\path (mod) edge [->,double] (shier);
\path (shier) edge [<->,double]node[left]{\parbox{1.5cm}{\small 辅助方程}} ($(Bri1.north)+(10mm,0)$);
\path ($(Bri1.south)+(10mm,0)$) edge [<->,double]node[left]{\parbox{2.5cm}{\small Tschirnhaus变换}} (qua);
\path (jie.south west) edge [<->,double]node[right]{\parbox{1.5cm}{\small J-B约化}} (qua.north east);
\end{tikzpicture}
\end{document}
