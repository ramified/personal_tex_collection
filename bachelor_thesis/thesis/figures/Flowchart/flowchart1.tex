\documentclass[border={0pt 0pt 0pt 0pt}]{standalone}
\usepackage{tikz-cd,tikz-3dplot} 
\usepackage{ctex}
\usepackage{CJKfntef}
\usetikzlibrary {calc,positioning,shapes.misc,graphs}
%\tikzset{
%	double -latex/.style args={#1 colored by #2 and #3}{    
%		-latex,line width=#1,#2,
%		postaction={draw,-latex,#3,line width=2*(#1)/3,shorten <=(#1)/6,shorten >=4.5*(#1)/6},
%	},
%}
\begin{document}
\begin{tikzpicture}[
double={blue!10}, double distance between line centers=3pt,
node distance=15mm,
terminal/.style={
	% The shape:
	rectangle,minimum size=6mm, minimum width=10mm,rounded corners=0mm,
	% The rest
	thick,draw=violet!50,
	top color=white,bottom color=violet!20,
	font=\ttfamily},
smaller/.style={
	% The shape:
rectangle,minimum size=4mm,rounded corners=0mm,
% The rest
thick,draw=red!50,
top color=white,bottom color=red!20,
font=\ttfamily},
every new ->/.style={shorten >=1pt},
>={Implies[fill=blue!80]},thick,blue!90,text=black,
%arrows = {-Latex[length=5pt 3 0]}
]

\node (ratfct) [terminal] {有理函数};
\node[smaller,right=of ratfct,yshift=-.45cm,xshift=1cm,font=\footnotesize]{扩张};
\node (fctfield) [terminal,right=of ratfct] {函数域};
\node[smaller,right=of fctfield,yshift=-.45cm,xshift=1.8cm,font=\footnotesize]{子群};
\node (group) [terminal,right=of fctfield,xshift=1cm] {群} ;
\node[smaller,below=of fctfield,yshift=-.49cm,xshift=1cm,font=\footnotesize]{分歧覆叠};
\node (RS) [terminal,below=of fctfield] {Riemann面};
\node (sym) [terminal,below=of group] {对称性};
\node (poly) [terminal,below=of sym] {正多面体};
\path (fctfield) edge [->,double]node[above]{\small PET} (ratfct);
\path (fctfield) edge [<->,double]node[above]{\small Galois} (group);
\path (fctfield) edge [<->,double] (RS);
\path (group) edge [<->,double] (sym);
\path (poly) edge [<->,double] (sym);
\draw[black,dashed] (-2.5,-1.5)--(8,-1.5);
\draw (-2,-1.5) node[anchor=south] {代数};
\draw (-2,-1.5) node[anchor=north] {几何};
%\path (shier) edge [<->,double]node[left]{\parbox{1.5cm}{\small 辅助方程}} ($(Bri1.north)+(10mm,0)$);
%\path ($(Bri1.south)+(10mm,0)$) edge [<->,double]node[left]{\parbox{2.5cm}{\small Tschirnhaus变换}} (qua);
%\path (jie.south west) edge [<->,double]node[right]{\parbox{1.5cm}{\small J-B约化}} (qua.north east);
\end{tikzpicture}

\end{document}

%\tiny
%\scriptsize
%\footnotesize
%\small
%\normalsize
%\large
%\Large
%\LARGE
%\huge
%\Huge