\chapter{构形空间与其紧化}

\section{定义}

 设$n,m$为满足$2n+m\ge 2$的非负整数.记\\
 $Conf_{n,m}=
\{(p_1,\dots,p_n;q_1,\dots,q_m)|\, p_i\in\H,\,\,q_j \in 
   \R,$\,\,$p_i, q_j$两两不同$\}$\\
为上半平面构形空间与实直线构形空间的积.\,$Conf_{n,m}$为(实)$2n+m$维光滑流形.\,$\C P^1$上保持上半平面和点$\infty$的全纯变换群$G^{(1)}$作用于$Conf_{n,m}$.\,\,$G^{(1)}$为$2$维连通李群,\,且同构于实直线上保持定向不变的仿射变换群$\{z\mapsto az+b| \,a,b\in \R, \,a>0\}$.

由$2n+m\ge 2$,\,$G^{(1)}$自由作用于$Conf_{n,m}$.商空间$C_{n,m}=Conf_{n,m}/G^{(1)}$为$2n+m-2$维流形.若$P=(p_1,\dots,p_n;q_1,\dots,q_m)$为$Conf_{n,m}$中一点,\,记$C_{n,m}$中对应的点为$[P]$.

类似地,\,对$n\ge 2$,\,定义构形空间\begin{center}
$Conf_n=\{(p_1,\dots,p_n)|\,p_i\in \C,$\,\,$p_i, q_j$两两不同$\},\,\,\,C_n=Conf_n/G^{(2)},$\\
\end{center}
其中$3$维李群$G^{(2)}=\{z\mapsto az+b|\, a\in \R, b \in \C, \,a>0\}.\,\,dim(C_n)=2n-3$.

我们将构造$C_{n,m}$的紧化${\overline C}_{n,m}$与$C_n$的紧化${\overline C}_n$,\,二者均为光滑带角流形.与带边流形类似,\,
一个($d$维)带角流形无非是局部为闭单纯锥$(\R_{\ge 0})^d$中开区域的第二可数Hausdorff空间.一个带角流形的例子是$[0,1]^d$.带角流形每个维数的面均为通常的流形.

首先给出${C}_{n}$的紧化.我们的做法是将$C_{n}$嵌入某个简单的带角流形,\,而后在其中取闭包.对$C_{n}$中任意的点$[(p_1,\dots,p_n)]$,\,$n(n-1)$个取值于$(\R /2\pi \Z)$的角$\left(Arg(p_i-p_j)\right)_{i\neq j}$与$n^2(n-1)^2$个非负实数$(|p_i-p_j|/|p_k-p_l|)_{i\neq j, k\neq l}$给出了$C_{n}$到$(\R /2\pi \Z)^{n(n-1)}\times [0,+\infty ]^{n^2(n-1)^2}$的嵌入.对于$\overline C_{n,m}$,\,首先将$C_{n,m}$通过
$$(p_1,\dots,p_n;q_1,\dots,q_m)\mapsto (p_1,\dots,p_n;\overline p_1,\dots,\overline p_n;q_1,\dots,q_m)$$
嵌入$C_{2n+m}$,\,再在$\overline C_{2n+m}$中取像的闭包.

对称群$S_n$与$S_n\times S_m$自然作用于$C_n$与$C_{n,m}$.有鉴于此,\,可以对有限集$A,B$定义$C_A$与$C_{A,B}$,\,其中$\#A\ge 2$或$2\#A+\#B\ge 2$.对集合的嵌入$A'\hra A$与$B'\hra B$,\,有自然的纤维丛(遗忘映射)\,$C_A\ra C_{A'}$与$C_{A,B}\ra C_{A',B'}$.



\section{具体实现}

本节考察构形空间${\overline C}_{n,m}$的结构.我们使用欧式几何$\C\simeq \R^2$代替Lobachevsky几何.

首先考虑$\R^2\simeq \C$上的构形.称构形$(p_1,\dots,p_n)$位于标准位置,\,若

(1)集合$\{p_1,\dots,p_n\}$的直径为$1$,\,

(2)包含$\{p_1,\dots,p_n\}$的最小圆圆心为$\in \C$.

$n\ge 2$时,\,$n$个两两不同的点的构形被$G^{(2)}$唯一的某个元素映到标准位置.

标准位置的构形给出了投影$Conf_n\ra C_n$的连续截面$s^{cont}$.

构形空间紧化时增加的低维部分来自构形中某些点互相靠近直到重合.考虑一个标准位置的构形.其中某些点由至少两个点在紧化后重合.放大这些部分,\,我们得到了新的构形,\,并将其置于标准位置.

重复这一过程,\,我们最终得到某个定向树$T$.\,$T$有一个根和叶$1,2,...,n$.对$T$每个非叶的点$v$,\,记$Star(v)$为始于$v$的边集.

$C_n$中互相靠近的点有以下参数化:\,

(1)对树$T$的每个非叶点$v$,\,一个标准位置的构形$c_v$,\,其点以$Star(v)$标记,\,

(2)对每个非叶非根的点$v$,\,标量$s_v>0$.将$c_v$置于构形$c_u$的点$p_v\in \C$,\,其中$(u,v)\in E_T$.

更精确地,\,$G^{(2)}$的元素$(z\mapsto s_v z+p_v)$作用于构形$c_v$.

$s_v$为小的正常数.形式地允许某些$s_v$取$0$,\,我们得到了紧化$\overline{C}_n$.

这样,\,我们得到了一个带角紧拓扑流形,\,其每个部分$C_T$来自(叶为$1,2,...,n$的)树$T$.每个$C_T$典范同构于$\prod_{v\notin \{leaves\}} C_{Star(v)}$.\,$C_T$来自所有$s_v=0$.

作为集合,\,$\overline{C}_n=\bigsqcup_{{\rm trees}\,\,\,T}\,\, \prod_{v\in V_T\setminus
              \{\rm leaves\}} C_{Star(v)}$.

为在$\overline{C}_n$上引入光滑结构,\,选取投影$Conf_n\ra C_n$的某个$S_n$-等变光滑截面$s^{smooth}$,\,而非构形标准位置给出的截面$s^{cont}$.\,$C_T$中某点附近的局部坐标为接近$0$的标量$s_v\in\R_{\ge 0}$与$C_{Star(v)}$的局部坐标,\,其中$v\in V_T\setminus
              \{\rm leaves\}$.我们得到的带角流形结构与光滑截面$s^{smooth}$的选取无关.

其次,\,考虑$\H\cup \R$的构形.称$\H\cup\R$上的点的非空有限集$S$位于标准位置,\,若

(1)$S$的凸包到直线$\R\subset \C\simeq \R^2$上的投影为一元集$\{0\}$或中心为$0$的区间,\,

(2)$S$的直径与$S$到$\R$距离的较大者为$1$.

对$2n+m\ge 2$,\,$\H$上$n$个点和$\R$上$m$个点组成的构形被$G^{(1)}$唯一的某个元素映到标准位置.为得到一个光滑结构,\,我们重复类似的操作.

此时有两种区域需要处理.第一种情况为$\H$上的一些点互相靠近.我们得到的是${\overline C}_{n'}$,\,$n'\le n$.第二种情况为$\H$上的某个点靠近$\R$.

考虑低维的$C_{n,m}$.\,$C_{1,0}={\overline C}_{1,0}$为一元集.\,$C_{0,2}=\OC_{0,2}$为二元集.\,$C_{1,1}$为开区间,\,其闭包$\OC_{1,1}$为闭区间.

在$G^{(1)}$作用下,\,可以将点$p_1$置于$i=\sqrt{-1}\in \H$处,\,故$C_{2,0}$微分同胚于$\H\setminus \{0+1\cdot i\}$.闭包$\OC_{2,0}$形如两条边围成的曲面去掉内部的小圆.

遗忘映射自然延拓为紧化空间上的光滑映射.


\subsection{边界项}
本节给出$\OC_{A,B}$中余维$1$的项.

(S1)对$i\in S\subseteq A$,\,点$p_i\in\H$互相靠近但远离$\R$.这里$\#S\ge 2$.

(S2)对$i\in S\subseteq A$与$j\in S'\subseteq B$,\,点$p_i\in \H$与$q_j\in\R$互相靠近且靠近$\R$.这里$2\#S+\# S'\ge 2$且$\#S+\#S'\le \#A+\#B-1$.

S1对应
$$\p_S \OC_{A,B}\simeq
                C_{S}\times C_{(A
               \setminus S)\sqcup \{pt\}, B},$$
其中$\{pt\}$为一元集,\,其元素代表$\H$中的点$(p_i)_{i\in S}$.

S2对应
$$\p_{S,S'} \OC_{A,B}\simeq
                 C_{S,S'}\times C_{A\setminus S,
                  (B\setminus S')\sqcup
                  \{pt\} }.$$











