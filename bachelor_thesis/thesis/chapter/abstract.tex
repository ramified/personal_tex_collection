\begin{cnabstract}
这篇文章主要讨论正多面体与模曲线之间的关系,作为副产品,我们将用正二十面体方程或$j$-函数来解一般的五次方程.在第一部分,我们讨论相对古典的理论:从正多面体群所对应的分歧覆叠开始,我们算出正多面体方程,并在最后将一般的五次方程化为Brioschi结式.在第二部分,我们先复习经典模形式和$\theta$-函数的理论,然后描述$X(\Gamma(5))$和$X(\Gamma(7))$的射影嵌入,从中得到它们的更多性质.

\keywords{正多面体,\enskip 分歧覆叠,\enskip 预解式,\enskip 求根公式,\enskip 模形式,\enskip Theta函数,\enskip Klein四次曲线
}
\end{cnabstract}

\begin{enabstract}
This article mainly concerned about relations between platonic solid and modular curve, as a byproduct, we will solve the equation of degree $5$ by icosahedron equation or $j$-function. In the first part, we talk about relatively classical theory: we begin with ramified covering concerned about the platonic rotation groups, and then gives equations of platonic solid. Finally, we reduce a general quintic equation to Brioschi resolvent. In the second part, we give a review of the classical modular forms and the $\theta$-functions in the beginning, then describe a relatively canonical projective embedding of $X(\Gamma(5))$ and $X(\Gamma(7))$, from which we extract more informations of them.
 
\enkeywords{platonic solid,\enskip ramified covering,\enskip resolvent,\enskip root formula,\enskip modular form,\enskip Theta fuction,\enskip Klein quartic
}
\end{enabstract}
