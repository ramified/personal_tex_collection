% Gemini theme
% https://github.com/anishathalye/gemini

\documentclass[final]{beamer}

% ====================
% Packages
% ====================

\usepackage[T1]{fontenc}
\usepackage{lmodern}
\usepackage[orientation=portrait, size=a0, scale=1.25]{beamerposter}
\usetheme{gemini}
\usecolortheme{mit}
\usepackage{graphicx}
\usepackage{booktabs}
%\usepackage{tikz}
\usepackage{pgfplots}
\pgfplotsset{compat=1.14}
\usepackage{anyfontsize}
\usepackage{tikz-cd}
\usepackage{quiver}
\usepackage{tcolorbox}
%\usepackage[dvipsnames]{xcolor}
% ====================
% Lengths
% ====================

\numberwithin{equation}{section}

\theoremstyle{plain}

\newtheorem{setting}[theorem]{Setting}
\newtheorem{proposition}[theorem]{Proposition}
\newtheorem{conjecture}[theorem]{Conjecture}

\newtheorem{claim}[theorem]{Claim}
\newtheorem{eg}[theorem]{Example}
\newtheorem{ex}[theorem]{Exercise}
\newtheorem{ques}[theorem]{Question}
\newtheorem{answ}[theorem]{Answer}
\newtheorem{warning}[theorem]{Warning}
\newtheorem{notation}[theorem]{Notations}



% If you have N columns, choose \sepwidth and \colwidth such that
% (N+1)*\sepwidth + N*\colwidth = \paperwidth
\newlength{\sepwidth}
\newlength{\colwidth}
\setlength{\sepwidth}{0.024\paperwidth}
\setlength{\colwidth}{0.464\paperwidth}

\newcommand{\separatorcolumn}{\begin{column}{\sepwidth}\end{column}}

% ====================
% Title
% ====================



\title{Subvarieties in Complex Abelian Varieties}

\author{Xiaoxiang Zhou \\ Advisor: Prof. Dr. Thomas Krämer}

\institute[shortinst]{Humboldt-Universität zu Berlin}

% ====================
% Footer (optional)
% ====================

\footercontent{
  \href{https://ramified.github.io/}{https://ramified.github.io/} \hfill
  Topology of Algebraic Varieties \hfill
  \href{mailto:xiaoxiang.zhou@student.hu-berlin.de}{xiaoxiang.zhou@student.hu-berlin.de}}
% (can be left out to remove footer)

% ====================
% Logo (optional)
% ====================

% use this to include logos on the left and/or right side of the header:
% \logoright{\includegraphics[height=4cm]{figures/BMS_logo.png}}
% \logoleft{\includegraphics[height=7cm]{logo2.pdf}}

% ====================
% Body
% ====================


%%%%%%%newcommand

\definecolor{BrickRed}{rgb}{0.65, 0.12, 0.21}
\definecolor{BrickBlue}{rgb}{0, 0.1, 0.3}
\DeclareMathOperator{\sky}{\operatorname{sky}}
\DeclareMathOperator{\Hcohom}{\operatorname{H}}
\DeclareMathOperator{\BM}{\operatorname{BM}}
\DeclareMathOperator{\cpt}{\operatorname{c}}
\DeclareMathOperator{\Or}{\operatorname{Or}}
\DeclareMathOperator{\pt}{\operatorname{pt}}
\DeclareMathOperator{\constructable}{\operatorname{cons}}
\DeclareMathOperator{\IC}{\operatorname{IC}}
\DeclareMathOperator{\Perv}{\operatorname{Perv}}
\DeclareMathOperator{\cone}{\operatorname{cone}}
\DeclareMathOperator{\CC}{\operatorname{CC}}
\DeclareMathOperator{\NMD}{\operatorname{NMD}}
\DeclareMathOperator{\sm}{\operatorname{sm}}
\DeclareMathOperator{\Gr}{\operatorname{Gr}}
\DeclareMathOperator{\Hom}{\operatorname{Hom}}
\DeclareMathOperator{\End}{\operatorname{End}}
\DeclareMathOperator{\Aut}{\operatorname{Aut}}
\DeclareMathOperator{\Img}{\operatorname{Im}}
\DeclareMathOperator{\Ob}{\operatorname{Ob}}
\DeclareMathOperator{\Mor}{\operatorname{Mor}}
\DeclareMathOperator{\Pic}{\operatorname{Pic}}
\DeclareMathOperator{\Jac}{\operatorname{Jac}}
\DeclareMathOperator{\Gal}{\operatorname{Gal}}
\DeclareMathOperator{\Mon}{\operatorname{Mon}}
\DeclareMathOperator{\AJ}{\operatorname{AJ}}
\DeclareMathOperator{\AP}{\operatorname{AP}}
\newcommand{\covermap}{h}
\DeclareMathOperator{\Prym}{\operatorname{Prym}}
\DeclareMathOperator{\Nm}{\operatorname{Nm}}
\DeclareMathOperator{\gon}{\operatorname{gon}}
\DeclareMathOperator{\univ}{\operatorname{univ}}
\DeclareMathOperator{\Abel}{\operatorname{Abel}}
\DeclareMathOperator{\pr}{\operatorname{pr}}
\DeclareMathOperator{\aff}{\operatorname{aff}}
\DeclareMathOperator{\charcycle}{\operatorname{cc}}
\DeclareMathOperator{\twist}{\operatorname{tw}}
\DeclareMathOperator{\Hnm}{\operatorname{H}}
\newcommand{\bigast}{\mathop{\scalebox{1.5}{\raisebox{-0.2ex}{$\ast$}}}}%

%https://stackoverflow.com/questions/57382286/bibliography-style-and-line-breaks-in-beamer-poster
\setbeamertemplate{bibliography entry article}{}
\setbeamertemplate{bibliography entry title}{}
\setbeamertemplate{bibliography entry location}{}
\setbeamertemplate{bibliography entry note}{}

\begin{document}
\begin{frame}[t]
\begin{columns}[t]
\separatorcolumn

\begin{column}{\colwidth}



  \begin{block}{Tangent Gauss Map}
Let $A/\mathbb{C}$ be an abelian variety of dimension $n$, and let $Z \subset A$ be a non-degenerate closed subvariety of dimension $r$.\\[15mm]
 
  To understand the geometry of $Z$, we encode the variation of its tangent spaces via the tangent Gauss map
 $$\phi_Z: Z^{\sm}\longrightarrow \Gr(r,T_0A) \qquad p \longrightarrow T_pZ \subset T_pA \cong T_0A.$$
Its differential
$$d_p\phi_Z: T_pZ \longrightarrow \Hom_{\mathbb{C}}(T_pZ, N_pZ)$$
is the second fundamental form, which captures information about the curvature of $Z$.

\end{block}
\begin{block}{Generic Injectivity of the Tangent Gauss Map}

Clearly the Gauss map has degree $> 1$ whenever the subvariety $Z \subset A$ is stable under a nontrivial translation or symmetric up to a translation, as shown below:
$\;$\\[-30mm] $\,$
  \begin{figure}[th]
  \begin{minipage}[t]{.45\textwidth}
    \vspace{0pt}
  	\centering
  	\includegraphics[width=.7\textwidth]{figures/translation_invariant.png}
  \end{minipage}
  \begin{minipage}[t]{.45\textwidth}
    \vspace{0pt}
  	\centering
  	\includegraphics[width=.7\textwidth]{figures/reflection_invariant.png}
  \end{minipage}
  \end{figure}
Are these the only cases where the Gauss map fails to be generically injective?

We specialize to the case where $Z=C$ is a curve. When $n=2$, $\phi_C\!\!: C^{\sm} \longrightarrow \mathbb{P}^1$ typically fails to be generically injective.

  
\begin{tcolorbox}[
  colback=gray!15,
  colframe=gray!60,
  boxrule=0.5pt,
  arc=4pt,
  left=12pt,right=12pt,top=12pt,bottom=12pt
]
\textbf{Question 1.}
Let $C$ be a non-degenerate curve on an abelian variety $A$ of dimension $n > 2$. If $C$ is not invariant under any non-trivial translation or reflection, then $\phi_C$ is generically injective.
\end{tcolorbox}

  $\;$\\[-45mm] $\,$
    \begin{figure}[th]
    \begin{minipage}[t]{.45\textwidth}
      \vspace{0pt}
    	\centering
    	\includegraphics[width=.7\textwidth]{figures/rotation_invariant.png}
    \end{minipage}
    \end{figure}
\begin{tcolorbox}[
  colback=red!10,
  colframe=gray!60,
  boxrule=0.5pt,
  arc=4pt,
  left=12pt,right=12pt,top=12pt,bottom=12pt
]
\textbf{Example 1.}
For any $n$ the power $A = E_{\rho}^{\oplus n}$ comes with a natural diagonal action of $\mathbb{\mu}_3 \cong \mathbb{Z}/3\mathbb{Z}$.
Computer experiments yield a non-degenerate $\mathbb{\mu}_3$-invariant curve $C \subset A$, for which $\phi_C$ is not generically injective.
\end{tcolorbox}
We have found no counterexample to Question 1 when $A$ is not isogenous to $E_{\mathfrak{i}}^{\oplus n}$ or $E_{\rho}^{\oplus n}$. This suggests the following refinement:
\begin{tcolorbox}[
  colback=gray!15,
  colframe=gray!60,
  boxrule=0.5pt,
  arc=4pt,
  left=12pt,right=12pt,top=12pt,bottom=12pt
]
\textbf{Conjecture 2.}
Let $C$ be a non-degenerate curve on an abelian variety $A$ of dimension $n>2$. Then $\phi_C$ is generically injective unless there exists a non-trivial automorphism $\tau \in \Aut(A)$ which preserves $C$ and acts via a scalar on $T_0A$.
\end{tcolorbox}
One may restate the conjecture using Gauss curvature, yielding a slightly stronger statement:
\begin{tcolorbox}[
  colback=gray!15,
  colframe=gray!60,
  boxrule=0.5pt,
  arc=4pt,
  left=12pt,right=12pt,top=12pt,bottom=12pt
]
\textbf{Conjecture 3.}
Let $C$ be a non-degenerate curve on an abelian variety $A$ of dimension $n > 2$. Then for general $p \in \Img \phi_C$, the Gauss curvature of $C \subset A$ is the same at all points of the fiber $\phi_C^{-1}(p)$.
\end{tcolorbox}

  \end{block}
  \begin{block}{Known Cases}
  \begin{enumerate}
  \item
  If $A=\Jac(C)$ and $C$ is embedded via the Abel-–Jacobi map, then $\phi_C=|\omega_C|$ is the canonical map, and one may check Conjectures 2 and 3 by direct inspection.
%  $\;$\\[-25mm] $\,$
%  \begin{itemize}
%  \item When $C$ is hyperelliptic, $C$ is invariant under the hyperelliptic involution, and $\deg \phi_C=2$;
%  \item When $C$ is non-hyperelliptic, $\phi_C$ is an embedding.
%  \end{itemize}
  
  \item Let $h:C \longrightarrow C'$ be a cyclic $k$-fold cover defined by $\eta \in \Pic(C')$ with $\eta^{\otimes k} \cong \mathcal{O}_{C'}(B)$. If $A = \Prym(C/C')$ and $C \to A$ is the Abel--Prym map, then
%  $$T_0A \cong \Hnm^0(\omega_C)/\Hnm^0(\omega_{C'}) \cong \bigoplus_{i=1}^{k-1} \Hnm^0(\omega_{C'} \otimes \eta^i)$$
  $$\phi_C: C \longrightarrow \mathbb{P} T_0A \cong \mathbb{P} \left( \bigoplus_{i=1}^{k-1} \Hnm^0(\omega_{C'} \otimes \eta^i)  \right)$$
  $\;$\\[-20mm] $\,$
  \begin{itemize}
  \item $k=2$: $C$ is invariant under the Prym involution, and $\phi_C=|\omega_{C'} \otimes \eta| \circ h$.\\
  If $C'$ is non-hyperelliptic with $g(C') \geq 4$, then $\deg \phi_C = 2$ or $4$, and
  $$\deg \phi_C=4 \Longleftrightarrow B=\varnothing, C' \text{ is bielliptic and $\eta$ pulled back from EC.}$$
  \item $k>2$: if $g(C') \geq 1$ and $|\omega_{C'} \otimes \eta|$ is generically injective, then $\phi_C$ is generically injective.
  \end{itemize}
  
\item If $C \subset A$ is smooth and either $\deg \phi_C = 2$ or $\phi_C$ is unramified, Conjecture 3 also holds.
  \end{enumerate}
  \end{block}



\end{column}

\separatorcolumn

\begin{column}{\colwidth}

\begin{block}{Conormal Gauss Map}
Consider the conormal variety $\Lambda_{Z} \subset T^*A \;\cong\; A \times T^{*}_{0}A$. The natural projection is the conormal Gauss map 
$\gamma_Z^{(\aff)}: \Lambda_{Z} \longrightarrow T^{*}_{0}A,$ which is generically finite whenever $Z$ is of general type.

  $\;$\\[-48mm] $\,$
    \begin{figure}[th]
    \begin{minipage}[t]{.97\textwidth}
      \vspace{-1cm}
    	\centering
    	\includegraphics[height=23.7cm]{figures/3dTikZ/combi_of_conormal_tangent_map.pdf}
    \end{minipage}
    \end{figure}
$\;$\\[-35mm] $\,$
\begin{tcolorbox}[
  colback=gray!15,
  colframe=gray!60,
  boxrule=0.5pt,
  arc=4pt,
  left=12pt,right=12pt,top=12pt,bottom=12pt
]



\textbf{Conjecture 4.}
Suppose $A$ is not isogenous to $E_{\mathfrak{i}}^{\oplus n}$ or $E_{\rho}^{\oplus n}$, and let $C \subset A$ be a non-degenerate curve which is not stable under any non-trivial translation or reflection. Then the monodromy group $\Gal(\gamma_C)$ is big — namely, a Weyl group of type $A$, $C$, or $D$.
\end{tcolorbox}

When $n>2$, Conjecture 4 follows from Conjecture 3.

The monodromy group $\Gal(\gamma_C)$ helps us to determine controls the Tannaka group of the tensor category generated by the IC sheaf on $C$; see \cite[Theorem 2.1]{Kr22GaussI}.

  \end{block}
%  $\;$\\[-35mm] $\,$
  \begin{block}{The Subvariety $Z^{(m)}$}
  
  The convolution structure on perverse sheaves gives rise to numerous cycles in $A$. They admit a simple geometric description: by fiberwise summing points in $\gamma_Z^{-1}(\xi_0)$ and projecting, one obtains new subvarieties of varying dimensions.
  
  Fix a general point $\xi_0 \in T_0^*A$ and choose an ordering
  $\gamma_Z^{-1}(\xi_0)= \left\{ p_1,\ldots,p_d \right\} \subset Z$. For $(m)=(m_1,\ldots,m_d)\in\mathbb{Z}^d$, let $Z^{(m)}$ denote the irreducible component of the subvariety obtained by this construction that contains the point $\sum_i m_i p_i$.
  
  \begin{tcolorbox}[
    colback=red!10,
    colframe=gray!60,
    boxrule=0.5pt,
    arc=4pt,
    left=12pt,right=12pt,top=12pt,bottom=12pt
  ]
  \textbf{Theorem 1.}
  Let $c_i := c_{M,i}(\Lambda_Z)$ be the Segre class of the cone $\Lambda_Z$, and let $*$ denote the Pontryagin product.
  
  When $\Gal(\gamma_{Z})=S_d$, the Segre classes of $\Lambda_{Z^{(m)}}$ can be written as
  \begin{equation*}
  \begin{aligned}
    c_{M,l}\left(\Lambda_{Z^{(m)}}\right)=\;& \frac{1}{d_Z^{(m)}} \sum_{\lambda \dashv l} \;\mu_d^{\lambda} \left( \bigast_{i=1}^{k'} c_{\lambda_i}\right) \\ 
  \end{aligned}
  \end{equation*}
  where $\lambda= [\lambda_1,\ldots,\lambda_{k'}]$ ranges over all partitions of $l$ and where
  \begin{itemize}
  \item $d_Z^{(m)} \in \mathbb{N}_{>0}$ is the degree of a certain addition map;
  \item $\mu_d^{\lambda}= \sum_{\alpha \in \mathcal{P}(d)} \sum_{\substack{\boldsymbol{l} \dashv \lambda\\\boldsymbol{l}:\text{ length }k}}  \mu(\hat{0},\alpha)    \alpha(m)^{2\boldsymbol{l}} d^{k-k'} \in \mathbb{Z}[m_1,\ldots, m_d]^{S_d}$
  \item $\alpha=\{A_1,\ldots,A_k\} \in \mathcal{P}(d)$, $\boldsymbol{l}=(l_1,\ldots,l_k) \in \mathbb{Z}^k$;
  \item $\mu(\hat{0}, \alpha) = (-1)^{d-k} \prod_{i=1}^{k}\, \left(\rule{0mm}{3mm}|A_i|-1\right)$;
  \item $\alpha(m)^{2\boldsymbol{l}}=  \left( \sum_{i \in A_1}m_i \right)^{2l_1} \cdots \left( \sum_{i \in A_k}m_i \right)^{2l_k}$.
  \end{itemize}
  \end{tcolorbox}
  
  \begin{tcolorbox}[
    colback=BrickBlue!10,
    colframe=gray!60,
    boxrule=0.5pt,
    arc=4pt,
    left=12pt,right=12pt,top=12pt,bottom=12pt
  ]
  \textbf{Remarks.}
  \begin{enumerate}
  \item One can recover both $\dim Z$ and $[Z] \in \Hnm^{2(n - \dim Z)}(A)$ from the Segre classes of $\Lambda_{Z}$: 
    $$\dim Z=\; \max \left\{ i \in \mathbb{Z} \;\middle|\; c_i \neq 0 \right\}, \qquad
   [Z]=\; c_{\dim Z}. $$
   \item If $Z = -Z$, then a similar formula also holds for $\Gal(\gamma_Z)=W(C_{d/2})$,\\ but the method does not extend to the case where $\Gal(\gamma_Z)=W(D_{d/2})$.
   \item The formula simplifies a lot for curves inside their Jacobian.\\
   For example, if $C$ is non-hyperelliptic, we obtain:
   \begin{equation*}
   \begin{aligned}
     c_l\left(\Lambda_{Z^{(m)}}\right)=\;& \frac{1}{c_Z^{(m)}} \frac{1}{2^{l}(g-l)!} \sum_{\sigma \in S_{2g-2}} \prod_{i=1}^{l} \left( m_{\sigma(2i-1)}-m_{\sigma(2i)} \right)^2 \cdot \Theta^{g-l} \\ 
     \dim Z^{(m)} =\;& \min_{k \in \mathbb{Z}} \left\{ \rule{0mm}{10mm} g-1, \# \left\{ i \in [2g-2] \;\middle|\; m_i \neq k \right\} \right\} \\
   \end{aligned}
   \end{equation*}
  \end{enumerate}
  \end{tcolorbox}

  \end{block}
%  $\;$\\[-35mm] $\,$
  \begin{block}{References}
  $\;$\\[-30mm] $\,$
    \nocite{Kr20,Kr22GaussI,sub_AV26}
    \footnotesize{
    
%    \setbeamertemplate{bibliography item}{}
%    \setlength{\itemsep}{0pt}
%    \setlength{\parskip}{0pt}
%    \renewcommand{\newblock}{\hskip .11em}
    \bibliographystyle{alpha}\bibliography{reference}}
  \end{block}



\end{column}

\separatorcolumn

\end{columns}
\begin{tikzpicture}[remember picture, overlay]
\node at ($(current page.north east)+(-8cm,-9cm)$)
  {\includegraphics[height=5cm]{figures/BMS_logo.png}};
\end{tikzpicture}
\end{frame}
\end{document}
