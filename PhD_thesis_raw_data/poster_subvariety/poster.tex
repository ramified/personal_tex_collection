% Gemini theme
% https://github.com/anishathalye/gemini

\documentclass[final]{beamer}

% ====================
% Packages
% ====================

\usepackage[T1]{fontenc}
\usepackage{lmodern}
\usepackage[orientation=portrait, size=a0, scale=1.25]{beamerposter}
\usetheme{gemini}
\usecolortheme{mit}
\usepackage{graphicx}
\usepackage{booktabs}
%\usepackage{tikz}
\usepackage{pgfplots}
\pgfplotsset{compat=1.14}
\usepackage{anyfontsize}
\usepackage{tikz-cd}
\usepackage{quiver}
\usepackage{tcolorbox}
%\usepackage[dvipsnames]{xcolor}
% ====================
% Lengths
% ====================

\numberwithin{equation}{section}

\theoremstyle{plain}

\newtheorem{setting}[theorem]{Setting}
\newtheorem{proposition}[theorem]{Proposition}
\newtheorem{conjecture}[theorem]{Conjecture}

\newtheorem{claim}[theorem]{Claim}
\newtheorem{eg}[theorem]{Example}
\newtheorem{ex}[theorem]{Exercise}
\newtheorem{ques}[theorem]{Question}
\newtheorem{answ}[theorem]{Answer}
\newtheorem{warning}[theorem]{Warning}
\newtheorem{notation}[theorem]{Notations}



% If you have N columns, choose \sepwidth and \colwidth such that
% (N+1)*\sepwidth + N*\colwidth = \paperwidth
\newlength{\sepwidth}
\newlength{\colwidth}
\setlength{\sepwidth}{0.024\paperwidth}
\setlength{\colwidth}{0.464\paperwidth}

\newcommand{\separatorcolumn}{\begin{column}{\sepwidth}\end{column}}

% ====================
% Title
% ====================



%Title: A User-Friendly Introduction to Six-Functor Formalism
%
%Abstract: Instead of delving into abstract and highly general formulations, this poster aims to introduce traditional six-functor formalism in an accessible manner. It presents a user-friendly toolkit designed to simplify the understanding and memorization of six-functor formalisms. Additionally, the poster concludes with various applications of six-functor formalism, demonstrating its versatility and utility in different contexts.
%
%For further details and supplementary materials, please visit my GitHub repository: https://github.com/ramified/personal_handwritten_collection/tree/main/applied_six_functor_formalism


\title{Subvarieties in Complex Abelian Varieties}

\author{Xiaoxiang Zhou \\ Advisor: Prof. Dr. Thomas Krämer}

\institute[shortinst]{Humboldt-Universität zu Berlin}

% ====================
% Footer (optional)
% ====================

\footercontent{
  \href{https://ramified.github.io/}{https://ramified.github.io/} \hfill
  Topology of Algebraic Varieties \hfill
  \href{mailto:xiaoxiang.zhou@student.hu-berlin.de}{xiaoxiang.zhou@student.hu-berlin.de}}
% (can be left out to remove footer)

% ====================
% Logo (optional)
% ====================

% use this to include logos on the left and/or right side of the header:
% \logoleft{\includegraphics[height=5cm]{BMS_logo.png}}
% \logoleft{\includegraphics[height=7cm]{logo2.pdf}}

% ====================
% Body
% ====================


%%%%%%%newcommand

\definecolor{BrickRed}{rgb}{0.65, 0.12, 0.21}
\definecolor{BrickBlue}{rgb}{0, 0.1, 0.3}
\DeclareMathOperator{\sky}{\operatorname{sky}}
\DeclareMathOperator{\Hcohom}{\operatorname{H}}
\DeclareMathOperator{\BM}{\operatorname{BM}}
\DeclareMathOperator{\cpt}{\operatorname{c}}
\DeclareMathOperator{\Or}{\operatorname{Or}}
\DeclareMathOperator{\pt}{\operatorname{pt}}
\DeclareMathOperator{\constructable}{\operatorname{cons}}
\DeclareMathOperator{\IC}{\operatorname{IC}}
\DeclareMathOperator{\Perv}{\operatorname{Perv}}
\DeclareMathOperator{\cone}{\operatorname{cone}}
\DeclareMathOperator{\CC}{\operatorname{CC}}
\DeclareMathOperator{\NMD}{\operatorname{NMD}}
\DeclareMathOperator{\sm}{\operatorname{sm}}
\DeclareMathOperator{\Gr}{\operatorname{Gr}}
\DeclareMathOperator{\Hom}{\operatorname{Hom}}
\DeclareMathOperator{\End}{\operatorname{End}}
\DeclareMathOperator{\Aut}{\operatorname{Aut}}
\DeclareMathOperator{\Img}{\operatorname{Im}}
\DeclareMathOperator{\Ob}{\operatorname{Ob}}
\DeclareMathOperator{\Mor}{\operatorname{Mor}}
\DeclareMathOperator{\Pic}{\operatorname{Pic}}
\DeclareMathOperator{\Jac}{\operatorname{Jac}}
\DeclareMathOperator{\Gal}{\operatorname{Gal}}
\DeclareMathOperator{\Mon}{\operatorname{Mon}}
\DeclareMathOperator{\AJ}{\operatorname{AJ}}
\DeclareMathOperator{\AP}{\operatorname{AP}}
\newcommand{\covermap}{h}
\DeclareMathOperator{\Prym}{\operatorname{Prym}}
\DeclareMathOperator{\Nm}{\operatorname{Nm}}
\DeclareMathOperator{\gon}{\operatorname{gon}}
\DeclareMathOperator{\univ}{\operatorname{univ}}
\DeclareMathOperator{\Abel}{\operatorname{Abel}}
\DeclareMathOperator{\pr}{\operatorname{pr}}
\DeclareMathOperator{\aff}{\operatorname{aff}}
\DeclareMathOperator{\charcycle}{\operatorname{cc}}
\DeclareMathOperator{\twist}{\operatorname{tw}}
\DeclareMathOperator{\Hnm}{\operatorname{H}}

\begin{document}

\begin{frame}[t]
\begin{columns}[t]
\separatorcolumn

\begin{column}{\colwidth}



  \begin{block}{Tangent Gauss Map}
Let $A/\mathbb{C}$ be an abelian variety of dimension $n$, and let $Z \subset A$ be a non-degenerate closed subvariety of dimension $r$.\\[15mm]
 
  To understand the geometry of $Z$, we encode the variation of its tangent spaces via the tangent Gauss map
 $$\phi_Z: Z^{\sm}\longrightarrow \Gr(r,T_0A) \qquad p \longrightarrow T_pZ \subset T_pA \cong T_0A.$$
Its differential
$$d_p\phi_Z: T_pZ \longrightarrow \Hom_{\mathbb{C}}(T_pZ, N_pZ)$$
is the second fundamental form, from which curvature invariants can be extracted.\\[15mm]

Besides the obvious examples (illustrated below), when can the Gauss map $\phi_Z$ fail to be generically injective?
$\;$\\[-30mm] $\,$
  \begin{figure}[th]
  \begin{minipage}[t]{.45\textwidth}
    \vspace{0pt}
  	\centering
  	\includegraphics[width=.7\textwidth]{figures/translation_invariant.png}
  \end{minipage}
  \begin{minipage}[t]{.45\textwidth}
    \vspace{0pt}
  	\centering
  	\includegraphics[width=.7\textwidth]{figures/reflection_invariant.png}
  \end{minipage}
  \end{figure}

We specialize to the case where $Z=C$ is a curve. When $n=2$, $\phi_C\!\!: C^{\sm} \longrightarrow \mathbb{P}^1$ typically fails to be generically injective.

  
\begin{tcolorbox}[
  colback=gray!15,
  colframe=gray!60,
  boxrule=0.5pt,
  arc=4pt,
  left=12pt,right=12pt,top=12pt,bottom=12pt
]
\textbf{Conjecture 1.}
Let $C \subset A$ be a non-degenerate curve, $n>2$. If $C$ is not invariant under any non-trivial translation or reflection, then $\phi_C$ is generically injective.
\end{tcolorbox}
  \end{block}
  \begin{block}{A Counterexample for Conjecture 1}
  $\;$\\[-45mm] $\,$
    \begin{figure}[th]
    \begin{minipage}[t]{.45\textwidth}
      \vspace{0pt}
    	\centering
    	\includegraphics[width=.7\textwidth]{figures/rotation_invariant.png}
    \end{minipage}
    \end{figure}
\begin{tcolorbox}[
  colback=blue!5,
  colframe=gray!60,
  boxrule=0.5pt,
  arc=4pt,
  left=12pt,right=12pt,top=12pt,bottom=12pt
]
\textbf{Example 1.}
For $A = E_{\rho}^{\oplus n}$, $\zeta_3$ acts on $A$ (and hence on $T_0A$) by scalar multiplication.
Computer experiments yield a non-degenerate $\zeta_3$-invariant curve $C \subset A$, for which $\phi_C$ is not generically injective.
\end{tcolorbox}
We have found no counterexample to Conjecture 1 when $A$ is not isogenous to $E_{\mathfrak{i}}^{\oplus n}$ or $E_{\rho}^{\oplus n}$. This suggests the following refinement:
\begin{tcolorbox}[
  colback=gray!15,
  colframe=gray!60,
  boxrule=0.5pt,
  arc=4pt,
  left=12pt,right=12pt,top=12pt,bottom=12pt
]
\textbf{Conjecture 2.}
Let $C \subset A$ be a non-degenerate curve, $n>2$. If no non-trivial $\tau \in \Aut(A)$ preserves $C$ and acts by scalar multiplication on $T_0A$, then $\phi_C$ is generically injective.
\end{tcolorbox}
One may restate the conjecture using Gauss curvature, yielding a slightly stronger statement:
\begin{tcolorbox}[
  colback=gray!15,
  colframe=gray!60,
  boxrule=0.5pt,
  arc=4pt,
  left=12pt,right=12pt,top=12pt,bottom=12pt
]
\textbf{Conjecture 3.}
Let $C \subset A$ be a non-degenerate curve, $n>2$. For a general point $p \in \Img \phi_C$, all points in $\phi_C^{-1}(p)$ exhibit the same Gauss curvature.
\end{tcolorbox}

  \end{block}
  \begin{block}{Known Cases}
  \begin{enumerate}
  \item
  If $A=\Jac(C)$ and $C$ is embedded via the Abel-–Jacobi map, then $\phi_C=|\omega_C|$ is the canonical map:
%  $\;$\\[-25mm] $\,$
  \begin{itemize}
  \item When $C$ is hyperelliptic, $C$ is invariant under the hyperelliptic involution, and $\deg \phi_C=2$;
  \item When $C$ is non-hyperelliptic, $\phi_C$ is an embedding.
  \end{itemize}
  
  \item Let $h:C \longrightarrow C'$ be a cyclic $k$-fold cover defined by $\eta \in \Pic(C')$ with $\eta^{\otimes k} \cong \mathcal{O}_{C'}(B)$. If $A = \Prym(C/C')$ and $C \to A$ is the Abel--Prym map, then
  $$T_0A \cong \Hnm^0(\omega_C)/\Hnm^0(\omega_{C'}) \cong \bigoplus_{i=1}^{k-1} \Hnm^0(\omega_{C'} \otimes \eta^i)$$
  $$\phi_C: C \longrightarrow \mathbb{P} T_0A \cong \mathbb{P} \left( \bigoplus_{i=1}^{k-1} \Hnm^0(\omega_{C'} \otimes \eta^i)  \right)$$
  $\;$\\[-25mm] $\,$
  \begin{itemize}
  \item $k=2$: $C$ is invariant under the Prym involution, and $\phi_C=|\omega_{C'} \otimes \eta| \circ h$.\\
  If $C'$ is non-hyperelliptic with $g(C') \geq 4$, then $\deg \phi_C = 2$ or $4$, and
  $$\deg \phi_C=4 \Longleftrightarrow B=\varnothing, C' \text{ is bielliptic and $\eta$ pulled back from EC.}$$
  \item $k>2$: if $g(C') \geq 1$ and $|\omega_{C'} \otimes \eta|$ is generically injective, then $\phi_C$ is generically injective.
  \end{itemize}
  
\item If $C \subset A$ is smooth and either $\deg \phi_C = 2$ or $\phi_C$ is unramified, Conjecture 3 also holds.
  \end{enumerate}
  \end{block}



\end{column}

\separatorcolumn

\begin{column}{\colwidth}

\begin{block}{Perverse Sheaf}

We will mix the usage of sheaves and complexes. For simplicity, let us fix a Whitney stratification $\mathcal{S}$:
\newcommand\largesubset{\mathrel{\scalebox{1.5}[1]{\(\subset\)}}}
$$\varnothing \overset{\;\,U_0}{\largesubset} Z_0 \overset{\;\,U_1}{\largesubset}\cdots  \overset{\;\,U_n}{\largesubset} Z_n = X$$
Denote $D_{\constructable,\mathcal{S}}^{b}(X)$ as the category of constructible sheaves over $X$ with respect to $\mathcal{S}$.

\heading{Definition}

Roughly speaking, a perverse sheaf is a type of sheaf that lies between $\pi^* \mathbb{Q}$ and $\pi^! \mathbb{Q}$. More rigorously, a perverse sheaf is a complex that belongs to the heart of the perverse $t$-structure. 

We say that $\mathcal{F} \in D_{\constructable,\mathcal{S}}^{b}(X)$ is perverse if 
$$
\begin{cases}
\mathcal{H}^i\left(\iota_{U_j}^* \mathcal{F}\right) = 0, & \text{for any } i > -j\\
\mathcal{H}^i\left(\iota_{U_j}^! \mathcal{F}\right) = 0, & \text{for any } i < -j\\
\end{cases}
$$




\heading{Deligne's construction}
Any local system $\mathcal{L}$ supported on $U_i$ can be converted into a perverse sheaf through truncations. This process is known as \textbf{Deligne's construction}, and the resulting perverse sheaf is called the intersection cohomology complex, or the IC sheaf, denoted by $\IC(\mathcal{L})$. IC sheaves are the simple objects in the category $\Perv_{\mathcal{S}}(X)$.

To determine whether a complex $\mathcal{F}$ is {\color{BrickRed}perverse} or an {\color{BrickBlue}IC sheaf}, one simply needs to complete {\color{BrickRed}Table 2}.

  \end{block}
  
  \begin{block}{Nearby Cycle}
A perverse sheaf may not be so ``perverse", but a nearby cycle is definitely ``nearby".

% https://q.uiver.app/#q=WzAsOSxbMCwxLCJcXHswXFx9Il0sWzEsMSwiXFxtYXRoYmJ7Q30iXSxbMiwxLCJcXG1hdGhiYntDfV57XFx0aW1lc30iXSxbMywxLCJcXHdpZGV0aWxkZXtcXG1hdGhiYntDfX1ee1xcdGltZXN9Il0sWzgsMCwiXFxtYXRoY2Fse0Z9Il0sWzQsMSwiXFxtYXRoYmJ7Q30iXSxbNSwwLCJpXipcXG1hdGhjYWx7Rn0iXSxbNiwwLCJcXHBzaSBcXG1hdGhjYWx7Rn0iXSxbNywwLCJcXHZhcnBoaSBcXG1hdGhjYWx7Rn0iXSxbMCwxLCJpIiwwLHsic3R5bGUiOnsidGFpbCI6eyJuYW1lIjoiaG9vayIsInNpZGUiOiJ0b3AifX19XSxbMiwxLCJqIiwyLHsic3R5bGUiOnsidGFpbCI6eyJuYW1lIjoiaG9vayIsInNpZGUiOiJib3R0b20ifX19XSxbMywyLCJwIiwyXSxbMSw0LCIiLDIseyJzdHlsZSI6eyJoZWFkIjp7Im5hbWUiOiJub25lIn19fV0sWzMsNSwiXFxjb25nIiwxLHsic3R5bGUiOnsiYm9keSI6eyJuYW1lIjoibm9uZSJ9LCJoZWFkIjp7Im5hbWUiOiJub25lIn19fV0sWzYsMCwiIiwwLHsic3R5bGUiOnsiaGVhZCI6eyJuYW1lIjoibm9uZSJ9fX1dLFs3LDAsIiIsMCx7InN0eWxlIjp7ImhlYWQiOnsibmFtZSI6Im5vbmUifX19XSxbOCwwLCIiLDAseyJzdHlsZSI6eyJoZWFkIjp7Im5hbWUiOiJub25lIn19fV1d
\[\begin{tikzcd}[ampersand replacement=\&]
	\&\&\&\&[-30mm]\&[-270mm] {\color{BrickRed}i^*\mathcal{F}} \&[-35mm] {\color{BrickRed}\psi \mathcal{F}} \&[-35mm] {\color{BrickRed}\varphi \mathcal{F}} \& {\mathcal{F}} \\[-15mm]
	{\{0\}} \& {\mathbb{C}} \& {\mathbb{C}^{\times}} \& {\widetilde{\mathbb{C}}^{\times}} \& {\mathbb{C}}
	\arrow[no head,color=BrickRed, from=1-6, to=2-1]
	\arrow[no head,color=BrickRed, from=1-7, to=2-1]
	\arrow[no head,color=BrickRed, from=1-8, to=2-1]
	\arrow["i", hook, from=2-1, to=2-2]
	\arrow[no head, from=2-2, to=1-9]
	\arrow["j"', hook', from=2-3, to=2-2]
	\arrow["p"', from=2-4, to=2-3]
	\arrow["\cong"{description}, draw=none, from=2-4, to=2-5]
\end{tikzcd}\]
Given $\mathcal{F} \in D^b(\mathbb{C})$, one can construct the \textbf{nearby cycle} 
$${\color{BrickRed}\psi \mathcal{F}}:= i^* Rj_* p_* p^* j^* \mathcal{F} \in D^b(\{0\}),$$
 which can be roughly viewed as the fiber $\mathcal{F}_x$ for $x$ sufficiently close to $0$. By quotienting out the \textbf{non-vanishing cycle} {\color{BrickRed}$i^*\mathcal{F}$}, one obtains the \textbf{vanishing cycle} $${\color{BrickRed}\varphi \mathcal{F}}:= \cone \left[ i^*\mathcal{F} \overset{sp}{\longrightarrow} \psi \mathcal{F} \right] \in D^b(\{0\}).$$
 
 
% https://q.uiver.app/#q=WzAsOSxbMCwyLCJcXHswXFx9Il0sWzEsMiwiRCJdLFsyLDIsIkReKiJdLFszLDIsIlxcd2lkZXRpbGRle0R9XioiXSxbMCwxLCJYXzAiXSxbMSwxLCJYIl0sWzIsMSwiWF4qIl0sWzMsMSwiXFx3aWRldGlsZGV7WH1eKiJdLFs0LDAsIlxcbWF0aGNhbHtGfSJdLFs3LDNdLFs2LDJdLFs1LDFdLFs0LDBdLFs0LDUsImkiLDAseyJzdHlsZSI6eyJ0YWlsIjp7Im5hbWUiOiJob29rIiwic2lkZSI6InRvcCJ9fX1dLFs2LDUsImoiLDIseyJzdHlsZSI6eyJ0YWlsIjp7Im5hbWUiOiJob29rIiwic2lkZSI6ImJvdHRvbSJ9fX1dLFsyLDEsIiIsMSx7InN0eWxlIjp7InRhaWwiOnsibmFtZSI6Imhvb2siLCJzaWRlIjoiYm90dG9tIn19fV0sWzAsMSwiIiwxLHsic3R5bGUiOnsidGFpbCI6eyJuYW1lIjoiaG9vayIsInNpZGUiOiJ0b3AifX19XSxbNyw2LCJwIiwyXSxbMywyXSxbNSw4LCIiLDIseyJzdHlsZSI6eyJoZWFkIjp7Im5hbWUiOiJub25lIn19fV1d
\[\begin{tikzcd}[ampersand replacement=\&]
	\&\&\&\&[-150mm] {\mathcal{F}} \\[-15mm]
	{X_0} \& X \& {X^*} \& {\widetilde{X}^*} \\
	{\{0\}} \& D \& {D^*} \& {\widetilde{D}^*}
	\arrow["i", hook, from=2-1, to=2-2]
	\arrow[from=2-1, to=3-1]
	\arrow[no head, from=2-2, to=1-5]
	\arrow[from=2-2, to=3-2]
	\arrow["j"', hook', from=2-3, to=2-2]
	\arrow[from=2-3, to=3-3]
	\arrow["p"', from=2-4, to=2-3]
	\arrow[from=2-4, to=3-4]
	\arrow[hook, from=3-1, to=3-2]
	\arrow[hook', from=3-3, to=3-2]
	\arrow[from=3-4, to=3-3]
\end{tikzcd}\]
  In general, $\mathcal{C}$ can be replaced by any disk $\mathcal{D}$, as the problem is local, and $\mathcal{F}$ can be a sheaf over any space $X$ over $\mathcal{D}$.
  
 The same construction yields a distinguished triangle in $D^b(X_0)$:
 % https://q.uiver.app/#q=WzAsNCxbMCwwLCJpXiogXFxtYXRoY2Fse0Z9Il0sWzEsMCwiXFxwc2lfZiBcXG1hdGhjYWx7Rn0iXSxbMiwwLCJcXHZhcnBoaV9mIFxcbWF0aGNhbHtGfSJdLFszLDBdLFswLDFdLFsxLDJdLFsyLDMsIisxIl1d
 \[\begin{tikzcd}[ampersand replacement=\&]
 	{i^* \mathcal{F}} \& {\psi_f \mathcal{F}} \& {\varphi_f \mathcal{F}} \& {}
 	\arrow[from=1-1, to=1-2]
 	\arrow[from=1-2, to=1-3]
 	\arrow["{+1}", from=1-3, to=1-4]
 \end{tikzcd}\]
  \end{block}




\end{column}

\separatorcolumn

\end{columns}
\end{frame}

\end{document}
