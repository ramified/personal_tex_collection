
\documentclass[reqno, UTF8]{amsart}
%Typical documenttypes: article/book
%some examples:
%\documentclass[reqno,11pt]{book}   %%%for books
%\documentclass[]{minimal}			%%%for Minimal Working Example


%for beamers, you have to change a lot. Especially, remove the package enumitem!!!



%%%%%%%%%%%%%%%%%%%% setting for fast compiling

%\special{dvipdfmx:config z 0}		% no compression

%\includeonly{chapters/chapter9}		% In practice, use an empty document called "chapter9"	% usually for printing books






%%%%%%%%%%%%%%%%%%%% here we include packages

%%%basic packages for math articles
\usepackage{amssymb}
\usepackage{amsthm}
\usepackage{amsmath}
\usepackage{amsfonts}
\usepackage[shortlabels]{enumitem}	% It supersedes both enumerate and mdwlist. The package option shortlabels is included to configure the labels like in enumerate.

%%%packages for special symbols
\usepackage{pifont}					% Access to PostScript standard Symbol and Dingbats fonts
\usepackage{wasysym}				% additional characters
\usepackage{bm}						% bold fonts: \bm{...}
\usepackage{extarrows}				% may be replaced by tikz-cd
%\usepackage{unicode-math}			% unicode maths for math fonts, now I don't know how to include it
%\usepackage{ctex}					% Chinese characters, huge difference.


%%%basic packages for fancy electronic documents
\usepackage[colorlinks]{hyperref}
\usepackage[table,hyperref]{xcolor} 			% before tikz-cd. 
%\usepackage[table,hyperref,monochrome]{xcolor}	% disable colored output (black and white)

%%%packages for figures and tables (general setting)
\usepackage{float}				%Improved interface for floating objects
\usepackage{caption,subcaption}
\usepackage{adjustbox}			% for me it is usually used in tables 
\usepackage{stackengine}		%baseline changes

%%%packages for commutative diagrams
\usepackage{tikz-cd}
\usepackage{quiver}			% see https://q.uiver.app/

%%%packages for pictures
\usepackage[width=0.5,tiewidth=0.7]{strands}
\usepackage{graphicx}			% Enhanced support for graphics

%%%packages for tables and general settings
\usepackage{array}
\usepackage{makecell}
\usepackage{multicol}
\usepackage{multirow}
\usepackage{diagbox}
\usepackage{longtable}

%%%packages for ToC, LoF and LoT







 %https://tex.stackexchange.com/questions/58852/possible-incompatibility-with-enumitem










%%%%%%%%%%%%%%%%%%%% here we include theoremstyles

\numberwithin{equation}{section}

\theoremstyle{plain}
\newtheorem{theorem}{Theorem}[section]

\newtheorem{setting}[theorem]{Setting}
\newtheorem{definition}[theorem]{Definition}
\newtheorem{lemma}[theorem]{Lemma}
\newtheorem{proposition}[theorem]{Proposition}
\newtheorem{corollary}[theorem]{Corollary}
\newtheorem{conjecture}[theorem]{Conjecture}

\newtheorem{claim}[theorem]{Claim}
\newtheorem{eg}[theorem]{Example}
\newtheorem{ex}[theorem]{Exercise}
\newtheorem{fact}[theorem]{Fact}
\newtheorem{ques}[theorem]{Question}
\newtheorem{warning}[theorem]{Warning}



\newtheorem*{bbox}{Black box}
\newtheorem*{notation}{Conventions and Notations}


\numberwithin{equation}{section}


\theoremstyle{remark}

\newtheorem{remark}[theorem]{Remark}
\newtheorem*{remarks}{Remarks}

%%% for important theorems
%\newtheoremstyle{theoremletter}{4mm}{1mm}{\itshape}{ }{\bfseries}{}{ }{}
%\theoremstyle{theoremletter}
%\newtheorem{theoremA}{Theorem}
%\renewcommand{\thetheoremA}{A}
%\newtheorem{theoremB}{Theorem}
%\renewcommand{\thetheoremB}{B}







%%%%%%%%%%%%%%%%%%%% here we declare some symbols

%%%%%%%DeclareMathOperator
%see here for why newcommand is better for DeclareMathOperator: https://tex.stackexchange.com/questions/67506/newcommand-vs-declaremathoperator

%%%%%basic symbols. Keep them!

%%%symbols for sets and maps
\DeclareMathOperator{\pt}{\operatorname{pt}}	%points. Other possibilities are \{pt\}, \{*\}, pt, * ...
\DeclareMathOperator{\Id}{\operatorname{Id}}	%identity in groups.
\DeclareMathOperator{\Img}{\operatorname{Im}}

\DeclareMathOperator{\Ob}{\operatorname{Ob}}
\DeclareMathOperator{\Mor}{\operatorname{Mor}}	%difference of Mor and Hom: Hom is usually for abelian categories
\DeclareMathOperator{\Hom}{\operatorname{Hom}}	\DeclareMathOperator{\End}{\operatorname{End}}
\DeclareMathOperator{\Aut}{\operatorname{Aut}}

%%%symbols for linear algebras and 
%%linear algebras
\DeclareMathOperator{\tr}{\operatorname{tr}}
\DeclareMathOperator{\diag}{\operatorname{diag}}	%for diagonal matrices

%%abstract algebras
\DeclareMathOperator{\ord}{\operatorname{ord}}
\DeclareMathOperator{\gr}{\operatorname{gr}}
\DeclareMathOperator{\Frac}{\operatorname{Frac}}

%%%symbols for basic geometries
\DeclareMathOperator{\vol}{\operatorname{vol}}	%volume
\DeclareMathOperator{\dist}{\operatorname{dist}}
\DeclareMathOperator{\supp}{\operatorname{supp}}

%%%symbols for category
%%names of categories
\DeclareMathOperator{\Mod}{\operatorname{Mod}}
\DeclareMathOperator{\Vect}{\operatorname{Vect}}


%%%symbols for homological algebras
\DeclareMathOperator{\Tor}{\operatorname{Tor}}
\DeclareMathOperator{\Ext}{\operatorname{Ext}}
\DeclareMathOperator{\gldim}{\operatorname{gl.dim}}
\DeclareMathOperator{\projdim}{\operatorname{proj.dim}}
\DeclareMathOperator{\injdim}{\operatorname{inj.dim}}
\DeclareMathOperator{\rad}{\operatorname{rad}}


%%%symbols for algebraic groups
\DeclareMathOperator{\GL}{\operatorname{GL}}
\DeclareMathOperator{\SL}{\operatorname{SL}}

%%%symbols for typical varieties
\DeclareMathOperator{\Gr}{\operatorname{Gr}}
\DeclareMathOperator{\Flag}{\operatorname{Flag}}

%%%symbols for basic algebraic geometry
\DeclareMathOperator{\Spec}{\operatorname{Spec}}
\DeclareMathOperator{\Coh}{\operatorname{Coh}}
\newcommand{\Dcoh}{\mathcal{D}_{\operatorname{Coh}}}%%%This one shows the difference between \DeclareMathOperator and \newcommand
\DeclareMathOperator{\Pic}{\operatorname{Pic}}
\DeclareMathOperator{\Jac}{\operatorname{Jac}}

%%%%%advanced symbols. Choose the part you need!

%%%symbols for algebraic representation theory
\DeclareMathOperator{\ind}{\operatorname{ind}}	%\ind(Q) means the set of  equivalence classes of finite dimensional indecomposable representations
\DeclareMathOperator{\Res}{\operatorname{Res}}
\DeclareMathOperator{\Ind}{\operatorname{Ind}}
\DeclareMathOperator{\cInd}{\operatorname{c-Ind}}


\DeclareMathOperator{\Rep}{\operatorname{Rep}}
\DeclareMathOperator{\rep}{\operatorname{rep}} %usually rep means the category of finite dimensional representations, while Rep means the category of representations.
\DeclareMathOperator{\Irr}{\operatorname{Irr}}
\DeclareMathOperator{\irr}{\operatorname{irr}}
\DeclareMathOperator{\Adm}{\operatorname{\Pi}}
\DeclareMathOperator{\Char}{\operatorname{Char}}
\DeclareMathOperator{\WDrep}{\operatorname{WDrep}}

%%%symbols for algebraic topology
\DeclareMathOperator{\EGG}{\operatorname{E}\!}
\DeclareMathOperator{\BGG}{\operatorname{B}\!}

\DeclareMathOperator{\chern}{\operatorname{ch}^{*}}
\DeclareMathOperator{\Td}{\operatorname{Td}}
\DeclareMathOperator{\AS}{\operatorname{AS}}	%Atiyah--Segal completion theorem 

%%%symbols for Auslander--Reiten theory 
\DeclareMathOperator{\Modup}{\overline{\operatorname{mod}}}
\DeclareMathOperator{\Moddown}{\underline{\operatorname{mod}}}
\DeclareMathOperator{\Homup}{\overline{\operatorname{Hom}}}
\DeclareMathOperator{\Homdown}{\underline{\operatorname{Hom}}}


%%%symbols for operad
\DeclareMathOperator{\Com}{\operatorname{\mathcal{C}om}}
\DeclareMathOperator{\Ass}{\operatorname{\mathcal{A}ss}}
\DeclareMathOperator{\Lie}{\operatorname{\mathcal{L}ie}}
\DeclareMathOperator{\calEnd}{\operatorname{\mathcal{E}nd}} %cal=\mathcal


%%%%%personal symbols. Use at your own risk!

%%%symbols only for master thesis
\DeclareMathOperator{\ptt}{\operatorname{par}}	%the partition map
\DeclareMathOperator{\str}{\operatorname{str}}	%strict case
\DeclareMathOperator{\RRep}{\widetilde{\operatorname{Rep}}}
\DeclareMathOperator{\Rpt}{\operatorname{R}}
\DeclareMathOperator{\Rptc}{\operatorname{\mathcal{R}}}
\DeclareMathOperator{\Spt}{\operatorname{S}}
\DeclareMathOperator{\Sptc}{\operatorname{\mathcal{S}}}
\DeclareMathOperator{\Kcurl}{\operatorname{\mathcal{K}}}
\DeclareMathOperator{\Hcurl}{\operatorname{\mathcal{H}}}
\DeclareMathOperator{\eu}{\operatorname{eu}}
\DeclareMathOperator{\Eu}{\operatorname{Eu}}
\DeclareMathOperator{\dimv}{\operatorname{\underline{\mathbf{dim}}}}
\DeclareMathOperator{\St}{\mathcal{Z}}

%%%%%symbols which haven't been classified. Add your own math operators here!

\DeclareMathOperator{\Perv}{\operatorname{Perv}}
\DeclareMathOperator{\Alb}{\operatorname{Alb}}
\DeclareMathOperator{\Sp}{\operatorname{Sp}}
\DeclareMathOperator{\SO}{\operatorname{SO}}
\DeclareMathOperator{\E6}{\operatorname{E}_6}
\DeclareMathOperator{\cc}{\operatorname{cc}}
\DeclareMathOperator{\Hnm}{\operatorname{H}}
\DeclareMathOperator{\-mod}{\!\operatorname{-mod}}
\DeclareMathOperator{\divisor}{\operatorname{div}}
\DeclareMathOperator{\rank}{\operatorname{rank}}
\DeclareMathOperator{\CH}{\operatorname{CH}}
\DeclareMathOperator{\sign}{\operatorname{sign}}
\DeclareMathOperator{\longleftmapsto}{\rotatebox[origin=c]{180}{$\;\longmapsto\;$}}






%%%%%%%newcommand

%%%basic symbols
\newcommand{\norm}[1]{\Vert{#1}\Vert}





%%%%%%%%%%%%%%%%%%%% here we make some blocks for special features. 

%%%% todo notes %%%%
\usepackage[colorinlistoftodos,textsize=footnotesize]{todonotes}
\setlength{\marginparwidth}{2.5cm}
\newcommand{\leftnote}[1]{\reversemarginpar\marginnote{\footnotesize #1}}
\newcommand{\rightnote}[1]{\normalmarginpar\marginnote{\footnotesize #1}\reversemarginpar}









%%%%%%%%%%%%%%%%%%%% here we make some global settings. Understand everything here before you make a document!

\usepackage[a4paper,left=3cm,right=3cm,bottom=4cm]{geometry}
\usepackage{indentfirst}	% Indent first paragraph after section header

\setcounter{tocdepth}{1}


%https://latexref.xyz/_005cparindent-_0026-_005cparskip.html
\setlength{\parindent}{15pt}	
\setlength{\parskip}{0pt plus1pt}

%\setlength\intextsep{0cm}
%\setlength\textfloatsep{0cm}
\def\arraystretch{1}
%\setcounter{secnumdepth}{3}

\allowdisplaybreaks


\begin{document}

% The beginning depends on the documentclass. Rewrite this part if you use different documentclass!
\date{\today}

\title
{the Dimension of $Z_{\chi}$, draft
}
\author{Xiaoxiang Zhou}
\address{Institut für Mathematik\\
Humboldt-Universität zu Berlin\\
Berlin, 12489\\ Germany\\} 
\email{email:xiaoxiang.zhou@hu-berlin.de}


\maketitle
\tableofcontents


\section{Background}
In this section, we establish notation and provide background on the question. Experts may wish to skip the first two sections, which are likely to be revised later.

For simplicity, we work over the base field $\kappa = \mathbb{C}$. Let $A$ denote a fixed complex abelian variety, and let $\Perv(A)$ denote the category of perverse sheaves on $A$ with coefficients in $\mathbb{Q}$. For any algebraic group $G$, we denote by $\Rep(G)$ the category of algebraic representations of $G$.

Following the approach of \cite{KW15vanishing}, we work in the quotient category $\overline{\Perv}(A) = \Perv(A) / N(A)$, where $N(A) \subset \Perv(A)$ is the Serre subcategory of negligible complexes. A complex $\mathcal{F}$ is defined to be negligible if $\chi(A, \mathcal{F}) = 0$. This quotient category admits a natural convolution structure, and every finitely generated tensor subcategory of it is Tannakian, with a reductive Tannaka group $G$ (see \cite[Thm 7.1 \& Cor 9.2]{KW15vanishing}). In particular, for any perverse sheaf $\delta \in \overline{\Perv}(A)$, the full subcategory generated by $\delta$ is categorically equivalent to the representation category of an algebraic group $G$:
$$\left< \delta, * \right> \cong \Rep(G).$$

Examples are abundant but intricate. For reference, we provide a brief list of known cases:

\begin{proposition}[see {\cite[Theorem 6.1]{KW15small}, \cite[Theorem 2]{Kr16cubicthreefold} and \cite[Theorem 1.4]{KW15ppav}} for details]
For any smooth projective variety $X$ over $\mathbb{C}$, let $A := \Alb(X)$ be its Albanese variety. When the Albanese map
$$\alpha: X \longrightarrow Alb(X)$$
is a closed embedding, this map defines a perverse sheaf 
$$\delta_X:= \alpha_* (\underline{\mathbb{Q}}[\dim X]) \in \overline{\Perv}(A).$$ 
In several cases, the Tannaka group is already well understood, as follows:
%https://tex.stackexchange.com/questions/342702/removing-brace-from-the-left-of-dcases
$$
\left< \delta_X, * \right> \cong
\left\{
\begin{aligned}
&\Rep( \SL_{2g-2} (\mathbb{C}) ), && X=C \text{ non-hyperelliptic}\\
&\Rep( \Sp_{2g-2} (\mathbb{C}) ), && X=C \text{ hyperelliptic}\\
&\Rep( \E6 (\mathbb{C}) ), && X=S \text{ Fano surface}\\
&\Rep( \SO_{g!} (\mathbb{C}) ), && X=\Theta,\; g \text{ odd}\\
&\Rep( \Sp_{g!} (\mathbb{C}) ), && X=\Theta,\; g \text{ even}\\
\end{aligned}
\right.
\hspace{28mm}
\textcolor{gray}{
\begin{aligned}
&A_{2g-3}\\
&C_{g-1} \\
&E_6 \\
&D_{g!/2}\\
&C_{g!/2}
\end{aligned}
}
\hspace{-33mm}
$$

Here, $g:= \dim_{\mathbb{C}}(A)$, and
\begin{itemize}
\item $C$ is a smooth projective curve over $\mathbb{C}$ with genus $g \geqslant 2$;
\item $S$ is the Fano surface of a smooth cubic threefold;
\item $\Theta$ is the smooth theta divisor of a general principally polarized abelian variety.
\end{itemize}
\end{proposition}
In \cite[2.c]{Kr20}, any perverse sheaf $\mathcal{F}$ can be associated with its clean characteristic cycle
$$\cc(\mathcal{F})= \sum_Z m_{\mathcal{F}}(Z) [\Lambda_Z].$$
This coinsides with the weight decomposition for $V \in \Rep(G)$:
$$V = \bigoplus_{\chi \in X^*(T)} V_{\chi} = \bigoplus_{[\chi] \in X^*(T)/W} \left(\bigoplus_{\chi \in [\chi]} V_{\chi} \right)$$

By comparing the following formulas (and applying induction on highest weight representations), we can associate each weight orbit $[\chi]$ with a subvariety $Z$ of $A$:
$$
\left\{
\begin{aligned}
\chi(\mathcal{F})&= \sum_Z \deg \Lambda_Z \cdot m_{\mathcal{F}}(Z)\\
\dim_{\mathbb{C}} V &= \sum_{[\chi]} \# [\chi] \cdot \dim_{\mathbb{C}} V_{\chi} 
\end{aligned}
\right.
$$
We may later denote this subvariety as $Z_{\chi}$ to indicate its correspondence with the weight orbit where $\chi$ lies.


In the case of curves, the conormal cone $\Lambda_Z$ of $Z = Z_{\chi}$ has an explicit description as a Lagrangian cycle:
$$\Lambda_Z \subset T^* A \cong A \times \Hnm^0(C,\omega_C).$$
In the next section, we will describe this Lagrangian cycle in detail, leading to an explicit description of $Z_{\chi}$.



\section{Description of $Z_{\chi}$ via correspondence}

From now on, we focus on the curve case, where $G = \SL_{2g-2}(\mathbb{C})$ or $\Sp_{2g-2}(\mathbb{C})$. We fix a maximal torus $T$ of $G$ and denotes by $\varpi \in X^*(T)$ the highest weight of the standard representation. In both cases, $\delta$ corresponds to the minuscule representation $L(\varpi)$. The Weyl group $W = S_{2g-2}$ or $S_{g-1} \times (\mathbb{Z}/2\mathbb{Z})^{g-1}$ acts on the character lattice $X^*(T)$. Letting $$[\varpi] = \{\lambda_1,\ldots, \lambda_{2g-2} \} \subset X^*(T)$$ 
denote the orbit of $\varpi$, we have 
$$X^*(T) = \left< \lambda_1,\ldots, \lambda_{2g-2} \right>_{\mathbb{Z}\-mod}.$$
In other words, any character $\chi \in X^*(T)$ can be written as $\chi = \sum_{i=1}^{2g-2} m_i \lambda_i$ for some tuple $(m) = (m_1, \ldots, m_{2g-2}) \in \mathbb{Z}^{2g-2}$.

For any $(m) \in \mathbb{Z}^k$, we can construct a map 
% https://q.uiver.app/#q=WzAsNyxbMCwwLCJhXnsobSl9OiJdLFsxLDAsIkNeayJdLFsyLDAsIlxcUGljXntcXHN1bSBtX2l9KEMpIl0sWzMsMCwiQSJdLFsyLDEsIlxcc3VtX3tpPTF9XntrfSBtX2lwX2kiXSxbMSwxLCIocF8xLFxcbGRvdHMscF9rKSJdLFszLDEsIlxcc3VtX3tpPTF9XntrfSBtX2kocF9pLXBfMCkiXSxbNSw0LCIiLDAseyJzdHlsZSI6eyJ0YWlsIjp7Im5hbWUiOiJtYXBzIHRvIn19fV0sWzQsNiwiIiwwLHsic3R5bGUiOnsidGFpbCI6eyJuYW1lIjoibWFwcyB0byJ9fX1dLFsxLDJdLFsyLDMsIlxcY29uZyIsMSx7InN0eWxlIjp7ImJvZHkiOnsibmFtZSI6Im5vbmUifSwiaGVhZCI6eyJuYW1lIjoibm9uZSJ9fX1dXQ==
\[\begin{tikzcd}[row sep=0mm]
	{a^{(m)}:} &[-10mm] {C^k} & {\Pic^{\scriptstyle\Sigma m_i}\!(C)} &[-5mm] A\hspace{20mm} \\
	& {(p_1,\ldots,p_k)} & {\sum_{i=1}^{k} m_ip_i} & {\sum_{i=1}^{k} m_i(p_i-p_0)}
	\arrow[from=1-2, to=1-3]
	\arrow["\cong"{description}, draw=none, from=1-3, to=1-4]
	\arrow[maps to, from=2-2, to=2-3]
	\arrow[maps to, from=2-3, to=2-4]
\end{tikzcd}\]
For simplicity, we write 
$a := a^{(1, \ldots, 1)}$
and let 
$$K \in \Pic^{2g-2}(C) \cong A$$
denote the class corresponding to the line bundle $\omega_C$ of degree $2g-2$.

\begin{proposition}
Assume the curve is non-hyperelliptic. For $\chi \in X^*(T)$, express $\chi$ as $\chi = \sum_{i=1}^{2g-2} m_i \lambda_i$ for some tuple $(m) \in \mathbb{Z}^{2g-2}$.
\begingroup
\upshape
%\setlist{itemsep=-0.4em}
\renewcommand\labelenumi{(\theenumi)}
\begin{enumerate}[(1)]
\item The conormal cone $\Lambda_{Z_{\chi}}$ is given by 
$$\Lambda_{Z_{\chi}}= \left\{\rule{0mm}{4.4mm} \left(a^{(m)}(p), \eta \right) \in A \times \Hnm^0 (C, \omega_C) \;\middle|\; p \in C^{2g-2},\textstyle \sum p_i=\divisor \eta  \right\}.$$
\item The subvariety $Z_{\chi}$ is described by $Z_{\chi}=a^{(m)}\left(a^{-1}(K)\right)$.
\end{enumerate}
\endgroup
\end{proposition}

\begin{proof}\
\begin{enumerate}[(1)]
\item This can first be checked on the fundamental weights and then extended linearly.
\item Take the projection $\pi_A: A \times \Hnm^0(C, \omega_C) \longrightarrow A$, then
\begin{equation*}
\begin{aligned}
 Z_{\chi} =\;&   \pi_A(\Lambda_{Z_{\chi}})\\ 
  =\;& \left\{a^{(m)}(p) \in A \;\middle|\;\divisor \eta= \textstyle \sum p_i \text{ for some } \eta \in \Hnm^0(C,\omega_C)  \right\} \\ 
  =\;& \left\{a^{(m)}(p) \in A \;\middle|\; a(p)=K  \right\} \\ 
  =\;& a^{(m)}\left(a^{-1}(K)\right)  \\ 
\end{aligned}
\end{equation*}
\end{enumerate}
\end{proof}
In the hyperelliptic case, the statement differs slightly. Assume that $X^*(T) = \bigoplus_{i=1}^{g-1} \mathbb{Z} \lambda_i$ with $\lambda_{i+g-1} = -\lambda_i$. For $\chi = \sum_{i=1}^{g-1} m_i \lambda_i + \sum_{i=g}^{2g-2} 0 \cdot \lambda_i$, let $(n) = (m_1, \ldots, m_{g-1}) \in \mathbb{Z}^{g-1}$. Then the normal cone $\Lambda_{Z_{\chi}}$ is given by 
\begin{equation*}
\begin{aligned}
  \Lambda_{Z_{\chi}}=\;& \left\{\rule{0mm}{4.4mm} \left(a^{(m)}(p), \eta \right) \in A \times \Hnm^0 (C, \omega_C) \;\middle|\; p \in C^{2g-2}, p_{i+g}= -p_i ,\textstyle \sum p_i=\divisor \eta  \right\} \\ 
  =\;& \left\{\rule{0mm}{4.4mm} \left(a^{(n)}(p), \eta \right) \in A \times \Hnm^0 (C, \omega_C) \;\middle|\; p \in C^{g-1},\textstyle \sum 2 p_i=\divisor \eta  \right\} \\ 
\end{aligned}
\end{equation*}
and the subvariety $Z_{\chi}$ is described by 
$$Z_{\chi} = \Img \left( a^{(n)}: C^{g-1} \longrightarrow A \right).$$


The primary difference here arises from the distinct symmetry type. The divisor $\operatorname{div}(\eta)$ exhibits certain internal constraints; the closer the Weyl group is to the full symmetric group, the fewer such constraints we observe. Fortunately, most results for hyperelliptic curves have already been discussed in detail in \cite[Section 3]{Kr20}. For this reason, we will mainly focus on the non-hyperelliptic case from now on.

In the remainder of this document, we will address one central question: 
	\begin{center}
			\fbox{What is the dimension of $Z_{\chi}$?}
	\end{center}
We will analyze this question from two perspectives: one that focuses on the local geometry of $Z_{\chi}$ and another that examines its global characteristics.

\section{Description of the tangent map}


The first approach attempts to determine $\dim_{\mathbb{C}} Z_{\chi}$ by analyzing its tangent space.

For a general point $p = (p_1, \ldots , p_{2g-2})$ in $a^{-1}(K)$, the fiber $a^{-1}(K)$ is smooth at $p$ and the space $Z_{\chi}$ is smooth at $q := a^{(m)}(p)$. This allows us to derive the diagram
% https://q.uiver.app/#q=WzAsMTQsWzEsMCwiYV57LTF9KEspIl0sWzYsMCwiVF9wIGFeey0xfShLKSJdLFswLDEsIlxce0tcXH0iXSxbMiwxLCJaX3tcXGNoaX0iXSxbNSwxLCIwIl0sWzcsMSwiVF9xIFpfe1xcY2hpfSJdLFs0LDJdLFszLDJdLFswLDQsIkEiXSxbMSwzLCJDXnsyZy0yfSJdLFsyLDQsIkEiXSxbNSw0LCJUX3tLfUEiXSxbNiwzLCJUX3BDXnsyZy0yfSJdLFs3LDQsIlRfcSBBIl0sWzcsNiwiIiwwLHsic3R5bGUiOnsiYm9keSI6eyJuYW1lIjoic3F1aWdnbHkifX19XSxbMiw4XSxbMCw5XSxbMywxMF0sWzQsMTFdLFsxLDEyXSxbNSwxM10sWzEyLDExLCJkX3AgYSIsMix7ImxhYmVsX3Bvc2l0aW9uIjozMH1dLFsxLDVdLFsxMiwxMywiZF9wIGFeeyhtKX0iLDAseyJsYWJlbF9wb3NpdGlvbiI6MzB9XSxbMCwzXSxbMCwyXSxbOSw4LCJhIiwyLHsibGFiZWxfcG9zaXRpb24iOjMwfV0sWzksMTAsImFeeyhtKX0iLDAseyJsYWJlbF9wb3NpdGlvbiI6MzB9XSxbMSw0XV0=
\[\begin{tikzcd}[column sep=2.25em,row sep=tiny]
	& {a^{-1}(K)} &&[-5mm]&&[-5mm]& {T_p a^{-1}(K)} \\
	{\{K\}} && {Z_{\chi}} &&& 0 && {T_q Z_{\chi}} \\
	&&& {} & {} \\[-5mm]
	& {C^{2g-2}} &&&&& {T_pC^{2g-2}} \\
	A && A &&& {T_{K}A} && {T_q A}
	\arrow[from=1-2, to=2-1]
	\arrow[from=1-2, to=2-3]
	\arrow[from=1-2, to=4-2]
	\arrow[from=1-7, to=2-6]
	\arrow[from=1-7, to=2-8]
	\arrow[from=1-7, to=4-7]
	\arrow[from=2-1, to=5-1]
	\arrow[from=2-3, to=5-3]
	\arrow[from=2-6, to=5-6]
	\arrow[from=2-8, to=5-8]
	\arrow[squiggly, from=3-4, to=3-5]
	\arrow["a"'{pos=0.3}, from=4-2, to=5-1]
	\arrow["{a^{(m)}}"{pos=0.25}, from=4-2, to=5-3]
	\arrow["{d_p a}"'{pos=0.3}, from=4-7, to=5-6]
	\arrow["{d_p a^{(m)}}"{pos=0.25}, from=4-7, to=5-8]
\end{tikzcd}\]
and get
$$T_q Z_{\chi} = d_p a^{(m)}(T_p a^{-1}(K)).$$

In the remainder of this section, we will analyze $T_q Z_{\chi}$ for general points $p \in a^{-1}(K)$, breaking down the process into three steps:

\begingroup

\setlist{itemindent=5mm}
\setlist{itemsep=0em}
\renewcommand\labelenumi{(\theenumi)}
\begin{enumerate}[\bfseries{Step} 1.]
\item Analyze $d_p a^{(m)}$.
\item Verify that $T_p a^{-1}(K) = \ker d_p a$.
\item Reduce the computation of $\dim_{\mathbb{C}} T_q Z_{\chi}$ to a linear algebra question, which will be dealt with in the next section.
\end{enumerate}
\endgroup

$\;$\\
{\bfseries Step 1.} The following lemma provides a foundational result for analyzing the tangent map up to scalar.

\begin{lemma}[see {\cite[Proposition 11.1.4]{BL04}}]\label{lemma:diff_of_AJ}
The projectivized differential of the Abel--Jacobi map $\iota_C: C \longrightarrow A$ is the canonical embedding $\varphi_C: C \longrightarrow \mathbb{P}^{g-1}$, i.e.,
$$\varphi_C(p) = \Img (d_p \iota_C) \in \mathbb{P}(T_p A) \cong \mathbb{P}\left(\Hnm^0 (C, \omega_C)^*\right).$$
\end{lemma}

For convenience, at each point $p = (p_1, \ldots, p_{2g-2}) \in C^{2g-2}$, we select nonzero elements $\alpha_i \in T_{p_i}C \subset \oplus_i T_{p_i} C$. Then, $\{\alpha_i\}_{i=1}^{2g-2}$ forms a basis for $\oplus_i T_{p_i} C$, and $\operatorname{Im} d_p a$ is generated by
$$\beta_i:= d_p \iota_C (\alpha_i) \in \Hnm^0(C, \omega_C)^*.$$
By Lemma \ref{lemma:diff_of_AJ}, $[\beta_i] = \varphi_C(p_i)$.

\begin{lemma}\label{lemma:diff_of_am}
For any tuple $(m) \in \mathbb{Z}^{2g-2}$, the differential of the map $a^{(m)}: C^{2g-2} \longrightarrow A$ at $p$ is given by
%\begin{equation*}
%\begin{aligned}
% d_p a^{(m)}=\;& \left[\left( m_i d_{p_i} \iota_C \right)_{i=1}^{2g-2} : \bigoplus_i T_{p_i}C \longrightarrow  \Hnm^0(C, \omega_C)^*\right]\\ 
% =\;& \left[\left( m_i \beta_i \right)_{i=1}^{2g-2} : \bigoplus_i \mathbb{C} \longrightarrow  \Hnm^0(C, \omega_C)^*\right]\\ 
%\end{aligned}
%\end{equation*}
\begin{equation*}
\begin{aligned}
 d_p a^{(m)}=\; \left( m_i d_{p_i} \iota_C \right)_{i=1}^{2g-2}  
 =\; \left( m_i \beta_i \right)_{i=1}^{2g-2} :\qquad T_p C^{2g-2} \longrightarrow  \Hnm^0(C, \omega_C)^*
\end{aligned}
\end{equation*}
under the identification
$$
 T_p C^{2g-2} \cong \bigoplus_{i=1}^{2g-2} T_{p_i}C \cong \bigoplus_{i=1}^{2g-2} \mathbb{C}.
$$
\end{lemma}

\begin{proof}
This is done by first checking it for  $(m)=(0,\ldots,1,\ldots,0)$, and then extending linearly.
\end{proof}


\begin{corollary}\label{cor:hyperplanemain}\
\begingroup
\upshape
%\setlist{itemsep=-0.4em}
\renewcommand\labelenumi{(\theenumi)}
\begin{enumerate}[(1)]
\item For any point $p \in a^{-1}(K)$, the associated differential $\eta \in H^0(C, \omega_C)$ determines a hyperplane $H$ in $H^0(C, \omega_C)^*$, which contains the image of $d_p a$. \label{cor: hyperplane}
\item For a general point $p \in a^{-1}(K)$, any selection of $g-1$ elements from $\{\beta_1, \ldots, \beta_{2g-2}\}$ is linearly independent and spans the hyperplane $H$.\label{cor: genpos}
\end{enumerate}
\endgroup
\end{corollary}

\begin{proof}\
\begin{enumerate}[(1)]
\item This follows from Lemma \ref{lemma:diff_of_AJ} and Lemma \ref{lemma:diff_of_am}.
\item This is a consequence of the general position theorem; see [Ar85, p109] for further details.
\end{enumerate}
\end{proof}
$\;$\\
{\bfseries Step 2.} It is easy to check that $T_p a^{-1}(K) \subseteq \ker d_p a$, and the equality is established through dimension counting. Observe that Lemma 3 implies $\operatorname{Im} d_p a = H$ for a generic point $p \in a^{-1}(K)$.
 
$\;$\\
{\bfseries Step 3.} 
Recall we have a surjection
$$d_p a^{(m)}: T_p a^{-1}(K) \longrightarrow T_q Z_{\chi}$$
for a generic point $p \in a^{-1}(K)$. Therefore,
\begin{equation*}
\begin{aligned}
  \dim_{\mathbb{C}} T_q Z_{\chi}=\;& \dim_{\mathbb{C}} T_p a^{-1}(K) - \dim_{\mathbb{C}} \ker \left( d_p a^{(m)} |_{T_p a^{-1}(K)} \right)   \\ 
  =\;& g-1 - \dim_{\mathbb{C}} \left( \ker  d_p a^{(m)} \cap \ker  d_p a \right) \qquad\qquad \text{ {by \bfseries Step 2}}\\ 
  =\;& g-1 - \dim_{\mathbb{C}} \ker
  \begin{pmatrix}
   d_p a^{\phantom{()}} \\  d_p a^{(m)}
  \end{pmatrix} \\ 
  =\;& \rank
    \begin{pmatrix}
     d_p a^{\phantom{()}} \\  d_p a^{(m)}
    \end{pmatrix} - (g-1) \\ 
\end{aligned}
\end{equation*}
where the map
$$
\begin{pmatrix}
     d_p a^{\phantom{()}} \\  d_p a^{(m)}
\end{pmatrix}
: \mathbb{C}^{2g-2} \longrightarrow H^0(C, \omega_C)^*  \oplus H^0(C, \omega_C)^*
$$
has target $ H \oplus H \cong \mathbb{C}^{2g-2}$ by Corollary \ref{cor:hyperplanemain} \ref{cor: hyperplane}. Explicitly, by Lemma \ref{lemma:diff_of_am}, it
has the matrix coefficient expression
\setlength{\arraycolsep}{2pt}
$$
\begin{pmatrix}
     \beta_1 & \cdots & \beta_{2g-2} \\[1mm]
     m_1\beta_1 & \cdots & m_{2g-2}\beta_{2g-2} \\
\end{pmatrix} \in M^{(2g-2) \times (2g-2)}(\mathbb{C}).
$$
Here, $\beta_{1},\ldots,\beta_{2g-2}$ are seen as vectors in $\mathbb{C}^{g-1}$ via a fixed choice of an isomorphism $H \cong \mathbb{C}^{g-1}$.


Now it is time to work on some special cases.

\begin{eg}\
\begingroup
\upshape
%\setlist{itemsep=-0.4em}
\renewcommand\labelenumi{(\theenumi)}
\begin{enumerate}[(1)]
\item \label{eg: case1} If for some $i \leqslant g-1$ and some $a \in \mathbb{Z}$, each of $m_1, \ldots, m_i$ differs from $a$ and $m_{i+1}=\cdots=m_{2g-2}=a$, then
\begin{equation*}
\begin{aligned}
  \rank\begin{pmatrix}
       \beta_1 & \cdots & \beta_{2g-2} \\[1mm]
       m_1\beta_1 & \cdots & m_{2g-2}\beta_{2g-2} \\
  \end{pmatrix} =\;& 
  \rank\begin{pmatrix}
       \beta_1 & \cdots & \beta_{i} & \beta_{i+1} & \cdots & \beta_{2g-2}\\[2mm]
       \rule{4mm}{0mm}m_1\beta_1\rule{4mm}{0mm} & \cdots &\rule{4mm}{0mm} m_{i}\beta_{i}\rule{4mm}{0mm} & a\beta_{i+1} & \cdots & a\beta_{2g-2} \\
  \end{pmatrix}
     \\ 
  =\;& 
    \rank\begin{pmatrix}
         \beta_1 & \cdots & \beta_{i} & \beta_{i+1} & \cdots & \beta_{2g-2}\\[2mm]
         (m_1-a)\beta_1 & \cdots & (m_{i}-a)\beta_{i} & 0 & \cdots & 0 \\
    \end{pmatrix}
       \\ 
     =\;& 
       \rank\begin{pmatrix}
             \beta_{i+1} & \cdots & \beta_{2g-2}
       \end{pmatrix}
       +
   \rank\begin{pmatrix}
        m_1\beta_1 & \cdots & m_{i}\beta_{i} 
   \end{pmatrix}      
          \\ 
      =\;& g-1+i \hspace{40mm} \text{by Corollary \ref{cor:hyperplanemain} \ref{cor: genpos}}
\end{aligned}
\end{equation*}
As a result, we have $\dim_{\mathbb{C}} T_q Z_{\chi}=i$.
\item In cases where at least $g-1$ of the $m_j$'s are equal, let $i$ be the numbers of the remaining elements, then after relabelling we are in situation \ref{eg: case1}, and hence one also gets $\dim_{\mathbb{C}} T_q Z_{\chi}=i$.
\end{enumerate}
\endgroup
\end{eg}

These examples illustrate only a fraction of the possible cases. In the next section, we will address all other cases and prove that they are all divisors. In particular, this shows:

\begin{corollary}
For a non-hyperelliptic curve, where $\chi = \sum_{i=1}^{2g-2} m_i \lambda_i$ with no $g-1$ of $m_i$ equal to each other, we have $\dim_{\mathbb{C}} Z_{\chi} = g-1$.
\end{corollary}


\section{First proof by representation theory}

In this section, we assume that at most $g-2$ of the values $m_i$ are equal. Let $H=\mathbb{C}^{g-1}$ and let $\beta_1, \ldots, \beta_{2g-2} \in H$ be given such that any $g-1$ of them are linearly independent. The goal is to establish the following linear algebra result:
\begin{theorem}\label{thm:A}
There exists a permutation $\sigma \in S_{2g-2}$ such that 
$$\det
\begin{pmatrix}
     \beta_{\sigma(1)} & \cdots & \beta_{\sigma(2g-2)} \\[2mm]
     m_1\beta_{\sigma(1)} & \cdots & m_{2g-2}\beta_{\sigma(2g-2)} \\
\end{pmatrix}
\neq 0.
$$
\end{theorem}


To prove this theorem, several preliminary steps are required.

\begin{definition}
For $\sigma \in S_{2g-2}$, define
$$
f_{\sigma}: (\mathbb{C}^{g-1})^{2g-2} \longrightarrow \mathbb{C} \qquad (v_1, \ldots, v_{2g-2}) \longmapsto \det (v_{\sigma(1)}\cdots v_{\sigma(g-1)}) \det (v_{\sigma(g)}\cdots v_{\sigma(2g-2)})
$$
as a polynomial in $(g-1) \times (2g-2)$ variables, and let 
$$V:= \left< f_{\sigma} \right>_{\sigma \in S_{2g-2}}$$
denote the vector subspace of the polynomial ring generated by $f_{\sigma}$. The symmetric group $S_{2g-2}$ acts naturally on $V$, defined by
$$(\sigma f)(v_1,\ldots , v_{2g-2}) =  f(v_{\sigma(1)},\ldots , v_{\sigma(2g-2)}).$$
\end{definition}

\begin{remark}
The $S_{2g-2}$-representation $V$ is irreducible. In fact, it is exactly the Specht module associated with the Young diagram of shape $(2, 2, \ldots, 2)$, see \cite[\textsection 7.2]{Fu97}.
\end{remark}

\begin{lemma}
The polynomial
$$
f^{(m)}:= \det \begin{pmatrix}
     v_1 & \cdots & v_{2g-2} \\[2mm]
     m_1v_1 & \cdots & m_{2g-2}v_{2g-2} \\
\end{pmatrix} \in V
$$
is nonzero.
\end{lemma}
\begin{proof}
The multilinearity property of the determinant allows us to show that $f^{(m)} \in V$. For proving that $f^{(m)} \neq 0$, we evaluate $f^{(m)}$ at specially chosen values. Since at most $g - 2$ of the $m_i$ terms are identical, we can always select a permutation $\sigma \in S_{2g-2}$ such that 
$$\prod_{l=1}^{g-1} \left( m_{\sigma(2l-1)}- m_{\sigma(2l)} \right) \neq 0.$$
 Let $e_l=(0,\ldots,\underset{\underset{l\text{-th}}{\uparrow}}{1},\ldots,0)^T$, setting $v_{\sigma(2l - 1)} = v_{\sigma(2l)} = e_l$ implies that 
 $$\det\begin{pmatrix}
      v_1 & \cdots & v_{2g-2} \\[2mm]
      m_1v_1 & \cdots & m_{2g-2}v_{2g-2} \\
 \end{pmatrix}\Bigg|_{\substack{v_{\sigma(2l - 1)} = e_l\\ v_{\sigma(2l)} = e_l}}
  =\pm \prod_{l=1}^{g-1} \left( m_{\sigma(2l-1)}- m_{\sigma(2l)} \right) \neq 0.$$
\end{proof}

\begin{proof}[Proof of Theorem \ref{thm:A}]
We prove it by contradiction. If
$$\det \begin{pmatrix}
     \beta_{\sigma(1)} & \cdots & \beta_{\sigma(2g-2)} \\[2mm]
     m_1\beta_{\sigma(1)} & \cdots & m_{2g-2}\beta_{\sigma(2g-2)} \\
\end{pmatrix}
= 0\quad \text{ for all }\sigma \in S_{2g-2},$$
then
$$\left(\sigma(f^{(m)})\right) (\beta_1,\ldots, \beta_{2g-2})
= 0\quad \text{ for all }\sigma \in S_{2g-2}.$$
Since $f^{(m)} \neq 0$ and $V$ is irreducible, we have $V= \left< \sigma(f^{(m)}) \right>_{\sigma \in S_{2g-2}}$, hence
$$f (\beta_1,\ldots, \beta_{2g-2})
= 0\quad \text{ for all }f \in V.$$
Taking $f=f_{\Id}$ implies that 
$$\det(\beta_1,\ldots, \beta_{g-1})\det(\beta_{g},\ldots, \beta_{2g-2})=0.$$ 
However, this contradicts the fact that any arbitrary selection of $g-1$ elements from $\{\beta_1, \ldots, \beta_{2g-2}\} \subset H$ must be linearly independent.
\end{proof}


\section{Description of homology class}

In the second argument, we seek to identify the homology class $[Z_{\chi}]$. When $Z_{\chi}$ is a divisor, 
$$a^{(m)}_* [a^{-1}(K)]=\deg \left( a^{(m)}\big|_{a^{-1}(K)} \right) \cdot [Z_{\chi}] \in \Hnm^2(A;\mathbb{Z}).$$
While I have not developed a way to calculate this degree, we do have a clear description of the homology class $a^{(m)}_* [a^{-1}(K)]$:

\begin{theorem}\label{thm:B}
There exists a homology class $[Z] \in \Hnm^2 (A;\mathbb{Q})$, such that for any tuple $(m) \in \mathbb{Z}^{2g-2}$, 
$$a^{(m)}_* [a^{-1}(K)]=\left( \frac{1}{2^{g-1}} \sum_{\sigma \in S_{2g-2}} \prod_{l=1}^{g-1} \left( m_{\sigma(2l-1)}-m_{\sigma(2l)} \right)^2  \right) \cdot [Z].$$
\end{theorem}

\begin{remark}
Take $m=( \underbrace{1, \ldots, 1}_{g-1}, \underbrace{0, \ldots, 0}_{g-1} )$, then Theorem \ref{thm:B} tells us that
\begin{equation}\label{eq:theta1}
a^{(m)}_* [a^{-1}(K)]= \left( \rule{0mm}{3.0mm} (g-1)! \right)^2 \cdot [Z].
\end{equation}
In the same time, $Z_{\chi} = \Theta$ and $\deg \left( a^{(m)}\big|_{a^{-1}(K)} \right) = \left( \rule{0mm}{3.0mm} (g-1)! \right)^2$, so
\begin{equation}\label{eq:theta2}
a^{(m)}_* [a^{-1}(K)]= \left( \rule{0mm}{3.0mm} (g-1)! \right)^2 \cdot [\Theta].
\end{equation}
Combining \eqref{eq:theta1} and \eqref{eq:theta2}, one gets $[Z]=[\Theta]$ in the main theorem.

Notice that the homology calculation aligns with the dimension calculation, as shown by the equivalence:
$$\dim_{\mathbb{C}} Z_{\chi} = g-1 \quad \Longleftrightarrow \quad a^{(m)}_* [a^{-1}(K)] \neq 0.$$
\end{remark}

To prove Theorem \ref{thm:B}, we start with a few preparatory steps. We start by fixing a basis for the cohomology and then express both the pullback and pushforward with respect to this basis. The homology class $[a^{-1}(K)] \in \Hnm^{2g-2}(C^{2g-2};\mathbb{Z})$ is not yet fully understood, adding complexity to the problem. We address this issue using certain symmetry methods in Section \ref{sec: sym_fct}.

\subsection{Basis}
The cohomology ring of a curve $C/\mathbb{C}$ is well-known:
$$\Hnm^*(C;\mathbb{Z}) =\; \mathbb{Z}\;\oplus\; \bigoplus_{i=1}^{g} \left( \rule{0mm}{3mm} \mathbb{Z}a_i \oplus \mathbb{Z}b_i \right) \;\oplus \;\mathbb{Z}e.$$
In this structure, $e = a_i \cup b_i = -b_i \cup a_i$ are the only non-trivial cup products. To simplify, we relabel $(a_1, b_1, \ldots, a_g, b_g)$ as $(c_1, c_2, \ldots, c_{2g})$, while also setting $c_0 = 1$ and $c_{2g+1} = e$.

Observe that the Abel--Jacobi map $\iota_C: C \longrightarrow A$ yields an isomorphism
$$\iota_C^* : \Hnm^1(A;\mathbb{Z}) \longrightarrow \Hnm^1(C;\mathbb{Z}) \cong \bigoplus_{i=1}^{2g} \mathbb{Z}c_i.$$
We retain the notation $\{c_i\}_{i=1}^{2g}$ as a basis of $\Hnm^1(A; \mathbb{Z})$ as well, and define 
$$c_I:= c_{i_1} \cup \cdots \cup c_{i_d} \in \Hnm^d(A;\mathbb{Z})$$
for $I = \{i_1,\ldots,i_d\} \subseteq \{1,\ldots, 2g\}$, $i_1 < \cdots < i_d$. This sets up the basis for our calculations:
$$
\begin{cases}
\displaystyle\Hnm^*(C;\mathbb{Z}) \cong \hspace{4mm}\bigoplus_{i=0}^{2g+1} \hspace{4mm}\mathbb{Z}c_i\\[5mm]
\displaystyle\Hnm^*(A;\mathbb{Z}) \cong \bigoplus_{I  \subseteq \{1,\ldots, 2g\}} \mathbb{Z}c_I\\
\end{cases}
$$
By Künneth formula, one can directly write down a basis for the cohomology of product spaces:
$$
\begin{cases}
\displaystyle\Hnm^*(C^k;\mathbb{Z}) \cong \bigoplus_{(i_1,\ldots,i_k)}^{\phantom{1}} \mathbb{Z}\left(\rule{0mm}{3mm} c_{i_1} \otimes \cdots \otimes c_{i_k}  \right)\\[5mm]
\displaystyle\Hnm^*(A^k;\mathbb{Z}) \cong \bigoplus_{(I_1,\ldots,I_k)} \mathbb{Z}\left(\rule{0mm}{3mm} c_{I_1} \otimes \cdots \otimes c_{I_k}  \right)\\
\end{cases}
$$

There's a small issue here. Due to the non-commutativity of the cohomology ring, expressions often involve $-1$ coefficients, which, while algebraically necessary, can obscure the core insights. To simplify matters, we adopt a modified basis: 
\begin{equation*}
\begin{aligned}
 \left(\rule{0mm}{3mm} c_{i_1} \otimes \cdots \otimes c_{i_k}  \right)^c :=\;&  \;(-1)^{\sign}\left(\rule{0mm}{3mm} c_{i_1} \otimes \cdots \otimes c_{i_k}  \right)\\ 
 \left(\rule{0mm}{3mm} c_{I_1} \otimes \cdots \otimes c_{I_k}  \right)^c :=\;&  \;(-1)^{\sign}\left(\rule{0mm}{3mm} c_{I_1} \otimes \cdots \otimes c_{I_k}  \right)\\ 
\end{aligned}
\end{equation*}
where the sign accounts for all coefficients arising when rearranging terms to the standard grading order $(c_0, \ldots, c_{2g+1})$. With this adjustment, the permutation action no longer introduces any extraneous coefficients:
$$\sigma \cdot \left(\rule{0mm}{3mm} c_{i_1} \otimes \cdots \otimes c_{i_k}  \right)^c = \left(\rule{0mm}{3mm} c_{\sigma(i_1)} \otimes \cdots \otimes c_{\sigma(i_k)}  \right)^c. $$

\subsection{Pullback}
The pullback is more straightforward to compute than the pushforward, as we only need to determine it on the basis elements. We collect the building blocks here: \footnote{For 2), it reduces to the addition map $S^1 \times S^1 \longrightarrow S^1$.}

\begin{equation*}
\begin{aligned}
&1) && \iota_C: & C\;& \longrightarrow A\qquad & c_i &\longleftmapsto c_i   \\ 
&2) && +: & A \times A\;& \longrightarrow A\qquad & c_i \otimes 1 + 1 \otimes c_i  &\longleftmapsto c_i   \\ 
&3) && \Delta: & A\;& \longrightarrow A \times A\qquad & c_i &\longleftmapsto  c_i \otimes 1,\;  1 \otimes c_i   \\ 
&4) && [m]: &  A\;& \longrightarrow A\qquad & mc_i &\longleftmapsto c_i   \\ 
&5) && +: & A^k\;& \longrightarrow A\qquad & \sum_{j=1}^{k} 1 \otimes \cdots \underset{\underset{j\text{-th}}{\uparrow}}{c_i} \cdots \otimes 1 &\longleftmapsto c_i   \\ 
\end{aligned}
\end{equation*}

Now, for a tuple $(m) \in \mathbb{Z}^{k}$, the map
$$a^{(m)}: C^k \longrightarrow A \qquad (p_1,\ldots , p_k) \longrightarrow \sum_{i=1}^{k} m_i (p_i-p_0)$$
can be written as compositions of basic functions:
% https://q.uiver.app/#q=WzAsNCxbMCwwLCJhXnsobSl9OiBDXmsiXSxbMSwwLCJBXmsiXSxbMiwwLCJBXmsiXSxbMywwLCJBIl0sWzAsMSwiKFxcaW90YV9DLFxcbGRvdHMsXFxpb3RhX0MpIl0sWzEsMiwiKG1fMSwgXFxsZG90cywgbV9rKSJdLFsyLDMsIisiXV0=
\[\begin{tikzcd}[ampersand replacement=\&,column sep=huge]
	{a^{(m)}: C^k} \& {A^k} \& {A^k} \& A
	\arrow["{(\iota_C,\ldots,\iota_C)}", from=1-1, to=1-2]
	\arrow["{(m_1, \ldots, m_k)}", from=1-2, to=1-3]
	\arrow["{+}", from=1-3, to=1-4]
\end{tikzcd}\]
We get
$$a^{(m),*} c_i = \sum_{j=1}^k 1 \otimes \cdots \underset{\underset{j\text{-th}}{\uparrow}}{m_ic_i} \cdots \otimes 1$$
As a result,
\begin{equation*}
\begin{aligned}
 a^{(m),*} c_I =\;& \sum_{I= \sqcup I_i} \left(m_1^{|I_1|} \iota_C^* c_{I_1} \otimes \cdots \otimes m_k^{|I_k|} \iota_C^* c_{I_k}\right)^c \\
 =\;& \sum_{\substack{I= \sqcup I_i \\ |I_i|\leqslant 2}}\left( \prod_{i=1}^k m_i^{|I_i|}\right)  \left(c_{I_1} \otimes \cdots \otimes  c_{I_k}\right)^c \\ 
\end{aligned}
\end{equation*}
where $c_{\{i,j\}} = c_i \cup c_j \in \Hnm^0(C;\mathbb{Z})$ in the last expression.

\subsection{Pushforward}

To determine the pushforward, we apply the projection formula
$$
f_* \alpha \cup \beta = f_*(\alpha \cup f^* \beta).
$$
Notably, this pushforward yields an isomorphism on the top cohomology:
\begin{equation*}
\begin{aligned}
  &\iota_{C, *}: &\Hnm^2(C;\mathbb{Z})& \longrightarrow \Hnm^{2g}(A;\mathbb{Z}) \qquad& e & \longmapsto c_{\{1,\ldots, 2g\}}\\ 
  &a^{(m)}_{*}:& \Hnm^{4g-4}(C^{2g-2};\mathbb{Z})& \longrightarrow \Hnm^{2g}(A;\mathbb{Z}) \qquad& e \otimes \cdots \otimes e & \longmapsto c_{\{1,\ldots, 2g\}}\\ 
\end{aligned}
\end{equation*}
Let us proceed with this method, first on the Abel--Jacobi map $\iota_C$, and then on $a^{(m)}$.

\begin{eg}[Pushforward of $\iota_C$]
Let $\{c_I^*\} \subset \Hnm^* (A;\mathbb{Z})$ denote the dual basis of $\{c_I\}$ with respect to the inner product induced by the cup product. Specifically, this basis satisfies 
$$c_I^* \cup c_J = \delta_{I,J} \cdot c_{\{1,\ldots, 2g\}}.$$

For $i, j \in \{ 1,\ldots,2g \}$, we have 
$$
\iota_{C, *}(c_i) \cup c_j = \iota_{C, *}(c_i \cup c_j) = 
\begin{cases}
c_{\{1,\ldots, 2g\}}, & i \text{ odd }, j=i+1\\
-c_{\{1,\ldots, 2g\}}, & i \text{ even }, j=i-1\\
0, & \text{otherwise}
\end{cases}
$$
which implies that
$$
\iota_{C, *}(c_i)  = 
\begin{cases}
c_{i+1}^*, & i \text{ odd }\\
-c_{i-1}^*, & i \text{ even }
\end{cases}
$$

For $i < j$, we have 
$$
\iota_{C, *}(1) \cup c_{\{i,j\}} = \iota_{C, *}(c_i \cup c_j) = 
\begin{cases}
c_{\{1,\ldots, 2g\}}, & i \text{ odd }, j=i+1\\
0, & \text{otherwise}
\end{cases}
$$
which implies that
$$
\iota_{C, *}(1)  = 
\sum_{l=1}^{g} c_{\{2l-1,2l\}}^*.
$$

In summary, the pushforward can now be written in terms of the elements $a_i$ and $b_i$:
\begin{equation*}
\left\{
\begin{aligned}
 \iota_{C, *}(e)\; =\;&  1^* \\ 
  \iota_{C, *}(a_i) =\;&  b_i^* \\ 
   \iota_{C, *}(b_i) =\;&  -a_i^* \\ 
    \iota_{C, *}(1)\; =\;&  \textstyle\sum_i (a_i \cup b_i)^* 
\end{aligned}
\right.
\end{equation*}

\end{eg}

Using a similar approach, we can explicitly express the pushforward of the map $a^{(m)}$ and qualitatively analyze its properties, as discussed in the following proposition:
\begin{proposition}
For any cohomology class $\alpha \in \sum_{i=1}^{k} \Hnm^{d_i} (C;\mathbb{Z}) \subset \Hnm^{|d|} (C^k;\mathbb{Z})$, $d_i \in \{0,1,2\}$, there exists a unique class $[Z_{\alpha}] \in \Hnm^{2g-2k+|d|}(A;\mathbb{Z})$ such that 
$$
a^{(m)}_* \alpha = \left( \prod_{i=1}^{k}  m_i^{2-d_i}  \right) \cdot [Z_{\alpha}].
$$
Moreover, $[Z_{\sigma(\alpha)}]=[Z_{\alpha}]$ for all permutation $\sigma \in S_k$.
\end{proposition}

\begin{proof}
For any $\beta \in  \Hnm^{2k-|d|} (A;\mathbb{Z})$, there exists $e_{\beta} \in \mathbb{Z}$ such that 
\begin{equation*}
\begin{aligned}
  \;& \left(a^{(m)}_* \alpha \right) \cup \beta = a^{(m)}_* \left(  \alpha \cup a^{(m),*} \beta \right) = \left( \prod_{i=1}^{k} m_i^{2-d_i}\right) e_{\beta}\; c_{\{1,\ldots,2g\}} \\ 
  \Rightarrow\;&  a^{(m)}_* \alpha =  \left( \prod_{i=1}^{k} m_i^{2-d_i}\right) \left( \sum_{\beta\text{:basis}} e_{\beta} \beta^* \right).\\ 
\end{aligned}
\end{equation*}
Moreover,
\begin{equation*}
\begin{aligned}
  \;& \left(a^{(m)}_* \sigma(\alpha) \right) \cup \beta = a^{(m)}_* \left(  \sigma(\alpha) \cup a^{(m),*} \beta \right) = \left( \prod_{i=1}^{k} m_{\sigma(i)}^{2-d_i}\right) e_{\beta}\; c_{\{1,\ldots,2g\}} \\ 
  \Rightarrow\;&  a^{(m)}_* \sigma(\alpha) =  \left( \prod_{i=1}^{k} m_{\sigma(i)}^{2-d_i}\right) \left( \sum_{\beta\text{:basis}} e_{\beta} \beta^* \right).\\ 
\end{aligned}
\end{equation*}
\end{proof}

For a polynomial $f$, the exponent refers to the largest degree of any one variable in $f$. For instance, the polynomial $\prod_{i=1}^{k} m_i^{2-d_i}$ has exponent $\leqslant 2$,

\begin{corollary}\label{cor: push_sym}
For any cohomology class $\alpha \in \Hnm^{d} (C^k;\mathbb{Z})^{S_k}$, we can associate a homogeneous symmetric polynomial $f_I^{\alpha} \in \mathbb{Z}[m_1, \ldots, m_k]^{S_k}$ of degree $2k - d$ and exponent at most $2$ to each subset $I \subseteq \{1, 2, \ldots, 2g\}$ of cardinality $2k - d$, such that 
$$a^{(m)}_*\alpha =\sum_{\substack{I\subseteq \{1,\ldots,2g\} \\ |I|=2k-d}} f_I^{\alpha}(m_1,\ldots,m_k) \cdot c_I^*.$$
\end{corollary}

\section{Second proof by symmetric functions}\label{sec: sym_fct}

Ideally, if we could express $[a^{-1}(K)] \in \Hnm^{2g-2} (C^{2g-2};\mathbb{Z})$ as a linear combination of the basis elements $c_{i_1} \otimes \cdots \otimes c_{i_{2g-2}}$, we could then compute it directly to obtain the answer. As of now, however, this formulation is unknown.

Fortunately, there are several symmetries that simplify calculations. For instance, the subset $a^{-1}(K) \subset C^{2g-2}$ is preserved under the action of $S_{2g-2}$.

\begin{proposition}\label{prop: symfct}\
\begingroup
\upshape
%\setlist{itemsep=-0.4em}
\renewcommand\labelenumi{(\theenumi)}
\begin{enumerate}[(1)]
\item for any $\{i,j \} \subset \{1,\ldots , 2g\}$, $i<j$, there exists  a homogeneous symmetric polynomial $f_{ij}^{\alpha} \in \mathbb{Z}[m_1, \ldots, m_{2g-2}]^{S_{2g-2}}$ of degree $2g-2$ and exponent at most $2$, such that 
\begin{equation}\label{eq: symfct1}
\begin{aligned}
  a^{(m)}_*[a^{-1}(K)] =\sum_{i<j} f_{ij}(m_1,\ldots,m_{2g-2}) \cdot \left( c_i \cup c_j \right)^*.
\end{aligned}
\end{equation}
\item For any $t \in \mathbb{Z}$, 
\begin{equation}\label{eq: symfct2}
\begin{aligned}
f_{ij}(m_1+t,\ldots,m_{2g-2}+t) = f_{ij}(m_1,\ldots,m_{2g-2}).
\end{aligned}
\end{equation}
\item The polynomial $f_{ij}$ is uniquely determined up to a scalar multiple, and it takes the form 
\begin{equation}\label{eq: symfct3}
\begin{aligned}
  f_{ij}(m_1,\ldots,m_{2g-2}) = \frac{e_{ij}}{2^{g-1}} \sum_{\sigma \in S_{2g-2}} \prod_{l=1}^{g-1} \left( m_{\sigma(2l-1)}-m_{\sigma(2l)} \right)^2
\end{aligned}
\end{equation}
for some $e_{ij} \in \mathbb{Q}$. 
\end{enumerate}
\endgroup
\end{proposition}

Combining \eqref{eq: symfct1} and \eqref{eq: symfct3}, one gets
\begin{equation*}
\begin{aligned}
a^{(m)}_* [a^{-1}(K)]=\left( \frac{1}{2^{g-1}} \sum_{\sigma \in S_{2g-2}} \prod_{l=1}^{g-1} \left( m_{\sigma(2l-1)}-m_{\sigma(2l)} \right)^2  \right) \cdot \left( \sum_{i<j} e_{ij} \left( c_i \cup c_j \right)^* \right).
\end{aligned}
\end{equation*}
This proves Theorem \ref{thm:B}.


\begin{proof}[Proof of Proposition \ref{prop: symfct}]\
\begingroup
\upshape
%\setlist{itemsep=-0.4em}
\renewcommand\labelenumi{(\theenumi)}
\begin{enumerate}[(1)]
\item Take $k=2g-2$, $d=2g-2$, $\alpha=[a^{-1}(K)]$ in Corollary \ref{cor: push_sym}.
\item  Notice that $(m_1+t,\ldots,m_{2g-2}+t)$ and $(m_1,\ldots,m_{2g-2})$ define the same homology class.
\item The polynomial $f_{ij}$ must be of the form
$$
f_{ij}(m_1,\ldots,m_{2g-2}) = \sum_{s=1}^{g-1} t_s \left( \sum_{\substack{\text{distinct} \\ \text{sym sum}}} m_1^2 \cdots m_s^2 m_{s+1} \cdots m_{2g-2-l} \right)
$$
for some $t_s \in \mathbb{Z}$. The condition \eqref{eq: symfct2} implies that $\sum_{l=1}^{2g-2} \partial_l f_{ij}=0$, which uniquely determines all $t_s$'s up to a scalar.
\end{enumerate}
\endgroup
\end{proof}

\begin{remark}
Applying the same method as for the map $\iota: C^{g-1} \to A$, we obtain that 

\begin{equation*}
\begin{aligned}
  \,[\Theta]=\;&  \frac{1}{(g-1)!}\; \iota_* [C^{g-1}]\\ 
  =\;&  \frac{1}{(g-1)!} (g-1)!\; \sum_{l=1}^g c_{\{1,\ldots, 2g\} \smallsetminus \{2l-1,2l\}}^*\\
   =\;&  \sum_{l=1}^g c_{2l-1} \cup c_{2l}\\
   =\;&  \sum_{l=1}^g a_{l} \cup b_{l} \qquad\qquad \in \Hnm^2(A;\mathbb{Z}).\\
\end{aligned}
\end{equation*}
However, we still cannot determine $[a^{-1}(K)] \in \Hnm^{2g-2}(C^{2g-2}; \mathbb{Z})$ explicitly. The choice of $a_i$ and $b_i$ might affect how $[a^{-1}(K)]$ is expressed. (Now we may have some new methods)

This argument cannot be extended to the Chow group, as we lack a satisfactory understanding of the Chow group $\CH^{g-1}(C^{2g-2};\mathbb{Z})$, and the Künneth formula no longer holds in this context.
\end{remark}

\begin{ques}\
\begingroup
\upshape
%\setlist{itemsep=-0.4em}
\renewcommand\labelenumi{(\theenumi)}
\begin{enumerate}[(1)]
\item How is the homology class of the special fiber generally computed? 
\item Extend this approach to the $E_6$ case. (It's doable, and now we are working over divisor cases.)
\end{enumerate}
\endgroup
\end{ques}
%\nocite{Eberhardt2022Koszul}	% cite articles which are not cited in the document yet

% Remember to protect the uppercase of people's name and LaTeX symbols

\bibliographystyle{plain}
\bibliography{reference}
\end{document}