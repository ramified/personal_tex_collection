\documentclass[pdf]{beamer}
\mode<presentation>{
	\usetheme{Ilmenau}
	
}
\usecolortheme{dolphin}
%\usepackage[UTF8,indent]{ctexcap}%中文
\usepackage{amssymb}
\usepackage{amsmath}
\usepackage{amsfonts}
%\usepackage{graphicx}
\usepackage{amsthm}
\usepackage{indentfirst}
\usepackage{enumerate}
\usepackage{extpfeil}
\usepackage{tikz-cd}
\usepackage{longtable}
\usepackage{makecell}
\usepackage{array}
\usepackage{xcolor}
\usetikzlibrary {calc,positioning,shapes.misc,graphs,decorations.pathreplacing}


\numberwithin{equation}{section}

\theoremstyle{plain}
\newtheorem{proposition}[theorem]{Proposition}
\newtheorem{claim}[theorem]{Claim}
\newtheorem{defn}[theorem]{Definition}
\newtheorem{eg}[theorem]{Example}
\newtheorem{pf}[theorem]{Proof}
\newtheorem{cor}[theorem]{Corollary}
\newtheorem{setting}[theorem]{Setting}
\newtheorem{goal}[theorem]{Goal}

\newtheorem{tabloid}[theorem]{Tabloid: equivalence class of standard filling}


\theoremstyle{plain}
\newtheorem{exercise}{Exercise}[section]


\theoremstyle{remark}
\newtheorem{remark}[theorem]{Remark}
\newtheorem{remarks}{Remarks}
\newtheorem{ex}[theorem]{Exercise}
\newtheorem{question}[theorem]{Question}
\newtheorem{facts}[theorem]{Facts}
\newtheorem{short}{ }



\newcommand*{\thick}[1]{\text{\boldmath$#1$}}
\newcommand*{\cir}[1]{\;$\ding{19#1}$\;}%临时使用
\newcommand*{\norm}[1]{\lVert#1\rVert}
\newcommand*{\ignore}[1]{\textcolor{gray}{#1}}
\newcommand*{\stress}[1]{\textcolor{red}{#1}}
\newcommand*{\bgpicb}[1]{\usebackgroundtemplate{%
	\begin{tikzpicture}[path image/.style={
		path picture={
			\node at (path picture bounding box.center) {
				\includegraphics[height=10cm]{#1}
			};
	}}]
	
	\draw [path image]
	(current page.north west) rectangle
	(current page.south east);
	
	\end{tikzpicture}
}}
\newcommand*{\bgpica}[1]{\usebackgroundtemplate{%
		\begin{tikzpicture}[path image/.style={
			path picture={
				\node at (path picture bounding box.center) {
					\includegraphics[height=7.5cm]{#1}
				};
		}}]
		
		\draw [path image]
		(current page.north west) rectangle
		(current page.south east);
		
		\end{tikzpicture}
}}


\DeclareMathOperator{\supp}{supp}
\DeclareMathOperator{\dist}{dist}
\DeclareMathOperator{\vol}{vol}
\DeclareMathOperator{\diag}{diag}
\DeclareMathOperator{\tr}{tr}
\DeclareMathOperator{\Proj}{\operatorname{Proj}}
\DeclareMathOperator{\Aut}{\operatorname{Aut}}
\DeclareMathOperator{\Img}{\operatorname{Im}}
\DeclareMathOperator{\Sym}{\operatorname{Sym}}
\DeclareMathOperator{\sgn}{\operatorname{sgn}}
\DeclareMathOperator{\Id}{\operatorname{Id}}
\DeclareMathOperator{\ques}{\;?\;}
\DeclareMathOperator{\Fl}{\mathcal{F\ell}}
\DeclareMathOperator{\Jac}{\operatorname{Jac}}
\DeclareMathOperator{\Prym}{\operatorname{Prym}}
\DeclareMathOperator{\AJ}{\operatorname{AJ}}
\DeclareMathOperator{\AP}{\operatorname{AP}}
\DeclareMathOperator{\univ}{\operatorname{univ}}
%\setlength{\parindent}{1em}
\newcommand{\character}[2]{\left[\begin{array}{c}{#1} \\ {#2}\end{array}\right]}
\newcommand{\normalcharacter}{\character{\epsilon}{\epsilon'}}



\setbeamertemplate{caption}[numbered]
% 设置图形文件的搜索路径
\graphicspath{{figures/}}
\title{Subvarieties in Abelian Variety}
\author[Xiaoxiang Zhou]{Xiaoxiang Zhou\\[10mm]{\small Supervisor: Thomas Krämer
}}
\institute[HU berlin]{Humboldt-Universität zu Berlin}
\date{\today}
\tikzset{
	invisible/.style={opacity=0,text opacity=0},
	visible on/.style={alt=#1{}{invisible}},
	alt/.code args={<#1>#2#3}{%
		\alt<#1>{\pgfkeysalso{#2}}{\pgfkeysalso{#3}} % \pgfkeysalso doesn't change the path
	},
}
%\setbeamercolor{section number projected}{fg=white!90!blue, bg=red!90!black}
\definecolor{goodblue}{RGB}{71,71,186}
\setbeamercolor{block body}{fg=black,bg=gray!10}
\setbeamercolor{block title}{fg=white, bg=goodblue}
\usefonttheme[onlymath]{serif}
\usepackage[T1]{fontenc}
\usepackage{lmodern}

\setbeamertemplate{headline}{
	\begin{beamercolorbox}[wd=\paperwidth,ht=2.5ex,dp=1.125ex]{section in head/foot}%
		\hspace{3ex}{\insertsectionhead}
	\end{beamercolorbox}
	%	\begin{beamercolorbox}[ht=2.5ex,dp=1.125ex,leftskip=.3cm,rightskip=.3cm plus1fil]{subsection in head/foot}
	%		\usebeamerfont{subsection in head/foot}\insertsubsectionhead
	%\end{beamercolorbox}
}%删除点
\begin{document}
\begin{frame}
	\titlepage
\end{frame}
\begin{frame}[fragile]
\begin{short}
Setting:
\begin{itemize}
	\item $A/\mathbb{C}$: an abelian variety of dim $n$
	\item $Z \subset A$: a (nondegenerate) subvariety of dim $r$

\ignore{$Z$ is a curve $C$ in our talk.}
	
\end{itemize}
\end{short}

%\begin{short}
%Goal:
%\begin{itemize}
%	\item Construct a family of subvarieties in $A$.
%	\item Find their dimension and homology class.
%	
%\end{itemize}
%\end{short}

\begin{goal}
\begin{itemize}
	\item Construct a family of subvarieties in $A$.
	\item Find their dimension and homology class.
	
\end{itemize}
\end{goal}
\end{frame}

\begin{frame}[fragile]
\begin{eg}[Jacobian case]
When $C$ is a smooth projective curve over $\mathbb{C}$ of genus $g \geqslant 2$, 
\begin{equation*}
\begin{aligned}
 A:=\Jac(C) &\qquad\qquad \text{the Jacobian of $C$}  \\ 
 \AJ_C:C \hookrightarrow A &\qquad\qquad \text{Abel--Jacobi map}
\end{aligned}
\end{equation*}
\end{eg}

\begin{eg}[Prym case]
When $h:C \longrightarrow C'$ is an unramified double cover of smooth projective curves, we can define
\begin{equation*}
\begin{aligned}
 A:=\Prym(C/C') &\qquad\qquad \text{the Prym variety of $h$}  \\ 
 \AP_{C/C'}:C \longrightarrow A &\qquad\qquad \text{Abel--Prym map}
\end{aligned}
\end{equation*}
We need to assume $C$ is non-hyperelliptic so that $\AP_{C/C'}$ is injective.
\end{eg}
\end{frame}


\begin{frame}[fragile]{Construct new subvarieties}
Since $A$ has addition structure, one defines
\begin{equation*}
\begin{aligned}
  C+C:=\;& \left\{ p+q \,\middle|\, p,q \in C  \right\} &&\subseteq A  \\ 
  2C:=\;& \left\{ 2p \,\middle|\, p \in C  \right\} &&\subseteq A  \\ 
\end{aligned}
\end{equation*}
and so on.

\begin{remark}
Since $C$ is nondegenerate,
$$\underbrace{C + C + \cdots + C}_{\text{$\geqslant n$ many}} = A.$$
\end{remark}

\end{frame}

\begin{frame}[fragile]{Construct new subvarieties}
\begin{question}
Can we define a family of subvarieties
$$\left\{ m_1C+ \cdots + m_dC \subseteq A  \;\middle|\; m_1,\ldots,m_k \in \mathbb{Z} \right\}$$
more respect to the addition structures?

\ignore{They should not be $A$.}
\end{question}
In fact, we can construct a family of subvarieties
$$\left\{ Z_{\chi} \subseteq A  \;\middle|\; \chi \in \mathbb{Z}^d \right\}$$
via the conormal variety.
\end{frame}

\begin{frame}[fragile]{Conic Lagrangian cycle}
For a (smooth) subvariety $Z \subset A$, one can define the conormal variety $\Lambda_Z \subset T^*A \cong A \times T_0^*A$  by 
$$\Lambda_Z: = \left\{ (p,\xi) \in T^*A  \;\middle|\; \xi|_{T_p^* Z} = 0 \right\}.$$ 

\begin{facts}
\begin{itemize}
	\item $\Lambda_Z$ is a conic Lagrangian cycle in $T^*A$;
	\item We have one-to-one correspondence
% https://q.uiver.app/#q=WzAsNixbMCwwLCJcXGxlZnRcXHsgXFx0ZXh0e2lyciBjb25pYyBMYWdyYW5naWFuIGN5Y2xlcyBpbiAkVF4qXFwhQSR9IFxccmlnaHRcXH0iXSxbMiwwLCJcXGxlZnRcXHsgXFx0ZXh0e2lyciBzdWJ2YXJpZXRpZXMgaW4gJEEkfSBcXHJpZ2h0XFx9Il0sWzEsMCwiXFxjb25nIl0sWzEsMSwiXFxsb25nbGVmdHJpZ2h0YXJyb3ciXSxbMCwxLCJcXExhbWJkYV9aIl0sWzIsMSwiWiJdXQ==
\[\begin{tikzcd}[ampersand replacement=\&,column sep={-4mm}, row sep={-1mm}]
	{\left\{ \text{irr conic Lagrangian cycles in $T^*\!A$} \right\}} \& \cong \& {\left\{ \text{irr subvarieties in $A$} \right\}} \\
	{\Lambda_Z} \& \longleftrightarrow \& Z
\end{tikzcd}\]
	\item The map $\gamma_Z: \Lambda_Z \subset A \times T_0^*A \longrightarrow T_0^*A $ is a generically finite map, when $Z$ is nondegenerate.
	
\end{itemize}
\end{facts}
\end{frame}
\begingroup
\setlength{\abovedisplayskip}{4pt}
\setlength{\belowdisplayskip}{4pt}
\begin{frame}[fragile]{Family of subvarieties}
\begin{definition}
Fix a general point $\xi_0 \in T_0^*A$, and \ignore{$d:=\deg \gamma_Z$,}
$$\gamma_Z^{-1}(\xi_0):= \left\{ p_1,\ldots,p_d \right\} \subset Z.$$
Denote $\Lambda_Z^{\univ}$ as the irreducible component of 
$$\underbrace{\Lambda_Z \times_{T_0^* A} \cdots \times_{T_0^* A} \Lambda_Z}_{\text{$d$ many}} \subset A \times \cdots \times A \times T_0^* A$$
containing the point $(p_1,\ldots,p_d, \xi_0)$.
\\[3mm]

For $\chi=(m_1,\ldots,m_d) \in \mathbb{Z}^d$, define $\Lambda_{Z_{\chi}}:= f(\Lambda_Z^{\univ})$, where 
% https://q.uiver.app/#q=WzAsNSxbMSwxLCIocV8xLFxcbGRvdHMscV9kLFxceGkpIl0sWzIsMSwiXFxsZWZ0KCBcXHN1bV9pIG1faXFfaSwgXFx4aSBcXHJpZ2h0KSJdLFsxLDAsIkEgXFx0aW1lcyBcXGNkb3RzIFxcdGltZXMgQSBcXHRpbWVzIFRfMF4qIEEiXSxbMiwwLCJBIFxcdGltZXMgVF8wXiogQSJdLFswLDAsImY6Il0sWzIsM10sWzAsMSwiIiwwLHsic3R5bGUiOnsidGFpbCI6eyJuYW1lIjoibWFwcyB0byJ9fX1dXQ==
\[\begin{tikzcd}[ampersand replacement=\&,column sep={between origins, 40mm},row sep={-1mm}]
	{f:} \&[-21mm] {A \times \cdots \times A \times T_0^* A} \& {A \times T_0^* A} \\
	\& {(q_1,\ldots,q_d,\xi)} \& {\left( \sum_i m_iq_i, \xi \right)}
	\arrow[from=1-2, to=1-3]
	\arrow[maps to, from=2-2, to=2-3]
\end{tikzcd}\]
$Z_{\chi}$ is then the corresponding subvariety of $\Lambda_{Z_{\chi}}$.
\end{definition}
\end{frame}
\endgroup

\begin{frame}[fragile]{Our work}
We determine $\dim Z_{\chi}$ and $[Z_{\chi}] \in H_*(A;\mathbb{Z})$ in special cases.
\begin{eg}
In the Jacobian case, $d=\deg \gamma_C = 2g-2$.

Assume that $C$ is non-hyperelliptic. For $\chi=(m_1,\ldots,m_d) \in \mathbb{Z}^d$, when no $g$ of $m_i$ equal to each other, we get 
\begin{equation*}
\begin{aligned}
  &\hspace{10mm}\dim Z_{\chi} =\; g-1 \\ 
  [Z_{\chi}]=\;\frac{1}{\deg f|_{\Lambda_Z^{\univ}}}&\left( \frac{1}{2^{g-1}} \sum_{\sigma \in S_{2g-2}} \prod_{l=1}^{g-1} \left( m_{\sigma(2l-1)}-m_{\sigma(2l)} \right)^2  \right) \cdot [\Theta]
\end{aligned}
\end{equation*}
\end{eg}
\end{frame}

\begin{frame}[fragile]{Q \& A}
Thank you for your listening! 

Any questions?


\end{frame}
\end{document}


%%%\stress -> \stress


%\begin{equation*}
%\begin{aligned}
%内容...
%\end{aligned}
%\end{equation*}
