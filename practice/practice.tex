\documentclass[UTF8]{ctexart}
   \author{大二新生}
   \title{代数语言的初步认识}
   \usepackage{amsmath}
   \usepackage{amssymb}
   \usepackage{pifont}
   \usepackage[all]{xy}
   \usepackage{mathrsfs}
   \usepackage{tikz}
   \usepackage{hyperref}
   \usepackage{CJK}
    \usetikzlibrary{positioning}
    \tikzset{>=stealth}
    \newcommand{\tikzmark}[3][]
  {\tikz[remember picture, baseline]
    \node [anchor=base,#1](#2) {#3};}

\begin{document}
    \maketitle
    \tableofcontents
本篇毒性较弱,请放心食用!

我们从集合和映射开始入手。

 会不会感觉整个代数方向,几乎所有的东西都是集合 or 映射?他们的普遍性注定了自身的不平凡。集合是静态的,而映射是动态的,之间关系错综复杂。

 (假定已知“$\in$”、“$\notin$”、“$\subseteq$”、“$\varsubsetneqq$”、“$\not\subseteq$”、“$\cap$”、“$\cup$”)

\section{集合}

单独来看集合,我们主要看两点。\ding{172}元素的个数,\ding{173}集合构成的集合,谓之族。

\subsection{定义}

先不考虑集合的完备性,考虑朴素的集合观点,即“袋子”。一个“袋子”用“$\lbrace\ \rbrace$” 表示,如:$A=\lbrace 1,2,3, \varnothing\rbrace$表示$A$ 中有元素。还有一个什么也没有装的袋子,另一种为描述法,如$\lbrace x\in A \big| x \not= 2\rbrace$,竖线左边为“取元素的总范围($A$中)”及“元素在右边的表示方法($x$)”,右边为“元素满足的条件”。

\subsection{集合的势}

“数数”是大多数人小时候就学会的东西。其抽象化,即为“皮亚诺公理”——你的数都是一个个数出来的。来数$\mathbb{N}$ 的元素个数:无限个,和地球上的沙子一样多,还是与天上的星星一样?我不知道,但我知道数学家们数数时就拿$\mathbb{N}$作标准,如果这个集合中有一种方法,使得每个元素都有唯一($\exists !$)的$\mathbb{N}$中的元素一一对应,就说它是可数集。为了让大家听不懂,他们把集合的元素个数称做集合的势。$\mathbb{N}$的势为$\aleph^0$,$\mathbb{R}$的势为$2^{\aleph^0} = \aleph^1$(见后)

\paragraph{小练习}

验证$\mathbb{N}^+$为可数集(见后),$\mathbb{Z}$、$\mathbb{Q}$、$\mathbb{A}$ 为可数集。

已知:$|A|=t \in \mathbb{N}^+$,求$B=\{X \big| X \subseteq A \}$的势。

\subsection{集合构成集合}

看这么一个集合:$$A=\Big\{ \varnothing , \big\{ \varnothing\big\} ,\big\{\{ \varnothing\}\big\} , \big\{ \varnothing , \{ \varnothing\} \big\}\Big\}$$ 糊涂了?其实就是“袋子套袋子”的小游戏。这个集合有四个元素:$ \varnothing , \big\{ \varnothing\big\} ,\big\{\{ \varnothing\}\big\} , \big\{ \varnothing , \{ \varnothing\} \big\}$;其中一个是空袋子;一个是套着空袋子的袋子,等等。若$A=\big\{ \{0\} , 2 \big\}$,则$0 \notin A$,因为0 不在“”袋子里”,不是它的元素,就算装着0的袋子是$A$的元素,即$ \{0\} \in A$。

已知$A=\{ 0, 1\}$,写出$B=\{X \big| X\varsubsetneqq A \}$、$C=\{Y \big| Y \subseteq B \}$。

这么着,看这个:
$$
A=\Big\{ \varnothing , \big\{ \varnothing\big\} , \big\{ \varnothing , \{ \varnothing\} \big\} , \big\{ \varnothing ,\{ \varnothing\} , \{ \varnothing , \{ \varnothing\}\} \big\} \Big\}
$$

有这个: $x\in A \Rightarrow x \subseteq A$,有意思吧!

构造一个集合$A$满足$|A|=5$ 且$x\in A \Rightarrow x \subseteq A$,也不再那么的困难了。

\subsection{补充}

构造集合的方式:

人们常说的“数”,即数学家们习惯于操纵的“对象”,已经不断扩充致复数域:
$$
\xymatrix{
 & & & & & \mathbb{A} \ar[rd] & \\
\varnothing \ar[r]^{\exists} & \lbrace 0 \rbrace \ar[r]^{+ \ \times} & \mathbb{N} \ar[r]^{-} & \mathbb{Z} \ar[r]^{\div} & \mathbb{Q} \ar[r]^{lim} \ar[ru] & \mathbb{R} \ar[r]^{root} & \mathbb{C}
}
$$
这其中每一步都成为后人不断发展的源泉,作为读者,应对各个扩充(后两个除外)有一些直观的了解。

另外,还可以通过取笛卡尔积(其中$(a,b)=(c,d) \Leftrightarrow a=c,c=d$)
$$
A \times B= \big\{(a,b)\big|a\in A,b\in B \big\}
$$

取商模掉等价类($A/_\sim$)

\section{映射}

接着来看映射,即两组集合间的关系。

\subsection{表示方式}

看些例子。
\begin{center}
    \begin{tabular}{rcl}
   “加1映射” $\mathscr{A}:\mathbb{N}$&$\rightarrow$&$\mathbb{N}$  \\
    $i$&$\mapsto$ &$i+1$ \\
    ($\mathscr{A}(i)$&=&$i+1$ )
    \end{tabular}
\end{center}

高中学的大部分函数。因为这里不知道什么映射方式,记为$f(x)$。
\begin{center}
    \begin{tabular}{rcl}
    $f:D \subseteq \mathbb{R}$&$\rightarrow$&$\mathbb{R}$  \\
    $x$ &$\mapsto$&$f(x)$
    \end{tabular}
\end{center}

来看一下对此的解释。
\begin{center}
        \begin{tabular}{crcl}
            映射名称&定义域&&映到的可能范围    \\
            \rotatebox[origin=c]{90}{$\Rsh$}&$\mathscr{A}:\mathbb{N}$&$\rightarrow$&$\mathbb{N}$  \\
            &i&$\mapsto$ &i+1
        \end{tabular}
\end{center}

也可以举例:
\begin{center}
    \begin{tabular}{rcl}
    $\mathscr{A}:\{ 0,1,2 \}$&$\rightarrow$&$\{ 0,1,2 \}$  \\
    0 &$\mapsto$&1\\
    1 &$\mapsto$&0\\
    2 &$\mapsto$&1
    \end{tabular}
\end{center}

注意符号:
单射(\rotatebox[origin=c]{180}{$\hookleftarrow$})、满射($\twoheadrightarrow$)、一一映射($\leftrightarrow$)。

\subsection{映射运算}

映射的复合:
$$
\xymatrix @R=0.2pc{
**[l]g \circ f:A \ar[r]^-{f} & B \ar[r]^{g} & C \\
x \ar @{|->}[r] & f(x) \ar @{|->}[r]& g\big(f(x)\big)
}
$$

如果这两组集合是同一个,那自然可以有多次映射,即映射的幂次。
$$
\xymatrix @C=2pc@R=0.2pc{
**[l]\mathscr{A}^3:\mathbb{N} \ar[r]^-{\mathscr{A}} & \mathbb{N} \ar[r]^{\mathscr{A}} & \mathbb{N} \ar[r]^{\mathscr{A}} & \mathbb{N} \\
i \ar @{|->}[r]& i+1 \ar @{|->}[r]& (i+1)+1 \ar @{|->}[r]& i+3
}
$$

\section{集合与映射}

\subsection{补充}

补充关于群、环、域的知识。(希望去看Artin,比较详细)

\begin{tabular}{lll}
集合&$A$&$(A)$\\
群&$G$&$(G,\circ)$\\
环&$R$&$(R,+,-,\times ,1)$\\
域&$F$&$(F,+,-,\times,\div)$\\
R-模&$M$&$(M,+,·)$\\
线性空间&$V$&$(V,+,·)$
\end{tabular}

\subsection{映射构成的集合}

映射可以构成集合。(其实不止)而有时,我们会选择满足一定条件的映射构成集合。

\begin{tabular}{l l}
$f:A \rightarrow B$&所有的$f$构成集合$A^B$。\\

$\mathscr{A}:\mathbb{F}^m\rightarrow\mathbb{F}^n$&所有的线性映射构成集合。\\

$f:A \leftrightarrow A$&所有的一一映射构成集合。
\end{tabular}

以上的这种集合可能出现在另一个映射的定义域或者值域中。

抽象群$G \rightarrow GL(V)$ 群表示

\subsection{代数结构}

一个集合自身到自身的映射,可以看成集合自身的性质(比较好的 or 映射满足一些性质的)
(有时是自身的笛卡尔积)

从此,集合$\xrightarrow{\text{升级}}$代数结构,集合的各元素之间开始产生了一些关系,集合动起来了。

而这些些集合的子集,由于满足了一定的运算封闭性,自然满足代数结构的运算律,成为了子群、子环、子空间、子模。
子集至原集合有一个自然的嵌入映射:
\begin{center}
    \begin{tabular}{rcl}
    $i:A \subseteq B$&$\rotatebox[origin=c]{180}{$\hookleftarrow$}$&$B$  \\
    $a$ &$\mapsto$&$a$
    \end{tabular}
\end{center}

子结构往往容易构造(生成元生成),不需要多余的$\exists !$性的定理,故遇到它、利用它往往是幸福的、简单的。就仿佛已经搭好了框架,在大框架内部处理比较轻松愉快。

然而新的东西、构造、外推、扩充往往是困难的。
为愚蠢的我们着想,在许多时候(一般大一时),数学家已经给出了一个大框架,如$\mathbb{C}$:这货对加减乘除封闭,非常数多项式还一定有根,可以很方便的处理问题啦!

\subsection{等价}

什么是等价?天上有层云、层积云、雨层云、雾、高积云、高层云、卷云、卷层云,它们都是云。有钢笔、圆珠笔、橡皮、水彩笔、圆规,它们都属于文具,但当谈到笔时,橡皮和圆规就和其他的不一样。水彩笔也有各个色系、各种品牌的。一样事物,拥有数不清的信息,但是我们真正需求的信息并不多。单无视部分差异,两种物品无本质差别时,我们就称它们是“等价的”。等价的东西放在一起就称为一个“等价类”。

跑跑题,先来看特征函数。

设$A\subseteq K$

\begin{center}
    \begin{tabular}{rcl}
    $f:K$&$\rightarrow$&$\{ 0, 1\}$  \\
    $a$ &$\mapsto$&1$\ $若$a \in A$  \\
    $a$ &$\mapsto$&0$\ $若$a \notin A$
    \end{tabular}
\end{center}

一个子集唯一定义了一个特征函数,你可以看得到$K$的子集与特征函数的一一对应。

等价关系:
\begin{center}
    \begin{tabular}{rcl}
    $f:A \times A$&$\rightarrow$&$\{ 0, 1\}$  \\
    $(a,b)$&$\mapsto$&$f(a,b)$\\
    \end{tabular}
\end{center}
\begin{center}
$f(a,b)=1 \Leftrightarrow a \sim b$
\end{center}
满足:
\begin{center}
    \begin{tabular}{|l|l|l|}
    \hline
自反性&$f(a,a)=1$&$a \sim a$\\
\hline
对称性&$f(a,b)=f(b,a)$&$a \sim b\Rightarrow b \sim a$\\
\hline
传递性&$f(a,b)=f(b,c)=1 \Rightarrow f(a,c)=1$&$a \sim b,b \sim c \Rightarrow a \sim c$\\
\hline
    \end{tabular}
\end{center}

等价关系确定了一个等价类,等号为最精确的等价类。

一般而言,在同一个等价类中的两个数有一些共同的性质,换言之,它们的本质是一样的。

很容易看出:($X$、$Y$的元素才是本质)
\begin{equation}
\begin{split}
f^{-1} (X \cap Y)=f^{-1} (X) \cap f^{-1} (Y)\\
f^{-1} (X \cup Y)=f^{-1} (X) \cup f^{-1} (Y)
\end{split}
\end{equation}

等价类构成了一个集合$\overline{A}$。关于这个集合,有一个自然的映射关系(是满射!):
\begin{center}
    \begin{tabular}{rcl}
    $\pi:A$&$\twoheadrightarrow$&$\overline{A}$  \\
    $a$&$\mapsto$&$\overline{a}$
    \end{tabular}
\end{center}

下面来看一个一般映射所诱导出来的等价关系。
\begin{center}
    \begin{tabular}{rcl}
    $f:A$&$\rightarrow$&$B$  \\
    a &$\mapsto$&$f(a)$
    \end{tabular}
\end{center}

考虑定义:$a \sim b \Leftrightarrow f(a) = f(b)$。例如:$f$为取余数的映射。这样做的原因是:如果我们关心的是$f$映射后得到的结果,那么是$a$是$b$ 就变的不再那么重要。

这个等价关系诱导出一个单射$\Rightarrow$化为一一映射
$$
\xymatrix @=4pc{
A \ar[r]^{f} \ar @{->>}[d]_{\pi}& B \\
\overline{A} \ar @{_{(}->}[ru]^{\exists !}_{f'} \ar @{<->}[r]_{\overline{a}} & im f \ar@{_{(}->}[u]_{\text{包含}}
}
$$

当$f$为群同态、环同态、模同态、线性映射时,此即第一同构定理。(事实上,集合的同态即为一一对应)

有了等价类我们自然会关心以下问题:
\ding{172}等价类的个数?
\ding{173}两元素是否等价?

对于\ding{172},$B$集合往往易知,此时只要对$B$集合进行分析即可。有时从每个等价类中各取出一个特殊、容易处理的元素,(有意而为之,称为标准型)处理问题就以其为模板。

反过来,当有了等价关系后,我们下意识就会去尝试去找其所隐含的函数的显式表达。找到后即可宣称\ding{173}问题的解决。

来放一个小毒:
在$C[0,1]$(定义在$[0,1]$上的连续实值函数)上定义等价关系
\begin{center}
$f \sim g \Leftrightarrow \big( \exists\ \ \delta > 0 \ \ s.t.$对$\ \ \forall x \in (a- \delta, a+ \delta)$,有$f(x)=g(x) \big)$
\end{center}

该等价类是一个含幺交换环,称作连续函数芽环。目前我还没有见到所隐含的函数的显式表达。这是曲助教放的毒……

\subsection{应用:作用}

一直在大谈特谈理论性的东西,我们来看上述的几个实例吧:

先来看笛卡尔积自然包含的映射:

$$
\xymatrix{
A \ar@{_{(}->}[rd]_{i}&  & A\\
 & A \times B \ar @{->>}[ru]_{p} \ar @{->>}[rd]_{p'}& \\
B \ar@{_{(}->}[ru]_{i} &  & B
}
$$

其中$i$、$i'$是嵌入,$p$、$p'$是投影。注意:$p \circ i = id_{A}$,而一般情况下,$i \circ p \not= id_{A \times B}$。

来看“二元运算”作为定义域的映射。

\begin{center}
    \begin{tabular}{rcl}
    $f:A \times B$&$\rightarrow$&$C$  \\
    $(a,b)$ &$\mapsto$&$ab$
    \end{tabular}
\end{center}

固定$a \in A$,此映射诱导出一个新的映射:

\begin{center}
    \begin{tabular}{rcl}
    $\varphi_a:B$&$\rightarrow$&$C$  \\
    $b$ &$\mapsto$&$ab$
    \end{tabular}
\end{center}

$\varphi_A:=\big\{\varphi_a \big| a \in A \big\}$构成一个新的集合。自然有同构:$A \leftrightarrow \varphi_A$,映$a \mapsto \varphi_a$。

当$C=B$时,即可以称作$A$对集合$B$的作用。

单$A=G$为群时,若满足:\ding{172}(单位元)$\varphi_1 =id_B$
\ding{173}(结合律),则称之为群$G$对集合$B$的作用。

\section{后记}

本小说历时5日,现在终于烂尾了,也算是完成我学\LaTeX 的一个小练笔吧。说明自己对数学的深入理解还有很长的一段路要走。

向我的各位数学助教致敬。没有你们,我不会了解到这样丰富有情趣的小世界。

另外,文中必有不正确,欠妥当的地方,请不吝赐教。

之后补充作用部分;摘要;加参考文献;加图片;对符号总览分栏。

\begin{appendix}
\section{符号总览}

\begin{tabular}{c  l}
    \ding{172}&项目列表\\

    $\exists !$&存在唯一 \\

    $\Rightarrow$&推出\\

    $\Leftrightarrow$&互推\\

    $\varnothing$&空集 \\

    $x \in A$&属于 \\

    $x \notin A$&不属于 \\

    $X \subseteq A$&子集 \\

    $X \not\subseteq A$&非子集\\

    $X \not\subseteq A$&真子集\\

    $A \cap B$&集合的交\\

    $A \cup B$&集合的并\\

    $A \sqcup B$&集合的无交并\\

    $A \bigotimes B$&笛卡尔积\\
\end{tabular}
\begin{tabular}{c  l}
    $\rightarrow$&一般映射\\

    $\mapsto$&一般映射2\\

    \rotatebox[origin=c]{180}{$\hookleftarrow$}& 单射\\

    $\twoheadrightarrow$& 满射\\

    $\leftrightarrow$&一一映射\\

    $\circ$&复合运算\\

    $\mathbb{N}$&字体1\\

    $\mathscr{A}$&字体2\\

    $a \sim b$&等价关系\\

    $\overline{a}$&等价类\\

    $\times$&乘法\\

    $\div$&除法\\

    $\pi$&圆周率(此刻用作典范映射)\\

    $\varepsilon - \delta$&语言
\end{tabular}

\end{appendix}
\end{document}
