
\documentclass[UTF8]{amsart}
%Typical documenttypes: article/book
%some examples:
%\documentclass[reqno,11pt]{book}   %%%for books
%\documentclass[]{minimal}			%%%for Minimal Working Example


%for beamers, you have to change a lot. Especially, remove the package enumitem!!!



%%%%%%%%%%%%%%%%%%%% setting for fast compiling

%\special{dvipdfmx:config z 0}		% no compression

\includeonly{chapters/chapter9}		% In practice, use an empty document called "chapter9"	% usually for printing books






%%%%%%%%%%%%%%%%%%%% here we include packages

%%%basic packages for math articles
\usepackage{amssymb}
\usepackage{amsthm}
\usepackage{amsmath}
\usepackage{amsfonts}
\usepackage[shortlabels]{enumitem}	% It supersedes both enumerate and mdwlist. The package option shortlabels is included to configure the labels like in enumerate.

%%%packages for special symbols
\usepackage{pifont}					% Access to PostScript standard Symbol and Dingbats fonts
\usepackage{wasysym}				% additional characters
\usepackage{bm}						% bold fonts: \bm{...}
\usepackage{extarrows}				% may be replaced by tikz-cd
%\usepackage{unicode-math}			% unicode maths for math fonts, now I don't know how to include it
%\usepackage{ctex}					% Chinese characters, huge difference.


%%%basic packages for fancy electronic documents
\usepackage[colorlinks]{hyperref}
\usepackage[table,hyperref]{xcolor} 			% before tikz-cd. 
%\usepackage[table,hyperref,monochrome]{xcolor}	% disable colored output (black and white)

%%%packages for figures and tables (general setting)
\usepackage{float}				%Improved interface for floating objects
\usepackage{caption,subcaption}
\usepackage{adjustbox}			% for me it is usually used in tables 
\usepackage{stackengine}		%baseline changes

%%%packages for commutative diagrams
\usepackage{tikz-cd}
%\usepackage{quiver}			% see https://q.uiver.app/

%%%packages for pictures
\usepackage[width=0.5,tiewidth=0.7]{strands}
\usepackage{graphicx}			% Enhanced support for graphics

%%%packages for tables and general settings
\usepackage{array}
\usepackage{makecell}
\usepackage{multicol}
\usepackage{multirow}
\usepackage{diagbox}
\usepackage{longtable}

%%%packages for ToC, LoF and LoT







 %https://tex.stackexchange.com/questions/58852/possible-incompatibility-with-enumitem










%%%%%%%%%%%%%%%%%%%% here we include theoremstyles

\numberwithin{equation}{section}

\theoremstyle{plain}
\newtheorem{theorem}{Theorem}[section]

\newtheorem{setting}[theorem]{Setting}
\newtheorem{definition}[theorem]{Definition}
\newtheorem{lemma}[theorem]{Lemma}
\newtheorem{proposition}[theorem]{Proposition}
\newtheorem{corollary}[theorem]{Corollary}
\newtheorem{conjecture}[theorem]{Conjecture}

\newtheorem{claim}[theorem]{Claim}
\newtheorem{eg}[theorem]{Example}
\newtheorem{ex}[theorem]{Exercise}
\newtheorem{fact}[theorem]{Fact}
\newtheorem{ques}[theorem]{Question}
\newtheorem{warning}[theorem]{Warning}



\newtheorem*{bbox}{Black box}
\newtheorem*{notation}{Conventions and Notations}


\numberwithin{equation}{section}


\theoremstyle{remark}

\newtheorem{remark}[theorem]{Remark}
\newtheorem*{remarks}{Remarks}

%%% for important theorems
%\newtheoremstyle{theoremletter}{4mm}{1mm}{\itshape}{ }{\bfseries}{}{ }{}
%\theoremstyle{theoremletter}
%\newtheorem{theoremA}{Theorem}
%\renewcommand{\thetheoremA}{A}
%\newtheorem{theoremB}{Theorem}
%\renewcommand{\thetheoremB}{B}







%%%%%%%%%%%%%%%%%%%% here we declare some symbols

%%%%%%%DeclareMathOperator
%see here for why newcommand is better for DeclareMathOperator: https://tex.stackexchange.com/questions/67506/newcommand-vs-declaremathoperator

%%%%%basic symbols. Keep them!

%%%symbols for sets and maps
\DeclareMathOperator{\pt}{\operatorname{pt}}	%points. Other possibilities are \{pt\}, \{*\}, pt, * ...
\DeclareMathOperator{\Id}{\operatorname{Id}}	%identity in groups.
\DeclareMathOperator{\Img}{\operatorname{Im}}

\DeclareMathOperator{\Ob}{\operatorname{Ob}}
\DeclareMathOperator{\Mor}{\operatorname{Mor}}	%difference of Mor and Hom: Hom is usually for abelian categories
\DeclareMathOperator{\Hom}{\operatorname{Hom}}	\DeclareMathOperator{\End}{\operatorname{End}}
\DeclareMathOperator{\Aut}{\operatorname{Aut}}

%%%symbols for linear algebras and 
%%linear algebras
\DeclareMathOperator{\tr}{\operatorname{tr}}
\DeclareMathOperator{\diag}{\operatorname{diag}}	%for diagonal matrices

%%abstract algebras
\DeclareMathOperator{\ord}{\operatorname{ord}}
\DeclareMathOperator{\gr}{\operatorname{gr}}
\DeclareMathOperator{\Frac}{\operatorname{Frac}}

%%%symbols for basic geometries
\DeclareMathOperator{\vol}{\operatorname{vol}}	%volume
\DeclareMathOperator{\dist}{\operatorname{dist}}
\DeclareMathOperator{\supp}{\operatorname{supp}}

%%%symbols for category
%%names of categories
\DeclareMathOperator{\Mod}{\operatorname{Mod}}
\DeclareMathOperator{\Vect}{\operatorname{Vect}}
\DeclareMathOperator{\rep}{\operatorname{rep}} %usually rep means the category of finite dimensional representations, while Rep means the category of representations.
\DeclareMathOperator{\Rep}{\operatorname{Rep}}


%%%symbols for homological algebras
\DeclareMathOperator{\Tor}{\operatorname{Tor}}
\DeclareMathOperator{\Ext}{\operatorname{Ext}}
\DeclareMathOperator{\gldim}{\operatorname{gl.dim}}
\DeclareMathOperator{\projdim}{\operatorname{proj.dim}}
\DeclareMathOperator{\injdim}{\operatorname{inj.dim}}
\DeclareMathOperator{\rad}{\operatorname{rad}}


%%%symbols for algebraic groups
\DeclareMathOperator{\GL}{\operatorname{GL}}
\DeclareMathOperator{\SL}{\operatorname{SL}}

%%%symbols for typical varieties
\DeclareMathOperator{\Gr}{\operatorname{Gr}}
\DeclareMathOperator{\Flag}{\operatorname{Flag}}

%%%symbols for basic algebraic geometry
\DeclareMathOperator{\Spec}{\operatorname{Spec}}
\DeclareMathOperator{\Coh}{\operatorname{Coh}}
\newcommand{\Dcoh}{\mathcal{D}_{\operatorname{Coh}}}%%%This one shows the difference between \DeclareMathOperator and \newcommand
\DeclareMathOperator{\Pic}{\operatorname{Pic}}
\DeclareMathOperator{\Jac}{\operatorname{Jac}}

%%%%%advanced symbols. Choose the part you need!

%%%symbols for algebraic representation theory
\DeclareMathOperator{\Irr}{\operatorname{Irr}}
\DeclareMathOperator{\ind}{\operatorname{ind}}	%\ind(Q) means the set of  equivalence classes of finite dimensional indecomposable representations
\DeclareMathOperator{\Res}{\operatorname{Res}}
\DeclareMathOperator{\Ind}{\operatorname{Ind}}
\DeclareMathOperator{\cInd}{\operatorname{c-Ind}}


%%%symbols for algebraic topology
\DeclareMathOperator{\EGG}{\operatorname{E}\!}
\DeclareMathOperator{\BGG}{\operatorname{B}\!}

\DeclareMathOperator{\chern}{\operatorname{ch}^{*}}
\DeclareMathOperator{\Td}{\operatorname{Td}}
\DeclareMathOperator{\AS}{\operatorname{AS}}	%Atiyah--Segal completion theorem 

%%%symbols for Auslander--Reiten theory 
\DeclareMathOperator{\Modup}{\overline{\operatorname{mod}}}
\DeclareMathOperator{\Moddown}{\underline{\operatorname{mod}}}
\DeclareMathOperator{\Homup}{\overline{\operatorname{Hom}}}
\DeclareMathOperator{\Homdown}{\underline{\operatorname{Hom}}}


%%%symbols for operad
\DeclareMathOperator{\Com}{\operatorname{\mathcal{C}om}}
\DeclareMathOperator{\Ass}{\operatorname{\mathcal{A}ss}}
\DeclareMathOperator{\Lie}{\operatorname{\mathcal{L}ie}}
\DeclareMathOperator{\calEnd}{\operatorname{\mathcal{E}nd}} %cal=\mathcal


%%%%%personal symbols. Use at your own risk!

%%%symbols only for master thesis
\DeclareMathOperator{\ptt}{\operatorname{par}}	%the partition map
\DeclareMathOperator{\str}{\operatorname{str}}	%strict case
\DeclareMathOperator{\RRep}{\widetilde{\operatorname{Rep}}}
\DeclareMathOperator{\Rpt}{\operatorname{R}}
\DeclareMathOperator{\Rptc}{\operatorname{\mathcal{R}}}
\DeclareMathOperator{\Spt}{\operatorname{S}}
\DeclareMathOperator{\Sptc}{\operatorname{\mathcal{S}}}
\DeclareMathOperator{\Kcurl}{\operatorname{\mathcal{K}}}
\DeclareMathOperator{\Hcurl}{\operatorname{\mathcal{H}}}
\DeclareMathOperator{\eu}{\operatorname{eu}}
\DeclareMathOperator{\Eu}{\operatorname{Eu}}
\DeclareMathOperator{\dimv}{\operatorname{\underline{\mathbf{dim}}}}
\DeclareMathOperator{\St}{\mathcal{Z}}

%%%%%symbols which haven't been classified. Add your own math operators here!


\DeclareMathOperator{\Modr}{\operatorname{-Mod}}





%%%%%%%newcommand

%%%basic symbols
\newcommand{\norm}[1]{\Vert{#1}\Vert}

%%%symbols only for master thesis
\newcommand{\dimvec}[1]{\mathbf{#1}}
\newcommand{\abdimvec}[1]{|\dimvec{#1}|}
\newcommand{\ftdimvec}[1]{\underline{\dimvec{#1}}}

\newcommand{\absgp}[1]{\mathbb{#1}}
\newcommand{\WWd}{\absgp{W}_{\abdimvec{d}}}
\newcommand{\Wd}{W_{\dimvec{d}}}
\newcommand{\MinWd}{\operatorname{Min}(\absgp{W}_{\abdimvec{d}},W_{\dimvec{d}})}
\newcommand{\Compd}{\operatorname{Comp}_{\dimvec{d}}}
\newcommand{\Shuffled}{\operatorname{Shuffle}_{\dimvec{d}}}

\newcommand{\Omcell}{\Omega}
\newcommand{\OOmcell}{\boldsymbol{\Omega}}
\newcommand{\Vcell}{\mathcal{V}}
\newcommand{\VVcell}{\boldsymbol{\mathcal{V}}}
\newcommand{\Ocell}{\mathcal{O}}
\newcommand{\OOcell}{\boldsymbol{\mathcal{O}}}
\newcommand{\preimage}[1]{\widetilde{#1}}
\newcommand{\orde}{\operatorname{ord}_e}
\newcommand{\fakestar}{*}

%as the subscription of Hom
\newcommand{\Alggp}{\text{-Alg gp}}







%%%%%%%%%%%%%%%%%%%% here we make some blocks for special features. 

%%%% todo notes %%%%
\usepackage[colorinlistoftodos,textsize=footnotesize]{todonotes}
\setlength{\marginparwidth}{2.5cm}
\newcommand{\leftnote}[1]{\reversemarginpar\marginnote{\footnotesize #1}}
\newcommand{\rightnote}[1]{\normalmarginpar\marginnote{\footnotesize #1}\reversemarginpar}









%%%%%%%%%%%%%%%%%%%% here we make some global settings. Understand everything here before you make a document!

\usepackage[a4paper,left=3cm,right=3cm,bottom=4cm]{geometry}
\usepackage{indentfirst}	% Indent first paragraph after section header

\setcounter{tocdepth}{2}


%https://latexref.xyz/_005cparindent-_0026-_005cparskip.html
\setlength{\parindent}{15pt}	
\setlength{\parskip}{0pt plus1pt}

%\setlength\intextsep{0cm}
%\setlength\textfloatsep{0cm}
\def\arraystretch{1}
%\setcounter{secnumdepth}{3}

\allowdisplaybreaks


\begin{document}

% The beginning depends on the documentclass. Rewrite this part if you use different documentclass!
\date{\today}

\title
{\LaTeX\;Template
}
\author{Xiaoxiang Zhou}
\address{School of Mathematical Sciences\\
University of Bonn\\
Bonn, 53115\\ Germany\\} 
\email{email:xx352229@mail.ustc.edu.cn}


\maketitle
\tableofcontents


\section{Introduction}
This is a document for beginning with ease. Sometimes I felt disturbed by the structures of the \LaTeX\;document. I don't know how to reset the arranges among paragraphs, and some environments crash with each other.

中文尝试


The structure of documents:
\begin{enumerate}
\item document class;
\item packages;
\item symbols, containing math operators and other symbols;
\item global settings;
\item blocks for special features;
\end{enumerate}
\begingroup
\renewcommand{\arraystretch}{1.2}

% https://q.uiver.app/?q=WzAsMTAsWzAsMCwiXFxSZXBfe1xcTGFtYmRhfShLWikiXSxbMCwxLCJcXFJlcF97XFxMYW1iZGF9KEtaKV8wIl0sWzAsMiwiXFxtYXRoY2Fse0J9Il0sWzEsMiwiXFxtYXRoY2Fse0N9Il0sWzAsMywiXFxtYXRoY2Fse0J9XzEiXSxbMSwzLCJcXG1hdGhjYWx7Q31fMSciXSxbMSwwLCJcXFJlcF97XFxMYW1iZGF9KEcpIl0sWzEsMSwiXFxSZXBfe1xcTGFtYmRhfShHKV8wIl0sWzIsMywiXFxNb2QoXFxHYW1tYV97XFxwaV8xfSkiXSxbMywzLCJcXG1hdGhjYWx7Q31fMSJdLFs0LDUsIlxcc2ltIl0sWzIsMywiXFxzaW0gXFx0ZXh0e2ZvciBmLmwufSJdLFsxLDddLFswLDYsIlxcY0luZF97S1p9XntHfSJdLFsxLDAsIlxcc3Vic2V0IiwxLHsic3R5bGUiOnsiYm9keSI6eyJuYW1lIjoibm9uZSJ9LCJoZWFkIjp7Im5hbWUiOiJub25lIn19fV0sWzIsMSwiXFxzdWJzZXQiLDEseyJzdHlsZSI6eyJib2R5Ijp7Im5hbWUiOiJub25lIn0sImhlYWQiOnsibmFtZSI6Im5vbmUifX19XSxbMyw3LCJcXHN1YnNldCIsMSx7InN0eWxlIjp7ImJvZHkiOnsibmFtZSI6Im5vbmUifSwiaGVhZCI6eyJuYW1lIjoibm9uZSJ9fX1dLFs3LDYsIlxcc3Vic2V0IiwxLHsic3R5bGUiOnsiYm9keSI6eyJuYW1lIjoibm9uZSJ9LCJoZWFkIjp7Im5hbWUiOiJub25lIn19fV0sWzQsMiwiXFxzdWJzZXQiLDEseyJzdHlsZSI6eyJib2R5Ijp7Im5hbWUiOiJub25lIn0sImhlYWQiOnsibmFtZSI6Im5vbmUifX19XSxbNSwzLCJcXHN1YnNldCIsMSx7InN0eWxlIjp7ImJvZHkiOnsibmFtZSI6Im5vbmUifSwiaGVhZCI6eyJuYW1lIjoibm9uZSJ9fX1dLFs1LDgsIlxcY29uZyIsMSx7InN0eWxlIjp7ImJvZHkiOnsibmFtZSI6Im5vbmUifSwiaGVhZCI6eyJuYW1lIjoibm9uZSJ9fX1dLFs4LDksIlxcY29uZyIsMSx7InN0eWxlIjp7ImJvZHkiOnsibmFtZSI6Im5vbmUifSwiaGVhZCI6eyJuYW1lIjoibm9uZSJ9fX1dLFs5LDcsIlxcc3Vwc2V0IiwyLHsic3R5bGUiOnsidGFpbCI6eyJuYW1lIjoiaG9vayIsInNpZGUiOiJib3R0b20ifX19XV0=
\[\begin{tikzcd}[column sep={between origins, 12mm},row sep=small]
	{\Rep_{\Lambda}(KZ)} & [37mm]{\Rep_{\Lambda}(G)} \\
	{\Rep_{\Lambda}(KZ)_0} & {\Rep_{\Lambda}(G)_0} &&[7mm]\\
	{\mathcal{B}} & {\mathcal{C}} \\
	{\mathcal{B}_1} & {\mathcal{C}_1'} & {\!\!\End_G(\Pi_1)\Modr} & {\mathcal{C}_1}
	\arrow["\sim", from=4-1, to=4-2]
	\arrow["{\sim \text{ for f.l.}}", from=3-1, to=3-2]
	\arrow[from=2-1, to=2-2]
	\arrow["{\cInd_{KZ}^{G}}", from=1-1, to=1-2]
	\arrow["\subset"{description}, sloped, draw=none, from=2-1, to=1-1]
	\arrow["\subset"{description}, sloped, draw=none, from=3-1, to=2-1]
	\arrow["\subset"{description}, sloped, draw=none, from=3-2, to=2-2]
	\arrow["\subset"{description}, sloped, draw=none, from=2-2, to=1-2]
	\arrow["\subset"{description}, sloped, draw=none, from=4-1, to=3-1]
	\arrow["\subset"{description}, sloped, draw=none, from=4-2, to=3-2]
	\arrow["\cong"{description}, draw=none, from=4-2, to=4-3]
	\arrow["\cong"{description}, draw=none, from=4-3, to=4-4]
	\arrow["\supset"{sloped}, hook', from=4-4, to=2-2]
\end{tikzcd}\]

\begin{table}[ht]
\centering
\[
\begin{array}{c|c|c|c|c|c|c|c}
\hline
M                      & M(1) & M(2) & M(3) & M(4) & M(5) & M(6)  & \href{http://oeis.org}{OEIS} \\ \hline
\Com                   & 1    & 1    & 1    & 1    & 1    & 1     &      \\ \hline
\Ass                   & 1    & 2    & 6    & 24   & 120  & 720   &      \\ \hline
\Lie                   & 1    & 1    & 2    & 6    & 24   & 120   &      \\ \hline
\mathcal{T}(E_{\Com})  & 1    & 1    & 3    & 15   & 105  & 945   &  \href{http://oeis.org/A001147}{A001147}    \\ \hline
\mathcal{T}(E_{\Ass})  & 1    & 2    & 12   & 120  & 1680 & 30240 & \href{http://oeis.org/A001813}{A001813}     \\ \hline
\mathcal{T}(E_{\Lie})  & 1    & 1    & 3    & 15   & 105  & 945   & \href{http://oeis.org/A001147}{A001147}     \\ \hline
\left(R_{\Com}\right)  & 0    & 0    & 2    & 14   & 104  & 944   &      \\ \hline
\left(R_{\Ass}\right)  & 0    & 0    & 6    & 96   & 1560 & 29520 &      \\ \hline
\left(R_{\Ass}\right)  & 0    & 0    & 1    & 9    & 81   & 825   &      \\ \hline
\calEnd_{\mathbb{C}^k} & k^2  & 2k^2 & 3k^2 & 4k^2 & 5k^2 & 6k^2  &      \\ \hline
\Com \circ \Lie        &      &      &      &      &      &       &      \\ \hline
\vdots                 &      &      &      &      &      &       &      \\ \hline
                       &      &      &      &      &      &       &      \\ \hline
                       &      &      &      &      &      &       &      \\ \hline
\end{array}
\]
\end{table}
\endgroup
\section{Examples}
\subsection{Theorem environment}
\begin{theorem}[{see \cite[Theorem 18.5.1]{vakil2017rising}}]
内容...
\end{theorem}

\begin{setting}
内容...
\end{setting}

\begin{definition}
内容...
\end{definition}

\begin{lemma}
内容...
\end{lemma}

\begin{proposition}
内容...
\end{proposition}

\begin{corollary}
内容...
\end{corollary}
\begin{conjecture}
内容...
\end{conjecture}
\begin{claim}
内容...
\end{claim}
\begin{eg}
内容...
\end{eg}
\begin{ex}
内容...
\end{ex}
\begin{fact}
内容...
\end{fact}
\begin{ques}
内容...
\end{ques}
\begin{warning}
内容...
\end{warning}
\begin{bbox}
内容...
\end{bbox}
\begin{notation}
内容...
\end{notation}

\begin{remark}
内容...
\end{remark}

\begin{remarks}\
\begin{enumerate}[1.]
\item ...
\item ...
\end{enumerate}
\end{remarks}

\include{chapters/chapter9}

\nocite{Eberhardt2022Koszul}	% cite articles which are not cited in the document yet

% Remember to protect the uppercase of people's name and LaTeX symbols

\bibliographystyle{plain}
\bibliography{reference}
\end{document}