\documentclass[pdf]{beamer}
\mode<presentation>{
	\usetheme{Ilmenau}
	
}
\usecolortheme{dolphin}
%\usepackage[UTF8,indent]{ctexcap}%中文
\usepackage{amssymb}
\usepackage{amsmath}
\usepackage{amsfonts}
%\usepackage{graphicx}
\usepackage{amsthm}
\usepackage{indentfirst}
\usepackage{enumerate}
\usepackage{extpfeil}
\usepackage{tikz-cd}
\usepackage{longtable}
\usepackage{makecell}
\usepackage{TooYoung}
\usepackage{array}
\usepackage{xcolor}
\usepackage{hyperref}
\usepackage{arydshln}

\usetikzlibrary {calc,positioning,shapes.misc,graphs,decorations.pathreplacing}


\numberwithin{equation}{section}

\theoremstyle{plain}
\newtheorem{proposition}[theorem]{Proposition}
\newtheorem{claim}[theorem]{Claim}
\newtheorem{defn}[theorem]{Definition}
\newtheorem{eg}[theorem]{Example}
\newtheorem{pf}[theorem]{Proof}
\newtheorem{cor}[theorem]{Corollary}

\newtheorem{tabloid}[theorem]{Tabloid: equivalence class of standard filling}


\theoremstyle{plain}
\newtheorem{exercise}{Exercise}[section]


\theoremstyle{remark}
\newtheorem{remark}[theorem]{Remark}
\newtheorem{remarks}{Remarks}
\newtheorem{ex}[theorem]{Exercise}
\newtheorem{question}[theorem]{Questions}
\newtheorem{short}{ }



\newcommand*{\thick}[1]{\text{\boldmath$#1$}}
\newcommand*{\cir}[1]{\;$\ding{19#1}$\;}%临时使用
\newcommand*{\norm}[1]{\lVert#1\rVert}
\newcommand*{\ignore}[1]{\textcolor{lightgray}{#1}}
\newcommand*{\stress}[1]{\textcolor{red}{#1}}
\newcommand*{\bgpicb}[1]{\usebackgroundtemplate{%
	\begin{tikzpicture}[path image/.style={
		path picture={
			\node at (path picture bounding box.center) {
				\includegraphics[height=10cm]{#1}
			};
	}}]
	
	\draw [path image]
	(current page.north west) rectangle
	(current page.south east);
	
	\end{tikzpicture}
}}
\newcommand*{\bgpica}[1]{\usebackgroundtemplate{%
		\begin{tikzpicture}[path image/.style={
			path picture={
				\node at (path picture bounding box.center) {
					\includegraphics[height=7.5cm]{#1}
				};
		}}]
		
		\draw [path image]
		(current page.north west) rectangle
		(current page.south east);
		
		\end{tikzpicture}
}}


\DeclareMathOperator{\supp}{supp}
\DeclareMathOperator{\dist}{dist}
\DeclareMathOperator{\vol}{vol}
\DeclareMathOperator{\diag}{diag}
\DeclareMathOperator{\tr}{tr}
\DeclareMathOperator{\Proj}{\operatorname{Proj}}
\DeclareMathOperator{\Aut}{\operatorname{Aut}}
\DeclareMathOperator{\Img}{\operatorname{Im}}
\DeclareMathOperator{\Sym}{\operatorname{Sym}}
\DeclareMathOperator{\sgn}{\operatorname{sgn}}
\DeclareMathOperator{\Id}{\operatorname{Id}}
\DeclareMathOperator{\ques}{\;?\;}
\DeclareMathOperator{\Fl}{\mathcal{F\ell}}
%%%symbols for algebraic groups
\DeclareMathOperator{\GL}{\operatorname{GL}}
\DeclareMathOperator{\SL}{\operatorname{SL}}
\DeclareMathOperator{\PGL}{\operatorname{PGL}}
%%%symbols for typical varieties
\DeclareMathOperator{\Gr}{\operatorname{Gr}}
\DeclareMathOperator{\Flag}{\operatorname{Flag}}
\DeclareMathOperator{\Hom}{\operatorname{Hom}}
\DeclareMathOperator{\codim}{\operatorname{codim}}
\DeclareMathOperator{\ext}{\operatorname{ext}}
\DeclareMathOperator{\Zar}{\operatorname{Zar}}

%\setlength{\parindent}{1em}
\newcommand{\character}[2]{\left[\begin{array}{c}{#1} \\ {#2}\end{array}\right]}
\newcommand{\normalcharacter}{\character{\epsilon}{\epsilon'}}
\renewcommand{\Fontinbox}[1]{\scriptstyle #1}

\setlength{\abovedisplayskip}{2pt}   % space above the display
\setlength{\belowdisplayskip}{2pt}



\setbeamertemplate{caption}[]
% 设置图形文件的搜索路径
\graphicspath{{figures/}}
\title{Bruhat--Tits building}
\author{Xiaoxiang Zhou}
\institute[HU berlin]{Humboldt-Universität zu Berlin}
\date{\today}
\tikzset{
	invisible/.style={opacity=0,text opacity=0},
	visible on/.style={alt=#1{}{invisible}},
	alt/.code args={<#1>#2#3}{%
		\alt<#1>{\pgfkeysalso{#2}}{\pgfkeysalso{#3}} % \pgfkeysalso doesn't change the path
	},
}


%\setbeamercolor{section number projected}{fg=white!90!blue, bg=red!90!black}
\definecolor{goodblue}{RGB}{71,71,186}
\setbeamercolor{block body}{fg=black,bg=gray!10}
\setbeamercolor{block title}{fg=white, bg=goodblue}
\usefonttheme[onlymath]{serif}
\usepackage[T1]{fontenc}
\usepackage{lmodern}

\setbeamertemplate{headline}{
	\begin{beamercolorbox}[wd=\paperwidth,ht=2.5ex,dp=1.125ex]{section in head/foot}%
		\hspace{3ex}{\insertsectionhead}
	\end{beamercolorbox}
	%	\begin{beamercolorbox}[ht=2.5ex,dp=1.125ex,leftskip=.3cm,rightskip=.3cm plus1fil]{subsection in head/foot}
	%		\usebeamerfont{subsection in head/foot}\insertsubsectionhead
	%\end{beamercolorbox}
}%删除点
\begin{document}
\begin{frame}
	\titlepage
\end{frame}
\begin{frame}[fragile]{Figures of Bruhat--Tits building}
\only<1>{
  \begin{figure}
    \centering
    \includegraphics[height=60mm]{figure/pretty_building.png}
    \caption{$\mathcal{B}_{\SL_3(\mathbb{Q}_p)}$, from Annette Werner's \href{https://www.math-berlin.de/images/stories/bruhat-tits.pdf}{talk}}
  \end{figure}
}
\only<2>{
  \begin{figure}
    \centering
    \href{https://buildings.gallery/}{
    \includegraphics[height=50mm]{figure/gallary_figure.png}}
    \caption{$\mathcal{B}_{\SL_3(\mathbb{Q}_p)}$, from  \href{https://buildings.gallery/}{buildings.gallery}
    }
  \end{figure}
}
\only<3>{
  \begin{figure}
    \centering
    \href{https://buildings.gallery/}{
    \includegraphics[height=50mm]{figure/ABC_1.png}}
    \caption{$\mathcal{B}_{\SL_2(\mathbb{Q}_2)}$
    }
  \end{figure}
}
\only<4>{
  \begin{figure}
    \centering
    \href{https://buildings.gallery/}{
    \includegraphics[height=50mm]{figure/ABC_2.png}}
    \caption{$\mathcal{B}_{\SL_2(\mathbb{Q}_2)}$
    }
  \end{figure}
}		
\end{frame}




%\begin{frame}[fragile]{Standard subgroups}
%
%$$B={\setlength{\arraycolsep}{2pt}\renewcommand{\arraystretch}{0.8}\begin{pmatrix}
%* & * & * \\ &*&*\\&&*
%\end{pmatrix}}
%\qquad
%P = {\setlength{\arraycolsep}{2pt}\renewcommand{\arraystretch}{0.8}
%\left(
%\begin{array}{c;{2pt/2pt}cc}
%    * &  \;\;*\;\;  \\ \hdashline[2pt/2pt]
%      &  \rule{0mm}{5mm}\;\;*\;\;   \\
%\end{array}
%\right)
%}
%\qquad
%T = {\setlength{\arraycolsep}{1.5pt}\renewcommand{\arraystretch}{0.8}\begin{pmatrix}
%* &  &  \\[-2mm] &\ddots&\\[-1mm]&&*
%\end{pmatrix}}
%$$
%
%\begin{equation*}
%\begin{aligned}
%  \GL_n(\kappa)/B=\;& \left\{ \text{ flags of } \kappa^n  \right\} \\
%  =\;& \left\{ V_1 \subset \cdots \subset V_n = \kappa^n \;\middle| \; \dim V_i=i  \right\} 
%\end{aligned}
%\end{equation*}
%
%\begin{equation*}
%\begin{aligned}
%  \GL_n(\kappa)/P=\;& \Gr(r,n) \\
%  =\;& \left\{ V \subset \kappa^n \;\middle| \; \dim V=r  \right\} 
%\end{aligned}
%\end{equation*}
%
%\begin{equation*}
%\begin{aligned}
%  \GL_n(\kappa)/T
%  =\;& \left\{ \kappa^n= W_1\oplus \cdots \oplus W_n \;\middle| \; \dim W_i=1  \right\} 
%\end{aligned}
%\end{equation*}
%
%\end{frame}
\begin{frame}{Process}
\tableofcontents[hideallsubsections]
\end{frame}

%%%%%%%%%%%%%%%%%%%%%%%%%%%%%%%%%%%%%%%%%%%%%
\section{Spherical building}
\begin{frame}{Process}
\tableofcontents[currentsection,hideallsubsections]
\end{frame}

\begin{frame}[fragile]{Standard subgroups}
%\begin{table}[]
\[
\begin{array}{lll}
\visible<2->{B={\setlength{\arraycolsep}{2pt}\renewcommand{\arraystretch}{0.8}\begin{pmatrix}
* & * & * \\ &*&*\\&&*
\end{pmatrix}}} & \visible<3->{\!\!\rightsquigarrow\!\!} &  \visible<3->{\begin{aligned}
  \GL_n(\kappa)/B=\;& \left\{ \text{ flags of } \kappa^n  \right\} \\
  \visible<4->{=\;& \left\{ V_1 \subset \cdots \subset V_n = \kappa^n \;\middle| \; \dim V_i=i  \right\}} 
\end{aligned}}\\[8mm]
\visible<7->{P = {\setlength{\arraycolsep}{2pt}\renewcommand{\arraystretch}{0.8}
\left(
\begin{array}{c;{2pt/2pt}cc}
    * &  \;\;*\;\;  \\ \hdashline[2pt/2pt]
      &  \rule{0mm}{5mm}\;\;*\;\;   \\
\end{array}
\right)
}} & \visible<7->{\!\!\rightsquigarrow\!\!} & \visible<5->{\begin{aligned}
  \visible<6->{\GL_n(\kappa)/P=\;}& \Gr(r,n) \\
  \visible<5->{=\;}& \left\{ V \subset \kappa^n \;\middle| \; \dim V=r  \right\} 
\end{aligned}} \\[8mm]
\visible<8->{T = {\setlength{\arraycolsep}{1.5pt}\renewcommand{\arraystretch}{0.8}\begin{pmatrix}
* &  &  \\[-2mm] &\ddots&\\[-1mm]&&*
\end{pmatrix}}} & \visible<9->{\!\!\rightsquigarrow\!\!} &
\visible<9->{ \begin{aligned}
  \GL_n(\kappa)/T
  =\;& \left\{ \kappa^n= W_1\!\oplus \cdots \oplus\! W_n \;\middle| \; \dim W_i\!=\!1  \right\} }
\end{aligned}
\end{array}
\]
%\end{table}
\visible<10->{\begin{short}
$T$ is comm, so every rep decomposes as direct sum of $1$-dim reps.
\end{short}
}
\vspace{-5mm}
\begin{equation*}
\begin{aligned}
  \visible<11->{X^*(T):=\;& \Hom(T,\mathbb{G}_m) \cong \mathbb{Z}^n\quad} && \visible<12->{\text{\phantom{co}characters (1-dim reps)}} \\ 
  \visible<13->{X_*(T):=\;& \Hom(\mathbb{G}_m,T) \cong \mathbb{Z}^n} && \visible<14->{\text{cocharacters (1-parameter subgps)}}
\end{aligned}
\end{equation*}
\end{frame}

\begin{frame}[fragile]{Weyl group}
\visible<2->{\begin{definition}[Weyl group]
$\,$\\[-6mm]
$$W:= N_G(T)/T.$$
\end{definition}}
\visible<3->{\begin{eg}
When $G=\GL_n(\kappa)$, 
{
\setlength{\abovedisplayskip}{2pt}   % space above the display
\setlength{\belowdisplayskip}{2pt}
\begin{equation*}
\begin{aligned}
  N_G(T)\phantom{/T}=\;& \left\{ \text{ monoidal matrixes } \right\} \\ 
  N_G(T)/T \cong\;& \;S_n \qquad\qquad \text{\ignore{Weyl group of type $A$}} \\
\end{aligned}
\end{equation*}
}
\end{eg}}

\visible<4->{\begin{remark}
We have Bruhat decomposition \ignore{proved by Gauss elimination}
{
\setlength{\abovedisplayskip}{2pt}   % space above the display
\setlength{\belowdisplayskip}{2pt}
$$G=\bigsqcup_{\omega\in W} B \omega B.$$
}
So the Weyl group is the ``heart'' of the reductive group.
\end{remark}
}
\end{frame}

\begin{frame}[fragile]{Weyl group}

\begin{remark}
We have Bruhat decomposition \ignore{proved by Gauss elimination}
{
\setlength{\abovedisplayskip}{2pt}   % space above the display
\setlength{\belowdisplayskip}{2pt}
$$G=\bigsqcup_{\omega\in W} B \omega B.$$
}
So the Weyl group is the ``heart'' of the reductive group.
\end{remark}
 \visible<2->{ \begin{figure}
    \centering
    \includegraphics[height=35mm]{figure/Pinnedbutterfly.jpg}
    \caption{\href{https://jmilne.org/math/Books/iag.html}{Pinned butterfly}
    }
  \end{figure}}
\end{frame}


\begin{frame}[fragile]{Weyl group action on cocharacter lattices}
\visible<2->{
\begin{short}
 When $G=\GL_2(\kappa)$, $T= \left( \begin{smallmatrix}
a &  \\  & b 
\end{smallmatrix} \right)$, $X_*(T)=\mathbb{Z}\varepsilon_1 \oplus \mathbb{Z}\varepsilon_2$, where
\vspace{-5mm}
  \begin{figure}[th]
  \begin{minipage}[t]{.55\textwidth}
    \vspace{0pt}
  	\centering
  	 \visible<5->{\includegraphics[height=35mm]{figure/GL_2_apartment.png}}
  \end{minipage}
  \begin{minipage}[t]{.43\textwidth}
  $\,$\\[-4mm]
% https://q.uiver.app/#q=WzAsNSxbMCwwLCJcXHZhcmVwc2lsb25fMToiXSxbMSwwLCJcXG1hdGhiYntHfV9tIl0sWzIsMSwiXFxsZWZ0KCBcXGJlZ2lue3NtYWxsbWF0cml4fSB4ICYgIFxcXFwgICYgMSAgXFxlbmR7c21hbGxtYXRyaXh9IFxccmlnaHQpIl0sWzIsMCwiVCJdLFsxLDEsIngiXSxbNCwyLCIiLDAseyJzdHlsZSI6eyJ0YWlsIjp7Im5hbWUiOiJtYXBzIHRvIn19fV0sWzEsM11d
 \visible<3->{\[\begin{tikzcd}[ampersand replacement=\&,column sep=5mm,row sep=-2mm]
	{\varepsilon_1:} \&[-6mm] {\mathbb{G}_m} \& T \\
	\& x \& \begin{array}{c} \left( \begin{smallmatrix} x &  \\  & 1  \end{smallmatrix} \right) \end{array}
	\arrow[from=1-2, to=1-3]
	\arrow[maps to, from=2-2, to=2-3]
\end{tikzcd}\]
\[\begin{tikzcd}[ampersand replacement=\&,column sep=5mm,row sep=-2mm]
	{\varepsilon_2:} \&[-6mm] {\mathbb{G}_m} \& T \\
	\& x \& \begin{array}{c} \left( \begin{smallmatrix} 1 &  \\  & x  \end{smallmatrix} \right) \end{array}
	\arrow[from=1-2, to=1-3]
	\arrow[maps to, from=2-2, to=2-3]
\end{tikzcd}\]}
 \visible<4->{$$W=S_2=\{\Id,s_1 \}$$}
  \end{minipage}
  \end{figure}
  \vspace{-10mm}
\end{short}
}
 \visible<6->{
\begin{short}
When $G=\SL_2(\kappa)$, $T= \left( \begin{smallmatrix}
a &  \\  & a^{-1} 
\end{smallmatrix} \right)$, $X_*(T)=\mathbb{Z}\varepsilon$, where
\vspace{-5mm}
  \begin{figure}[th]
  \begin{minipage}[t]{.55\textwidth}
    \vspace{0pt}
  	\centering
  	 \visible<9->{\includegraphics[height=18mm]{figure/SL_2_apartment.png}}
  \end{minipage}
  \begin{minipage}[t]{.43\textwidth}
  $\,$\\[-4mm]
 \visible<7->{\[\begin{tikzcd}[ampersand replacement=\&,column sep=5mm,row sep=-2mm]
	{\varepsilon:} \&[-6mm] {\mathbb{G}_m} \& T \\
	\& x \& \begin{array}{c} \left( \begin{smallmatrix} x &  \\  & x^{-1}  \end{smallmatrix} \right) \end{array}
	\arrow[from=1-2, to=1-3]
	\arrow[maps to, from=2-2, to=2-3]
\end{tikzcd}\]}
\vspace{-5mm}
 \visible<8->{$$W=S_2=\{\Id,s_1 \}$$}
  \end{minipage}
  \end{figure}
\end{short}
}
\end{frame}

\begin{frame}[fragile]{Weyl group action on cocharacter lattices}
\begin{short}
When $G=\SL_2(\kappa)$, $T= \left( \begin{smallmatrix}
a &  \\  & a^{-1} 
\end{smallmatrix} \right)$, $X_*(T)=\mathbb{Z}\varepsilon$, where
\vspace{-5mm}
  \begin{figure}[th]
  \begin{minipage}[t]{.55\textwidth}
    \vspace{0pt}
  	\centering
  	\includegraphics[height=18mm]{figure/SL_2_apartment.png}
  \end{minipage}
  \begin{minipage}[t]{.43\textwidth}
  $\,$\\[-4mm]
\[\begin{tikzcd}[ampersand replacement=\&,column sep=5mm,row sep=-2mm]
	{\varepsilon:} \&[-6mm] {\mathbb{G}_m} \& T \\
	\& x \& \begin{array}{c} \left( \begin{smallmatrix} x &  \\  & x^{-1}  \end{smallmatrix} \right) \end{array}
	\arrow[from=1-2, to=1-3]
	\arrow[maps to, from=2-2, to=2-3]
\end{tikzcd}\]
\vspace{-5mm}
$$W=S_2=\{\Id,s_1 \}$$
  \end{minipage}
  \end{figure}
%  \vspace{-8mm}
\end{short}
\visible<2->{\begin{short}
When $G=\SL_3(\kappa)$, $T= \left( \begin{smallmatrix}
a &  \\  & b 
\end{smallmatrix} \right)$, $X_*(T)=\mathbb{Z}\varepsilon_1 \oplus \mathbb{Z}\varepsilon_2$, where
\vspace{-5mm}
  \begin{figure}[th]
  \begin{minipage}[t]{.55\textwidth}
    \vspace{0pt}
  	\centering
  	\visible<5->{\includegraphics[height=35mm]{figure/SL_3_apartment.png}}
  \end{minipage}
  \begin{minipage}[t]{.43\textwidth}
  $\,$\\[-4mm]
% https://q.uiver.app/#q=WzAsNSxbMCwwLCJcXHZhcmVwc2lsb25fMToiXSxbMSwwLCJcXG1hdGhiYntHfV9tIl0sWzIsMSwiXFxsZWZ0KCBcXGJlZ2lue3NtYWxsbWF0cml4fSB4ICYgIFxcXFwgICYgMSAgXFxlbmR7c21hbGxtYXRyaXh9IFxccmlnaHQpIl0sWzIsMCwiVCJdLFsxLDEsIngiXSxbNCwyLCIiLDAseyJzdHlsZSI6eyJ0YWlsIjp7Im5hbWUiOiJtYXBzIHRvIn19fV0sWzEsM11d
\visible<3->{\[\begin{tikzcd}[ampersand replacement=\&,column sep=5mm,row sep=-2mm]
	{\varepsilon_1:} \&[-6mm] {\mathbb{G}_m} \& T \\
	\& x \& \begin{array}{c} \left( \begin{smallmatrix} x & & \\ & x^{-1} & \\ & & 1  \end{smallmatrix} \right) \end{array}
	\arrow[from=1-2, to=1-3]
	\arrow[maps to, from=2-2, to=2-3]
\end{tikzcd}\]
\vspace{-3mm}
\[\begin{tikzcd}[ampersand replacement=\&,column sep=5mm,row sep=-2mm]
	{\varepsilon_2:} \&[-6mm] {\mathbb{G}_m} \& T \\
	\& x \& \begin{array}{c} \left( \begin{smallmatrix} 1 & & \\  & x & \\ & & x^{-1} \end{smallmatrix} \right) \end{array}
	\arrow[from=1-2, to=1-3]
	\arrow[maps to, from=2-2, to=2-3]
\end{tikzcd}\]}
\vspace{-3mm}
\visible<4->{$$W=S_3=\left< s_1,s_2 \right>$$}
  \end{minipage}
  \end{figure}
\end{short}}
\end{frame}

\begin{frame}[fragile]{Weyl group action on cocharacter lattices}
\begin{short}
When $G=\SL_2(\kappa)$, $T= \left( \begin{smallmatrix}
a &  \\  & a^{-1} 
\end{smallmatrix} \right)$, $X_*(T)=\mathbb{Z}\varepsilon$, where
\vspace{-5mm}
  \begin{figure}[th]
  \begin{minipage}[t]{.55\textwidth}
    \vspace{0pt}
  	\centering
  	\includegraphics[height=18mm]{figure/SL_2_apartment_Weyl.png}
  \end{minipage}
  \begin{minipage}[t]{.43\textwidth}
  $\,$\\[-4mm]
\[\begin{tikzcd}[ampersand replacement=\&,column sep=5mm,row sep=-2mm]
	{\varepsilon:} \&[-6mm] {\mathbb{G}_m} \& T \\
	\& x \& \begin{array}{c} \left( \begin{smallmatrix} x &  \\  & x^{-1}  \end{smallmatrix} \right) \end{array}
	\arrow[from=1-2, to=1-3]
	\arrow[maps to, from=2-2, to=2-3]
\end{tikzcd}\]
\vspace{-5mm}
$$W=S_2=\{\Id,s_1 \}$$
  \end{minipage}
  \end{figure}
\end{short}
\begin{short}
When $G=\SL_3(\kappa)$, $T= \left( \begin{smallmatrix}
a & & \\  & b & \\ & & a^{-1}b^{-1}
\end{smallmatrix} \right)$, $X_*(T)=\mathbb{Z}\varepsilon_1 \oplus \mathbb{Z}\varepsilon_2$, where
\vspace{-5mm}
  \begin{figure}[th]
  \begin{minipage}[t]{.55\textwidth}
    \vspace{0pt}
  	\centering
  	\includegraphics[height=35mm]{figure/SL_3_apartment_Weyl.png}
  \end{minipage}
  \begin{minipage}[t]{.43\textwidth}
  $\,$\\[-4mm]
% https://q.uiver.app/#q=WzAsNSxbMCwwLCJcXHZhcmVwc2lsb25fMToiXSxbMSwwLCJcXG1hdGhiYntHfV9tIl0sWzIsMSwiXFxsZWZ0KCBcXGJlZ2lue3NtYWxsbWF0cml4fSB4ICYgIFxcXFwgICYgMSAgXFxlbmR7c21hbGxtYXRyaXh9IFxccmlnaHQpIl0sWzIsMCwiVCJdLFsxLDEsIngiXSxbNCwyLCIiLDAseyJzdHlsZSI6eyJ0YWlsIjp7Im5hbWUiOiJtYXBzIHRvIn19fV0sWzEsM11d
\[\begin{tikzcd}[ampersand replacement=\&,column sep=5mm,row sep=-2mm]
	{\varepsilon_1:} \&[-6mm] {\mathbb{G}_m} \& T \\
	\& x \& \begin{array}{c} \left( \begin{smallmatrix} x & & \\ & x^{-1} & \\ & & 1  \end{smallmatrix} \right) \end{array}
	\arrow[from=1-2, to=1-3]
	\arrow[maps to, from=2-2, to=2-3]
\end{tikzcd}\]
\vspace{-3mm}
\[\begin{tikzcd}[ampersand replacement=\&,column sep=5mm,row sep=-2mm]
	{\varepsilon_2:} \&[-6mm] {\mathbb{G}_m} \& T \\
	\& x \& \begin{array}{c} \left( \begin{smallmatrix} 1 & & \\  & x & \\ & & x^{-1} \end{smallmatrix} \right) \end{array}
	\arrow[from=1-2, to=1-3]
	\arrow[maps to, from=2-2, to=2-3]
\end{tikzcd}\]
\vspace{-3mm}
$$W=S_3=\left< s_1,s_2 \right>$$
  \end{minipage}
  \end{figure}
\end{short}
\end{frame}


\begin{frame}[fragile]{Non-standard subgroups}
\visible<2->{
\begin{short}
The subgroup $T = {\setlength{\arraycolsep}{1.5pt}\renewcommand{\arraystretch}{0.8}\begin{pmatrix}
* &  &  \\[-2mm] &\ddots&\\[-1mm]&&*
\end{pmatrix}}$ is not the only maximal torus.
\end{short}}
\visible<3->{
\begin{fact}
All non-standard subgroups are conjugated to standard subgroups. Therefore,
\visible<4->{\begin{equation*}
\begin{aligned}
  \left\{ \text{ Borel subgroups } \right\} =\;& \left\{ gBg^{-1} \right\} \hspace{20mm}&&\visible<5->{\cong G/B} \\ 
  \left\{ \text{ parabolic subgroups } \right\} =\;& \left\{ gPg^{-1} \right\} &&\visible<6->{\cong G/P} \\ 
  \left\{ \text{ maximal tori } \right\} =\;& \left\{ gTg^{-1} \right\} &&\visible<7->{\cong G/N_G(T)} \\ 
\end{aligned}
\end{equation*}
}
\end{fact}
}
\end{frame}

\begin{frame}[fragile]{Non-standard subgroups}
\begin{short}
\vspace{-4mm}
\begin{equation*}
\begin{aligned}
  \left\{ \text{ Borel subgroups } \right\} =\;& \left\{ gBg^{-1} \right\} &&\cong G/B \\ 
  \left\{ \text{ parabolic subgroups } \right\} =\;& \left\{ gPg^{-1} \right\} &&\cong G/P \\ 
  \left\{ \text{ maximal tori } \right\} =\;& \left\{ gTg^{-1} \right\} &&\cong G/N_G(T) \\ 
  \visible<2->{\left\{ (B,T)\;\middle|\;B \supset T \right\} =\;& \left\{ (gB_0g^{-1},gT_0g^{-1}) \right\} &&\cong G/T_0  }
\end{aligned}
\end{equation*}
\end{short}
% https://q.uiver.app/#q=WzAsNSxbMCwwLCJcXGxlZnRcXHsgVCBcXDtcXG1pZGRsZXxcXDsgVCBcXHN1YnNldCBCXzAgIFxccmlnaHRcXH0iXSxbNCwwLCJcXGxlZnRcXHsgQiBcXDtcXG1pZGRsZXxcXDsgQiBcXHN1cHNldCBUXzAgIFxccmlnaHRcXH0iXSxbMiwxLCJcXGxlZnRcXHsgKEIsVCkgIFxccmlnaHRcXH0iXSxbMSwyLCJcXGxlZnRcXHsgQiAgXFxyaWdodFxcfSJdLFszLDIsIlxcbGVmdFxceyB0ICBcXHJpZ2h0XFx9Il0sWzIsM10sWzIsNF0sWzAsMiwiIiwyLHsic3R5bGUiOnsiaGVhZCI6eyJuYW1lIjoibm9uZSJ9fX1dLFsyLDEsIiIsMix7InN0eWxlIjp7ImhlYWQiOnsibmFtZSI6Im5vbmUifX19XV0=

\vspace{-5mm}
\visible<3->{
  \begin{minipage}[t]{.48\textwidth}
    \vspace{0pt}
  	\centering
\[\begin{tikzcd}[ampersand replacement=\&,column sep=0mm,row sep=5mm]
	{\scriptstyle\left\{ T \;\middle|\; T \subset B_0  \right\}} \&[-12mm]\&\&\&[-12mm] {\scriptstyle\left\{ B \;\middle|\; B \supset T_0  \right\}} \\[-3mm]
	\&\& {\left\{ (B,T)  \right\}} \\[2mm]
	\& {\left\{ B  \right\}} \&\& {\left\{ T  \right\}}
	\arrow[no head, from=1-1, to=2-3]
	\arrow[no head, from=2-3, to=1-5]
	\arrow[from=2-3, to=3-2]
	\arrow[from=2-3, to=3-4]
\end{tikzcd}\]
  \end{minipage}
  \begin{minipage}[t]{.48\textwidth}
    \vspace{0pt}
  	\centering
\[\begin{tikzcd}[ampersand replacement=\&,column sep=0mm,row sep=5mm]
	{\scriptstyle B_0/T_0} \&[-12mm]\&\&\&[-25mm] {\scriptstyle N_G(T_0)/T_0=W} \\[-3mm]
	\&\& G/T_0 \\[2mm]
	\& G/B_0 \&\& G/N_G(T_0)
	\arrow[no head, from=1-1, to=2-3]
	\arrow[no head, from=2-3, to=1-5]
	\arrow[from=2-3, to=3-2]
	\arrow[from=2-3, to=3-4]
\end{tikzcd}\]  	
  \end{minipage}}
\visible<4->{  $$\left\{ \text{ (Weyl) chambers } \right\} \quad\stackrel{1:1}{\longleftrightarrow}\quad W \quad\stackrel{1:1}{\longleftrightarrow}\quad  \left\{ B \;\middle|\; B \supset T_0  \right\}$$}

\end{frame}

\begin{frame}[fragile]{Weyl group action on cocharacter lattices(revisit)}
\begin{short}
When $G=\SL_2(\kappa)$, $T= \left( \begin{smallmatrix}
a &  \\  & a^{-1} 
\end{smallmatrix} \right)$, $X_*(T)=\mathbb{Z}\varepsilon$, where
\vspace{-5mm}
  \begin{figure}[th]
  \begin{minipage}[t]{.55\textwidth}
    \vspace{0pt}
  	\centering
  	\only<1>{\includegraphics[height=17mm]{figure/SL_2_apartment_Weyl.png}}
  	\only<2->{\includegraphics[height=18mm]{figure/SL_2_apartment_Borel.png}}
  \end{minipage}
  \begin{minipage}[t]{.43\textwidth}
  $\,$\\[-4mm]
\[\begin{tikzcd}[ampersand replacement=\&,column sep=5mm,row sep=-2mm]
	{\varepsilon:} \&[-6mm] {\mathbb{G}_m} \& T \\
	\& x \& \begin{array}{c} \left( \begin{smallmatrix} x &  \\  & x^{-1}  \end{smallmatrix} \right) \end{array}
	\arrow[from=1-2, to=1-3]
	\arrow[maps to, from=2-2, to=2-3]
\end{tikzcd}\]
\vspace{-5mm}
$$W=S_2=\{\Id,s_1 \}$$
  \end{minipage}
  \end{figure}
\end{short}
\begin{short}
When $G=\SL_3(\kappa)$, $T= \left( \begin{smallmatrix}
a & & \\  & b & \\ & & a^{-1}b^{-1}
\end{smallmatrix} \right)$, $X_*(T)=\mathbb{Z}\varepsilon_1 \oplus \mathbb{Z}\varepsilon_2$, where
\vspace{-5mm}
  \begin{figure}[th]
  \begin{minipage}[t]{.55\textwidth}
    \vspace{0pt}
  	\centering
  	\only<1-2>{\includegraphics[height=35mm]{figure/SL_3_apartment_Weyl.png}}
  	\only<3>{\includegraphics[height=35mm]{figure/SL_3_apartment_Borel.png}}
  	\only<4->{\includegraphics[height=35mm]{figure/SL_3_apartment_parabolic.png}}
  \end{minipage}
  \begin{minipage}[t]{.43\textwidth}
  $\,$\\[-4mm]
% https://q.uiver.app/#q=WzAsNSxbMCwwLCJcXHZhcmVwc2lsb25fMToiXSxbMSwwLCJcXG1hdGhiYntHfV9tIl0sWzIsMSwiXFxsZWZ0KCBcXGJlZ2lue3NtYWxsbWF0cml4fSB4ICYgIFxcXFwgICYgMSAgXFxlbmR7c21hbGxtYXRyaXh9IFxccmlnaHQpIl0sWzIsMCwiVCJdLFsxLDEsIngiXSxbNCwyLCIiLDAseyJzdHlsZSI6eyJ0YWlsIjp7Im5hbWUiOiJtYXBzIHRvIn19fV0sWzEsM11d
\[\begin{tikzcd}[ampersand replacement=\&,column sep=5mm,row sep=-2mm]
	{\varepsilon_1:} \&[-6mm] {\mathbb{G}_m} \& T \\
	\& x \& \begin{array}{c} \left( \begin{smallmatrix} x & & \\ & x^{-1} & \\ & & 1  \end{smallmatrix} \right) \end{array}
	\arrow[from=1-2, to=1-3]
	\arrow[maps to, from=2-2, to=2-3]
\end{tikzcd}\]
\vspace{-3mm}
\[\begin{tikzcd}[ampersand replacement=\&,column sep=5mm,row sep=-2mm]
	{\varepsilon_2:} \&[-6mm] {\mathbb{G}_m} \& T \\
	\& x \& \begin{array}{c} \left( \begin{smallmatrix} 1 & & \\  & x & \\ & & x^{-1} \end{smallmatrix} \right) \end{array}
	\arrow[from=1-2, to=1-3]
	\arrow[maps to, from=2-2, to=2-3]
\end{tikzcd}\]
\vspace{-3mm}
$$W=S_3=\left< s_1,s_2 \right>$$
  \end{minipage}
  \end{figure}
\end{short}
\end{frame}


\begin{frame}[fragile]
\begin{definition}[chamber, apartment and building]
Given a maximal torus $T$, the apartment is
$$\mathcal{A}_T:=X_*(T)_{\mathbb{R}} = \bigcup_{B \supset T} \mathcal{C}_B,$$
\visible<2->{and the building is 
$$\mathcal{B}:=\; \left(\bigsqcup_T \mathcal{A}_T\right)/_{\sim} \;= \;\bigcup_{B}\; \mathcal{C}_B.$$}
\end{definition}
\end{frame}

\begin{frame}[fragile]{Example of spherical building}
\visible<2->{
\begin{short}
When $G=\SL_2(\mathbb{F}_2)$, the building $\mathcal{B}$ has $3$ apartments and $3$ chambers. 
  \begin{figure}
    \centering
    \includegraphics[height=35mm]{figure/SL_2(F_2)_building.png}
    \caption{$\mathcal{B}_{\SL_2(\mathbb{F}_2)}$}
  \end{figure}
\end{short}}
\visible<3->{
\begin{short}
When $G=\SL_3(\mathbb{F}_2)$, the building \href{https://buildings.gallery/}{$\mathcal{B}$} has $28$ apartments and $21$ chambers. 
\end{short}
}
\end{frame}

\begin{frame}[fragile]
\visible<1->{\begin{remark}
$\mathcal{B}$ inherits the metric structure from $\mathcal{A}_T=X_*(T)_{\mathbb{R}}$.\\
\visible<2->{$\mathcal{B}$ has also polysimplicial complex structure.\\
When $\kappa=\mathbb{F}_p$, $\mathcal{B}$ is finite.}
\end{remark}}
\visible<3->{
\begin{proposition}
\begin{itemize}
\item Two chambers lie in one apartment.
\visible<4->{\item There is a unique geodesic passing any two points $p_1,p_2 \in \mathcal{B}$.}
\end{itemize}
}
\vspace{30mm}
\end{proposition}
\end{frame}

%%%%%%%%%%%%%%%%%%%%%%%%%%%%%%%%%%%%%%%%%%%%%
\section{p-adic building}
\begin{frame}{Process}
\tableofcontents[currentsection,hideallsubsections]
\end{frame}

\begin{frame}[fragile]{p-adic notation}
\centering
{\setlength{\arraycolsep}{50pt}\renewcommand{\arraystretch}{1.4}
\begin{table}[]
\begin{tabular}{l|l|l}
\hline
\multicolumn{1}{c|}{symbol}                          & \multicolumn{1}{c|}{name} & \multicolumn{1}{c}{example}        \\ \hline
$F$                                                  & NA local field               & \visible<2->{$\mathbb{Q}_p$}                     \\
$\mathcal{O}=\mathcal{O}_F$                          & integral ring             & \visible<2->{$\mathbb{Z}_p$}                     \\
$\mathfrak{p}=\mathfrak{p}_F$                        & maximal ideal             & \visible<2->{$p\mathbb{Z}_p$}                    \\
$\kappa=\mathcal{O}/\mathfrak{p}$                    & residue field             & \visible<2->{$\mathbb{F}_p$}                     \\
$\pi \in \mathfrak{p} \smallsetminus \mathfrak{p}^2$ & uniformizer               & \visible<2->{$p$}                                \\
$v:F^* \longrightarrow \mathbb{Z}$                   & valuation                 & \visible<2->{$v\left(\frac{a}{b} p^k \right)=k$} \\ \hline
\end{tabular}
\end{table}
}

\end{frame}

\begin{frame}[fragile]{standard subgroups in p-adic world}
\visible<2->{$$\pi: \GL_n(\mathcal{O}) \longrightarrow \GL_n(\kappa)$$}

\begin{equation*}
\begin{aligned}
  \visible<3->{I=\;& \pi^{-1}(B)= {\setlength{\arraycolsep}{2pt}\renewcommand{\arraystretch}{0.8}\begin{pmatrix}
  \mathcal{O} & \mathcal{O} & \mathcal{O} \\ \mathfrak{p}&\mathcal{O}&\mathcal{O}\\\mathfrak{p}&\mathfrak{p}&\mathcal{O}
  \end{pmatrix}}} \qquad&& \visible<4->{\text{Iwahori subgroup}} \\
  \visible<5->{\widetilde{P}=\;& \pi^{-1}(P)= {\setlength{\arraycolsep}{2pt}\renewcommand{\arraystretch}{0.8}
  \left(
  \begin{array}{c;{2pt/2pt}cc}
      \mathcal{O} &  \;\;\mathcal{O}\;\;  \\ \hdashline[2pt/2pt]
       \mathfrak{p} &  \rule{0mm}{5mm}\;\;\mathcal{O}\;\;   \\
  \end{array}
  \right)
  }} \qquad&& \visible<6->{\text{Parahoric subgroup}} \\ 
\end{aligned}
\end{equation*}
\visible<7->{\begin{remark}
They also have moduli intepretations. For example,
\begin{equation*}
\begin{aligned}
  \GL_n(F)/I\cong\;& \left\{ L=L_0 \subset L_1 \subset \cdots \subset L_n = \mathfrak{p}L  \;\middle| \; L_{i+1}/L_i \cong \kappa  \right\}\\
  = \;& \left\{ \mathcal{O}\text{-lattice chains in } F^n \right\}
\end{aligned}
\end{equation*}
\end{remark}}
\end{frame}

\begin{frame}[fragile]{Extended Weyl group}
\visible<2->{To get the Iwahori decompositionn
$$G(F)= \bigsqcup_{\varpi\in W_{\ext}} I \varpi I,$$
we define the extended Weyl group as
$$W_{\ext}:= N_G(T\ignore{(\mathcal{O})})/T\ignore{(\mathcal{O})} \cong X_*(T) \rtimes W_f.$$}
\visible<3->{\begin{eg}
When $G=\GL_n(F)$,
$$W_{\ext}=\left\{ \text{ monoidal matrixes } \right\} \Big/_{\setlength{\arraycolsep}{0pt}\renewcommand{\arraystretch}{0.8}\left(\begin{smallmatrix}
  \mathcal{O}^*\!\!\!\!\!\! & &  \\[-3mm] &\ddots&\\[-1mm] & &\!\!\mathcal{O}^*
  \end{smallmatrix}\!\right)} \cong \mathbb{Z}^n \rtimes S_n.$$
\end{eg}}
\end{frame}

\begin{frame}[fragile]{Extended Weyl group action}
\visible<2->{$W_{\ext}$ acts on $X_*(T)$ by ``twisted conjugation'':
$$W_{\ext} \times X_*(T) \longrightarrow X_*(T) \qquad (\mu \rtimes u, \lambda) \longmapsto \mu+u\lambda u^{-1}$$}
\visible<3->{\begin{short}
When $G=\SL_2(F)$, $W_{\ext}=\left< s_0, s_1 \right>$, where
\visible<4->{
$$s_1=\begin{pmatrix}
 & 1 \\ -1 &
\end{pmatrix} \qquad 
s_0=\begin{pmatrix}
 & \pi^{-1} \\ -\pi &
\end{pmatrix}
$$}
\vspace{-5mm}
  \begin{figure}
    \centering
    \only<5>{\includegraphics[width=\linewidth]{figure/SL_2(F)_apartment.png}}
    \only<6>{\includegraphics[width=\linewidth]{figure/SL_2(F)_apartment_Weyl.png}}
  \end{figure}
\end{short}
}
\end{frame}

\begin{frame}[fragile]{Extended Weyl group action}
\begin{short}
When $G=\SL_3(F)$, $W_{\ext}=\left< s_0, s_1, s_2 \right>$, where
\vspace{-3mm}
\visible<2->{$$s_1={\setlength{\arraycolsep}{0pt}\renewcommand{\arraystretch}{0.8}\begin{pmatrix}
 & \phantom{-}1 & \\ -1 & & \\ & & \phantom{-}1
\end{pmatrix} \quad 
s_2=\begin{pmatrix}
\,1\, &  & \\  & &\phantom{-}1 \\ & -1 & 
\end{pmatrix} \quad 
s_0=\begin{pmatrix}
 & & \phantom{-}\pi^{-1} \\ & \phantom{-}1 & \\ -\pi & &
\end{pmatrix}}
$$}
\vspace{-5mm}
  \begin{figure}
    \centering
    \only<1-3>{\visible<3>{\includegraphics[width=.45\linewidth]{figure/SL_3(F)_apartment.png}}}
    \only<4>{\includegraphics[width=.45\linewidth]{figure/SL_3(F)_apartment_Weyl.png}}
    \only<5>{\includegraphics[width=.45\linewidth]{figure/SL_3(F)_apartment_Weyl2.png}}
  \end{figure}
\end{short}
\end{frame}

\begin{frame}[fragile]{Extended Weyl group action}
  \begin{figure}[ht]
    \centering
    \includegraphics[width=.6\linewidth]{figure/Lievis.png}
    \caption{Reduced expressions labels, from  \href{https://www.jgibson.id.au/lievis/}{Lievis}}
  \end{figure}
\end{frame}

\begin{frame}[fragile]{Non-standard subgroups in p-adic world}
\visible<2->{\begin{short}
Similarly,
\vspace{-4mm}
\begin{equation*}
\begin{aligned}
  \left\{ \text{ Iwahori subgroups } \right\} =\;& \left\{ gI_0g^{-1} \right\} &&\cong G/I_0 \\ 
  \left\{ \text{ paraholic subgroups } \right\} =\;& \left\{ g\widetilde{P}_0g^{-1} \right\} &&\cong G/\widetilde{P}_0 \\ 
  \left\{ \text{ maximal tori over $\mathcal{O}$ } \right\} =\;& \left\{ gT_0g^{-1} \right\} &&\cong G/N_G(T_0\ignore{(\mathcal{O})}) \\ 
  \visible<3->{\left\{ (I,T)\;\middle|\;I \supset T \right\} =\;& \left\{ (gI_0g^{-1},gT_0g^{-1}) \right\} &&\cong G/T_0\ignore{(\mathcal{O})} } 
\end{aligned}
\end{equation*}
\end{short}}
\vspace{-7mm}
 \visible<4->{ \begin{minipage}[t]{.48\textwidth}
    \vspace{0pt}
  	\centering
\[\begin{tikzcd}[ampersand replacement=\&,column sep=0mm,row sep=5mm]
	{\scriptstyle\left\{ T \;\middle|\; T \subset I_0  \right\}} \&[-12mm]\&\&\&[-12mm] {\scriptstyle\left\{ I \;\middle|\; I \supset T_0  \right\}} \\[-3mm]
	\&\& {\left\{ (I,T)  \right\}} \\[2mm]
	\& {\left\{ I  \right\}} \&\& {\left\{ T  \right\}}
	\arrow[no head, from=1-1, to=2-3]
	\arrow[no head, from=2-3, to=1-5]
	\arrow[from=2-3, to=3-2]
	\arrow[from=2-3, to=3-4]
\end{tikzcd}\]
  \end{minipage}
  \begin{minipage}[t]{.48\textwidth}
    \vspace{0pt}
  	\centering
\[\begin{tikzcd}[ampersand replacement=\&,column sep=0mm,row sep=5mm]
	{\scriptstyle I_0/T_0} \&[-12mm]\&\&\&[-25mm] {\scriptstyle N_G(T_0)/T_0=W_{\ext}} \\[-3mm]
	\&\& G/T_0 \\[2mm]
	\& G/I_0 \&\& G/N_G(T_0)
	\arrow[no head, from=1-1, to=2-3]
	\arrow[no head, from=2-3, to=1-5]
	\arrow[from=2-3, to=3-2]
	\arrow[from=2-3, to=3-4]
\end{tikzcd}\]  	
  \end{minipage}}
 \visible<5->{ $$\left\{ \text{ chambers } \right\} \quad\stackrel{1:1}{\longleftrightarrow}\quad W_{\ext} \quad\stackrel{1:1}{\longleftrightarrow}\quad  \left\{ I \;\middle|\; I \supset T_0  \right\}$$ }
\end{frame}

\begin{frame}[fragile]{Comparison}
  \begin{figure}[ht]
    \centering
    \includegraphics[width=1.05\linewidth]{figure/comparison.png}
  \end{figure}
\end{frame}

\begin{frame}[fragile]{Extended Weyl group action(revisit)}
$W_{\ext}$ acts on $X_*(T)$ by ``twisted conjugation'':
$$W_{\ext} \times X_*(T) \longrightarrow X_*(T) \qquad (\mu \rtimes u, \lambda) \longmapsto \mu+u\lambda u^{-1}$$
\begin{short}
When $G=\SL_2(F)$, $W_{\ext}=\left< s_0, s_1 \right>$, where
$$s_1=\begin{pmatrix}
 & 1 \\ -1 &
\end{pmatrix} \qquad 
s_0=\begin{pmatrix}
 & \pi^{-1} \\ -\pi &
\end{pmatrix}
$$
\vspace{-5mm}
  \begin{figure}
    \centering
    \only<1>{\includegraphics[width=\linewidth]{figure/SL_2(F)_apartment_Weyl.png}}
    \only<2>{\includegraphics[width=\linewidth]{figure/SL_2(F)_apartment_Iwahori.png}}
    \only<3>{\includegraphics[width=\linewidth]{figure/SL_2(F)_apartment_Parahoric.png}}
  \end{figure}
\end{short}
\end{frame}

\begin{frame}[fragile]{Extended Weyl group action(revisit)}
\begin{short}
When $G=\SL_3(F)$, $W_{\ext}=\left< s_0, s_1, s_2 \right>$, where
\vspace{-3mm}
$$s_1={\setlength{\arraycolsep}{0pt}\renewcommand{\arraystretch}{0.8}\begin{pmatrix}
 & \phantom{-}1 & \\ -1 & & \\ & & \phantom{-}1
\end{pmatrix} \quad 
s_2=\begin{pmatrix}
\,1\, &  & \\  & &\phantom{-}1 \\ & -1 & 
\end{pmatrix} \quad 
s_0=\begin{pmatrix}
 & & \phantom{-}\pi^{-1} \\ & \phantom{-}1 & \\ -\pi & &
\end{pmatrix}}
$$
\vspace{-5mm}
  \begin{figure}
    \centering
    \only<1>{\includegraphics[width=.45\linewidth]{figure/SL_3(F)_apartment_Weyl2.png}}
%    \only<2>{\includegraphics[width=.45\linewidth]{figure/SL_3(F)_apartment_Iwahori.png}}
  \end{figure}
\end{short}
\end{frame}

\begin{frame}[fragile]{Extended Weyl group action(revisit)}
\begin{short}
  \begin{figure}
    \centering
    \only<1>{\includegraphics[width=.55\linewidth]{figure/SL_3(F)_apartment_Weyl3.png}}
    \only<2>{\includegraphics[width=.55\linewidth]{figure/SL_3(F)_apartment_Iwahori.png}}
	\only<3>{\includegraphics[width=.65\linewidth]{figure/SL_3(F)_apartment_parahoric.png}}
  \end{figure}
\end{short}
\end{frame}

\begin{frame}[fragile]{p-adic building}
\visible<2->{\begin{definition}[chamber, apartment and building]
Given a maximal torus $T$ over $\mathcal{O}$, the apartment is
$$\mathcal{A}_T:=X_*(T)_{\mathbb{R}} = \bigcup_{I \supset T} \mathcal{C}_I,$$
and the p-adic building is 
$$\mathcal{B}:=\; \left(\bigsqcup_T \mathcal{A}_T\right)/_{\sim} \;= \;\bigcup_{I}\; \mathcal{C}_I.$$
\end{definition}
}
\visible<3->{
\begin{remark}
Similarly, two chambers lie in one apartment,\\
and there is a unique geodesic passing $p_1,p_2 \in \mathcal{B}$.
\end{remark}
}
\end{frame}

%%%%%%%%%%%%%%%%%%%%%%%%%%%%%%%%%%%%%%%%%%%%%
\section{Gromov-Schoen theorem}
\begin{frame}{Process}
\tableofcontents[currentsection,hideallsubsections]
\end{frame}

\begin{frame}[fragile]{Gromov-Schoen theorem}
\visible<2->{\begin{theorem}
Let $F$ be a NA local field, $(M,g)$ be a cpt conn Riemannian manifold with the universal covering space $\widetilde{M}$.\\[3mm]

\visible<3->{For any reductive map
$$\rho: \pi_1(M) \longrightarrow \GL_n(F),$$
\visible<4->{there exists a $\pi_1(M)$-equivariant Lipschitz continuous regular harmonic map
$$h_{\rho}:\widetilde{M} \longrightarrow \mathcal{B}_{\GL_n(F)}$$
\vspace{-5mm}}}
\end{theorem}
\visible<5->{We call $\rho$ reductive when $\overline{\rho(\pi_1(M))}^{\Zar} \subseteq \GL_n(F)$ is reductive.
}
}
\end{frame}

\begin{frame}[fragile]{regularity}
\visible<2->{\begin{definition}
$h_{\rho}$ is regular at $x \in \widetilde{M}$ if 
$$\text{a neighbourhood of $x$ has image inside an apartment $\mathcal{A}_T$ of $\mathcal{B}$.}$$
\visible<3->{$h_{\rho}$ is regular if 
$$\codim_{\widetilde{M}} \left\{ x \in \widetilde{M}\;\middle|\;\text{ $h_{\rho}$ is not regular at $x$ } \right\} \geqslant 2.$$}
\vspace{-6mm}
\end{definition}
}
\visible<4->{\begin{eg}
The map
$$f: \mathbb{R}^2 \longrightarrow \left\{ y^2=x^2 \right\} \qquad (a,b) \longmapsto (a|b|,b|a|)$$
is regular.
\end{eg}}
\end{frame}

\begin{frame}[fragile]{Thanks for listening!}
You can get this slide at: \\
\href{https://github.com/ramified/personal_tex_collection/blob/main/Bruhat--Tits_building/Bruhat--Tits_building.pdf}{https://github.com/ramified/personal\_tex\_collection/raw/main/\\Bruhat--Tits\_building/Bruhat--Tits\_building.pdf}
\end{frame}


\end{document}


%%%\stress -> \stress


%\begin{equation*}
%\begin{aligned}
%内容...
%\end{aligned}
%\end{equation*}
